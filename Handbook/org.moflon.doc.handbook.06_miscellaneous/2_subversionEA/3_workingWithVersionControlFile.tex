\subsection{Working with a version-controlled EAP file}

\begin{enumerate}
  \item[$\blacktriangleright$] A \texttt{Check Out} retrieves the lock for a certain package and gives you exclusive access, i.e., no one else can change the
  package. Very important: if subpackages are also under version control, they are not affected by checking out the ``super''-package and remain locked.
  A \texttt{Check Out} also updates the package to the latest version.

\item[$\blacktriangleright$] A \texttt{Check In} commits your work to the server and gives up the lock on the package so others can work on it.
If you do not want to commit your changes, you can just use \texttt{Undo Check Out...} to revert all local changes.

\item[$\blacktriangleright$]  The corresponding \texttt{..Branch} options perform the actions for the current package and all subpackages.
Please note, this has nothing to do with ``branching'' in normal SVN lingo.

\item[$\blacktriangleright$] \texttt{Get Latest/Get All Latest} retrieves the latest version of the selected package / all packages.
This is basically an update but does not retrieve the lock for any package.

\item[$\blacktriangleright$] Conversely, \texttt{Put Latest} saves all your changes without giving up any locks.

\item[$\blacktriangleright$] \texttt{Compare with controlled version} can be used to review incoming changes.
Green elements will be added, red will be deleted.

\item[$\blacktriangleright$] \texttt{File History} gives you a summary of all commits made while you were lying on the beach.
For a useful file history, always use meaningful commit statements when checking in!
A date stamp is created automatically.
\end{enumerate}
