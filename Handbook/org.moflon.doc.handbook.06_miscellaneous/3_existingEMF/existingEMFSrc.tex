\newpage

\section{Using existing EMF projects in eMoflon}
\visHeader

This chapter contains stepwise instructions on how to use existing \mbox{EMF}/Ecore projects with an eMoflon project specified using the visual syntax via EA.
We will present an example of an existing metamodel which must be integrated with eMoflon before, for example, its transformation using SDMs can be
specified. The basic workflow for using an existing EMF project in eMoflon is described in the following and may of course be similarly applied to a metamodel
specified in the textual syntax via MOSL. 

We will begin by implementing a small subset of the \texttt{Ecore -> GenModel} transformation, where \texttt{GenModel} is part of the EMF/Ecore standard. The
\emph{GenModel} for a given Ecore model can be viewed as a \emph{wrapper} that contains additional generation-specific Java code details. These details are
separated from the Ecore model to keep it free of such ``low-level'' information and settings.

\subsection{Modelling relevant aspects in EA}
\visHeader

The first step is to get the existing metamodel into EA. A complete and automatic import of existing Ecore files in EA is currently not possible and therefore,
\emph{relevant parts} of the existing metamodel (\texttt{GenModel}) have to be modelled manually in EA. Although this might sound frightening (especially for
large, complex metamodels), the emphasis here on \emph{relevant} indicates that only elements that are used for the transformation have to be present in EA and
can be added iteratively as the transformation grows.

\begin{enumerate}

\item[$\blacktriangleright$] To specify our example transformation, open Eclipse and create a new metamodel project named \texttt{EcoreToGenModel}, and select
the \texttt{Add Demo Specification} option in the project wizard window. 

\item[$\blacktriangleright$] You can either delete or ignore the \texttt{DemoTestSuite} project raising errors, and switch to EA by double-clicking the created
\texttt{Ecore\-To\-Gen\-Model.eap} file.

\item[$\blacktriangleright$] Explore the project browswer and make of note of the packages already present in EA under \texttt{eMoflon Languages}, especially
\texttt{Ecore} which we shall use for this transformation.

\item[$\blacktriangleright$] Create a new package named \texttt{GenModel} in \texttt{eMoflon Languages}, and add a new Ecore diagram. Model the elements as
depicted in Fig.~\ref{fig_gMM}. You'll need to create the three EClasses on the left, but \texttt{Ecore::EPackage} and \texttt{Ecore::EClass} can be
drag-and-dropped, pasted as links from the project browser. 

\newpage

\begin{figure}[htbp]
\begin{center}  
	\includegraphics[width=1.0\textwidth]{CDGenmodel.pdf}
	\caption{Metamodel of \texttt{GenModel}}  
\label{fig_gMM}
\end{center}
\end{figure} 

\item[$\blacktriangleright$] Please note that the actual \texttt{GenModel} metamodel contains lots more elements, but this subset is sufficient for our task.

\item[$\blacktriangleright$] Create another \texttt{Ecore2GenModel} package and digram to contain the \texttt{Transformer} class with the methods as depicted in
Fig.~\ref{fig_e2gm}.

\item[$\blacktriangleright$] Implement their SDMs as depicted in Figs.~\ref{fig_pack2gm} and \ref{fig_transf}.
\end{enumerate}

\begin{figure}[htbp]
\begin{center}  
\includegraphics[width=0.8\textwidth]{CDTransformer.pdf}
\caption{Methods in \texttt{Transformer}}  
\label{fig_e2gm}
\end{center}
\end{figure} 

\begin{figure}[htbp]
\begin{center}  \includegraphics[width=0.8\textwidth]{SDMePackageToGenModel.pdf}
        \caption{Main method for \texttt{EPackage} to \texttt{GenModel} transformation}  
  \label{fig_pack2gm}
\end{center}
\end{figure} 

\begin{figure}[htbp]
\begin{center}  
\includegraphics[width=1.0\textwidth]{SDMtransformEpackageToGenPackage.pdf}
\caption{Helper function to transform all \texttt{EPackages} to \texttt{GenPackages}}  
\label{fig_transf}
\end{center}
\end{figure} 



\newpage

\subsection{Configuration for code generation in Eclipse}
\visHeader

As there is already generated code (provided via a plugin in Eclipse) for the existing \texttt{GenModel} metamodel, we do \emph{not} want to export our
incomplete subset of \texttt{GenModel} in EA.

\begin{enumerate}
\item[$\blacktriangleright$] To prevent this, right-click the \texttt{GenModel} package in EA and select ``Properties/Moflon'' and change the tagged value
\texttt{Moflon::Export} to \texttt{false} (Fig.~\ref{fig_customNS}).
\end{enumerate}

\begin{figure}[htb]
\begin{center}  \includegraphics[width=\textwidth]{8_nsUriPre_edited}
  \caption{Update the \texttt{GenMode} export option and create custom tags}  
  \label{fig_customNS}
\end{center}
\end{figure}

Furthermore, we have to set the ``real'' name and URI of the project to be used in Eclipse so that references are exported properly. 

\begin{enumerate}

\item[$\blacktriangleright$] In the ``Properties/Moflon" dialogue for \texttt{GenModel}, create the new tagged values \texttt{Moflon::CustomNsPrefix} and
\texttt{Moflon::CustomNsUri} and set them according to Fig.~\ref{fig_customNS}. These values can be determined by inspecting the corresponding values in the
existing .ecore file (i.e.,~the existing metamodel).

\item[$\blacktriangleright$] Export all projects as usual to your Eclipse workspace and update the metamodel project by pressing \texttt{F5}.

\item[$\blacktriangleright$] Convert the generated Eclipse project \texttt{Ecore2GenModel} to a \emph{plugin project} by right-clicking the project and
selecting ``Configure/Convert to Plug-in Projects...''. This makes it easier to set the required dependencies for code generation.

\item[$\blacktriangleright$] Now right-click \texttt{Ecore2GenModel} and choose ``Plug-in Tools/Open Manifest''. In the window that opens up, choose the
\texttt{Dependencies} tab, click \texttt{Add}, and type in \texttt{org.eclipse.emf.codegen.ecore} (which includes both the \texttt{Ecore} and \texttt{GenModel}
libraries as required).

\end{enumerate}

Although we have already specified the name and URI of the existing project (in our case \texttt{GenModel}) in EA, we now have to tell eMoflon where to find the
implementation (generated code) for the existing project.

\begin{enumerate}
\item[$\blacktriangleright$] Open the \texttt{moflon.properties} file located in your project folder and insert the following lines:\\
\end{enumerate}

\vspace{-1cm}
{\small \ttfamily \hspace{-2.5cm} ADDITIONAL\_DEPENDENCIES=platform:/plugin/org.eclipse.emf.codegen.ecore/model/GenModel.ecore} \\
\vspace{0.75cm}%
{\small \ttfamily \hspace{-2.5cm} ADDITIONAL\_USED\_GEN\_PACKAGES=platform:/plugin/org.eclipse.emf.codegen.ecore/model/GenModel.genmodel}

Finally, to compenstate for some cases where our naming conventions were violated, add the following mappings as corrections:

\begin{enumerate}
\item[$\blacktriangleright$] An \emph{import mapping} for correct generation of the required import:

{\small \ttfamily IMPORT\_MAPPINGS=genmodel-> org.eclipse.emf.codegen.ecore.genmodel}


\item [$\blacktriangleright$] A \emph{factory mapping} to ensure that \texttt{GenModelFactory} is used as the factory for creating elements in the
transformation instead of \texttt{Genmodel\-Factory}, which would be the default convention:

{\small \ttfamily FACTORY\_MAPPING=genmodel-> GenModelFactory}

\end{enumerate}


\newpage

Your \texttt{moflon.properties} file should now closely resemble Fig.~\ref{fig_mofProp}. Generate code one more time for the project and ensure (with a JUnit
test) that the transformation behaves as expected.

\vspace{0.5cm}

\begin{figure}[htbp]
\hspace*{-2cm}
\includegraphics[width=1.5\textwidth]{9_mofProperties}
  \caption{Additional properties for code generation}  
  \label{fig_mofProp}
\end{figure} 

