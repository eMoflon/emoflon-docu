\newpage
\subsection{Checking Cards (Textual)}
\texHeader
\hypertarget{checkCard tex}{}

 % introduce the tricks in THIS section?
 
 \texttt{}
 \emph{}
 
 \begin{itemize}
   
\item[$\blacktriangleright$] This time, we need to create three separate patterns - an assertion pattern, a pattern to promote a card, and another to return.
But, given that this is an assertion, we need to generate proper control flow; we need to create an \texttt{if/else}

\item[$\blacktriangleright$] in the \texttt{check} declaration, enter the following code: (Fig) \\ 
if \texttt{checkCard}  { \\ \texttt{promoteCard} return true \\ } else \\ \texttt{ [penalizeCard] return false} \\

\item[$\blacktriangleright$] Using the ``Quick Fix'' wizard, generate the pattern files for all three. Your package explorer should now resemble Fig.

\item[$\blacktriangleright$] In order to validate the guess, go to the \texttt{checkCard} pattern, and create a rule bound to the passed-in \texttt{card} (Fig).

\item[$\blacktriangleright$] Now that we have the card to be checked, we need to compare the user's guess to the value on the back of the card. To do this, we
need to specify an \emph{Attribute Constraint}. This is a non-structural condition that must be satisfied for a story pattern to match.

\item[$\blacktriangleright$] ATTRIB CONSTRAINT + RETURN == click to proceed to figure it out. (and then descr in the tutorial\ldots)

\item[$\blacktriangleright$] Promote card: three rules. create a bounded \texttt{card : Card} and  \texttt{this : Partition}, and a free  \texttt{next :
Partition} variable.

\item[$\blacktriangleright$] Just like \texttt{removeCard}, in the \texttt{@this} rule, enter  \texttt{--> card : card}. In order to create a new reference and
place the card in its new partition, enter  \texttt{++ -> card : Card} under \texttt{next}. Because a link was set to update a card anytime a partition
deletes or creates a reference to it, it would be redundant to put any commands under \texttt{card}.

\item[$\blacktriangleright$] Be sure to check each of your eclasses to make sure the reference names are correct, or else nothing will be generated.	

\item[$\blacktriangleright$] \texttt{penalizeCard} has the Same variable/rule set up as \texttt{promoteCard}, except the creation and deletion are opposite,
and a \texttt{previous : Partition}.

\item[$\blacktriangleright$] The final thing we need to do is return a final boolean statement depending on how the card was moved.

\item[$\blacktriangleright$] View the changes in ``PartitionImpl.Java''

 \end{itemize}
 