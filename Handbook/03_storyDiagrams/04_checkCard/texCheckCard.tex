\newpage
\hypertarget{checkCard tex}{}
\subsection{Implementing check}
\texHeader
 
\begin{itemize}
   
\item[$\blacktriangleright$] In this SDM, the guess assertion, card promotion, and card penalisation must each be implemented as patterns. Given that every
action is determined a the result of a conditional statement, we need an \emph{if/else} construct!

\item[$\blacktriangleright$] In the \texttt{check} method, create the basic \emph{if/else} construct with three patterns, as illustrated in
Fig.~\ref{fig:checkDec}. The auto completion feature includes a template for this.

\vspace{0.5cm}

\begin{figure}[htbp]
\begin{center}
  \includegraphics[width=0.65\textwidth]{eclipse_checkCardMethod}
  \caption{An if/else construct in \texttt{check} \update}
  \label{fig:checkDec}
\end{center}
\end{figure} 

\item[$\blacktriangleright$] Upon saving use the ``Quick Fix'' wizard again to generate the required pattern files. Your package explorer should now resemble
Fig.~\ref{fig:checkPatternsExplorer}.

\begin{figure}[htbp]
\begin{center}
  \includegraphics[width=0.55\textwidth]{eclipse_checkPackageExplorer}
  \caption{Project directory after creating patterns}
  \label{fig:checkPatternsExplorer}
\end{center}
\end{figure} 

\item[$\blacktriangleright$] Open the \texttt{checkCard} pattern. In order to validate the user's guess, we need to establish an \texttt{attribute contstraint}
between the \texttt{back} attribute of the current \texttt{card}, and the primitive \texttt{EString} parameter, \texttt{guess}. 

\item[$\blacktriangleright$] Referencing the \texttt{back} attribute can be done via `dot' operator, as used in Java.

\item[$\blacktriangleright$] Attribute constraints have similar operators as Java comparators, so we'll want to use \texttt{`=='} to equate the
values.\footnote{See the \hyperlink{quickRef}{Quick Reference} for a listing of all operators}

\item[$\blacktriangleright$] The tricky part of the statement is referencing the parameter value. Given that we have not created an object variable, we'll need
to use a \texttt{ParameterExpression}\define{ParameterExpression}to access it. Its syntax is as follows: \syntax{parameter\_expression := `\$'ID}

\item[$\blacktriangleright$] All combined, the complete attribute constraint statement is: \syntax{card.back == \$guess} Your pattern should now resemble
Fig.~\ref{fig:checkPattern}.

\begin{figure}[htbp]
\begin{center}
  \includegraphics[width=0.5\textwidth]{eclipse_checkPattern}
  \caption{Completed \texttt{checkCard} pattern}
  \label{fig:checkPattern}
\end{center}
\end{figure} 

\clearpage

\item[$\blacktriangleright$] Now let's specify the \texttt{promoteCard} pattern. It requires three object variables: the current partition (\texttt{this}),
the card to be promoted, and the partition the card will move to. Open the file, and the file so that it resembles Fig.~\ref{fig:promoteCardPattern}.

\begin{figure}[htbp]
\begin{center}
  \includegraphics[width=0.5\textwidth]{eclipse_promoteCardPattern}
  \caption{Object variables for promoting a card}
  \label{fig:promoteCardPattern}
\end{center}
\end{figure} 

\item[$\blacktriangleright$] Given that \texttt{this} is bound while \texttt{next} partition is free, we need to establish a reference between the \texttt{box}
and partition. This will allow the card move around. Create a simple outgoing link variable, \texttt{next},  with target object variable \texttt{nextPartition}.
Your file should now resemble Fig.~\ref{fig:promoteThisRule}.

\vspace{0.5cm}

\begin{figure}[htbp]
\begin{center}
  \includegraphics[width=0.5\textwidth]{eclipse_promoteCardThisRule}
  \caption{The \texttt{@this} object variable}
  \label{fig:promoteThisRule}
\end{center}
\end{figure} 

\item[$\blacktriangleright$] Finally, let's update the \texttt{cardContainer} reference of \texttt{card}. Simply delete the link to \texttt{this}, and create
a link to \texttt{nextPartition}.\footnote{There are several different ways you could have implemented this movement, updating the link from
\texttt{nextPartition} to \texttt{card}.} Your pattern is now complete, and should resemble Fig.~\ref{fig:completedPromote}.

\begin{figure}[htbp]
\begin{center}
  \includegraphics[width=0.6\textwidth]{eclipse_promoteCardCompleted}
  \caption{The finished \texttt{promoteCard} pattern}
  \label{fig:completedPromote}
\end{center}
\end{figure} 

\item[$\blacktriangleright$] Compare the differences between the \texttt{promoteCard} and \texttt{penalizeCard} concepts. They both do the exact same thing,
except the destination partition is different. Knowing this, try to complete \texttt{penalizeCard} entirely on your own. Your workspace should come to
resemble Fig.~\ref{fig:completedPatterns}.

\vspace{0.5cm}

\begin{figure}[htbp]
\begin{center}
  \includegraphics[width=\textwidth]{eclipse_movementPatternsCompleted}
  \caption{The finished movement patterns}
  \label{fig:completedPatterns}
\end{center}
\end{figure}


\item[$\blacktriangleright$] Did you notice that the order of the \texttt{this} and \texttt{card} object variable scopes are reversed in the figures above?
In general, for simple transformations like this, it doesn't matter in which order object variables are specified, or link variables manipulated in patterns.
Everything just needs to be correctly stated.

\clearpage

\item[$\blacktriangleright$] We nearly forgot to complete the control flow! We have specified the assertion and card movements, but we haven't returned a
\texttt{EBoolean} result as the method requires. This is also known as a \emph{LiteralExpression}.\define{LiteralExpression}This expression type can be used to
specify arbitrary text, but should really only be used for true literals like 42, ``foo'' or \texttt{true}. The syntax for any LiteralExpression is simply:
\syntax{LiteralExpression := boolean\_literal | integer\_literal | any\_literal}

\vspace{0.5cm}

\item[$\blacktriangleright$] Knowing this, and that \texttt{guess} was successful
when the card was promoted, return \texttt{true} beneath \texttt{promoteCard}. If it was penalized, the guess was \texttt{false}. Your \texttt{check} method
should now resemble Fig.\ref{fig:finalMethod}.

\vspace{0.5cm}

\begin{figure}[htbp]
\begin{center}
  \includegraphics[width=0.7\textwidth]{eclipse_checkMethodFinal}
  \caption{Completed control flow for \texttt{check} \update}
  \label{fig:finalMethod}
\end{center}
\end{figure}

\item[$\blacktriangleright$] Save and build your metamodel, then try viewing the changes in ``PartitionImpl.Java'' under \texttt{check}. You'll be able to
see the if/else construct, as well as the link manipulations. 

\vspace{0.5cm}

\item[$\blacktriangleright$] Great job! You have just implemented checking card guesses with a new control flow construct, \emph{if/else}, and two patterns that
move the cards appropriately. To see how this SDM is implemented visually, check out Fig.~\ref{fig:sdm_check_finish} for the control flow, and
Figs.~\ref{fig:sdm_check_complete_activity_node} and \ref{fig:sdm_check_complete_penalize} for the movement patterns, all in the previous section.

\end{itemize}
 