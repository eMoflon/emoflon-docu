\genHeader

{\scriptsize \texttt{Approximate time to complete: ?? minutes} }

% Rewrite and expand!
% Explain WHY we use SDMs?

% motivation for SDMS - from Part I. Make less jumpy..
Although getting the \emph{details} of mapping the static structure of our models to Java might be extremely difficult, it is actually straight forward.
A fantastic productivity boost in any case but (yawn) not exactly exciting.

% So we have declared n methods.. (uncessary?)
In this part, we introduce unidirectional model transformations via programmed graph transformations using Story Driven Modeling (SDM). SDMs focus on concrete
implementations, so we'll implement the methods signatures we previously declared as part of the system's abstract syntax. It other words, we'll build our
metamodels dynamic semantics! Don't let the sheer size of this part frighten you off - we've included deep, thorough explanations and an ample number of
images to ensure clarity. It's really not that bad.

We learned in the previous part that we can implement methods with fairly straightforward injections so, why bother with SDMs?

Visually, the advantage is obvious. Its a clear distinction from the original class diagram, and by using familiar UML conform activity diagrams, its easy to
understand.

Textually, it abides by the extended Object Oriented Paradigm, Model-Oritented Programming, by separating all classes, links, and activities into separate
files, folders, and patterns.

Overall, SDMs simply provide an alternate way to implementing methods, rather than verbose Java code.

If you're just joining us, read the next this section for a brief overview of the example so far and how to download the files that will let you start right away. 
If you're continuing from Part II, click the link below to continue with your constructed learning box model.

\begin{center}\texttt{$\triangleright$ \hyperlink{explanation}{Continue from Part II\ldots}}\end{center}
