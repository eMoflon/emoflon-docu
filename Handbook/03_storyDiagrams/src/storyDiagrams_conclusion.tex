\genHeader
\section{Conclusion and next steps}

\vspace{0.5cm}

Congratulations - you've reached the end of eMoflon's introduction to unidirectional model transformations! You've learnt that SDMs are declared inside method
\emph{activities}, which can be comprised of different \emph{activities} (\emph{pattern}s) and \emph{statements} (method calls). \emph{Patterns} are made up of
various \emph{Object} and \emph{Link Variables} with their own set of \texttt{rules}. These variables can be given different \emph{binding states}, depending on
their purpose. You have also learnt how to make a \emph{deterministic} pattern matcher with \emph{NACs}.

\vspace{0.5cm}

To test your story driven modelling skills, challenge yourself to try:
\begin{itemize}
\item Creating a method/pattern to eject the card from the final partition (to signal it's been learnt)
\item Editing \texttt{Partition.empty()} to include a method call to \texttt{removeCard} to reuse this previous SDM
\item Modifying the GUI source files to execute more methods
\end{itemize}

\vspace{0.5cm}
	
If you have any comments, suggestions, or concerns for this part, feel free to drop us a line at \href{mailto:contact@moflon.org}{contact@moflon.org}.
Otherwise, if you enjoyed this section, continue to Part IV to learn about Triple Graph Transformations, or even onto Part V for Model-to-Text Transformations.
The final part of this handbook (Part VI: Miscellaneous), goes over tips and tricks in EA which you might find useful when creating SDMs in the future. 

For more detailed information on each part, please refer to Part 0, which can be downloaded at \dlPartZero.
\vspace{0.5cm}

Cheers!
