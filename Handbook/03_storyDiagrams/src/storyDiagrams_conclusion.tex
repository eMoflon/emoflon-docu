\genHeader
\section{Conclusion and next steps}
\hypertarget{conclusion}{}

\vspace{0.5cm}

Congratulations - you've reached the end of eMoflon's unidirectional model transformations! You've learned that SDMs are declared inside method
\emph{activities}, are comprised of different \emph{pattern}s which themselves are made up of various \emph{Object} and \emph{Link Variables}. These
variables can be given different \emph{binding states}, depending on their purpose. You have also learned how to build \emph{NACs}, which make pattern
matches \emph{deterministic} by construction.

\vspace{0.5cm}

To test your Story Driven Modelling skills, challenge yourself to try:
\begin{itemize}
\item Creating a method/pattern to eject the card from the final partition (to signal it's been learnt)
\item Edit \texttt{Partition.empty} to include a method call to \texttt{removeCard} so you can reuse a previous SDM
\item Modifying the GUI source files to execute the other methods we have now established
\end{itemize}

\vspace{0.5cm}
	
If you have any comments, suggestions, or concerns for this part, feel free to drop us a line at \href{mailto:contact@moflon.org}{contact@moflon.org}.
Otherwise, if you enjoyed this section, continue to Part IV to learn about Triple Graph Transformations, or even onto Part V for Model-to-Text Transformations.
The final part of this handbook (Part VI, Miscellaneous), goes over tips and tricks in EA which you might find useful when creating SDMs in the future. As
always however, refer to Part 0 for a detailed description on each part of our handbook.

\vspace{0.5cm}


Cheers!
