\newpage
\section{Reviewing eMoflon's expressions}
\genHeader

We've encountered several different expression types throughout our SDMs so far, and all of them can be used for binding expresssions. Since each syntax has
used at least three of these once over the course of completing the metamodel, let's consider what each type would mean in this context:

\begin{description}
  
  \item[LiteralExpression:]~\\ 
  As usual this can be anything and is literally copied with a surrounding typecast 
  into the generated code.  Using \emph{LiteralExpression}s too often is usually a sign 
  for not thinking in a pattern oriented manner and is considered a \emph{bad smell}.
  
  \item[MethodCallExpression:]~\\ 
  This would allow invoking a method and binding its return value to the object variable.  
  This is how non-primitive return values of methods can be used safely in SDMs.
  
  \item[ParameterExpression:]~\\ 
  This could be used to bind the object variable to a parameter of the method.  
  If the object variable is of a different type than the parameter (e.g. a subtype), 
  this represents basically a successful typecast if the pattern matches.
  
  \item[ObjectVariableExpression:]~\\ 
  This can be used to refer to other object variables in preceding story nodes.  
  Just like \emph{ParameterExpression}s, this represents a simple typecast if the 
  types of the \texttt{target} and the object variable with the binding are different.

\end{description}

For those using the textual syntax, we have provided a \hyperlink{quickRef}{quick-reference guide} at the end of this part, along with the structure of
several other commands encountered so far. Part VI, Miscellaneous, contains a complete reference of all MOSL commands.
