\newpage
\hypertarget{invertCard tex}{}
\subsection{Implementing invert}
\texHeader

\begin{itemize}

\item[$\blacktriangleright$] You're no longer an SDM beginner, so create a simple control flow with two patterns in \texttt{card.eclass} under the invert
declaration, named \texttt{initializeTemp} and \texttt{swapVariable}. Don't forget to include a return statement!

\item[$\blacktriangleright$] Your declaration should now resemble Fig.~\ref{fig:eclipse_invert}

\begin{figure}[htbp]
\begin{center}
  \includegraphics[width=0.4\textwidth]{eclipse_invertControlFlow}
  \caption{Control flow for \texttt{card.invert}}  
  \label{fig:eclipse_invert}
\end{center}
\end{figure}

\item[$\blacktriangleright$] Create \texttt{this} and \texttt{temp} \texttt{Card} variables in each pattern until your workspace resembles
Fig.~\ref{fig:invertPatterns}.

\begin{figure}[htbp]
\begin{center}
  \includegraphics[width=0.5\textwidth]{eclipse_invertPatterns}
  \caption{Swapping the \texttt{card} values}  
  \label{fig:invertPatterns}
\end{center}
\end{figure}

\item[$\blacktriangleright$] Believe it or not, that's it! We recommend building at this point to confirm nothing has gone wrong with your metamodel. To
see this method in the visual syntax, review Fig.~\ref{fig:sdm_invertComplete} on the previous page.

\end{itemize}
