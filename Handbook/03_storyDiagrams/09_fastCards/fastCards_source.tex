\newpage
\section{Fast cards!}
\genHeader
\hypertarget{sec:fastCard}{}

Congratulations, you're almost there! This is the last SDM we need to build to make Leitener's Learning Box fully functional.

For very simple cards (i.e: words in different language that are similar), it might be a bit annoying to have to answer these cards again and again in every
partition. Such \emph{fast} cards should somehow be marked and handled differently! If a fast card is gotten right once, it should be immediately moved to the
last partition in the box. This way the card is reviewed once, and is tested only once more before finally being removed from the box (Fig.).

It makes sense that a \texttt{fastCard} is simply inherited from \texttt{Card}. It has the exact same structure as a standard card, it just has an extra marker
to show it gets moved differently (Fig.~\ref{fig:goal_fastCard}). This means we can extend our \texttt{check} SDM to get the result we desire!

\begin{figure}[htbp]
	\centering
    \includegraphics[width=0.7\textwidth]{goal_fastCard}
	\caption{Checking a fast card against a guess}
	\label{fig:goal_fastCard}
\end{figure}
\FloatBarrier

% Describe HOW we would extend it here? (they might be TOTALLY different for each, and so they NEED to be explained differently).

\vspace{0.5cm}
\begin{itemize}
\item[$\blacktriangleright$] You're nearly there! Click a link below to begin.
\end{itemize}

\begin{center} {$\triangleright$ \hyperlink{fastCard vis}{Fast cards: The visual syntax}}%
\\ \vspace{0.5cm}
{$\triangleright$ \hyperlink{fastCard tex}{Fast cards: The textual syntax}}\end{center} 
 
\newpage
\subsection{Implementing FastCards}
\visHeader
\hypertarget{fastCard vis}{}

\begin{itemize}

\item[$\blacktriangleright$] To introduce fast cards into your learning box, return to the metamodel diagram and create a new \texttt{eclass},
\texttt{FastCard}. Quick link to \texttt{Card} and choose \texttt{Create Inheritance} from the context menu. We only want to check the dynamic type of a
tested card at runtime, which means we don't need to override anything. Therefore, when the \texttt{Overrides \& Implementations} dialogue appears, make sure
nothing is selected (Fig.~\ref{ea:dialogue_override}). Your metamodel should then resemble Fig.~\ref{ea:metamodel_FastCard}.

\vspace{0.5cm}

% NOTE : NOT ACCURATE: MODIFIED TO REDUCE WHITE SPACE (original screenshot in visFCImages)
\begin{figure}[htp]
\begin{center}
  \includegraphics[width=0.6\textwidth]{ea_overrideDialogueModified}
  \caption{Selecting operations to override}  
  \label{ea:dialogue_override}
\end{center}
\end{figure}

\begin{figure}[htp]
\begin{center}
  \includegraphics[width=0.9\textwidth]{ea_EClassFastCard}
  \caption{Fast cards are a special kind of card}  
  \label{ea:metamodel_FastCard}
\end{center}
\end{figure}

\vspace{0.5cm}

\item[$\blacktriangleright$] Now return to the \texttt{check} SDM (in \texttt{Partition}) and extend the control flow as depicted in
Fig.~\ref{ea:extendCheck}.

 \vspace{0.5cm}
 
\begin{figure}[htbp]
\begin{center}
  \includegraphics[width=\textwidth]{ea_extendCheck}
  \caption{Extend check to handle fast cards.}  
  \label{ea:extendCheck}
\end{center}
\end{figure}
 

\item[$\blacktriangleright$] As you can see, you have created two new story nodes, \texttt{isFastCard}, and \texttt{promoteFastCard}.
 
\item[$\blacktriangleright$] Next, in order to complete the newest conditional, create a bound \texttt{FastCard} object variable, named \texttt{fastcard} in
\texttt{isFastCard} (Fig.~\ref{ea:fastCardBinding}).
 
\item[$\blacktriangleright$] To check the dynamic type, we'll need to create a binding of \texttt{card} (of type \texttt{Card}) to \texttt{fastcard} (of
type \texttt{FastCard}), so edit the \texttt{Binding} tab in the \texttt{Object Variable Properties} dialogue (Fig.~\ref{ea:fastCardBinding}). Please note that
this tab will not allow any changes unless the \texttt{bound} option in \texttt{Object Properties} is selected. As you can see, this set up configures the
pattern matcher to check for types, rather than \texttt{parameters} and \texttt{attributes} as we've previously encountered.

\vspace{0.5cm}

\begin{figure}[htbp]
\begin{center}
  \includegraphics[width=0.9\textwidth]{ea_fastCardBinding}
  \caption{Create a binding for \texttt{fastcard}}  
  \label{ea:fastCardBinding}
\end{center}
\end{figure}

\clearpage

In our case, we could use a \emph{ParameterExpression} or an \emph{ObjectVariableExpression}\define{ObjectVariable\-Expression} as \texttt{card} is indeed a
parameter \emph{and} has already been used in \texttt{checkIfGuessIsCorrect}. We haven't tried the latter yet, so let's use \emph{ObjectVariableExpression}.

\item[$\blacktriangleright$] Update the \texttt{fastcard} binding by switching the expression to 
\texttt{Object\-Vari\-able\-Ex\-pres\-sion}, with \texttt{card} as the target. Note that a binding could also use a \emph{MethodCallExpression} to invoke a
method whose return value would be the bound value. This is very useful as it allows invoking helper methods directly in patterns.

\item[$\blacktriangleright$] To finalize the SDM, extract the \texttt{promoteFastCard} story pattern and build the pattern according to
Fig.~\ref{ea:promoteFastCardPattern}. Compare this pattern to Figs.~\ref{ea:sdm_check_complete_activity_node} and \ref{ea:sdm_check_complete_penalize}, the
original promotion and penalizing card movements. As you can see, they're very similar, except \texttt{fastCard} is transferred from the current partition
(\texttt{this}) immediately to the last partition in \texttt{box}, identified as having no \texttt{nextPartition} with an appropriate NAC.

\begin{figure}[htbp]
\begin{center}
  \includegraphics[width=\textwidth]{ea_promoteFastCardPattern}
  \caption{Story pattern for handling fast cards.}  
  \label{ea:promoteFastCardPattern}
\end{center}
\end{figure}

\item[$\blacktriangleright$] Inspect Fig.~\ref{eclipse:promoFastCardFinal} to see how this is done in the textual syntax.

\item[$\blacktriangleright$] You have now implemented every method using SDMs -- fantastic work! Save, validate, and build your metamodel to see some new code.
Inspect the implementation for \texttt{check}.  Can you find the generated type casts for \texttt{fastcard}?

\item[$\blacktriangleright$] At this point, we encourage you to read each of the textual SDM instructions to try and understand the full scope of eMoflon's
features (which start on page~\hyperlink{page.9}{9}) but you are of course, free to carry on.

\jumpSingle{subsec:fastGUI}

\end{itemize}


\newpage
\subsection{Textual; Fast Cards}
\texHeader
\hypertarget{fastCard tex}{}

\begin{itemize}
  
\item[$\blacktriangleright$] Create a new eclass in ``LearningBoxLanguage" named \texttt{FastCard} that extends \texttt{Card}. It doesn't need any new
attributes, so leave its declaration empty (Fig.~\ref{fig:fastClass}).

\begin{figure}[htp]
\begin{center}
  \includegraphics[width=0.45\textwidth]{eclipse_fastCardClass}
  \caption{A new Data Type}
  \label{fig:fastClass}
\end{center}
\end{figure}

\item[$\blacktriangleright$] Return to \texttt{Partition} to \texttt{check(card, guess)}, and edit the control flow by adding a second \texttt{if/else}
construct. Call the assertion pattern \texttt{isFastCard}, and the action pattern \texttt{promoteFastCard}. Keep the original \texttt{[promoteCard]} pattern in
the \texttt{else} statement.

\item[$\blacktriangleright$] Your workspace should now resemble Fig.~\ref{fig:isFastCard}.

\begin{figure}[htp]
\begin{center}
  \includegraphics[width=0.7\textwidth]{eclipse_isFastCardFlow}
  \caption{Checking for \texttt{FastCard}}
  \label{fig:isFastCard}
\end{center}
\end{figure}

\item[$\blacktriangleright$] \texttt{isFastCard} is a simple, one line statement pattern. You need to create an assignment constraint to check a bounded object,
of type \texttt{FastCard}, against the type of card that was passed in through the parameter. Remember, to access paramter valuers, preface the name with a `\$'
symbol.

\item[$\blacktriangleright$] Your workspace should now resemble Fig.~\ref{fig:isFastCardPattern}.

\begin{figure}[htp]
\begin{center}
  \includegraphics[width=0.5\textwidth]{eclipse_isFastCardPattern}
  \caption{A \texttt{FastCard} attribute constraint}
  \label{fig:isFastCardPattern}
\end{center}
\end{figure}

\item[$\blacktriangleright$] To establish \texttt{promoteFastCard.pattern}, first create the four main object variables - \texttt{@fastCard}, \texttt{@this},
\texttt{lastPartition}, and \texttt{box}. Immediately under \texttt{lastPartition}, also create a \texttt{next} NAC.

\vspace{0.5cm}

\item[$\blacktriangleright$] Your workspace should now resemble Fig.~\ref{fig:objVarFastCard}.

\begin{figure}[htp]
\begin{center}
  \includegraphics[width=0.6\textwidth]{eclipse_promoteFastCardObjVars}
  \caption{Object variables for \texttt{promoteFastCard}}
  \label{fig:objVarFastCard}
\end{center}
\end{figure}

\item[$\blacktriangleright$] When creating the necessary references, remember - this is the pattern that will be invoked when the fast card status has been
confirmed! This means that, in the appropriate variables, you'll want to:
(1) Link the partition to the current box.
(2) Remove \texttt{fastCard} from its current partition, and insert it into \texttt{lastPartition}
(3) Confirm \texttt{lastPartition} is in a box, then check to see if it has a \texttt{next} value.\footnote{If you need help remembering how NACs work, review
section 7}

\vspace{0.5cm}

\item[$\blacktriangleright$] Your final workspace should resemble Fig.~\ref{fig:promoFastCardFinal}

\newpage

\vspace*{1cm}

\begin{figure}[htp]
\begin{center}
  \includegraphics[width=0.6\textwidth]{eclipse_promoFastCardFinal}
  \caption{The completed fast card promotion pattern}
  \label{fig:promoFastCardFinal}
\end{center}
\end{figure}

\vspace{0.5cm}

\item[$\blacktriangleright$] You have now completed \texttt{every} method signature from your abstract syntax using SDMs - fantastic work! Build your project to
confirm there aren't any errors, and review Fig.~\ref{fig:sdm_fastcard_5} to see how \texttt{FastCard}s are implemented visually. We encourage you to read each
visual SDM section to understand the full scope of eMoflon's features, which start on page~\hyperlink{page.12}{12} but you are, of course, free to carry on.
  
\end{itemize}
