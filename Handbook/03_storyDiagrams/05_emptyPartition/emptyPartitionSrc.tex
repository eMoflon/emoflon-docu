\newpage
\section{Emptying a partition of all cards}
\genHeader
\hypertarget{sec:emptyPartition}{}

This next SDM should \emph{empty} a partition of all its cards by removing every card within it. Since we can assume that there is more than one card in the
partition\footnote{If there was only one, we would just invoke \texttt{removeCard}.}, we obviously need some construct for repeatedly deleting each card in the
partition an unknown number of times (Fig.~\ref{fig:goal_empty}). 

\vspace{0.5cm}

\begin{figure}[htbp]
	\centering
  \includegraphics[width=0.25\textwidth]{empty.pdf}
	\caption{Emptying a partition of every card}
	\label{fig:goal_empty}
\end{figure}
\FloatBarrier

\vspace{0.5cm}

In SDM, this \note{For Each} construct is accomplished via a \emph{for each} story node. A
\emph{for each} story node performs the specified actions for \emph{every} match in its pattern. That means for every \texttt{Card} the pattern finds, it'll
peform an action set.

This construct however, gives us two very interesting points to dicuss. Firstly, how would the pattern be interpreted if the node was a normal story node, not a
\emph{for each}?

The pattern would specify that \emph{a} card should be matched and deleted from the current partition - that's it. The \emph{exact} card is not specified,
meaning that the actual choice of the card is \emph{non-deterministic} or random, and it is only done once. This randomness is a common property of both graph
transformations and pattern matching, and it's something that takes time getting used to.  In general, there are no guarantees concerning the choice and
order of valid matches. The \emph{for each} node ensures that \emph{all} cards will be matched and deleted.

Our second point is determining if we actually need to destroy the link between \texttt{this} and \texttt{card}. Would the pattern be interpreted differently if we
destroyed \texttt{card} and left the link?

The answer is no, the pattern would yield the same result, regardless of if the link is explicitly destroyed!  \note{Dangling Edges}This is due to the
transformation engine eMoflon uses\footnote{CodeGen2, a part of Fujaba \url{http://www.fujaba.de/}}. It ensures that there are never any \emph{dangling edges}
in a model. Since deleting just the \texttt{card} would result in a `dangling edge' attached to \texttt{this}, that link is deleted as well. Explicitly
destroying the link is therefore a matter of taste, so \ldots why not be as explicit as possible?

\vspace{1cm}
\begin{itemize}
\item[$\blacktriangleright$] Select a link below to begin this SDM.
\end{itemize}

\begin{center} {$\triangleright$ \hyperlink{emptyPartition vis}{Emptying a Partition: The visual syntax}}%
\\ \vspace{0.5cm}
{$\triangleright$ \hyperlink{emptyPartition tex}{Emptying a Partition: The textual syntax} }\end{center} 


\newpage
\hypertarget{emptyPartition vis}{}
\subsection{Implementing empty}
\visHeader

\begin{itemize}

\item[$\blacktriangleright$] Open a new activity diagram for \texttt{Partition.empty}. To begin building the \emph{for each} pattern, quick create a new story
node and edit its properties. Name it \texttt{deleteCardsInPartition} and change the \texttt{Type} from \texttt{StoryNode} to \texttt{ForEach}. You'll also want
to create the invoking \texttt{Partition} object, so ensure that too, is selected. As you can see, a \emph{for each} node is represented as a stacked node to
indicate the potential for repetition.

\begin{figure}[htbp]
\begin{center}
  \includegraphics[width=0.9\textwidth]{ea_sdmEmptyNew}
  \caption{Creating a looping story node}  
  \label{fig:sdm_foreach}
\end{center}
\end{figure}

\item[$\blacktriangleright$] Now create the neccessary \texttt{card} variable needed to complete this SDM. Unlike \texttt{removeCard}\footnote{See
Fig.~\ref{fig:sdm_complete_control_flow}} however, the goal of \texttt{emptyCards} is not just to remove the link between the selected partition and card, we
want the matched \texttt{card} to be \emph{completely} deleted. This means in the properties tab, after setting the name and binding state, you'll need to set
the \texttt{Binding Operator} to \texttt{Destroy} (Fig.~\ref{fig:sdm_bindingOperator}).

\item[$\blacktriangleright$] Complete the story pattern as indicated in Fig.~\ref{fig:sdm_end}. Notice that the guard that terminates the looping node has an
\texttt{[end]} edge guard. Indeed, a \emph{for each} story node \emph{must} execute an \texttt{end} activity when all matches in the pattern have been
handled. \texttt{empty} is defined as a \texttt{void} method, so don't worry about setting any return values.

\begin{figure}[htbp]
\begin{center}
  \includegraphics[width=0.65\textwidth]{ea_emptyBindingOperator}
  \caption{Editing \texttt{card} so the variable gets destroyed}  
  \label{fig:sdm_bindingOperator}
\end{center}
\end{figure}

\begin{figure}[htbp]
\begin{center}
  \includegraphics[width=0.8\textwidth]{ea_sdmEmptyComplete}
  \caption{Completed \texttt{empty} story pattern}  
  \label{fig:sdm_end}
\end{center}
\end{figure}
\FloatBarrier

\item[$\blacktriangleright$] Done! You've now learnt that in order to create a repeating action, all you need to do is change a standard standard story node
into a \texttt{for each} node, and include \emph{edge guards}. Inspect Fig.~\ref{fig:emptyControlFlow} and Fig.~\ref{fig:emptyPattern} to see how this is
implemented in the textual specifications.

\fancyfoot[R]{ $\triangleright$ \hyperlink{sec:invertCard}{Next}} 

\end{itemize}



\newpage
\subsection{Textual; emptying}
\texHeader
\hypertarget{emptyPartition tex}{}

\texttt{}
\emph{}

\begin{itemize}
 
\item[$\blacktriangleright$] Can use type completion here to set up the flow for your for each. Press ctrl space and select forEach

\item[$\blacktriangleright$] create a \texttt{deleteCardsInPartition} pattern. While similar to \texttt{removeCard}, this goes one step further by requesting a
full deletion/destruction of card. This means we need to destroy the obj. variable. Work until your workspace resembles (Fig).

\item[$\blacktriangleright$] That's it! Check out how it's done in the visual syntax.

\end{itemize}
