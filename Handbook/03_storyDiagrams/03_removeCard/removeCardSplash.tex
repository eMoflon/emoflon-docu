\newpage
\genHeader
\section{Removing a card}
\hypertarget{sec:remCard}{}

Since we're just getting started with SDMs,\footnote{As you may have already noticed, we use ``SDM'' interchangeably to mean both our graph transformation
language \emph{or} a concrete transformation (a story model) used to implement a method, consisting of an activity and pattern.} lets re-implement the method
previously completed with injections.\footnote{Refer to Part II, Section 6} The goal of this method is to remove a single card from its current partition, which
can be done by destroying the link between the two items (Fig.~\ref{fig:goal_removeCard}).

\vspace{1cm}

\begin{figure}[htbp]
	\centering
    \includegraphics[width=0.2\textwidth]{goal_remove}
	\caption{Removing a card from its partition}
	\label{fig:goal_removeCard}
\end{figure}
\FloatBarrier

\vspace{0.5cm}

According to the signature of the method \texttt{removeCard}, we should return the card that has been deleted. Although this might strike you as slightly odd,
considering that we already passed in the card as an argument, it still makes sense as it allows for chaining method calls:
\syntax{ aPartition.removeCard(aCard).invert()}

Before we implement this change as a pattern, lets remove the old injection content to avoid potential conflicts.

\begin{itemize}

\item[$\blacktriangleright$] Delete the \texttt{PartitionImpl.inject} file from your working set (Fig~\ref{fig:delete_injection}).

\item[$\blacktriangleright$] Now right-click on \texttt{PartitionImpl.java} and go to ``emoflon/Clean(and Build)" 

\item[$\blacktriangleright$] You'll be able to see the changes in \texttt{PartitionImpl.java}, found under ``MyWorkingSet.'' The \texttt{removeCard}
declaration should now be empty and look identical to the others.

\end{itemize}

\newpage

\begin{figure}[htbp]
	\centering
    \includegraphics[width=0.7\textwidth]{eclipse_removeInjection}
	\caption{Remove injection content}
	\label{fig:delete_injection}
\end{figure}

\vspace{1cm}

Thats it! We now have a fresh start for \texttt{removeCard}. Let's briefly discuss what we need to establish the transformation.

One of the goals of SDM is to allow you to focus less on \emph{how} a method will do something, but rather on \emph{what} the method will do.
Integrated as an atomic step in the overall control flow, a single graph transformation step (such as link deletion)\footnote{hint, hint!} can be embedded as a
\emph{story pattern}.

These patterns however, must establish\define{Object \\ Variables}\emph{object variables}, place holders for actual objects in a model (i.e., an object for each
element in your instance model). During \emph{pattern matching}, occurrences of objects in the current model are connected to the object variables in the
pattern according to the indicated type and other conditions.\footnote{We shall learn what conditions may be specified in later SDMs}

\clearpage

In \texttt{removeCard}, the SDM will requires just two object variables: a \texttt{this} partition (named according to Java convention) referring to the
object whose method is invoked, and \texttt{card}, the parameter object that will be removed.

When applicable, patterns must also establish \emph{link variables}\define{Link \\ Variables} to match references in the model. Given that
we're concerned with removing a certain card from a specific partition, \texttt{removeCard} will therefore have a single link variable to match these two
objects together.

Unfortunately, we know the pattern matcher uses a randomized system until it matches a valid object to each of its variables, and any new variables a pattern
creates simply won't have any default settings. How can this be changed so that, as required for \texttt{removeCard}, the matcher will retrieve the information
associated with specific model objects (i.e., match with the correct \texttt{card})?

The \emph{binding state}\define{Binding~State}of an object determines how its data is found. By default, every object is \emph{unbounded}, or a  \emph{free
variable}.\define{Free \\ Variable}Values for these objects can be both assigned and changed throughout an activity. By declaring a
\emph{bounded}\define{Bounded}object however, the item will have a fixed value from the context of the model.
\emph{Binding} is implicit via the \emph{name} of the bound object variable. As a rule, \texttt{this} variables, and any parameterized values (i.e.,
\texttt{card}) are always bounded.

On a final note, every object or link variable can also set its \emph{binding operator} to \texttt{Check Only, Create, or Destroy}. For a rule $r: (L,
R)$, as discussed in \hyperlink{explanation}{Section 2}, these marks the variable as belonging to the set of elements to be retained ($L\cap R$), the set of
elements to be newly created ($R\setminus L$), or the set of elements to be deleted ($L\setminus R$).

If you're feeling overwhelmed by all the new terms and concepts, don't worry! We will define them again in the context of your syntax with the example. For
quick reference, we have also defined each of the marginalized words at the end of this part in a \hyperlink{glossary}{glossary}. 

\jumpDual{remCard vis}{remCard tex}

\newpage
\hypertarget{remCard vis}{}
\subsection{Implementing removeCard}
\visHeader

\begin{itemize}

\item[$\blacktriangleright$] Open \texttt{LearningBoxLanguage.eap} in Enterprise Architect (EA) from Eclipse by double-clicking it in Eclipse. Carefully do the
following: (1) Click \emph{once} on \texttt{Partition} to select it, then (2) Click \emph{once} on the method \texttt{removeCard} to highlight it
(Fig.~\ref{fig:sdm_start}), and (3) \emph{Double-click} on the chosen method to indicate that you want to implement it.

\begin{figure}[htp]
\begin{center}
  \includegraphics[width=0.6\textwidth]{ea_startSDM}
  \caption{Double-click a method to implement it}  
  \label{fig:sdm_start}
\end{center}
\end{figure}
 
\item[$\blacktriangleright$] If you did everything right, a new \emph{activity diagram} should be created and open in a new tab with a cute anchor in
the corner, and a \emph{start node} labelled with the signature of the method (Fig.~\ref{fig:sdm_skeleton}).  

\begin{figure}[htp]
\begin{center}
 \includegraphics[width=1.0\textwidth]{ea_generatedSDM}
  \caption{Generated SDM diagram and start node}  
  \label{fig:sdm_skeleton}
\end{center}
\end{figure}

\vspace{0.5cm}

\item[$\blacktriangleright$] This diagram is where you'll model \texttt{removeCard}'s activities via a \emph{control flow}. In other words,
this \emph{removeCard}'s imperative top level. We refer to the whole activity diagram simply as the \emph{activity}, which always starts with a start node,
contains \emph{activity nodes} connected via \emph{activity edges}, then finally terminates with a \emph{stop node}. Before developing these however, let's
quickly familiarise ourselves the EA workspace.

\item[$\blacktriangleright$] First, inspect the project browser and notice that an \texttt{<<SDM Activity>>} container has been created for the method
\texttt{removeCard}. This container will eventually host every artifact related to this pattern (i.e., object variables, stop nodes, etc). Please note
that if you're ever unhappy with an SDM, you can always delete the appropriate container in the project browser (such as this one), and start from scratch.

\item[$\blacktriangleright$] Next, note the new \texttt{SDM} toolbox that has been automatically opened for the diagram and placed to the left above
the common toolbox. This provides quick access to SDM items that you'll frequently use in your diagram.

\item[$\blacktriangleright$] Finally, in the top left corner of the diagram, you'll notice a small anchor. Double click on this icon to quickly jump back to the
metamodel. From there, double click the method again to jump back to the SDM. This is just a small trick to help you quickly shift between diagrams.

\newpage

\item[$\blacktriangleright$] To begin, select the start node, and note the small black arrow that appears (Fig.~\ref{fig:sdm_quicklink}). 

\begin{figure}[htp]
\begin{center}
  \includegraphics[width=0.5\textwidth]{ea_sdmStartNode}
  \caption{Quick link in SDM diagram to create new activity node}  
  \label{fig:sdm_quicklink}
\end{center}
\end{figure}

\item[$\blacktriangleright$] Similar to quick linking,\footnote{Learnt in Part II, Section 2.5} a second fundamental gesture in EA is \emph{Quick
Create}.
To quick-create an element, pull the arrow and click on an empty spot in the diagram. This is basically ``quick linking'' to a non-existent element.

\item[$\blacktriangleright$] EA notices that there is nothing to quick-link to, and pops a small, context-sensitive dialogue offering to create an element which
can be connected to the source element.

\item[$\blacktriangleright$] As illustrated in Fig.~\ref{fig:sdm_new_activity_node}, choose \texttt{Append StoryNode} to create a \emph{Story Node}.

\begin{figure}[htp]
\begin{center}
  \includegraphics[width=0.8\textwidth]{ea_sdmQuickLinkStoryNode}
  \caption{Create new activity node}  
  \label{fig:sdm_new_activity_node}
\end{center}
\end{figure}

\item[$\blacktriangleright$] If you quick-created correctly, you should now have a start node, one node called \texttt{ActivityNode1}, and an edge
connecting the two items. Complete the activity by quick-creating a stop node (Fig.~\ref{fig:sdm_stop_node}).

\begin{figure}[htp]
\begin{center}
  \includegraphics[width=\textwidth]{ea_sdmAppendStopNode}
  \caption{Complete the activity with a stop node}  
  \label{fig:sdm_stop_node}
\end{center}
\end{figure}

\vspace{0.5cm}

\item[$\blacktriangleright$] If everything is correct, you should now have a fully constructed activity that models the method's process.

\item[$\blacktriangleright$] While a \emph{stop node} is rather self explanatory, you may be wondering about the differences between the other two menu options,
the\define{Story Node}\emph{story node} and \emph{statement node}.\define{Statement Node}Since not all activity nodes can contain story patterns (i.e., start
and stop nodes), those that \emph{can} are called story nodes. Statement nodes are just used to guarantee an action, such as method execution, and only happen
between story nodes. We'll encounter this type in a later SDM.

\item[$\blacktriangleright$] To complete this activity, double-click \texttt{ActivityNode1} to prompt the dialogue depicted in
Fig.~\ref{fig:story_pattern}. Enter \texttt{removeCardFromPartition} as the name of the story node, and select \texttt{Create this Object}.  Click
\texttt{OK}. The activity node now has a single \emph{bounded} \emph{object variable}, \texttt{this}.

\item[$\blacktriangleright$] To create a new object variable, either choose \texttt{SDM ObjectVariable} from the toolbox or press \texttt{ctrl}, then
click inside the activity node (Fig.~\ref{fig:tool_box}). A properties window will automatically appear (Fig.~\ref{fig:object_variable_properties}).

\begin{figure}[htpb]
\begin{center} 
  \includegraphics[width=0.6\textwidth]{ea_sdmEditActivityNode}
  \caption{Initializing a story node}  
  \label{fig:story_pattern}
\end{center}
\end{figure}

\begin{figure}[htp]
\begin{center}
  \includegraphics[width=0.8\textwidth]{ea_sdmNewObjVar}
  \caption{Add a new object variable from the toolbox}  
  \label{fig:tool_box}
\end{center}
\end{figure}

\newpage

\vspace{0.5cm}

\begin{figure}[htp]
\begin{center}
  \includegraphics[width=\textwidth]{ea_sdmPropertiesObjVar}
  \caption{Specify properties of the added object variable}  
  \label{fig:object_variable_properties}
\end{center}
\end{figure}


\item[$\blacktriangleright$] Using the drop-down menus, choose \texttt{card} as the name of the object, and set \texttt{Card} as its type.
Since \texttt{card} is a parameter of the method, it is offered as a possible name which can be directly chosen to prevent annoying typing mistakes.

\vspace{0.5cm}

\item[$\blacktriangleright$] In this dialogue, note that the \texttt{Bound} option is set. We have now seen two cases in this activity for bound object
variables: an assignment to \texttt{this}, and an assignment to a method parameter. Setting \texttt{card} to bound means that it will implicitly assign itself
to a parameter value of the same name.

\item[$\blacktriangleright$] To create a \emph{link variable} between the current partition and the card to be removed, choose the object variable \texttt{this}
and quick-link it to \texttt{card} (Fig.~\ref{fig:link_variable}).

\begin{figure}[htpb]
\begin{center}
  \includegraphics[width=0.6\textwidth]{ea_sdmCreateLinkVar}
  \caption{Create a link variable}   
  \label{fig:link_variable}
\end{center}
\end{figure}

\vspace{0.5cm}

\item[$\blacktriangleright$] According to the metamodel, there is only one possible link between a partition and card. Select this and set the
\emph{Binding Operator} to \texttt{Destroy} (Fig.~\ref{fig:link_variable_properties}). The reference names will automatically appear in the diagram.

\vspace{0.5cm}

% Had to force (h!) image to appear here; no other images were co-operating
\begin{figure}[h!]
\begin{center} 
 \includegraphics[width=0.6\textwidth]{ea_sdmBindLink}
  \caption{Specify properties for created link variable}  
  \label{fig:link_variable_properties}
\end{center}
\end{figure}

\vspace{0.5cm}

\item[$\blacktriangleright$] Remember how we said that this method should return the same card that was passed in? As luck would have it, a return value for any
SDM can be specified in the stop node. As depicted in Fig.~\ref{fig:stop_node_return_value}, double-click the stop node to prompt the \texttt{Edit StopNode} dialogue. 

\newpage

\begin{figure}[htbp]
\begin{center}
  \includegraphics[width=0.5\textwidth]{ea_sdmStopNodeExpr}
  \caption{Adding a return value to the stop node}  
  \label{fig:stop_node_return_value}
\end{center}
\end{figure}

\item[$\blacktriangleright$] In the \texttt{Expression} field, choose \texttt{ParameterExpression}\define{ParameterExpression}as the expression. Given that
\texttt{card} is the sole parameter, it will be automatically selected. A \emph{ParameterExpression} is an eMoflon (with EA) mechanism that exclusively accesses
parameter values.

\vspace{0.5cm}

We're nearly done! As you can see, by using several different dialouges, eMoflon employs a simple context-sensitive expression language for specifying  values.
We have intentionally avoided creating a full-blown sub-language, and limit expressions to a few simple types.\footnote{We also do not support nesting expressions} The philosophy here is to keep things simple and concentrate on what SDMs are good for -- expressing structural changes. Our approach is to
provide a clear and type-safe interface to a general purpose language (Java) and support a simple \emph{fallback} as soon as things get too low-level and
difficult to express as a pattern.

The alternative approach to eMoflon would be to support arbitrary expressions, for example, in a script language like JavaScript or in an appropriate
DSL\footnote{A DSL is a Domain Specific Language: a language designed for a specific task which is usually simpler than a general purpose language like Java and
more suitable for the exact task.} designed for this purpose. In the following SDM implementations, we'll learn the other expression types eMoflon supports,
and how to use them. 

\newpage

\item[$\blacktriangleright$] Returning to the activity, if you've done everything right, your first eMolfon SDM should resemble
Fig.~\ref{fig:sdm_complete_control_flow}, where \texttt{removeCard}'s entire pattern layer is modeled inside the sole \emph{activity node}. The method's return
value is now indicated below the stop node.

\vspace{0.5cm}

\begin{figure}[htbp]
\begin{center}
  \includegraphics[width=0.7\textwidth]{ea_sdmRemoveComplete}
  \caption{Complete SDM for \texttt{Partition::removeCard}}  
  \label{fig:sdm_complete_control_flow}
\end{center}
\end{figure}

\vspace{0.5cm}

\item[$\blacktriangleright$]  Don't forget to save your files, validate and export your pattern to the Eclipse workspace,\footnote{Go to to
``\texttt{Extensions}" and select \texttt{Add-In Windows} to activate eMoflon's console. If you're unsure how to validate, export, or use this window, review
Part II, section 3} then build your metamodel's code from the package explorer.

\item[$\blacktriangleright$] If you're unable to export or generate code successfully, compare your SDM carefully with Fig.~\ref{fig:sdm_complete_control_flow}
and make sure you haven't forgotten anything.

\item[$\blacktriangleright$] If you'd like to see how this SDM is implemented in the textual syntax, check out Fig.~\ref{fig:deleteReference} in the
next section.

\jumpSingle{remCard end}

\end{itemize}



\newpage
\hypertarget{remCard tex}{}
\subsection{Implementing removeCard}
\texHeader

\begin{itemize}

\item[$\blacktriangleright$] Open \texttt{Partition.eclass}, go to the \texttt{removeCard} signature and add a pair of curly brackets, so that it looks like a
proper method declaration. This entire space can be referred to the method's \emph{activity}, where the control flow, or the imperative top-layer of a
transformation is defined.

\item[$\blacktriangleright$] Complete the activity with a single pattern and return statement as depicted in Fig~\ref{fig:remCardDec}. Don't forget -- you can
use eMolfon's auto-complete features here as well! Press \texttt{ctrl + space} on an empty line, then select the pattern template.

\begin{figure}[htp]
\begin{center}
  \includegraphics[width=0.5\textwidth]{eclipse_removeCardDeclaration}
  \caption{Control flow for \texttt{removeCard} \update}
  \label{fig:remCardDec}
\end{center}
\end{figure}

\item[$\blacktriangleright$] It should be mentioned that MOSL limits method definitions exclusively to the method's control flow. All actual transformations are
modeled in separate \emph{pattern} files. In this case, \texttt{removeCard}'s link deletion will be modeled in \texttt{[deleteSingleCard]}.

\item[$\blacktriangleright$] Save the \texttt{Partition.eclass}. An error should immediately appear below the editor! In the ``Problems'' tab, the message
states that the builder cannot find the specified pattern file. Well, this makes sense. You haven't created it yet! Click this message and press \texttt{ctrl +
1} to invoke a ``Quick Fix'' dialogue (Fig~\ref{fig:quixFix}). It offers to create a pattern file for you. Given that's exactly what you'd like, select the option
and press \texttt{Finish}.

\begin{figure}[htp]
\begin{center}
  \includegraphics[width=0.6\textwidth]{eclipse_patternQuickFix}
  \caption{A quick fix to a missing pattern error}
  \label{fig:quixFix}
\end{center}
\end{figure}

\newpage

\item[$\blacktriangleright$] The new file will open in the editor, and you'll be able to see a new directory structure under ``LearningBoxLanguage/\_patterns''
(Fig.~\ref{fig:pattStruct}). In this case, \texttt{deleteSingleCard.pattern} is invoked by the method \texttt{removeCard}, which is in the \texttt{Partition}
EClass. \texttt{Partition} will eventually contain a folder for each method that needs a pattern.

\vspace{0.5cm}

\begin{figure}[htp]
\begin{center}
  \includegraphics[width=0.9\textwidth]{eclipse_patternStructure}
  \caption{Directory structure for a pattern}
  \label{fig:pattStruct}
\end{center}
\end{figure}

\item[$\blacktriangleright$] The content of any pattern file is simply a list \emph{object variable scopes}, and then,
within those declarations, operations such as `delete this reference.' Remember - the main goal of SDMs is to focus here is not on \emph{how}, but
\emph{what}.

\vspace{0.5cm}

\item[$\blacktriangleright$] Create two object variables, \texttt{@this:Partition} and~\texttt{@card:Card} (Fig.~\ref{fig:remCardObjVar}). In MOSL syntax,
`\texttt{@}' indicates that a variable is \emph{bound}. \texttt{this} is bound to the object who invoked the method, while \texttt{card} is bound to the
parameter value with the same name.

\begin{figure}[htp]
\begin{center}
  \includegraphics[width=0.6\textwidth]{eclipse_thisObjVar}
  \caption{Object variables for \texttt{removeCard}}
  \label{fig:remCardObjVar}
\end{center}
\end{figure}

\clearpage

\item[$\blacktriangleright$] Object variable scopes determine the changes to be applied to the \emph{outgoing link variables} of the variable. Therefore, add
\texttt{-- -> card:card} to \texttt{@this} to destroy the \texttt{card} reference targeting the \texttt{card} object. Your pattern should now resemble
Fig.~\ref{fig:deleteReference}. The syntax follows this format:

\syntax{[action]`->'nameOfOutgoingLV`:'targetOV\\
\\
With:\\
action := `- -' | `++' | `!'\\
nameOfOutgoingLV := STRING\\
targetOV := STRING
}

\vspace{0.5cm}

\begin{figure}[htp]
\begin{center}
    \includegraphics[width=0.6\textwidth]{eclipse_remCardObjVars}
  \caption{Destroy the link between a card and its partition}
  \label{fig:deleteReference}
\end{center}
\end{figure}

\item[$\blacktriangleright$] If you ever need to quickly remind yourself of specific navigation or attributes names, press \texttt{alt} and the \texttt{left}
arrow to jump back from your pattern to your EClass. Conversely, to quickly open or jump to a pattern, hover over the pattern name while holding \texttt{ctrl}
until it's underlined, then click!

\item[$\blacktriangleright$] Remember that links between classes can be specified as \emph{bidrectional EReferences},\footnote{Technically two
\emph{unidirectional EReferences}} linked together as opposites in ``LearningBoxLanguage/\_con\-straints.mconf.'' In this case, therefore, we don't need to
worry about declaring \texttt{-- -> cardContainer:Card} inside \texttt{card}, as it would be redundant.

\item[$\blacktriangleright$] Save and build your metamodel. If any errors occur, double check and make sure your activity and pattern match ours. 

\item[$\blacktriangleright$] To see how this same method is crafted in the visual syntax, check out Fig.~\ref{fig:sdm_complete_control_flow} from the previous
section.

\end{itemize}


\newpage
\genHeader
\hypertarget{remCard end}{}
\subsubsection{Concluding removeCard}

Let's take a step back and briefly review what we have specified:  if \texttt{p.remove\-Card(c)} is invoked for a partition \texttt{p}, with a card, \texttt{c},
as its argument, the specified pattern will \emph{match} only if that card is contained in the partition. After determining matches for all variables, the
link between the partition and the card is deleted, effectively ``removing'' the card from the partition. If the card is \emph{not} contained in the partition,
the pattern won't match, and nothing will happen. In both cases, the card that's passed in is returned.

\begin{itemize}

\item[$\blacktriangleright$] If your code generation was successful, navigate to 
``Learning\-Box\-Language/\-gen/\-Learning\-Box\-Language/\-impl/\-Partition\-Impl.java" to the \texttt{\-remove\-Card} declaration (approximately line 385).
Inspect the generated implementation for your method (Fig.~\ref{fig:remCardImpl}). Notice all the null checks that are automatically created - only a very conscientious
(and probably slightly paranoid) programmer would program so defensively!

\begin{figure}[htp]
\begin{center}
  \includegraphics[width=\textwidth]{eclipse_remCardImpl}
  \caption{Generated implementation code}
  \label{fig:remCardImpl}
\end{center}
\end{figure}

\item[$\blacktriangleright$] After using injections, you were able to test your implementation with \texttt{LeitnersBoxGui},\footnote{If you haven't downloaded
or used the GUI before, review Part II, Section 7} so lets test and make sure \emph{this} version works. Load and run the GUI, then go to any partition and
select \texttt{Remove Card}. If it doesn't work, make sure your metamodel is built, and double check your specification syntax.

\end{itemize}

