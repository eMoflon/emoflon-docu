\newpage
\subsubsection{Inversion review}
\genHeader
\hypertarget{invert close}{}

Before we start the next SDM, let's quickly review one point. Have you considered why the \texttt{temp} object variable is bound in the second pattern for
\texttt{invert}, (\texttt{swap variables}), but not where it's first defined in \texttt{initialize temp}?\footnote{See Fig.~\ref{fig:sdm_invertComplete}
(Visual) or Fig.~\ref{fig:invertPatterns} (Textual)} This is a new case for bound variables that we haven't treated yet!

Until now, we have seen object variables that can be bound to (1) an argument of the method (set when the method is invoked), or (2) the
current object (\texttt{this}) whose method is invoked. In both cases, the object to be matched is completely determined by the context of the method before
the pattern matcher starts. This means that it does not need to be determined or found by the pattern matcher.

Setting \texttt{temp} as bound in \texttt{Swap variables} is a third case in which an object variable is bound to a value determined in a \emph{previous}
activity node without using a special expression type. In this SDM, this means \texttt{temp} will be bound to the value determined for a variable
of the same name in the previous node, \texttt{Initialize temp}. This binding feature enables you to refer to previous matches for object variables in the
preceding control flow.

On a separate note, you're just over halfway through working through this part of the eMoflon handbook, so give your brain a small break. Take a walk, pour
yourself another coffee, and check out one of my favourite jokes:
\syntax{How do you wake up Lady Gaga?}

\vspace{0.5cm}

\syntax{Poke her face!}
