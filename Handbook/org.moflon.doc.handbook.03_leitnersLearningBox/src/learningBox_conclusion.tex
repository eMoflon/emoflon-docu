\newpage
\section{Conclusion and next steps}
\genHeader

\vspace{0.5cm}

Whoo, this has been quite a busy handbook. Great job, you've finished Part II! This part contained some key skills for eMoflon, as we learned how to create
the abstract syntax in Ecore for our learning box! Both the visual and textual specifications are now complete with all the classes, attributes, references, and
method signatures that make up the type graph for a working learning box. Additionally, we also learned how to insert a small handwritten method into generated
code, and tested all our work in an interactive GUI.

If you enjoyed this section and wish to fully develop \emph{all} the methods we just declared, we invite you to carry on to Part III: Story
Diagrams\footnote{Download: \dlPartThree}! Story Diagrams are a powerful feature of eMoflon as we can model a large part of a system's dynamic semantics via high-level pattern rules. 


Of course, you're always free to pick a different part of the handbook if you feel like skipping ahead and checking out some of the other features eMoflon has
to offer. Check out Triple Graph Grammars (TGGs)  in Part IV\footnote{Download: \dlPartFour}, or Model-to-Text Transformations in Part V\footnote{Download: \dlPartFive}. We'll provide instructions on how to easily download
all the required resources so you can start without having to complete the previous parts.

For a detailed description of all parts, please refer to Part 0\footnote{Download: \dlPartZero}.

\vspace{1.0cm}

Cheers!