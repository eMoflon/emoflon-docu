\newpage
\hypertarget{static:classes tex}{}
\subsection{Declaring classes and attributes}
\texHeader

\begin{itemize}

\item[$\blacktriangleright$] Right click your \texttt{LearningBoxLanguage} model and create your first EClass by navigating to ``New/EClass.'' Name it
\texttt{Box} (Fig.~\ref{eclipse:moslNewClass}).

\begin{figure}[htbp]
	\centering
	\subfloat{\includegraphics[width=0.7\textwidth]{eclipse_newClass}}		
	\subfloat{\includegraphics[width=0.4\textwidth]{eclipse_definingBoxEClass}}
	\caption{Creating a new EClass}
	\label{eclipse:moslNewClass}
\end{figure} 

\vspace{0.5cm}

\item[$\blacktriangleright$] The class editor should automatically open. Let's add the first two EAttributes of our \texttt{Box}, \texttt{name} and
\texttt{stringRep}. eMoflon offers auto-completion templates to help you with
this task. Go to an empty line and press \texttt{Ctrl + Space}. You'll be
provided with a short list of suggestions (Fig.~\ref{eclipse:typeCompTempl}).
The first four items are related to method control flow, so select \texttt{attribute} near
the bottom to create \texttt{name} of type \texttt{EString}.

\vspace{0.5cm}

\begin{figure}[htbp]
	\centering
  \includegraphics[width=0.8\textwidth]{eclipse_typeCompletionTemplates}
	\caption{eMoflon's auto-completion}
	\label{eclipse:typeCompTempl}
\end{figure} 

\vspace{0.5cm}

% Autocompletion is missing

%\item[$\blacktriangleright$] Auto-completion also supports you by suggesting a list of types. Start to create a second attribute, \texttt{stringRep}, but
%stop after typing the \texttt{`:'} operator and press the hotkeys. The pop-up list provides a list of all types currently available
%(Fig.~\ref{eclipse:typeCompTypes}) in both your metamodel, and eMoflon's standard metamodels that are included in every new project.

%\vspace{0.5cm}

\item[$\blacktriangleright$] Your workspace should now resemble (Listing~\ref{eclipse:boxDeclaration}).

\newpage 

\vspace{0.5cm}

\lstinputlisting[style=eclass, label=eclipse:boxDeclaration ,caption={Newly created \texttt{Box} EClass}]{../2_staticSemantics/2_definingClasses/dcTexCode/Box.txt} 

\FloatBarrier

\vspace{0.5cm}

\item[$\blacktriangleright$] Now create two empty EClasses in your model, \texttt{Partition} and \texttt{Card}.

\vspace{0.5cm}

\item[$\blacktriangleright$] In \texttt{Partition}, add two \texttt{EInt} attributes, \texttt{index} and \texttt{partitionSize} (Listing~\ref{eclipse:partitionDeclaration}).

\lstinputlisting[style=eclass, label=eclipse:partitionDeclaration ,caption={Newly created \texttt{Partition} EClass}]{../2_staticSemantics/2_definingClasses/dcTexCode/Partition.txt} 

\vspace{0.5cm}

\item[$\blacktriangleright$] In \texttt{Card}, create three \texttt{EString} attributes, \texttt{back}, \texttt{face} , and \texttt{part\-it\-ion\-His\-tory} (Listing~\ref{eclipse:cardDeclaration}).

\lstinputlisting[style=eclass, label=eclipse:cardDeclaration ,caption={Newly created \texttt{Card} EClass}]{../2_staticSemantics/2_definingClasses/dcTexCode/Card.txt}

\item[$\blacktriangleright$] If you've done everything correctly, your workspace should now resemble Fig.~\ref{eclipse:workspaceClassAttributes} and Listings~\ref{eclipse:boxDeclaration},~\ref{eclipse:partitionDeclaration}~and~\ref{eclipse:cardDeclaration} .

\begin{figure}[htbp]
	\centering
  \includegraphics[width=0.6\textwidth]{eclipse_workspaceTexClassAttributes}
	\caption{The project after creation of \texttt{Box}, \texttt{Partition} and \texttt{Card}}
	\label{eclipse:workspaceClassAttributes}
\end{figure} 

\vspace{0.5cm}

\item[$\blacktriangleright$] That's it for declaring class attributes! Feel free to build your project again and view the changes in the \texttt{.ecore}
mode, and the generated files in ``gen" and ``src." On a final note, while some languages (such as Java) allow the declaration of several small classes (such as
these three) in the same file, when tooling with eMoflon, we keep them separated. Don't worry -- we'll explain this later in the handbook. As for now, continue
to the next section to start creating references between these EClasses.

\end{itemize}
