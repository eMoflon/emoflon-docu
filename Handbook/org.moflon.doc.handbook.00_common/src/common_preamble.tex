\documentclass[11pt, a4paper, oneside]{book}

\usepackage{fancyhdr}
\pagestyle{fancy}
\fancyhf{}

\usepackage[utf8]{inputenc}
\usepackage[english]{babel}
\usepackage{graphicx}
\usepackage[colorlinks,hyperindex,plainpages=false,breaklinks]{hyperref}
\usepackage{amssymb}
\usepackage{wasysym} 
\usepackage{wrapfig}
\usepackage{enumerate}
\usepackage{placeins} % Necessary for \FloatBarrier
\usepackage{subfig}   % Necessary for subfloat (images next to each other)
\usepackage{color}
\usepackage[usenames,dvipsnames,svgnames]{xcolor}
\usepackage{listings}
\usepackage{upquote}
\usepackage{paralist}
\usepackage[nodayofweek]{datetime}
\usepackage{chngcntr}

%change the numbering of pictures and listings
\AtBeginDocument{\counterwithin{lstlisting}{section}}
\counterwithin{figure}{section}

\newcommand{\monthword}[1]{\ifcase#1\or January\or February\or March\or April\or May\or June\or July\or August\or September\or October\or November\or December\fi}

%\include{common_definitions}
\newcommand{\eMoflonContact}{\href{mailto:contact@emoflon.org}{contact@emoflon.org}}
\newcommand{\eMoflonWebsite}{\href{www.emoflon.org}{www.emoflon.org}}
\newcommand{\disclaimerForTextualSyntax}{\fbox{\parbox{\textwidth}{
Please note that the textual syntax is not as thoroughly tested as the visual syntax because most of our projects are built with the visual syntax.
This means: Whenever something goes wrong even though you are sure to have followed the instructions precisely, do not hesitate to contact us via \eMoflonContact.
}}}
\newcommand{\todo}[1]{} %\textbf{TODO: #1}}
\def\versionNumber{2.0.0}

\def\itemWithRightTriangle{\item[$\blacktriangleright$]}

\definecolor{RED}{RGB}{255, 0, 0}
\definecolor{GREEN}{RGB}{0, 128, 0}
\definecolor{DARKGREEN}{RGB}{0, 64, 0}
\definecolor{BLUE}{RGB}{0, 0, 255}
\definecolor{DARKBLUE}{RGB}{0,0,128}
\definecolor{GRAY}{RGB}{144,144,144}
\definecolor{BROWN}{RGB}{205, 133, 63}
\definecolor{ORANGE}{RGB}{255, 165, 0}
\definecolor{PURPLE}{RGB}{160, 32, 240}
\definecolor{VIOLET}{RGB}{149, 0, 85}
\definecolor{WHITE}{RGB}{255, 255, 255}
\definecolor{BLACK}{RGB}{0,0,0}

\definecolor{codelightgray}{rgb}{0.87,0.87,0.87}

\hypersetup{colorlinks=true,% 
	linkcolor=black,%
	citecolor=red,%
	filecolor=blue,% 
	menucolor=black,% 
	pagecolor=black,%
	urlcolor=black
}

\lstset{language=,
    literate={\\\-}{}{0\discretionary{-}{}{}}, %https://tex.stackexchange.com/questions/69346/how-to-deal-with-very-long-lstinline-phrases-like-long-class-names/69364#69364
    breaklines=true,
    keywordstyle=\color{blue},
    basicstyle=\scriptsize\ttfamily,
    showstringspaces=false,
    backgroundcolor=\color{codelightgray},
    backgroundcolor=\color{white},
    frame=topbottom,
    morekeywords={SELECT,FROM,WHERE,AND,OR,EClass}
}
%\makeatletter
%     \lst@ifdisplaystyle\scriptsize\fi
%\makeatother
\lstdefinelanguage{MOSLTGG}
{morekeywords={rule, refines, source, target, correspondence, constraints, operations},
	sensitive=false,
	morestring=[b]",
	morestring=[b]',
	morecomment=[s]{/*}{*/},
	morecomment=[l]{//}
}

\lstdefinestyle{TGGs}{
	captionpos=b,
	numbers=left,
	escapeinside={(*@}{@*)},
	extendedchars=true,
	breaklines=true,
	frame=single,
	backgroundcolor=\color{WHITE},
	language=MOSLTGG,
	keywordstyle=\bfseries\color{VIOLET},
	commentstyle=\itshape\color{DARKGREEN},
	stringstyle=\color{BLUE},
	morecomment=[l][\color{RED}]{--},
	morecomment=[l][\color{GREEN}]{++}
}

\lstdefinelanguage{MOSLPATTERN}
{morekeywords={pattern, this, @this, true, false},
	sensitive=false,
	morestring=[b]",
	morestring=[b]',
	morecomment=[s]{/*}{*/},
	morecomment=[l]{//}
}

\lstdefinestyle{pattern}{
	captionpos=b,
	numbers=left,
	extendedchars=true,
	breaklines=true,
    escapeinside={(*@}{@*)},
	frame=single,
	backgroundcolor=\color{WHITE},
	language=MOSLPATTERN,
	keywordstyle=\bfseries\color{VIOLET},
	commentstyle=\itshape\color{DARKGREEN},
	stringstyle=\color{BLUE},
	morecomment=[l][\color{RED}]{--},
	morecomment=[l][\color{GREEN}]{++}
}

\lstdefinelanguage{MOSLECLASS}
{morekeywords={class, return, abstract, extends, this, true, false, if, else},
	sensitive=false,
	morestring=[b]",
	morestring=[b]',
	morecomment=[s]{/*}{*/},
	morecomment=[l]{//}
}

\lstdefinestyle{eclass}{
	captionpos=b,
	numbers=left,
	extendedchars=true,
	breaklines=true,
	escapeinside={(*@}{@*)},
	backgroundcolor=\color{WHITE},
	frame=single,
	language=MOSLECLASS,
	keywordstyle=\bfseries\color{VIOLET},
	commentstyle=\itshape\color{DARKGREEN},
	stringstyle=\color{BLUE},
	morestring=[s][\itshape\color{BLUE}]{[}{]},
	morecomment=[s][\color{BLUE}]{<}{>}
}

\lstdefinelanguage{MOSLMCONF}
{morekeywords={opposites, foreign, user_defined_constraints, udc, import},
	sensitive=false,
	morestring=[b]",
	morestring=[b]',
	morecomment=[s]{/*}{*/},
	morecomment=[l]{//}
}

\lstdefinestyle{mconf}{
	captionpos=b,
	numbers=left,
	extendedchars=true,
	breaklines=true,
	escapeinside={(*@}{@*)},
	backgroundcolor=\color{WHITE},
	frame=single,
	language=MOSLMCONF,
	keywordstyle=\bfseries\color{VIOLET},
	commentstyle=\itshape\color{DARKGREEN},
	stringstyle=\color{BLUE},
	morestring=[s][\itshape\color{BLUE}]{[}{]}
}

\setcounter{tocdepth}{1}

\setlength{\parindent}{0pt} 
\setlength{\parskip}{0.3cm}

% Custom links created at: http://tinyurl.com/
\newcommand{\eMonflonHandbookURL}[1]{\url{http://tiny.cc/emoflon-rel-handbook/part#1.pdf}}
\newcommand{\dlPartZero}{\eMonflonHandbookURL{0}}
\newcommand{\dlPartOne}{\eMonflonHandbookURL{1}}
\newcommand{\dlPartTwo}{\eMonflonHandbookURL{2}}
\newcommand{\dlPartThree}{\eMonflonHandbookURL{3}}
\newcommand{\dlPartFour}{\eMonflonHandbookURL{4}}
\newcommand{\dlPartFive}{\eMonflonHandbookURL{5}}
\newcommand{\dlPartSix}{\eMonflonHandbookURL{6}}

\newcommand{\eMoflonUpdateSite}{\url{http://tiny.cc/emoflon-rel-update-site}}
\newcommand{\eMoflonEAAddin}{\url{http://tiny.cc/emoflon-rel-eaaddin}}

\newcommand{\timeOne}{1\,h}
\newcommand{\timeTwo}{2\,h 30\,min}
\newcommand{\timeThree}{4\,h 45\,min}
\newcommand{\timeFour}{3\,h}
\newcommand{\timeFive}{3\,h 30\,min}
\newcommand{\timeSix}{1\,h 20\,min}


% --- HEADER FUNCTIONS % --------------------------------------------------------------------------------------------------------------------------------------
% Default plain header; turn off all lines and colors; turn on page numbers for all
\newcommand{\noHeader}{
 	\fancyhead[R]{\thepage}
 	\fancyhead[L]{}
	\renewcommand{\headrulewidth}{0pt}
}

% Common instruction header; Black
\newcommand{\genHeader}{
 	\fancyhead[L]{}
	\renewcommand{\headrulewidth}{1.5pt}
 	\renewcommand{\headrule}{\hbox to\headwidth{%
  		\color{Black}\leaders\hrule height \headrulewidth\hfill}}
  	\fancyfoot{}
}

% Visual instructions; Red header
\newcommand{\visHeader}{
	\fancyhead[L]{\color{RedOrange}\tiny \bf VISUAL}
	\renewcommand{\headrulewidth}{1.5pt}
	\renewcommand{\headrule}{\hbox to\headwidth{%
  		\color{RedOrange}\leaders\hrule height \headrulewidth\hfill}}
  	\fancyfoot{}
}

% Text instructions; Blue header
\newcommand{\texHeader}{
	\fancyhead[L]{\color{CornflowerBlue}\tiny \bf TEXTUAL}
	\renewcommand{\headrulewidth}{1.5pt}
	\renewcommand{\headrule}{\hbox to\headwidth{%
  		\color{CornflowerBlue}\leaders\hrule height \headrulewidth\hfill}}
  	\fancyfoot{}
}


% --- FORMATTING COMMANDS --------------------------------------------------------------------------------------------------------------------------------------
% 'Next..' jump links (bottom right)
\newcommand{\jumpSingle}[1]{
\fancyfoot[OR]{$\triangleright$ \hyperlink{#1}{\texttt{Next}}}
}

\newcommand{\jumpDual}[2]{
\fancyfoot[RO]{ $\triangleright$ \hyperlink{#1}{\texttt{Next [visual]\hspace{0.2cm}}}%
 \\ $\triangleright$ \hyperlink{#2}{\texttt{Next [textual]}}}
}


% Required time for part (Should appear on introduction page)
\newcommand{\requiredTime}[1]{{\scriptsize \texttt{Approximate time to complete: #1} }}
\newcommand{\downloadLocation}[1]{{\scriptsize \texttt{URL of this document: #1}}}

% These words should appear in the glossary
\newcommand{\define}[1]{\marginpar{\small\emph{#1}}}

% Textual syntax EBNF statement format
\newcommand{\syntax}[1]{ \begin{quote} \small \texttt{#1} \end{quote}}	

\renewcommand{\thesection}{\arabic{section}}
