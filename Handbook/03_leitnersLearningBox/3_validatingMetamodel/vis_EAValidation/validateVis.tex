\visHeader

\hypertarget{validation vis}{Our} EA extension provides rudimentary support for validating both the static semantics (Ecore) and dynamic semantics (SDM) of
metamodels. Validation results are displayed and, in some cases, even ``quick fixes'' to automatically solve the problems are offered. In addition to reviewing
your model, the validation option automatically exports the current model to your eclipse workspace.

\subsection{eMoflon validation support in EA}

\begin{enumerate}
\item[$\blacktriangleright$] If not already active, make the eMoflon control panel visible in EA by choosing ``Extensions/\-Add-In Windows''. This should
display a new output window, as depicted in Fig.~\ref{fig:validation_output}. Many users prefer this interface, as it provides quick access to all of eMoflon's
features, as opposed to the drop down menu under ``Exensions/MOFLON::Ecore Addin" which only offers limited functionality.

\begin{figure}[htbp]
	\centering
  \includegraphics[width=0.9\textwidth]{ea_controlPanel}
	\caption{Activating the validation output window}
	\label{fig:validation_output}
\end{figure}
\FloatBarrier

\clearpage
\item[$\blacktriangleright$] To start the validation, choose ``Validate all'' in the ``Validate" section of the control panel
(Fig.~\ref{fig:validation_menu}). If you haven't made any mistakes while modelling your \texttt{LearningBoxLanguage} so far, the validation results window
should remain empty, indicating your metamodels are free of errors.

\begin{figure}[htbp]
	\centering
  \includegraphics[width=1.0\textwidth]{ea_startValidation}
	\caption{Starting the validation}
	\label{fig:validation_menu}
\end{figure}
\FloatBarrier
\end{enumerate}

If an error did appear, the validation system would try to suggest a ``Quick Fix.'' Why don't we try to get familiar and examine the validation and quick fix
features in detail? Let's add two small modelling errors in \texttt{LearningBoxLanguage}.

\begin{enumerate}
\item[$\blacktriangleright$] Create a new Eclass in the \texttt{Learning\-Box\-Language} diagram. You can retain the default name \texttt{EClass1}. Let's
assume, you wish to delete this class from your metamodel.

\item[$\blacktriangleright$] Select the rouge class in the diagram and press the \texttt{Delete} button on your keyboard. Note that \texttt{EClass1} still
exists in the project browser (and thus in your metamodel).

\item[$\blacktriangleright$] Run the validation test, and notice the new \texttt{Information} message in the validation output
(Fig.~\ref{fig:validation_information}).

\begin{figure}[htbp]
	\centering
  \includegraphics[width=1.0\textwidth]{EA_validationDeleteElement}
	\caption{Validation information error: element still exists}
	\label{fig:validation_information}
\end{figure}

This message informs you that \texttt{EClass1} is not on any diagram, and seeing as it is still in the model, that this \emph{could} be a mistake. As you
can see, just pressing the \texttt{Delete} button is not the proper way of removing a class from a metamodel - It only removes it from the current
diagram!\footnote{Deleting elements properly and other EA specific aspects are discussed in detail in Part VI: Micellaneous}

\item[$\blacktriangleright$] Suppose you were inpecting a different diagram, and were not on the current screen. To navigate to the problematic element in the
\texttt{Project Browser}, click \emph{once} on the information message.

\item[$\blacktriangleright$] To check to see if there are any quick fixes available, \emph{double} click the information message to invoke the ``QuickFix''
dialogue. In this case, there is one quick fix which suggests (properly) deleting the element from the model (Fig.~\ref{fig:quick-fix1}). Since this was indeed
the intent, click \texttt{Ok}.

\begin{figure}[htbp]
	\centering
  \includegraphics[width=0.55\textwidth]{ea_quickFixElements}
	\caption{Quick fix for elements that are not on any diagram}
	\label{fig:quick-fix1}
\end{figure}
\FloatBarrier

\item[$\blacktriangleright$] \texttt{EClass1} should now be correctly removed from your metamodel. Your metamodel should now be error-free again as indicated by
the validation output window.

\item[$\blacktriangleright$] To make an error that leads to a more critical message than ``information,'' double click the navigable reference end
\texttt{previous} of the class \texttt{Partition}, and delete its role name as depicted in Fig.~\ref{fig:delete-role-name}. Affirm with \texttt{OK}.

\begin{figure}[htbp]
    \centering
  \includegraphics[width=1.0\textwidth]{EA_validationDeleteRoleName}
    \caption{Deleting a navigable role name of a reference}
    \label{fig:delete-role-name}
\end{figure}

\item[$\blacktriangleright$] You should now see a new \texttt{Fatal Error} in the validation output, stating that a navigable end \emph{must} have a role name.
Double click the error to view the quick fix menu (Fig.~\ref{fig:fatal-error}). As navigable references are mapped to data members in a Java class, omitting the
name of a navigable reference makes code generation impossible (data members (i.e., class variables) must have a name).

\begin{figure}[htbp]
	\centering
  \includegraphics[width=0.6\textwidth]{EA_quickFixFatal}
	\caption{Fatal error after deleting a navigable role name}
	\label{fig:fatal-error}
\end{figure}

\item[$\blacktriangleright$] Given there are no automatic solutions, correct your metamodel manually by setting the name of the navigable reference back to
\texttt{previous}.

\item[$\blacktriangleright$] Ensure that your metamodel closely resembles Fig.~\ref{fig:metamodel_complete} again, and that there are no error messages before
you proceed with the rest of this handbook.
\end{enumerate}

\vspace*{1cm}

As you may have already noticed, eMoflon distinguishes between five different types of validation messages:
\begin{description}
  \item[Information:]~\\
  This is only a hint for the user and can be safely ignored if you know what you're doing.
  Export and code generation should be possible, but certain naming/modelling conventions are violated, or a problematic situation has been detected.
  
  \item[Warning:]~\\ Export and code generation is possible, but only with defaults and automatic corrections applied by the code generator.
  As this might not be what the user wants, such cases are flagged as warnings (e.g., omitting the multiplicity at references which is automatically set by the
  code generator to 1).
  Being as explicit as possible is often better than relying on defaults.
  
  \item[Error:]~\\ Although the metamodel can be exported from EA, it is not Ecore conform, and code generation will not be possible.
 
  \item[Fatal Error:]~\\ The metamodel cannot be exported as required information, such as names or classifiers of model elements, are incorrectly set or
  missing.
  
  \item[Eclipse Error:]~\\ Display error messages produced by our Eclipse plugin after an unsuccessful attempt to generate code. This is currently not actively
  used.

\end{description}

\fancyfoot[R]{ $\triangleright$ \hyperlink{sec:creatingInstance common}{Next task} }