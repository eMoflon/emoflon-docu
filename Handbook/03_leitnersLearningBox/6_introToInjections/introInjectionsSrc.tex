\newpage
\section{Intro to injections}
\genHeader

In this chapter you will learn how to implement small methods by adding handwritten code to classes created from your model.

Injections are inspired by partial classes in C\#, and are our preferred way of providing a clean separation of generated from handwritten code.

Open ``gen/LearningBoxLanguage/impl/BoxImpl.java'' and implement the code given below {\small (do we want this, and THEN explain, or should we give a long
winded explanation about how to d? they're modfying a java file\ldots) idea. KEEP IN MIND: textual; is there stuff they can copy paste? set it up so they can..}
Do not remove the comment which is necessary to indicate that this code is written by the user and needs to be extracted into our injection file.

// write code (Fig.)

Right-click on BoxImpl.java and choose ``eMoflon/Create Injection for class'' (Fig.) 

This creates a new file in the ``injection'' folder of your project with the same packages and name as the Java class but with \texttt{.inject} as extension
(Fig.). This file contains the definition of a \textit{partial class}~(Fig.).

You now have the choice of implementing your methods directly in the injection file, or in the corresponding Java file and generating the injection from it. We
recommend the latter as you can use the usual support from the Java editor. Just don't forget to update the injection before re-generating code! Now complete
the injection with the code given in Fig.

Rebuild your project (eMoflon $\rightarrow$ Clean and build) and this code will be injected in \texttt{LearningBoxLanguage.impl.BoxImpl.java} (Fig.). 

There are also some extra options you can incorporate with your injections. (Read A:5)

two additional comments? (injection members)

Task: Read the main chapter portion, and figure out how to implement with removeCard