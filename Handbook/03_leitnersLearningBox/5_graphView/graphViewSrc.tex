\newpage
\section{eMoflon's Graph Viewer}
\genHeader
\hypertarget{sec:Graph View}{}

At this point, you should have a small instance model completely built. While Eclipse's built-in editor offers a nice tree structure to view your model,
wouldn't it be nice to see depiction of your customized model? Recalling the definition of a model, we know its simply a graph that ahderes to a set of rules
and contraints. eMoflon's ``Graph View'' offers a visual depiction of this graph!

\begin{itemize}

\item[$\blacktriangleright$] When you first opened the eMoflon perspective, a small window to the right should have appeared with two tabs, ``Outline'' and
``Graph View.'' \footnote{If this window is not open, you can re-activate it by right-clicking on the ``eMoflon'' button on the main toolbar and pressing
``Reset,'' or by going to ``Window/Show View/Other..,'' then ``Other/Graph View''.} 

\item[$\blacktriangleright$] Activate the second tab, then drag a \texttt{Partition} instance into the window. 

\item[$\blacktriangleright$] Here you can see your model, fully expanded to display every element it contains, connected by a named reference line
(Fig.~\ref{fig:graphView_init}).

\begin{figure}[htbp]
	\centering
  \includegraphics[width=1.0\textwidth]{eclipse_graphViewInit}
	\caption{Comment}
	\label{fig:graphView_init}
\end{figure}

\item[$\blacktriangleright$] Clicking any item in the view will highlight it, and hovering over a node will quickly list all its properties. If you hover
over a card, for example, you should be able to see their \texttt{back} and \texttt{face} values. You can also drag elements to new positions.

\item[$\blacktriangleright$] Try dragging the second \texttt{Partition} element from the model to the viewer. As you can see, it doesn't replace the current
view, it simply places them side-by-side. To clear the screen, click the upside-down triangle in the top right corner of the window, and select ``Clear View.''

\item[$\blacktriangleright$] To make the graph bigger, increase the size of the window, then select ``Redraw Graph'' from the same upside-down arrow. As you can
see, the graph view is not automatically updated. This option is useful if you change a property of an element (ie., the \texttt{back} value of a \texttt{Card})
and wish to see it reflected in the graph.

\item[$\blacktriangleright$] Try experimenting with each of the different layouts and zoom settings. You'll notice for the ``Spring Layout'' that each time you
press ``Redraw Graph,'' the graph will re-arrange itself, even if you haven't updated any values or changed the window size.

\item[$\blacktriangleright$] This feature is not exclusively for models - Expand your platform in the ecore editor, and drag and drop the
\texttt{LearningBoxLanguage} package into the viewer. Your entire metamodel, or type graph, is now completely displayed (Fig.~\ref{fig:graphView_typeGraph})! We
have found that the ``Radial Layout'' works best for a model of this complexity.

\begin{figure}[htbp]
	\centering
  \includegraphics[width=1.0\textwidth]{eclipse_entireTypeGraph}
	\caption{Comment}
	\label{fig:graphView_typeGraph}
\end{figure}

\fancyfoot[R]{ $\triangleright$ \hyperlink{sec:injections common}{Next}}
\end{itemize}

Seque
