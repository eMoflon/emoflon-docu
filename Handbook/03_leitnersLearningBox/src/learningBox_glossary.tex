\section{Glossary of Terms}
\vspace{0.5cm}

\begin{itemize}

\item[\bf Concrete Syntax]
The implementation syntax (ie: visual or textual) of a model. \hfill \hyperlink{page.3}{3} 

\item[\bf Grammar] 
A set of rules a language must follow. \hfill \hyperlink{page.3}{3} 

\item[\bf Graph Grammar] 
Grammar 'rules' presented as a graph. \hfill \hyperlink{page.3}{3}

\item[\bf Type Graph] 
Define the types and relations that form a language. \hfill \hyperlink{page.3}{3}

\item[\bf Abstract Syntax] 
Strictly conforms to the type graph for a language; The types and relations defined here must exist without errors. \hfill \hyperlink{page.3}{3}

\item[\bf Static Semantics] 
An additional list of rules and constraints a language must obey.\hfill \hyperlink{page.3}{3}

\item[\bf Metamodel] 
Defines the abstract syntax of a language and includes some static semantics for a model. \hfill \hyperlink{page.3}{3}

\item[\bf Constraint Language] 
Complex constraints (static semantics) that cannot be expressed in a metamodel. \hfill \hyperlink{page.3}{3}

\item[\bf Model] 
Graphs which conform to some metamodel. \hfill \hyperlink{page.4}{4}

\item[\bf Unification]  
The ``Everything is a model.'' principle, even the metamodel which defines models.\hfill \hyperlink{page.7}{7}, \hyperlink{page.22}{22}

\item[\bf Meta-metamodel] 
A model that defines a \emph{modeling language} for metamodels.\hfill \hyperlink{page.7}{7}, \hyperlink{page.22}{22}

\item[\bf Meta-Language] 
A language used to define another language through a consistent set of rules. \hfill \hyperlink{page.7}{7}, \hyperlink{page.22}{22}

\item[\bf Modeling Language] 
See Meta-language; Graphical languages use diagrammed techniques, while textual languages use standardized keywords. \hfill \hyperlink{page.7}{7}, \hyperlink{page.22}{22}

\item[\bf Dynamic Semantics] 
Set of rules that define a system's behavior and reactions to external stimulus. \hfill \hyperlink{page.17}{17}, \hyperlink{page.26}{26}

\end{itemize}