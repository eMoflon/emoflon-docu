% make sure not a repeat header!
\subsection{Getting started in EA}
\visHeader
\hypertarget{static:starting vis}{}

\begin{itemize}
\item[$\blacktriangleright$]  If you have continued from Part I, you can begin modeling this project in two different ways. You can either continue in the same
workspace as the demo by opening the \texttt{demo.eap} file in EA, initializating a new diagram, and then working within that space, or you can start this
project with a fresh space as instructed below. If you choose to do this, please note that the steps are exactly the same, but our package explorers may not
exactly match. Keep a sharp eye out for footnotes which will explain the few differences. This handbook has assumed you prefer the latter.

\item[$\blacktriangleright$]  To begin, navigate to ``New Metamodel Project,'' and start a new visual project (without demo specifications) named `Leitners
Learning Box' (Fig.~\ref{fig:new_visModel}). Open the empty \texttt{.eap} file in EA.

\begin{figure}[htbp]
	\centering
  \includegraphics[width=0.8\textwidth]{eclipse_newMetamodelVisualPlain}
	\caption{Starting a new visual project {\bf update 'LeitnersBox'}}
	\label{fig:new_visModel}
\end{figure}

\newpage

\item[$\blacktriangleright$] From EA, select your working set and press the ``Add a Package'' button (Fig.~\ref{fig:new_package})\footnote{If using the previous
files, you won't have the \texttt{eMoflon Languages} package, since all the needed files were included in \texttt{Demo}. For this simple example, we won't be
using all the features of eMoflon, so this limited collection will suffice.}.

\begin{figure}[htbp]
	\centering
  \includegraphics[width=0.5\textwidth]{ea_addPackage}
	\caption{Add a new package to \texttt{MyWorkingSet}}
	\label{fig:new_package}
	\vspace{0.5cm}
\end{figure}

\vspace{1.0cm}

\item[$\blacktriangleright$] In the dialogue that pops up (Fig.~\ref{fig:new_package_name}), enter `Learning\-Box\-Language' as the name of the new package.
Make sure \texttt{Class View} is selected, and click \texttt{OK}.

\begin{figure}[htbp]
	\centering
    \includegraphics[width=0.33\textwidth]{ea_namePackage.png}
	\caption{Enter the name of the new package}
	\label{fig:new_package_name}
\end{figure}
\FloatBarrier

\vspace{1.0cm}

\item[$\blacktriangleright$] Your \texttt{Project Browser} should now resemble Fig.~\ref{fig:new_package_completed}.

\begin{figure}[htbp]
	\centering
  \includegraphics[width=0.5\textwidth]{ea_newPackage}
	\caption{State after creating the new package.}
	\label{fig:new_package_completed}
\end{figure}
\FloatBarrier

\clearpage
\item[$\blacktriangleright$] Now select your package and create a ``New Diagram'' (Fig.~\ref{fig:diagram}).

\begin{figure}[htbp]
	\centering
  \includegraphics[width=0.5\textwidth]{ea_addDiagram}
	\caption{Add a diagram.}
	\label{fig:diagram}
\end{figure}
\FloatBarrier

\item[$\blacktriangleright$] In the dialogue that appears, (Fig.~\ref{fig:diagram_type}), choose \texttt{eMoflon Ecore Diagrams} and press \texttt{OK}. 

\begin{figure}[htbp]
	\centering
  \includegraphics[width=0.8\textwidth]{ea_chooseDiagramType}
	\caption{Select the ecore diagram type}
	\label{fig:diagram_type}
\end{figure}
\FloatBarrier

 
\item[$\blacktriangleright$] After creating the new diagram, your  \texttt{Project Browser} should now resemble Fig.~\ref{fig:diagram_completed}. You'll notice
that your \texttt{LearningBoxLanguage} package has transformed into a container. This is now be the source location for any and all ecore diagrams for your
project; All generated files are derived from the pieces stored in this package.

\begin{figure}[htbp]
	\centering
  \includegraphics[width=0.5\textwidth]{ea_afterDiagramState}
	\caption{State after creating diagram}
	\label{fig:diagram_completed}
\end{figure}
\FloatBarrier

\newpage

\item[$\blacktriangleright$] To finalize the initalization of your metamodel, export your project to Eclipse\footnote{If unsure how to perform this step, please
refer to Section 2.1 in Part I.}, then refresh your \texttt{Package Explorer}. A new node, \texttt{My Working Set}\footnote{If you do not have the two distinct
nodes, make sure your ``Top Level Elements'' are set to \texttt{Working Sets}} should have appeared containing your adapted \texttt{Epackage}
(Fig.~\ref{fig:init_import}). You can see the that a \texttt{LearningBoxLanguage.ecore} file has been generated, and placed in ``model.'' This is the graph of
your metamodel that must adhere to all future types and restrictions you create in your diagrams.

\vspace{0.5cm}

\begin{figure}[htbp]
	\centering
  \includegraphics[width=0.55\textwidth]{eclipse_initExport}
	\caption{Inital import to Eclipse}
	\label{fig:init_import}
\end{figure}
\FloatBarrier

\vspace{0.5cm}

\item[$\blacktriangleright$] If you'd like to review the overall project structure, the purposes of certain files and folders, read Section 4 from
Part~I\footnote{\downLink} of this handbook. Otherwise, continue to the next section learn how to declare your classes and attributes.

\fancyfoot[R]{$\triangleright$ \hyperlink{static:classes vis}{Next}}
\end{itemize}
