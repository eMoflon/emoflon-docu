\newpage
\subsection{Method Signatures}
\visHeader
\hypertarget{static:methods vis}{}

Now we shall define the \emph{signatures} of some operations for our learning box. (Because\ldots)

\begin{enumerate}
  
\item[$\blacktriangleright$] Right-click \texttt{Partition} to invoke the context-menu depicted in Fig.~\ref{fig:add_operation} and choose ``Features \&
Properties/Operations..''

\begin{figure}[htbp]
	\centering
  \includegraphics[width=0.8\textwidth]{ea_contextAddOperation}
	\caption{Add an operation}
	\label{fig:add_operation}
\end{figure}
\FloatBarrier

\item[$\blacktriangleright$] In the dialogue that pops-up (Fig.~\ref{fig:operation_properties}), enter `empty' as the \texttt{Name} of the operation,
leave the \texttt{Return Type} as `void', and press \texttt{Save}.

\item[$\blacktriangleright$] In the same dialogue, press \texttt{New} to add further operations and enter the values in Fig.~\ref{fig:operation_parameters}. 
Parameters can be added by pressing \texttt{Edit}\footnote{You must save the operation before this option will become active.} and entering the name and
choosing the type of each parameter in a separate dialogue.

\begin{figure}[htbp]
	\centering
  	\includegraphics[width=0.9\textwidth]{ea_operationEmpty}
	\caption{Properties for operation}
	\label{fig:operation_properties}
\end{figure}
\FloatBarrier

\begin{figure}[htbp]
	\centering
  \includegraphics[width=0.9\textwidth]{ea_operationRemoveCard}
	\caption{Parameters and Return Type}
	\label{fig:operation_parameters}
\end{figure}
\FloatBarrier

\item[$\blacktriangleright$] Repeat the process for the \texttt{check} operation in Fig.~\ref{fig:operation_partition}. Notice that the \texttt{Return Type} can
be chosen by either the drop-down menu for primitives (e.g. \texttt{EBoolean}), or via the `\texttt{\ldots}' button (highlighted in
Fig.~\ref{fig:operation_properties}) for types you've established in the metamodel (e.g. \texttt{Card}).

\vspace{-.3cm}
\begin{quote}
{ \small
$\textbf{Please note:}$ Non-primitive types \emph{must} be chosen via the `\texttt{\ldots}' button. It allows you to browse for the corresponding elements in
your project. Simply typing them won't work!
}
\end{quote}

\item[$\blacktriangleright$] If you've done everything right, your dialogue should now contain three methods - \texttt{check}, \texttt{empty}, and
\texttt{removeCard} - each with the corresponding parameters and return types in Fig.~\ref{fig:operation_partition}.

\begin{figure}[htbp]
	\centering
  \includegraphics[width=0.9\textwidth]{ea_operationCheck}
	\caption{All operations in \texttt{Partition}}
	\label{fig:operation_partition}
\end{figure}

\item[$\blacktriangleright$] Add all operations analogously for \texttt{Box} and \texttt{Card} so that your metamodel closely resembles
Fig.~\ref{fig:metamodel_complete}\footnote{Please note that names of parameters may not be displayed by default in EA.}.

\begin{figure}[htbp]
	\centering
  \includegraphics[width=0.7\textwidth]{ea_metamodelComplete.png}
\caption[Complete metamodel for our learning box.]{Complete metamodel for our learning box}
	\label{fig:metamodel_complete}
\end{figure}

\pagebreak

\item[$\blacktriangleright$] To see how this complete metamodel is represented in the textual syntax, examine Fig.~\ref{fig:workspaceMethods} in the following
section.

\item[$\blacktriangleright$] To finish, try to export the metamodel for code generation in Eclipse.

\item[$\blacktriangleright$]  If code is generated successfully, take a look at all the stuff that's been generated under \texttt{/gen}. (Fig.). In particular,
look at the the default implementation for all methods who throw an \texttt{OperationNotSupported} exception. We shall shortly discover that the Eclipse
Modeling Foundation's (EMF) code generator supports injecting hand-written implementations of methods into these already generated methods and classes.
With eMoflon however, we can model a large part of the dynamic semantics and only need to implement small helper methods (such as string manipulation) by
hand.

\end{enumerate}

\fancyfoot[R]{$\triangleright$ \hyperlink{static review}{Next}}
