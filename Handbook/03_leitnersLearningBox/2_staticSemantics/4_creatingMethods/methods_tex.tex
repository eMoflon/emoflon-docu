\newpage
\subsection{Method Signatures}
\texHeader
\hypertarget{static:methods tex}{}

\begin{itemize}

\item[$\blacktriangleright$] We're nearing the end of our model creation! One of the last things we need to do is to make the program \emph{do} something. After
all, what good is a program that only stores attributes and references?

\item[$\blacktriangleright$] Remember when we mentioned Dynamic Semantics? Let's set up these operations by declaring their \emph{signatures}, starting with the
\texttt{Partition} class. We want a partition to be able to do three things: compare the answer on a \texttt{Card} with a guess and return a true/false
response, remove a specific card, or empty itself of all cards.

% Go over syntax..
\item[$\blacktriangleright$] Start with the \texttt{empty} method - it won't need any parameters, and it doesn't need to return anything. Declare this via an
\texttt{empty():void}.

% Except now we need to include parameters\ldots
\item[$\blacktriangleright$] Create two more functions for \texttt{Partition} the same way. We'll need a \texttt{removeCard} method that accepts and returns a
\texttt{Card}, as well as a EBoolean \texttt{check} method that accepts a \texttt{Card} and an \texttt{EString} guess. Your partition class should now resemble
Fig.~\ref{fig:partitionMethods}.

\begin{figure}[htbp]
	\centering
  \includegraphics[width=0.6\textwidth]{eclipse_partitionMethods}
	\caption{The completed \texttt{Partition} class}
	\label{fig:partitionMethods}
\end{figure}

\item[$\blacktriangleright$] What needs to be done in the \texttt{Card} class? Well, in order to check the card, we'll need to be able to look at the flip side.
We'll also need to print whatever is on the current side. Create two paramater-less void functions, \texttt{invert} and \texttt{printCard}.

\item[$\blacktriangleright$] Finally, what do we need to do with the largest object in our model, the \texttt{Box}? In summary, we want a \texttt{Box} to:

\begin{description}
	{\small
  \item[\texttt{determineNextSize():EInt }] find out how large the upcoming partition is
  \item[\texttt{grow():void}] increase in size to allow more partitions
  \item[\texttt{toString():EString}] Helper method for (below)
  \item[\texttt{addToStringRep(card:Card):void}] \ldots so we can represent them as a string
  }
\end{description}


\item[$\blacktriangleright$] Your workspace should now resemble Fig.~\ref{fig:workspaceMethods}.
\begin{figure}[htbp]
	\centering
  \includegraphics[width=1.0\textwidth]{eclipse_workspaceMethods}
	\caption{Completed method signatures}
	\label{fig:workspaceMethods}
\end{figure}


Congratulations! You've \emph{almost} completeley modeled Leitner's Learning Box using a concrete, textual syntax! To see how this appears in a class diagram,
check out Fig.~\ref{fig:metamodel_complete} from the visual syntax.

\item[$\blacktriangleright$]The very last thing we need to do is conduct a build and generate the required \texttt{.genmodel} and \texttt{.ecore} files. Beside
the \texttt{New Metamodel} button on the toolbar, you'll notice that there is a circular arrow button that offers to ``Build (without cleaning).''   %explain
this here? Press it, and wait for the package explorer to refresh.

\item[$\blacktriangleright$] If you've done everything correctly, a new \texttt{MyWorkingSet} node should have appeared, and your entire expanded explorer
should resemble Fig.~\ref{fig:builtModel}.

\begin{figure}[htbp]
	\centering
  \includegraphics[width=0.5\textwidth]{eclipse_finalPackageExplorer}
	\caption{Final project structure of our Static Semantics}
	\label{fig:builtModel}
\end{figure}

\item[$\blacktriangleright$] Examine the generated files in \texttt{gen} folder, especially the default implementation for all methods that just throw an
\texttt{OperationNotSupported} exception. We shall see in later parts of this handbook that the EMF code generator actually supports injecting hand-written
implementation of methods into generated methods and classes. With eMoflon however, we can model a large part of the dynamic semantics with ease, and only need
to implement small helper methods (such as those for string manipulation) by hand.

\item[$\blacktriangleright$] Finally, you can see that your two metamodels have been created and placed in the \texttt{model} folder. Together, they form
everything the EMF needs to generate your program. If you like, instead of viewing your \texttt{.ecore} model in a tree diagram, you can request eclipse to
build a visual diagram with its built-in visual modeler! Right click on \texttt{LearningBoxLanguage.ecore} and select ''Initilalize Ecore Diagram File.''

\fancyfoot[R]{ $\triangleright$ \hyperlink{static review}{Next}}

\item[$\blacktriangleright$] You're all done!

\end{itemize}