\newpage
\section{eMoflon in a Jar}
\genHeader

This section describes how to package code generated with eMoflon into
runnable Jar files, which is useful if you want to build applications for end-users.

We distinguish between repository, i.e., SDM-based, and integration,
i.e., TGG-based, projects.

\subsection{Packaging SDM projects into a Jar file}

The following explanations use the demo specification that is shipped with
eMoflon to explain the workflow of building a runnable Jar file.

\begin{itemize}
    
\item[$\blacktriangleright$] 
Open a fresh workspace and add to it the eMoflon Demo specification by selecting the ``Install, configure and deploy Moflon'' button and open the  ``Install Workspace'' menu bar. Select the ``Demo Workspace''.
    
\item[$\blacktriangleright$]
Generate code for the demo and verify the result by running the test cases in \emph{DemoTestSuite}.

\item[$\blacktriangleright$]
Add a suitable main method to \texttt{NodeTest}, for instance:
\begin{lstlisting}
public static void main(String[] args) {
    System.out.println("Begin of test runs");
    new NodeTest().testDeleteNode();
    new NodeTest().testInsertNodeAfter();
    new NodeTest().testInsertNodeBefore();
    System.out.println("End of test runs");
}
\end{lstlisting}

\item[$\blacktriangleright$]
Run \texttt{NodeTest} as ``Java Application" (\emph{not} as ``JUnit Test").
Now you have a new launch configuration named ``NodeTest".

\item[$\blacktriangleright$]
Now, select the repository project (containing the generated code) and the project \emph{DemoTestSuite}.
You do not need to add the project containing the EA project.
Right-click and select ``Export...".
Choose ``Runnable JAR file".
  
\item[$\blacktriangleright$]
On the next page, select the launch configuration you just created by running \texttt{NodeTest} and an appropriate target location for your Jar file.
The libraries should be packaged or extracted into the generated Jar file.

\item[$\blacktriangleright$]
Afterwards, open up a console in the folder containing the generated Jar file and execute it as follows:
\begin{lstlisting}
java -jar [GeneratedJarFile.jar]
\end{lstlisting}

    
\end{itemize}

\subsection{Packaging TGG projects into a Jar file}

In the following, you will create a runnable Jar from a TGG specification.
We assume that you have some existing TGG implementation and that you want to execute the \texttt{main} method in class \texttt{org.moflon.tie.MyIntegration\-Trafo}.

\emph{Note:}
The following instructions show how to use Eclipse's built-in facility for generating runnable Jars.
There are other build tools such as ant, Maven or Gradle that facilitate this process.

\begin{itemize}
    
\item[$\blacktriangleright$] 
Ensure that your TGG rules from within Eclipse.
For simplicity, we assume that your main method currently resembles the following snippet:
\begin{lstlisting}
public static void main(String[] args) throws IOException {
    // Set up logging
    BasicConfigurator.configure();

    // Forward Transformation
    MyIntegrationTrafo helper = 
        new MyIntegrationTrafo();
    helper.performForward("instances/fwd.src.xmi");
}
\end{lstlisting}
The default transformation helper should print a short success message when the forward transformation has finished.

\item[$\blacktriangleright$]
Before packaging your project, you have to change the ways how the TGG rules are being loaded (in the constructor of \texttt{MyIntegration\-Trafo}).
Replace this method call
\begin{lstlisting}
loadRulesFromProject("..");
\end{lstlisting}
with
\begin{lstlisting}
File jarFile = new File(MyIntegrationTrafo.class
    .getProtectionDomain().getCodeSource()
    .getLocation().toURI().getPath());
loadRulesFromJarArchive(
    jarFile,
    "/MyIntegration.sma.xmi");
\end{lstlisting}
This is a tiny trick to find out the name of the Jar file that you are about to build.
If you already know the name of your Jar file (e.g., ``tggInAJar.jar"), you could simply use the following code:
\begin{lstlisting}
loadRulesFromJarArchive(
    "tggInAJar.jar",
    "/MyIntegrationTrafo.sma.xmi");
\end{lstlisting}


\item[$\blacktriangleright$]
Next, make the ``model" directory a source folder by right-clicking it and selecting ``Build Path/Use as Source Folder".
This will make the contents of ``model" available in the Jar file to be built.


\item[$\blacktriangleright$]
Now, your projects are ready to be packaged.
Select all projects that are involved in your TGG, that is, the project of the source and target metamodel as well as the actual integration project.

Right-click the projects and select ``Export..." and then ``Java/Runnable JAR File".

\item[$\blacktriangleright$]
Select the appropriate launch configuration (named ``MyIntegrationTraf"), choose the export destination, and make sure that the library handling is set to ``Extract required libraries".

\item[$\blacktriangleright$]
After a successful export, locate the generated Jar file.
The program expects to find the source model of the transformation at the following path, relative to the folder containing your Jar file: ``instances/fwd.src.xmi".

Now, let's take the transformation for a spin:
\begin{lstlisting}
java -jar [GeneratedJarFile.jar]
\end{lstlisting}

\end{itemize}



   
