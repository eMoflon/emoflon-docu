\subsection{Positioning elements}

Layout is always an important factor when using a visual language:
A well laid-out diagram is easiest to understand and, by centralizing important
elements or clustering related elements, you can actually impart additional information.

\begin{stepbystep}
\item To select a group of elements, either drag a selection box around the items or hold \texttt{Ctrl} and select each element
one-by-one.

\item In the top right corner of the last selected element, a small colon-styled symbol will appear (\Cref{ea:layout1}). Click on
this for a context list of different options you can simultaneously apply to all active elements. The same list appears on the toolbar above the
diagram. 

\item Experiment to find out what effect each option has. The last symbol in the list opens a further drop-down menu with standard layout
algorithms to organize your diagram automatically.

\item Right-clicking any of the selected elements opens a different menu with a further set of layout options and their descriptions
(\Cref{ea:layout2}). \texttt{Align Centers} or \texttt{Same Height and Width} can be especially useful.

\begin{figure}[htbp]
\begin{center} 
  \includegraphics[width=0.65\textwidth]{ea_layoutElementsCommonContext}
  \caption{Setting the layout of multiple elements}  
  \label{ea:layout1}
\end{center}
\end{figure}

\begin{figure}[htbp]
\begin{center}  
  \includegraphics[width=0.7\textwidth]{layoutElements2}
  \caption{Further layout options}  
  \label{ea:layout2} 
\end{center}
\end{figure}

\end{stepbystep}
