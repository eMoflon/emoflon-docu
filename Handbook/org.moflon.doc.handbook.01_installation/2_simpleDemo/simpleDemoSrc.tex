\newpage
\genHeader

\section{Get a simple demo running}


\begin{stepbystep}
\hypertarget{simpleDemo common}{} 
\item
Open Eclipse to a clean, fresh workspace.
Go to \menuPath{Window \menuSep Open Perspective \menuSep Other\ldots} \footnote{A path given as \menuPath{foo \menuSep bar} indicates how to navigate in a series of menus and toolbars.
New definitions or concepts will be \newconcept{italicized}, and any data you're required to enter, open, or select will be given as \entity{command}.} and choose \texttt{eMoflon} (\Cref{eclipse:openPerspective}).

\begin{figure}[htbp]
	\centering
  \includegraphics[width=0.5\textwidth]{eclipse_openPerspective}
	\caption{Choose the eMoflon perspective}
	\label{eclipse:openPerspective}
\end{figure} 

\item
At either the far right or center of the toolbar, a new action set should have appeared.
Navigate to \menuPath{\enquote{eMoflon Cloud} \menuSep Install Workspace} \LK{Please update the screenshot.}
(\Cref{eclipse:newMetamodel}).
%
\vspace{0.5cm}
\begin{figure}[htbp]
	\centering
  \includegraphics[width= 0.7\textwidth]{eclipse_MoflonButton}
	\caption{Invoking \enquote{Install Workspace $\to$ eMoflon Examples $\to$ Demo (Double-Linked List)}}
	\label{eclipse:newMetamodel}
\end{figure}

\item
In this menu you can check out different workspaces for Eclipse. 
Here you can also check out workspaces for the handbook tutorials (\menuPath{eMoflon Examples}).
How you can create your own projects is described in Part I.
As the first tutorial select \menuPath{eMoflon Examples \menuSep Demo (Double-Linked List)}. 

All our handbook examples are provided via Git and are hosted on GitHub\footnote{See \url{https://github.com/eMoflon/emoflon-examples}.}.
If you encounter problems when fetching some handbook example, the reason may be that your checked out working copy is in a \enquote{dirty} state.
In this case, you may start over by 
\begin{inparaenum}
\item removing all affected projects from your workspace (do \emph{not} tick \menuPath{Delete project contents on disk}),
\item navigating to the \menuPath{Git} perspective in Eclipse (\menuPath{Window \menuSep Perspective \menuSep Open Perspective \menuSep Other\dots}) and
\item performing one of the following:
\end{inparaenum}
\begin{itemize}
\item The drastic solution:
Right-click your working copy \eMoflonExamplesRepo and choose \menuPath{Delete Repository...}.
Make sure to fully delete the repository by ticking all boxes.
\item The smooth solution:
Right-click your working copy \eMoflonExamplesRepo and select \menuPath{Clean\dots}.
Tick \menuPath{Clean selected untracked files and directories} and \menuPath{Include ignored resources}, and confirm with \menuPath{Finish}.
The previous command has purged all untracked and ignored files but modifications of files tracked by Git are still kept in their modified state.
To undo these changes, right-click your working \eMoflonExamplesRepo copy again and select \menuPath{Reset\dots}.
Select \menuPath{master} and tick \menuPath{Hard}, then confirm with \menuPath{Reset}.
\end{itemize}


\item
Another button in the new action set is \menuPath{View and configure logging} represented by an \texttt{L} (\Cref{eclipse:logger}).
Clicking this icon will open a \texttt{log4jConfig\-.properties} file where you can silence certain loggers, set the level of loggers, or configure other
settings.\footnote{If you're not sure how to do this, check out a short Log4j tutorial a \url{http://logging.apache.org/log4j/1.2/manual.html}} All of eMoflon's
messages appear in our console window, just below your main editor. This is automatically opened when you selected the \texttt{eMoflon} perspective and
contains important information for us if something goes wrong!

\begin{figure}[htbp]
	\centering
  \includegraphics[width=\textwidth]{eclipse_logger}
	\caption{The eMoflon console with log messages}
	\label{eclipse:logger}
\end{figure} 
\end{stepbystep}

\clearpage
\genHeader

\subsection{A first look at EA}

\begin{itemize}
\FloatBarrier
\hypertarget{simpleDemo vis}{}
\item[$\blacktriangleright$] Can you locate the new \texttt{Demo.eap} file in your package explorer? This is the EA project file you'll be
modelling in. Don't worry about any other folders at the moment - all problems will be resolved by the end of this section.

In the meantime, do not rename, move, or delete anything.

\item[$\blacktriangleright$] Double-click \texttt{Demo.eap} to start EA, and choose \texttt{Ultimate} when starting EA for the first time.

\item[$\blacktriangleright$] In EA, select ``Extensions/Add-In Windows'' (Fig.~\ref{ea:validate_dropdown}). This will activate our tool's full
control panel. If nothing happens the installation was probably not successful. Work again through the installation or have a look at the following site:
\newline
\href{https://github.com/eMoflon/emoflon/wiki/Using-EA-With-A-Second-Windows-Account}{https://github.com/eMoflon/emoflon/wiki/Using-EA-With-A-Second-Windows-Account}.

\vspace{0.5cm}

\begin{figure}[htbp]
	\centering
  \includegraphics[width=1\textwidth]{ea_extensionMenu}
	\caption{Export from EA} 
	\label{ea:validate_dropdown} 
\end{figure}

\item[$\blacktriangleright$] This tabbed control panel provides access to all of
eMoflon's functionality. This is where you can validate and export your complete project to Eclipse by pressing \texttt{All} (Fig.~\ref{ea:controlPanel}).

\begin{figure}[htbp]
	\centering
  \includegraphics[width=0.9\textwidth]{ea_controlPanelValidateAll}
	\caption{eMoflon's control panel in EA} 
	\label{ea:controlPanel} 
\end{figure}

\item[$\blacktriangleright$] Now try exploring the EA project browser! Try to navigate to the packages, classes, and diagrams. Don't worry if you don't
understand that much - we'll get to explaining everything in a moment. Just make sure not to change anything!

\item[$\blacktriangleright$] Switch back to Eclipse, choose your metamodel project, and press \texttt{F5} to refresh. 
The export from EA places all required files in a hidden folder (.temp) in the project.
A new, third project named \emph{org.moflon.demo.doublelinkedlist} is now being created.
Do not worry about the problem markers.

\item[$\blacktriangleright$] The three asterisks (Fig.~\ref{eclipse:dirty-project}) signal that the project still needs to be built.

\vspace{0.5cm}

\begin{figure}[htbp]
    \centering
    \includegraphics[width=0.35\textwidth]{eclipse_dirtyProject}
    \caption{Dirty projects are marked with ***} 
    \label{eclipse:dirty-project} 
\end{figure}

\vspace{0.5cm}

\item[$\blacktriangleright$] Now, right-click \emph{org.moflon.demo.doublelinkedlist} and choose ``eMoflon/Build" (or use the shortcut Alt+Shift+E,B\footnote{First press Alt+Shift+E, release, and press B.
By default, most shortcuts eMoflon start with Alt+Shift+E.}).

eMoflon now generates the Java code in your repository project.
You should be able to monitor the progress with the green bar in the lower right corner (Fig.~\ref{eclipse:build}). Pressing the
symbol opens a monitor view that gives more details of the build process. You don't need to worry about any of these details, just remember to \begin{inparaenum}[(i.)]
\item refresh your Eclipse workspace after an export, and
\item rebuild projects that bear a ``dirty" marker (\texttt{***}).
\end{inparaenum}

\begin{figure}[htbp]
    \centering
    \includegraphics[width=1\textwidth]{eclipse_buildingProgress}
    \caption{Eclipse workspace when using visual syntax} 
    \label{eclipse:build} 
\end{figure}

\item[$\blacktriangleright$] If you're ever worried about forgetting to refresh your workspace, or if you just don't want to bother with having to do this,
Eclipse does offer an option to do it for you automatically. To activate this, go to ``Window/Preferences/General/Workspace" and select \texttt{Refresh on
access}.

\end{itemize}

