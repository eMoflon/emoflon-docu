\newpage
\section{Model-To-Tree with TGGs}
\genHeader

For those who have read Part IV, do you remember what one of the goals of using TGGs was? We hoped that, by specifying one direction of the transformation, we
could get the other free. That is exactly what's happened here! The final output of the forward transformation was the model specified in
\texttt{tree.xmi\_fwd.xmi}. This was used as input in the reverse transformation, whose final tree output is \texttt{tree.xmi\_FWD.xmi\_BWD.xmi}. If our
bidirectional transformation was successful, this tree should match the original text instance, \texttt{tree.xmi}. Let's compare the two.

\vspace{0.5cm}

\begin{figure}[htpb]
\begin{center}
  \includegraphics[width=\textwidth]{eclipse_generatedBackwardsModel}
  \caption{needs refinement\ldots}
  \label{eclipse:generatedBkwrdMdl}
\end{center}
\end{figure}

\vspace{0.5cm}

It's close, but not perfect. You can see that some things need to be refined. The ``DICTIONARY'' and ``ENTRY'' labels, for example, are missing from the
major nodes. You'll also notice that, as you scroll through each \texttt{Node}, the title and author have the correct \texttt{index} values of 0 and 1, but
unfortunately, \emph{every} other node is also set to 0. We need to bind each \texttt{entry} to a value so avoid potential conflicts when executing our TGG
transformation again, where our \texttt{NodeToDictionaryRule} assumes whatever node has this value must be the \texttt{authorNode}. Luckily, both of these are
very simple, quick fixes. We just need some more attribute constraints!

\jumpDual{m2tvis}{m2ttex}

\subsection{Vis model to tree}
\visHeader

I BELIEVE IN THE ..


\newpage
\hypertarget{m2ttex}{}
\subsection{Polishing the TGG Transformation}
\texHeader

\begin{itemize}

\item[$\blacktriangleright$] Open \texttt{NodeToDictionaryRule}, and add the following constraint to \texttt{dictionaryNode} as shown in
Fig.~\ref{eclipse:NodeToDictionaryRuleUpdated}.

\vspace{0.5cm}

\begin{figure}[htp]
\begin{center}
  \includegraphics[width=0.7\textwidth]{eclipse_NodeToDictionaryUpdate}
  \caption[labelInTOC]{Updating \texttt{NodeToDictionaryRule}}
  \label{eclipse:NodeToDictionaryRuleUpdated}
\end{center}
\end{figure}


\item[$\blacktriangleright$] Similarly, open \texttt{ForAllEntryRule} and edit \texttt{entryNode} (Fig.~\ref{eclipse:ForAllEntryRuleUpdated}).\footnote{Please
note that the MOSL parser requires all attribute constraints to be declared before link variables.}

\vspace{0.5cm}

\begin{figure}[htp]
\begin{center}
  \includegraphics[width=0.7\textwidth]{eclipse_ForAllEntryRuleUpdated}
  \caption[labelInTOC]{Updating \texttt{ForAllEntryRule}}
  \label{eclipse:ForAllEntryRuleUpdated}
\end{center}
\end{figure} 

\item[$\blacktriangleright$] Now let's introduce our custom constraint to handle each \texttt{entryNode.index}
value. Update the \texttt{constraints} scope in \texttt{ForAllEntryRule} as shown in Fig.~\ref{eclipse:newEntryConstraint}.\footnote{ To review the purpose of
constraints and discuss the significance of each of the attribute options, refer to Part IV, Section 4.7.}

\vspace{0.5cm}

\begin{figure}[htbp]
\begin{center}
  \includegraphics[width=0.7\textwidth]{eclipse_SetDefaultNumberConstraint}
  \caption{A final constraint for each \texttt{entry}}
  \label{eclipse:newEntryConstraint}
\end{center}
\end{figure}

\item[$\blacktriangleright$] That's everything! Save and build your updated \texttt{DictionaryCodeAdapter}, but don't run the TGG yet -- we haven't provided any
implementation for \texttt{setDefaultNumber}.

\end{itemize}


\newpage
\hypertarget{common cspConstraint}{}
\subsection{Implementing SetDefaultNumber}
\genHeader

\begin{itemize}

\item[$\blacktriangleright$]  Navigate and expand ``DictionaryCodeAdapter/src/csp.constraints.'' Open \texttt{SetDefaultNumber.java} and edit this file until it
matches Fig.~\ref{eclipse:setDefaultImpl}.

\vspace{0.5cm}

\begin{figure}[htbp]
\begin{center}
  \includegraphics[width=0.9\textwidth]{eclipse_setDefaultNumberImplementation}
  \caption{Completing the custom \texttt{setDefaultNumber} constraint}
  \label{eclipse:setDefaultImpl}
\end{center}
\end{figure}

\item[$\blacktriangleright$] After saving, and run \texttt{TGGMain} one more time. The initial and final \texttt{tree} variants should now be nearly
identical! The only difference should be that, instead of each \texttt{entry.index} value increasing (as seen in \texttt{tree.xmi}), each value should now be
set to exactly 2. Their order may be different, depending on how the transformation processed them, but their \texttt{index} values should all  be correct.

\vspace{0.5cm}

\item[$\blacktriangleright$] Your transformation is nearly complete! The only remaining step is to unparse this final tree structure into an output filesystem.
Before we move on however, let's reflect on how easy and short it was to implement this `backwards' transformation. If we were to use another method (such as
SDMs), we would have had to create at least six more independent rules to handle this. Instead, TGGs gave us this direction for free!

\end{itemize}

