\newpage
\hypertarget{finalStep}{}
\section{Tree-to-text transformation}
\genHeader

We've finally reached the last step, transforming our resulting tree, \texttt{tree\-.xmi\-\_FWD\-.xmi\-\_BWD\-.xmi} back into a filesystem with
\texttt{.dictionary} files identical to our original input, the \texttt{myLibrary} filesystem. Note that in an actual application, we would do something useful
with the model before transforming it back to text. Or the dictionary might have been produced from a learning box, i.e., the textual syntax representation
wouldn't exist yet. One of the coolest things about ANTLR is that the same parsing technology that we used in Section 2 can be used to \emph{unparse} the tree.

Analogously to parsing text with a lexer and parser grammar to produce a tree, a tree is unparsed to text using a \emph{tree grammar} and \emph{templates}. A
tree grammar is similar to EBNF, consisting of rules (\texttt{main}, \texttt{entry}) that each match a tree fragment and evaluate a template, as
opposed to rules that match text fragments and build a tree. For further details concerning tree grammars, we refer to \cite{ANTLR} and the ANTLR
website \url{www.antlr.org}.

\begin{itemize}

\item[$\blacktriangleright$] Expand ``src/org.moflon.moca.dictionary.unparser'', open \texttt{Dict\-ion\-ary\-Tree\-Gram\-mar.g}, and edit the contents as
depicted in Fig.~\ref{eclipse:treeGrammar}. 

\vspace{0.5cm}

\begin{figure}[htpb]
\begin{center}
  \includegraphics[width=0.8\textwidth]{eclipse_dictionaryTreeGrammar}
  \caption{Tree grammar for the unparser}
  \label{eclipse:treeGrammar}
\end{center}
\end{figure}

\newpage


\item[$\blacktriangleright$] Next, open \texttt{Dict\-ion\-ary\-Un\-pars\-er\-Ad\-ap\-ter.java} (Fig~\ref{eclipse:unparserCommented}). You'll notice that this
file contains a (commented) \texttt{StringTemplateGroup} method for retrieving a group of templates and needs to be implemented. The comments explain how to use
either a folder containing different template files, or a single file containing all templates. The latter is better for numerous small templates, while the
former makes sense when the templates contain a lot of static text.

\vspace{0.5cm}

\begin{figure}[htpb]
\begin{center}
  \includegraphics[width=\textwidth]{eclipse_DictionaryUnparserAdapterUnimplemented}
  \caption{Two options of how to store templates}
  \label{eclipse:unparserCommented}
\end{center}
\end{figure}


\item[$\blacktriangleright$] For this small example, a single file with all templates is ideal. Uncomment line 44 (the option for a group file) and remove
the line throwing an \texttt{Un\-sup\-port\-ed\-Op\-er\-at\-ion\-Ex\-cep\-tion}.

\item[$\blacktriangleright$] Create a template file by navigating to the empty ``templates'' folder of your adapter project, and creating a new file
named \texttt{Dictionary.stg} (as demanded in \texttt{Dict\-ion\-ary\-Un\-pars\-er\-Ad\-ap\-ter.java}). Complete it as specified in
Fig.~\ref{eclipse:dictionaryTemplate}.

\vspace{0.5cm}

\begin{figure}[htpb]
\begin{center}
  \includegraphics[width=0.5\textwidth]{eclipse_dictionaryTemplate}
  \caption{The \texttt{dictionary} template}
  \label{eclipse:dictionaryTemplate}
\end{center}
\end{figure}

% \item[$\blacktriangleright$] Remember how you removed the unparsing command from \texttt{TGGMain.java} at the start of this part? Return to the file and
% uncomment line 46.  As you can see, it will create and place the results of this action into a single \texttt{out} folder in the instance directory. 

\item[$\blacktriangleright$] Save and run your transformation again. Inspect and compare your input and output files (Fig.~\ref{eclipse:unparseResult}). Are
they the same?

\vspace{0.5cm}

\begin{figure}[htpb]
\begin{center}
  \includegraphics[width=0.5\textwidth]{eclipse_finalInstancesHierarchy}
  \caption{The final input and output filesystems}
  \label{eclipse:unparseResult}
\end{center}
\end{figure}

\item[$\blacktriangleright$] If everything succeeded, your transformation is now complete in both directions! Feel free to play around with
changing some files such as a the unparser template, or the content of the original files. How are the changes propagated through the transformation?
How about implementing an SDM to refactor or extend the library model in some useful way before transforming it to text?

\end{itemize}