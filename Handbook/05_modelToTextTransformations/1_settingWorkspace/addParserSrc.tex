\newpage
\hypertarget{subSec:setupParser}{}
\subsection{Setting up the Parser}
\genHeader

Before moving on, it should be quickly mentioned that our conventions and workflow state that the \emph{code adapter} package that we established in the
previous step is a package that contains the tree-to-model transformation logic. This logic \emph{could} be integrated directly in the corresponding metamodel
(\texttt{Dic\-tion\-ary\-Language}), but a separation makes sense here as there could be \emph{different} code adapters for the \emph{same} language.

Anyway - The final thing our transformation needs in order to translate from a textual instance into a model is an ANTLR parser/unparser tool.

\begin{itemize}

\item[$\blacktriangleright$] Right-click on \texttt{DictionaryCodeAdapter} and navigate to ``eMolfon/ Add Parser/Unparser''
(Fig~\ref{eclipse:contextParser}).

\vspace{0.5cm}

\begin{figure}[htpb]
\begin{center}
  \includegraphics[width=0.8\textwidth]{eclipse_contextAddParserUnparser}
  \caption{figureCaption}
  \label{eclipse:contextParser}
\end{center}
\end{figure}


\item[$\blacktriangleright$] In the parser settings window, enter \texttt{dictionary} as the \texttt{File extension}, and confirm the \texttt{Create Parser},
\texttt{Create Unparser}, and \texttt{ANTLR} options are selected as the corresponding technology for each case (Fig~\ref{eclipse:wizardParser}). Affirm by
pressing \texttt{Finish}.

\begin{figure}[htpb]
\begin{center}
  \includegraphics[width=0.8\textwidth]{eclipse_wizardParser}
  \caption{figureCaption}
  \label{eclipse:wizardParser}
\end{center}
\end{figure}

\vspace{0.5cm}

\item[$\blacktriangleright$] If everything was completed without error, parser and unparser stubs should generate and appear in the \texttt{src}
package, where \texttt{ANTLR} automatically built the corresponding Java packages (Fig.~\ref{eclipse:generatedParser}). In addition, a new
\texttt{in} folder should appear under \texttt{instances} which will store any source text files. Your setup is now complete!

\begin{figure}[htpb]
\begin{center}
  \includegraphics[width=0.4\textwidth]{eclipse_generatedParser}
  \caption{figureCaption}
  \label{eclipse:generatedParser}
\end{center}
\end{figure}

\end{itemize}
