\genHeader

Welcome to Part V, an introduction to unidirectional transformations via Triple Graph Grammars (TGGs). If you're just joining our handbook in
this part, and haven't completed any of the previous parts, we recommend working through at least Part I for the required setup and installation instructions to
ensure eMoflon is working correctly, and strongly encourage finishing Part IV to master the basics of TGGs. That part will be a key reference if you're ever
unsure how to use a TGG feature here. Otherwise, we have assumed as little as possible from previous parts before that and include appropriate
references where necessary.

As we've mentioned, we introduced the basics of TGGs in Part IV and plan to expand that knowledge here. In parts before that, we created
\texttt{LeitnersLearningBox}, a memorization tool that stored cards (with keywords on the front, and definitions on the back) in different partitions, which
could then move throughout the box based on a set of rules. This metamodel was used in Part IV as a source language in a
\emph{bidirectional}\define{Bidirectional Transformation}  TGG transformation, where each card was translated into an entry (with a sole content attribute
storing all of a card's information) into a single dictionary container. In this part, instead of transforming elements from one instance model into another, we
want to perform a \emph{unidirectional}\define{Unidirectional Trans\-form\-at\-ion} transformation from a textual instance of a model into a
\texttt{Dictionary}. We'll use a popular \texttt{ANTLR} parser and unparser, then define a set of rules to handle the information using eMoflon's standard
\texttt{MocaTree} language.

When establishing a model-driven solution, \emph{model transformations} usually play a central and important role. They could be used for specifying dynamic
semantics (as done with our learning box) or, more generally, for transforming a certain model to another model to achieve some goal (i.e., consistency, adding
or abstracting from platform details, \ldots).

There are many \emph{types} of model transformations and \cite{CH03,Mens_Gorp_2006} give a nice and detailed classification along a set of different dimensions.
In this part, we shall explore some of these dimensions and learn how \emph{model-to-text} transformations can be achieved with a nice mixture of \emph{string
grammars} and \emph{graph grammars}.

For the rest of this part, a model transformation is to be denoted as:
\begin{displaymath}
 	\Delta: m_{src} \rightarrow m_{trg}
\end{displaymath}
where the source model $m_{src}$ is to be transformed to the target model $m_{trg}$. Let's review the four primary ways in which $\Delta$ can be classified.

$\Delta$ is \emph{endogenous}\define{Endogenous}, if $m_{src}$ and $m_{trg}$ conform to the same metamodel. Each story driven model (SDM) built in Part III for
\texttt{LeitnersLearningBox} are examples of \emph{endogenous} transformations.

$\Delta$ is \emph{exogenous}\define{Exogenous}, if $m_{src}$ and $m_{trg}$ are instances of different metamodels. For example: A dictionary is used to learn new
words (similar to a learning box), but is more suitable for use as a reference (i.e., one already knows the words, but may occasionally need a specific
definition). In contrast, a learning box is geared towards the actual memorization process. Therefore, one could start with a learning box and, once all the
words have been memorized, transform it into a personalised dictionary for future reference. If too many words become forgotten, the dictionary should be
transformed back to a learning box. The learning box to dictionary transformation and vice-versa are therefore examples of \emph{exogenous}
transformations, and we created this by complementing our \texttt{LeitnersLearningBox} with a simple language for \texttt{Dictionary} in Part IV.

$\Delta$ operates \emph{in-place}\define{In-place} if $m_{src}$ is \emph{destructively} transformed to $m_{trg}$. The SDMs for our learning box (e.g.
\texttt{grow} or \texttt{check}) are examples of \emph{in-place} transformations as they perform destructive changes directly to a source model, thus
transforming it into the target model.

Finally, $\Delta$ is \emph{out-place}\define{Out-place} if $m_{src}$ is left intact and is unchanged by the transformation which creates $m_{trg}$. A file
instance to dictionary transformation and vice-versa (which we will implement in this part) are examples of \emph{out-place} transformations.

Although \emph{endogenous} + \emph{in-place} is the natural case for SDMs (as was the case for our learning box), \emph{exogenous} and/or \emph{out-place}
transformations can still be specified. Using the basics introduced in Part IV, we will implement this latter type with TGGs.
 
\newpage
 
To twist your brain a bit, here are a few interesting statements:
\begin{enumerate}

\item[$\blacktriangleright$] \emph{Out-place} transformations can be \emph{endogenous} or \emph{exogenous}.

\item[$\blacktriangleright$] \emph{In-place} transformations can usually only be \emph{endogenous}. \emph{Exo\-gen\-ous} transformations are consequently,
always \emph{out-place}.  Why?

\end{enumerate}  

It should be noted that $\Delta$ can be further classified as \emph{horizontal} if $m_{src}$ and $m_{trg}$ are on the same \emph{abstraction level}, or
\emph{vertical} if they are not. Unfortunately, this \emph{abstraction level} dimension is a bit `fuzzy,' but we will explore and work on these different levels
by establishing a textual concrete syntax for \texttt{Dictionary}. We shall learn how graph transformations can be used in combination with parser generators
and template languages to implement model-to-text and text-to-model transformations that are typically \emph{vertical} (text is normally on a lower abstraction
level than a model). On the other hand, the overall learning box to dictionary transformation completed in the previous part is \emph{horizontal} as the models
represent the \emph{same} information differently, and can thus be considered to be on the same abstraction level.

In the following, the \emph{\bf Mo}flon \emph{\bf C}ode \emph{\bf A}dapter (\emph{Moca}) framework refers to:
\begin{enumerate}

 \item The approach we use to integrate string grammars, graph grammars and template languages, 

 \item how we separate a transformation into different modular steps, 

 \item the use of a generic and simple tree to consolidate different platforms, and 

 \item the actual tool support that acts as glue to hold all the different parts together.

\end{enumerate}

% Fig.~\ref{fig:moca-overview} gives a ``big picture'' of what we plan to achieve in this part of the eMoflon handbook. All explanations are integrated right with
% the figure, so take your time to read it carefully and let it sink in. We'll be zooming in the bits and pieces of each step in the following sections to clear
% up any confusion.

% \begin{figure}[htp]
% \begin{center}
%  \includegraphics[angle=90, height=\textheight]{pics/moca/text-to-model}
%   \caption{Overview of model-to-text with the MOCA framework}
%   \label{fig:moca-overview}
% \end{center}
% \end{figure} 
