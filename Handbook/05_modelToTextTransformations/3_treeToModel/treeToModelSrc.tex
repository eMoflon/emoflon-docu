\newpage
\section{Tree-to-Model transformation with TGGs}
\genHeader

Our goal in this section is to break down the broad \texttt{MocaTree} to \texttt{Dictionary} transformation into smaller, modular steps. More specifically, we
want separate rules for transforming a \texttt{Folder} into its appropriate container element (i.e., \texttt{Library} or \texttt{Shelf}), then individual rules
to handle whatever \texttt{File} and \texttt{Node} elements they contain (such as transforming each of the \texttt{.dictionary} files into \texttt{dictionary},
\texttt{author}, and \texttt{entry} elements).

\vspace{0.5cm}

\begin{figure}[htp]
\begin{center}
  \includegraphics[width=0.7\textwidth]{goal}
  \caption{Transforming a \texttt{myLibrary} tree into an instance model}
  \label{fig:treeToDictionary}
\end{center}
\end{figure}

\vspace{0.5cm}

When creating \emph{any} rule, it's important to keep in mind any flexibility the rule may require. In future \texttt{tree} instances for example, we
may not know how many exactly subfolders \texttt{myLibrary} will contain, whether or not each \texttt{.dictionary} file will have an author node (as seen in
\texttt{unknown.dictionary}), or how many entries each \texttt{dictionary} will contain. Just like SDMs, it's key to avoid situation-specific rules and
patterns.

\jumpDual{treeToModel vis}{treeToModel tex}

\newpage
\hypertarget{treeToModel vis}{}
\subsection{The visual transformation rules}
\visHeader

Some sort of content to keep from jumping immediately into an item. We need to break it down, we want to separate multiple instances. Remember -- they follow a
precondition/postcondition format\ldots While it may seem like there are a lot of rules, you'll notice that they are all neccessary and make the transformation
both easy to execute and understand.

\begin{itemize}

\subsubsection{FolderToLibraryRule} % ---------------------------------

\item[$\blacktriangleright$] Expand the \texttt{<<Rules Package>>} node in EA and open the \texttt{Rules} diagram. Create a new rule named
\texttt{FolderToLibraryRule}, double-clicking the new element to open its diagram. Complete the rule as depicted in Fig.~\ref{ea:FolderIntoLibrary_Complete}.
Remember -- we established that first correspondence type when creating the TGG schema in Section 1.

\vspace{0.5cm}

\begin{figure}[htbp]
\begin{center}
  \includegraphics[width=\textwidth]{ea_FolderToLibraryRule}
  \caption{completed folder into library}
  \label{ea:FolderIntoLibrary_Complete}
\end{center}
\end{figure}

\item[$\blacktriangleright$] We're able to use this entire rule as context for the next rule in order to handle the creation of shelves. Select
\texttt{inputFolder}, \texttt{in\-put\-Fol\-der\-To\-Lib\-rary,} and \texttt{library}, then use the eMoflon control panel to \texttt{derive} a new rule. Name
this \texttt{ForAllShelfRule}. % Integrating green with black..

\subsubsection{ForAllShelfRule} % ---------------------------------

\item[$\blacktriangleright$] This will open a new diagram with three black objects, representing the context. This rule is remarkably similar to
\texttt{FolderToLibraryRule}, except it will need two green links connecting the new elements to their respective container. Complete \texttt{ForAllShelfRule}
as depicted in Fig.~\ref{ea:ForAllShelves_Complete}; You'll need to create a new correspondence type in either the schema (as we did
in the beginning) or on-the-fly by selecting \texttt{Create new Link} in the quick-link dialogue.\footnote{see Part, Section \update}

\clearpage

\begin{figure}[htbp]
\begin{center}
  \includegraphics[width=0.8\textwidth]{ea_ForAllShelfRule}
  \caption{completed ForAllShelves}
  \label{ea:ForAllShelves_Complete}
\end{center}
\end{figure}

\subsubsection{NodeToDictionaryRule} % ---------------------------------

\item[$\blacktriangleright$] Now we can handle the dictionary \texttt{File} elements. Analogously to how you began the previous rule, select
\texttt{shelfFolder}, \texttt{FolderToShelf}, and \texttt{shelf}, and derive \texttt{NodeToDictionaryRule}.

\item[$\blacktriangleright$] Build it as shown in Fig.~\ref{ea:NodeToDictionary_Complete}. As you can see, this rule creates a consistency between
\texttt{dictionaryNode} and the \texttt{dictionary} instance, and only handles the first \texttt{titleNode} in the tree structure. Nearly every element is
involved in order to correctly set the \texttt{dictionary} and \texttt{dictionaryFile} names in two different constraints!

\begin{figure}[htbp]
\begin{center}
  \includegraphics[width=\textwidth]{ea_NodeToDictionaryRule}
  \caption{completed NodeToDictionary}
  \label{ea:NodeToDictionary_Complete}
\end{center}
\end{figure}


\item[$\blacktriangleright$] Please note that the \texttt{index} \emph{attribute constraint} is required in order to ensure that the node with the title
information is correctly matched. We could have also included a \texttt{node} to handle the author, a third, fourth, or even tenth \texttt{node} connected to
\texttt{DictionaryNode}, but that would mean the pattern absolutely has to match to an author and ten elements, which may not always exist. Instead, we'll
create separate rules for each of these types which can be called as many times as necessary.

\subsubsection{ForAllEntryRule} % ---------------------------------

\item[$\blacktriangleright$] Let's handle the \texttt{entry} elements first. Create and complete \texttt{For\-All\-Ent\-ry\-Rule} and depicted in
Fig.~\ref{ea:ForAllEntry_Complete}. We needed to match both a \texttt{contentNode} and \texttt{indexNode} to each \texttt{entryNode}, bound by their
\texttt{index} values in order to ensure the correct EString attributes were set to an \texttt{entry}'s \texttt{content} and \texttt{level} values.

\vspace{0.5cm}

\begin{figure}[htbp]
\begin{center}
  \includegraphics[width=\textwidth]{ea_ForAllEntryRule}
  \caption{completed ForAllEntry}
  \label{ea:ForAllEntry_Complete}
\end{center}
\end{figure}

\clearpage

\item[$\blacktriangleright$] Return to \texttt{NodeToDictionaryRule}. We need to think about what context elements we'll need for our next rule to handle
authors. Not only will we need \texttt{DictionaryNode} and \texttt{dictionary} as we did in \texttt{ForAllEntry}, we'll also need \texttt{shelf} and
\texttt{library} in order to satisfy the \texttt{Dictionary} metamodel, where each \texttt{author} is linked to both individual \texttt{dictionary} and
\texttt{library} elements. Derive and create \texttt{AuthorRule} as depicted below Fig.~\ref{ea:AuthorRule}

\subsubsection{AuthorRule} % ---------------------------------

\begin{figure}[htbp]
\begin{center}
  \includegraphics[width=\textwidth]{ea_AuthorRule}
  \caption{completed AuthorRule}
  \label{ea:AuthorRule}
\end{center}
\end{figure}

\end{itemize}

There is another part of handling authors when transforming from our tree to our instance model. Right now, we have specified that
for every \texttt{authorNode} the rule finds, create an \texttt{author} instance. This would be fine if we had unique authors for \emph{every} dictionary
\texttt{File}, but look at both of the french \texttt{numbers} files -- they both have the same contact information. This means that our \texttt{Dictionary}
will have two identical author instances for one Library. 

Some users may be okay with this, and not care about multiples, so long as all the correct information is there, while others would prefer a minimalist
structure. How can we refine this rule so that the user can quickly implement one or othe other?

eMolfon's visual syntax has a cool \emph{refinement} feature which enables you to adjust specific elements in a rule. This is exactly what we're looking for --
our rules to handle either creating or checking for an existing author are nearly \emph{identical} to \texttt{AuthorRule}, save for the binding (and therefore
reference links) on \texttt{authorNode}.

\begin{itemize}

\item[$\blacktriangleright$] Return to the \texttt{Rules} diagram. Since we're no longer looking to implement \texttt{AuthorRule} directly, we need to update
its definition to an \texttt{abstract} class. Select the rule, then hit \texttt{alt + enter} top open its properties dialogue.

\item[$\blacktriangleright$] Switch to the \texttt{details} tab, and select \texttt{Abstract} from the list of class types (Fig.~\ref{ea:abstractDetails}). 
Affirm and close by pressing \texttt{Ok}.

\begin{figure}[htbp]
\begin{center}
  \includegraphics[width=\textwidth]{ea_abstractDetails}
  \caption{completed AuthorRule}
  \label{ea:abstractDetails}
\end{center}
\end{figure}

\end{itemize}

Now to develop our two rules. The key idea when building refinements is to imagine the rules are being placed directly over the
pattern they inherit from, similar to a transparency sheet. Theses rules will execute \texttt{AuthorRule} exactly, except for whatever modficiations you make.

Let's create the rule to handle an already existing author first. Inspecting \texttt{AuthorRule}, we still want the rule to match a new \texttt{authorNode}
and create a link between \texttt{author} and \texttt{dictionary}, but the link between \texttt{author} variable and the link connecting it to \texttt{library}
should already exist, i.e., be 'black.'

\begin{itemize}

\subsubsection{ExistingAuthorRule} % ---------------------------------

\item[$\blacktriangleright$] In \texttt{AuthorRule}'s diagram, select \texttt{authorNode} and \texttt{library}, and under the ``eMoflon TGG Functions'' tab in
the eMoflon control panel, press \texttt{Derive}. 

\item[$\blacktriangleright$] Enter \texttt{ExistingAuthorRule} as the rule's name but, given that we're just wanting to replace the selected elements, and not
use them as context elements, be sure to select the exact copy option (Fig.~\ref{ea:deriveRefinement}).

\begin{figure}[htbp]
\begin{center}
  \includegraphics[width=0.6\textwidth]{ea_deriveRefinement}
  \caption{completed AuthorRule}
  \label{ea:deriveRefinement}
\end{center}
\end{figure}

\item[$\blacktriangleright$] The rule diagram will open in the editor, with the elements in the same place you copied them from. You can move the model around
if the white space is bothering you, but the ``transparency'' will  be much easier to visualise if you leave it as-is.

\item[$\blacktriangleright$] Change the 'green' binding operator on \texttt{author} and the link to \texttt{Check Only}. That's everything!

\subsubsection{NewAuthorRule} % ---------------------------------

\item[$\blacktriangleright$] Return to \texttt{AuthorRule}. For reasons that will be explained shortly in the next section, derive a copy of \texttt{authorNode}
and \texttt{library}. Don't modify either variable -- since we have now defined \texttt{AuthorRule} as an abstract class, it needs a concrete rule to execute
it.

\item[$\blacktriangleright$] Open the \texttt{Rule} diagram one last time. We need the new rules to inherit from the \texttt{AuthorRule}. Quick-link
each to the root rule, choosing \texttt{Create Refinement Link} from the context menu. Your diagram should now resemble Fig.~\ref{ea:refinementClasses}.

\begin{figure}[htbp]
\begin{center}
  \includegraphics[width=0.6\textwidth]{ea_refinementLinks}
  \caption{Finished \texttt{Rules} diagram}
  \label{ea:refinementClasses}
\end{center}
\end{figure}


\item[$\blacktriangleright$] You're nearly done! Make sure everything is saved, and validate your TGG. If a dialogue appears saying the attempt was
unsuccessful, you may simply need to update the schema diagram which may not contain the new correspondence types you created on the fly. To do so, open
\texttt{dictionaryCodeAdapter}, right click anywhere in the diagram and add any missing elements by navigating to ``Insert Existing Element''
(Fig.~\ref{ea:insertContext}), and selecting the missing correspondence types from the root package's tree (Fig.~\ref{ea:insertTree}).

\begin{figure}[htbp]
   \centering
      \subfloat[Right-click to open the context menu]{
        \includegraphics[width=0.7\textwidth]{ea_InsertExistingElements}
        \label{ea:insertContext}
      }
      \\
      \subfloat[Check your \texttt{TGG}, \texttt{DictionaryLanguage}, and \texttt{MocaTree} packages]{
        \includegraphics[width=0.7\textwidth]{ea_insertElementTree}
        \label{ea:insertTree}
      }
      % \caption{Updating your schema with missing elements}
\end{figure}

\item[$\blacktriangleright$] Your schema diagram should resemble Fig.~\ref{ea:Schema_Complete} upon exit. Validate the project once again, then switch to the
Eclipse workspace and refresh the package explorer to generate new code. Great work!

\newpage

\vspace*{3cm}

\begin{figure}[htbp]
\begin{center}
  \includegraphics[width=\textwidth]{ea_finalSchema}
  \caption{Completed TGG schema diagrams}
  \label{ea:Schema_Complete}
\end{center}
\end{figure}

\jumpSingle{t2m close}

\end{itemize}


\newpage
\hypertarget{treeToModel tex}{}
\subsection{Tree To model text source}
\texHeader

Please note that we have presented the instructions for the visual specification's TGG transformation rules in the same headings construct as this section for
easy comparison. We suggest viewing your rule's visual equivalent in \hyperlink{treeToModel tex}{Section 3.1} once finished.

\begin{itemize}

\subsubsection{FolderToLibraryRule} % ---------------------------------

\item[$\blacktriangleright$] Expand \texttt{DictionaryCodeAdapter}, right-click on the \texttt{Rules} folder. Navigate to ``New/TGG Rule," and name this
initializing rule \texttt{Folder\-To\-Lib\-rary\-Rule}.

\item[$\blacktriangleright$] All this simple rule needs to do is match to a \texttt{Folder} instance (i.e., ``myLibrary'') and create its equivalent
\texttt{Library} instance, create a correspondence link between them to ensure they remain equivalent, and use a single constraint equating the \texttt{name} of
the folder to the \texttt{name} of the library. Build your rule until it resembles Fig.~\ref{eclipse:FolderToLibraryRule}.

\vspace{0.5cm}

\begin{figure}[htbp]
\begin{center}
  \includegraphics[width=\textwidth]{eclipse_FolderToLibraryRule}
  \caption{The TGG transformation will begin with this rule.}
  \label{eclipse:FolderToLibraryRule}
\end{center}
\end{figure}

\vspace{-0.5cm}

\subsubsection{ForAllShelfRule} % ---------------------------------

\item[$\blacktriangleright$] Let's use some elements from the previous rule to help us define how to handle creating shelves for our library. Copy and paste the
elements from each correspondence to a new \texttt{FolderToLibraryRule}, adding a \texttt{shelfFolder} and \texttt{shelf} and described in
Fig.~\ref{eclipse:ForAllShelvesRule}.

\vspace{0.5cm}

\begin{figure}[htbp]
\begin{center}
  \includegraphics[width=0.9\textwidth]{eclipse_ForAllShelfRule}
  \caption{Completed \texttt{ForAllShelfRule}}
  \label{eclipse:ForAllShelvesRule}
\end{center}
\end{figure}

\item[$\blacktriangleright$] Don't forget that in order to successfully create the correspondence link between these elements, you need to add the
correspondence type to your \texttt{schema} (Fig.~\ref{eclipse:updatedSchema}).

\vspace{0.5cm}

\begin{figure}[h!]
\begin{center}
  \includegraphics[width=0.7\textwidth]{eclipse_updatedSchema}
  \caption{Update the TGG \texttt{schema} with any new correspondence types}
  \label{eclipse:updatedSchema}
\end{center}
\end{figure}

\newpage

Now that we have the containers, we can handle the dictionary \texttt{File} elements. We know from our generated tree model that each dictionary will always
have a title node, but we're unsure if an author will be included, and we won't know how many entries are involved. Therefore, we should create at least three
different rules to handle this stage of the transformation. 

\subsubsection{NodeToDictionaryRule} % ---------------------------------

\item[$\blacktriangleright$] Create a rule name \texttt{NodeToDictionaryRule}, and build it as indicated in Fig.~\ref{eclipse:NodeToDictionaryRule}.

\vspace{0.5cm}

\begin{figure}[htbp]
\begin{center}
  \includegraphics[width=\textwidth]{eclipse_NodeToDictionaryRule}
  \caption{\texttt{NodeToDictionaryRule} handling only \texttt{titleNode}s}
  \label{eclipse:NodeToDictionaryRule}
\end{center}
\end{figure}

\newpage

\item[$\blacktriangleright$] As you can see, this rule demands that a \texttt{shelfFolder} and \texttt{shelf} already exist before executing, implying that this
rule can only be called after executing \texttt{ForAllShelfRule}. An attribute constraint is used with \texttt{titleNode} to ensure that the correct child
\texttt{Node} is matched from \texttt{dictionaryNode}, and not accidentally to an author or entry node, which will have different indices.

\item[$\blacktriangleright$] This rule also imposes two constraints, one for each direction. In the forward transformation, we want to add the name of
the shelf to the dictionary file's title (i.e., ``english'' and ``numbers1-10'') and set it as the target's \texttt{name}. Though this may be somewhat abstract
at first, imagine this constraint is actually called ``add/remove Suffix'' since, during the inverse transformation, this constraint also will ensure the name
of the shelf is not present. Similarly, in the backwards transformation, the transformation needs to generate a valid \texttt{dictionaryFile} name, and it does
so by appending \texttt{.dictionary} to the end of \texttt{titleNode.name}.

\subsubsection{ForAllEntryRule} % ---------------------------------

\item[$\blacktriangleright$] Let's handle the \texttt{Entry} elements next. We can demand \texttt{Node\-To\-Dict\-ion\-ary\-Rule} be executed first by needing
context information from \texttt{dictionaryNode}. Create \texttt{ForAllEntryRule}, and build it so it matches
Fig.~\ref{eclipse:ForAllEntryRule}. Each \texttt{entryNode} will have a child \texttt{indexNode} and \texttt{contentNode} whose attribute constraints, and
transformation constraint ensure the correct information is set to its equivalent \texttt{entry} instance.

\begin{figure}[htbp]
\begin{center}
  \includegraphics[width=\textwidth]{eclipse_ForAllEntryRule}
  \caption{Completed \texttt{ForAllEntryRule}}
  \label{eclipse:ForAllEntryRule}
\end{center}
\end{figure}

\newpage


The last thing we need to specify is how to handle \texttt{author}s. Transforming forwards from a \texttt{authorNode} to an \texttt{author} isn't as simple as
an \texttt{entryNode}, where you create an \texttt{entry} every time you find a valid match. Instead, we have to account for the possibility of a single author
for multiple dictionaries in a \texttt{Library}. While some users may not care about duplicate information, why not also provide a rule for the case when the
library needs to be optimized with only unique authors in a single \texttt{Library} instance?

\subsubsection{ForAllNewAuthorRule} % ---------------------------------

\item[$\blacktriangleright$] Create \texttt{ForAllNewAuthorRule}, and complete it as depicted in Fig.~\ref{eclipse:ForAllNewAuthorRule}. This is our one-to-one
correspondence rule, where every \texttt{authorNode} instance the rule is able to find, an equivalent \texttt{author} is created.

\begin{figure}[htbp]
\begin{center}
  \includegraphics[width=\textwidth]{eclipse_ForAllNewAuthorRule}
  \caption{Creating new authors in \texttt{ForAllNewAuthorRule}}
  \label{eclipse:ForAllNewAuthorRule}
\end{center}
\end{figure}

\subsubsection{ExistingAuthorRule} % ---------------------------------

\item[$\blacktriangleright$] Similarly, create \texttt{ForExistingAuthorRule} as specified in Fig.~\ref{eclipse:ForExistingAuthorRule}. You should be able to
copy and paste the majority of the previous rule. In fact, the only thing you need to change are two small characters in front of \texttt{author} and its
\texttt{dictionary} reference, forcing the rule to assume and match to some context \texttt{author}, but still create a new link between the current dictionary
file and previous author.

\begin{figure}[htbp]
\begin{center}
  \includegraphics[width=\textwidth]{eclipse_ForExistingAuthorRule}
  \caption{Checking for existing authors in \texttt{ForExistingAuthorRule}}
  \label{eclipse:ForExistingAuthorRule}
\end{center}
\end{figure}

\newpage

\item[$\blacktriangleright$] Great work! You have now specified six different rules (with a focus on the forward direction) to perform a text-to-model
transformation! For confirmation, your final schema and package explorer should now resemble Fig.~\ref{eclipse:schemaFinal}.

\vspace{0.5cm}

\begin{figure}[htbp]
\begin{center}
  \includegraphics[width=\textwidth]{eclipse_finalSchema}
  \caption{Your final \texttt{rules} project structure and \texttt{schema}}
  \label{eclipse:schemaFinal}
\end{center}
\end{figure}

\vspace{0.5cm}

\item[$\blacktriangleright$] Given that everything has been done correctly, and MOSL hasn't reported any errors, build your TGG transformation. If
problems arise, be sure to double-check your files for spelling or character symbol mistakes.

\end{itemize}


\newpage
\hypertarget{t2m close}{}
\subsection{Additional author handling}
\genHeader

\begin{itemize}

\item[$\blacktriangleright$] If your project's build succeeded, run the transformation file again and examine the successful forward output, 
\texttt{fwd.trg.xmi}, a little closer (Fig.~\ref{eclipse:generatedFwdTrsfm}).

\vspace{0.5cm}

\begin{figure}[htbp]
\begin{center}
  \includegraphics[width=0.7\textwidth]{eclipse_generatedForwardTransformation}
  \caption{\texttt{Dictionary} result of the forward transformation}
  \label{eclipse:generatedFwdTrsfm}
\end{center}
\end{figure}

\vspace{0.5cm}

\item[$\blacktriangleright$] Your output may or may not resemble ours. In fact, there's a 50/50 chance that some of the \texttt{author} nodes are ever created!
Let's run the integrator on \texttt{fwd.corr.xmi} to find out why.\footnote{If you haven't already, read Part IV, Section 6 for details on how the
integrator can help you visualise and debug a transformation}

\vspace{0.5cm}

\item[$\blacktriangleright$] Proceed through the transformation until you reach the first match to an author node (Fig.~\ref{eclipse:fwdIntegrator}). Keep an
eye on the small window below -- it states that there are two possible rules that apply to the match!

\begin{figure}[htbp]
\begin{center}
  \includegraphics[width=0.8\textwidth]{eclipse_integratorAuthorChoice}
  \caption{The transformation is able to find two possible rules}
  \label{eclipse:fwdIntegrator}
\end{center}
\end{figure}

% \clearpage
\vspace{0.5cm}

\item[$\blacktriangleright$] At run-time, the transformation is presented with a choice between two rules to apply to the matched \texttt{authorNode}. The
resulting choice is entirely random, meaning that your output is likely to be different each time you run the transformation. For a deterministic
transformation, you therefore need to force a preferred decision. There are two ways to do this: (1) at run-time, where users will be able to decide for
themselves what they would prefer to use, or (2) at design-time, where you make the decision a part of the TGG.

\vfill

\end{itemize}

\subsubsection{Option 1: Run-time decision}

The advantage with this option is that you give users the choice of what they prefer. Some users don't mind having multiple authors, while others might prefer a
minimalist design. They can easily change their preference possibly on a case-by-case basis, by implementing a TGG rule configurator.

\begin{itemize}

\item[$\blacktriangleright$] Navigate to ``DictionaryCodeAdapter/src/org.moflon.tie.'' Right-click this package and create a new java class named
\texttt{Author\-Config\-ur\-at\-or}, then complete the override as described in Fig.~\ref{eclipse:authorConfig}. Be careful not to make any mistakes --
Eclipse's auto-completion can help you here.

\begin{figure}[htbp]
  \includegraphics[width=1.1\textwidth]{eclipse_authorConfiguratorClass}
  \caption{Setting a preference for \texttt{NewAuthorRule}}
  \label{eclipse:authorConfig}
\end{figure}

\item[$\blacktriangleright$] As you can see on Line 21, this simple configurator prefers to create a new author every time there is a choice. Users can
edit this value to ``ExistingAuthorRule'' to avoid redundant authors.

\item[$\blacktriangleright$] You still need to set the configurator to be used for the transformation. This decision will only arise in the forward direction,
so declare it once in \texttt{Dict\-ion\-ar\-y\-Code\-Ad\-ap\-ter\-Traf\-o/\-per\-form\-For\-ward} as depicted in Fig.~\ref{eclipse:editTGGMain} (on Line 59).

\newpage

\vspace*{0.5cm}

\begin{figure}[htbp]
\begin{center}
  \includegraphics[width=0.8\textwidth]{eclipse_authorConfiguratorTGGMain}
  \caption{Setting the configurator to control the run-time decision}
  \label{eclipse:editTGGMain}
\end{center}
\end{figure}

\item[$\blacktriangleright$] Save and run the transformation a few times, using the integrator to confirm your preference is enforced each time.

\end{itemize}

\subsubsection{Option 2: Design-time decision}

It is also possible to set a preference as part of the actual design of the transformation -- users will not be able to modify this. In our example, this
preference can be enforced using a NAC which checks to see if there is already an author with the same email in the library.

\begin{itemize}

\item[$\blacktriangleright$] Open and update either \texttt{NewAuthorRule} (visual) as shown in Fig.~\ref{ea:existingAuthorNAC} or edit the target domain in 
Eclipse (textual) as depicted in Fig.~\ref{eclipse:existingAuthorNAC}.

\newpage

\vspace*{0.5cm}

\begin{figure}[htbp]
\begin{center}
  \includegraphics[width=\textwidth]{ea_NewAuthorRuleNAC}
  \caption{Adjust \texttt{NewAuthorRule} by adding a NAC}
  \label{ea:existingAuthorNAC}
\end{center}
\end{figure}

\vspace{0.5cm}

\begin{figure}[htbp]
\begin{center}
  \includegraphics[width=0.7\textwidth]{eclipse_ForAllNewAuthorNAC}
  \caption{Add a NAC to \texttt{NewAuthorRule}}
  \label{eclipse:existingAuthorNAC}
\end{center}
\end{figure}

\vspace{0.5cm}

\item[$\blacktriangleright$] Save and rebuild the TGG, then run the transformation a few times. Confirm your preference is enforced each time -- With this NAC,
the configurator won't be given the chance to decide anymore!

\end{itemize}

