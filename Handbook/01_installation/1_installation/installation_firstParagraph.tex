{\small \texttt Approximate time: Just a few minutes \ldots}

This part provides a very simple example and a JUnit test to check the installation and configuration of eMoflon.

After working through this part, you should have an installed and tested eMoflon working for a trivial example.
We also explain the general workflow, the different workspaces involved and general useage of each syntax.

This part can be considered \emph{mandatory} if you are new to eMoflon, but we reccommend working through it anyway.

Don't forget the way we've presented this handbook! Here we introduce the red, blue, and black headers to separate the visual, textual, and common instructions. At the bottom of each page, you'll find a link that looks like \mbox{$\triangleright$ { \textt \emph{label} } }. This is the link that will take you to your next appropriate step. If there's no link, just proceed to the next page. You are still welcome to go through the entire handbook page by page, but be warned! If what you're doing isn't matching the instructions, you may be reading the wrong information. 

If, however, you're finding that the screenshots we've taken aren't matching your screen and you ARE in the right place, please send us an email at \href{mailto:contact@moflon.org}{contact@moflon.org} and let us know. They get outdated so fast! They just grow up, move on, start doing their own thing and \ldots uh, wait a second. We're talking about pictures here.

Feel free to also contact us if you have any questions, concerns, or suggestions on ways we can improve.

\pagebreak