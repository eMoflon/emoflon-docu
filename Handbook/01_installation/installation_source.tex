
{\bf \huge Part I:}
\vspace{0.7cm}

{\bf \Huge Installation \& Setup }
\vspace{0.5cm}
\genHeader

We start with a common step. Unfortunately, it's only a small step, so we do apologize for the white space. This is done throughout the entire tutorial to make the distinction between the two eMoflon syntaxes clearer than crystal. In the future, keep an eye out small notes that tell you if and where you can skip ahead if you don't feel like flipping through white pages.

\section{Install our plugin for Eclipse}
 
\begin{itemize}
\item[$\blacktriangleright$] Download and install Eclipse for Modelling ``Eclipse Modeling Tools (includes incubating components)''\footnote{Please note that you \emph{have to} install \emph{Eclipse Modeling Tools}, from the Juno download packages, or nothing will work.  Do not choose a different Eclipse package!  Although different versions and {\bf constellations} might work, eMoflon is currently tested for Eclipse Juno and Java 1.7.} from \url{http://www.eclipse.org/downloads/packages/} (Fig.~\ref{fig_downloadModelingPackage}).

\begin{figure}[htbp]
	\centering
  	\includegraphics[width=0.86\textwidth]{eclipse_modelingpackage}
	\caption{Download Eclipse Modeling Tools.}
	\label{fig_downloadModelingPackage}
\end{figure}

\item[$\blacktriangleright$] Install our Eclipse Plugin from the following update site\footnote{For a detailed tutorial on how to install Eclipse and Eclipse Plugins please refer to \url{http://www.vogella.de/articles/Eclipse/article.html}} 
\footnote{Please note: Calculating requirements and dependencies when installing the plugin might take quite a while depending on your internet connection.}:
\url{http://www.moflon.org/fileadmin/download/moflon-ide/eclipse-plugin/update-site2}

\end{itemize}


% -----------------------------------------------------------------------------------------------------------------------------------
% -------- VISUAL -------------------------------------------------------------------------------------------------------------------
\newpage

\section{Install our extension for Enterprise Architect}

% No New page; first instance
\visHeader

Our EA extension provides rudimentary support for validating the static semantics (Ecore) and dynamic semantics (SDM) of metamodels.
Validation results are displayed and, in some cases, even ``quick fixes'' to automatically solve the problems are offered.
Our validation framework is still work in progress so if you find an error that is not/wrongly detected feel free to send us a brief description.

\begin{enumerate}
\item[$\blacktriangleright$] To make the validation output window visible in EA, choose ``Extensions/\-Add-In Windows''.
This should display a new output window as depicted in Fig.~\ref{fig:validation_output}. This control panel also contains shortcuts for all the functionality available via the extensions file menu and many users actually prefer this.

\begin{figure}[htbp]
	\centering
  \includegraphics[width=0.6\textwidth]{pics/memBoxBilder/memBox40}
	\caption{Activating the validation output window}
	\label{fig:validation_output}
\end{figure}
\FloatBarrier

\item[$\blacktriangleright$] To start the validation choose ``Extensions/\-MOFLON::Ecore Addin/\-Validate all'' in EA (Fig.~\ref{fig:validation_menu}).
If you haven't already made any mistakes while modelling your \texttt{LearningBoxLanguage} in the last chapter, the output should resemble Fig.~\ref{fig:validation_output}, indicating that your metamodels are free of errors (at least according to our validation).

\begin{figure}[htbp]
	\centering
  \includegraphics[width=0.6\textwidth]{pics/memBoxBilder/memBox41}
	\caption{Starting the validation}
	\label{fig:validation_menu}
\end{figure}
\FloatBarrier
\end{enumerate}


To get familiar with our validation and quick fix features, let's add two small modelling errors in \texttt{LearningBoxLanguage}.

\begin{enumerate}
\item[$\blacktriangleright$] Create a new class in the \texttt{Learning\-Box\-Language} diagram.
You can retain the default name \texttt{EClass1}.
Let's assume, you wish to delete this class from your metamodel.
\item[$\blacktriangleright$] Select it (\texttt{EClass1}) in the diagram and press the \texttt{Delete} button.
Note that \texttt{EClass1} still exists in the project browser (and thus in your metamodel) and that you have one new \texttt{Information} message in the validation output (Fig.~\ref{fig:validation_information}).

\begin{figure}[htbp]
	\centering
  \includegraphics[width=0.7\textwidth]{pics/memBoxBilder/memBox43}
	\caption{Output indicating that an element is present in the metamodel, which is not in any diagram and should probably be deleted completely}
	\label{fig:validation_information}
\end{figure}

The message informs you that \texttt{EClass1} is not on any diagram, and as it is still in the model, that this could be a mistake.
Just pressing the \texttt{Delete} button is apparently not the proper way of \emph{deleting} \texttt{EClass1} from the metamodel and only removes it from the current diagram.
Deleting elements properly and other EA specific aspects are discussed in detail in Chapter~\ref{chap:Tips and Tricks}.


\item[$\blacktriangleright$] To navigate to the problematic element click once on the information message in the output window.
EA should navigate automatically to \texttt{EClass1} and highlight it in the project browser.
\item[$\blacktriangleright$] To check if there are any quick fixes available, double click the information message to invoke the quick fix dialogue.
In this case, there is one quick fix which suggests simply deleting the element from the model (Fig.~\ref{fig:quick-fix1}) as this is probably what was intended.
\item[$\blacktriangleright$] Click \texttt{OK} and \texttt{EClass1} will be deleted correctly from your model.
Your metamodel should now be error-free again as indicated by the validation output window.

\begin{figure}[htbp]
	\centering
  \includegraphics[width=0.55\textwidth]{pics/memBoxBilder/memBox45}
	\caption{Quick fix for elements that are not on any diagram}
	\label{fig:quick-fix1}
\end{figure}
\FloatBarrier

\item[$\blacktriangleright$] To make an error that leads to a more critical message than ``information'',
double click the navigable reference end \texttt{previous} of the class \texttt{Partition}, and delete its role name as depicted in Fig.~\ref{fig:delete-role-name}.
Affirm with \texttt{OK}.

You should now see a new \texttt{Fatal Error} in the validation output, stating that a navigable end \emph{must} have a role name (Fig.~\ref{fig:fatal-error}).
As navigable references are mapped to data members in a Java class, omitting the name of a navigable reference makes code generation impossible (data members, i.e., class variables must have a name).

\begin{figure}[htbp]
    \centering
  \includegraphics[width=0.85\textwidth]{pics/memBoxBilder/memBox46}
    \caption{Deleting a navigable role name of a reference}
    \label{fig:delete-role-name}
\end{figure}

\item[$\blacktriangleright$] As before, clicking the error message selects the problematic connector in the diagram, and double clicking reveals that there are no quick fixes for this problem (Fig.~\ref{fig:fatal-error}).
\item[$\blacktriangleright$] Correct your metamodel manually by setting the name of the navigable reference back to \texttt{previous}.
\clearpage
\item[$\blacktriangleright$] Ensure that your metamodel closely resembles Fig.~\ref{fig:metamodel_complete} again, and that there are no error messages before you proceed with the rest of the chapter.

\begin{figure}[htbp]
	\centering
  \includegraphics[width=0.7\textwidth]{pics/memBoxBilder/memBox47}
	\caption{Fatal Error after deleting a navigable role name}
	\label{fig:fatal-error}
\end{figure}
\end{enumerate}

As you've probably already noticed, we distinguish between five different types of validation messages:
\begin{description}
  \item[Information:]~\\
  This is only a hint for the user and can be safely ignored if you know what you're doing.
  Export and code generation should be possible, but certain naming/modelling conventions are violated, or a problematic situation has been detected.
  \item[Warning:]~\\ Export and code generation is possible, but only with defaults and automatic corrections applied by the code generator.
  As this might not be what the user wants, such cases are flagged as warnings (e.g., omitting the multiplicity at references which is automatically set by the code generator to 1).
  Being as explicit as possible is often better than relying on defaults.
  \item[Error:]~\\ Although the metamodel can be exported from EA, it is not Ecore conform and code generation will not be possible.
  \item[Fatal Error:]~\\ The metamodel cannot be exported as required information such as names or classifiers of model elements are incorrectly set or missing.
  \item[Eclipse Error:]~\\ Error messages produced by our Eclipse plugin after an unsuccessful attempt to generate code.
  This is currently not actively used.
   %This will be discussed in detail in Sect~\ref{par:validation_in_eclipse}.

\end{description}

% No New page; first instance
\visHeader

Our EA extension provides rudimentary support for validating the static semantics (Ecore) and dynamic semantics (SDM) of metamodels.
Validation results are displayed and, in some cases, even ``quick fixes'' to automatically solve the problems are offered.
Our validation framework is still work in progress so if you find an error that is not/wrongly detected feel free to send us a brief description.

\begin{enumerate}
\item[$\blacktriangleright$] To make the validation output window visible in EA, choose ``Extensions/\-Add-In Windows''.
This should display a new output window as depicted in Fig.~\ref{fig:validation_output}. This control panel also contains shortcuts for all the functionality available via the extensions file menu and many users actually prefer this.

\begin{figure}[htbp]
	\centering
  \includegraphics[width=0.6\textwidth]{pics/memBoxBilder/memBox40}
	\caption{Activating the validation output window}
	\label{fig:validation_output}
\end{figure}
\FloatBarrier

\item[$\blacktriangleright$] To start the validation choose ``Extensions/\-MOFLON::Ecore Addin/\-Validate all'' in EA (Fig.~\ref{fig:validation_menu}).
If you haven't already made any mistakes while modelling your \texttt{LearningBoxLanguage} in the last chapter, the output should resemble Fig.~\ref{fig:validation_output}, indicating that your metamodels are free of errors (at least according to our validation).

\begin{figure}[htbp]
	\centering
  \includegraphics[width=0.6\textwidth]{pics/memBoxBilder/memBox41}
	\caption{Starting the validation}
	\label{fig:validation_menu}
\end{figure}
\FloatBarrier
\end{enumerate}


To get familiar with our validation and quick fix features, let's add two small modelling errors in \texttt{LearningBoxLanguage}.

\begin{enumerate}
\item[$\blacktriangleright$] Create a new class in the \texttt{Learning\-Box\-Language} diagram.
You can retain the default name \texttt{EClass1}.
Let's assume, you wish to delete this class from your metamodel.
\item[$\blacktriangleright$] Select it (\texttt{EClass1}) in the diagram and press the \texttt{Delete} button.
Note that \texttt{EClass1} still exists in the project browser (and thus in your metamodel) and that you have one new \texttt{Information} message in the validation output (Fig.~\ref{fig:validation_information}).

\begin{figure}[htbp]
	\centering
  \includegraphics[width=0.7\textwidth]{pics/memBoxBilder/memBox43}
	\caption{Output indicating that an element is present in the metamodel, which is not in any diagram and should probably be deleted completely}
	\label{fig:validation_information}
\end{figure}

The message informs you that \texttt{EClass1} is not on any diagram, and as it is still in the model, that this could be a mistake.
Just pressing the \texttt{Delete} button is apparently not the proper way of \emph{deleting} \texttt{EClass1} from the metamodel and only removes it from the current diagram.
Deleting elements properly and other EA specific aspects are discussed in detail in Chapter~\ref{chap:Tips and Tricks}.


\item[$\blacktriangleright$] To navigate to the problematic element click once on the information message in the output window.
EA should navigate automatically to \texttt{EClass1} and highlight it in the project browser.
\item[$\blacktriangleright$] To check if there are any quick fixes available, double click the information message to invoke the quick fix dialogue.
In this case, there is one quick fix which suggests simply deleting the element from the model (Fig.~\ref{fig:quick-fix1}) as this is probably what was intended.
\item[$\blacktriangleright$] Click \texttt{OK} and \texttt{EClass1} will be deleted correctly from your model.
Your metamodel should now be error-free again as indicated by the validation output window.

\begin{figure}[htbp]
	\centering
  \includegraphics[width=0.55\textwidth]{pics/memBoxBilder/memBox45}
	\caption{Quick fix for elements that are not on any diagram}
	\label{fig:quick-fix1}
\end{figure}
\FloatBarrier

\item[$\blacktriangleright$] To make an error that leads to a more critical message than ``information'',
double click the navigable reference end \texttt{previous} of the class \texttt{Partition}, and delete its role name as depicted in Fig.~\ref{fig:delete-role-name}.
Affirm with \texttt{OK}.

You should now see a new \texttt{Fatal Error} in the validation output, stating that a navigable end \emph{must} have a role name (Fig.~\ref{fig:fatal-error}).
As navigable references are mapped to data members in a Java class, omitting the name of a navigable reference makes code generation impossible (data members, i.e., class variables must have a name).

\begin{figure}[htbp]
    \centering
  \includegraphics[width=0.85\textwidth]{pics/memBoxBilder/memBox46}
    \caption{Deleting a navigable role name of a reference}
    \label{fig:delete-role-name}
\end{figure}

\item[$\blacktriangleright$] As before, clicking the error message selects the problematic connector in the diagram, and double clicking reveals that there are no quick fixes for this problem (Fig.~\ref{fig:fatal-error}).
\item[$\blacktriangleright$] Correct your metamodel manually by setting the name of the navigable reference back to \texttt{previous}.
\clearpage
\item[$\blacktriangleright$] Ensure that your metamodel closely resembles Fig.~\ref{fig:metamodel_complete} again, and that there are no error messages before you proceed with the rest of the chapter.

\begin{figure}[htbp]
	\centering
  \includegraphics[width=0.7\textwidth]{pics/memBoxBilder/memBox47}
	\caption{Fatal Error after deleting a navigable role name}
	\label{fig:fatal-error}
\end{figure}
\end{enumerate}

As you've probably already noticed, we distinguish between five different types of validation messages:
\begin{description}
  \item[Information:]~\\
  This is only a hint for the user and can be safely ignored if you know what you're doing.
  Export and code generation should be possible, but certain naming/modelling conventions are violated, or a problematic situation has been detected.
  \item[Warning:]~\\ Export and code generation is possible, but only with defaults and automatic corrections applied by the code generator.
  As this might not be what the user wants, such cases are flagged as warnings (e.g., omitting the multiplicity at references which is automatically set by the code generator to 1).
  Being as explicit as possible is often better than relying on defaults.
  \item[Error:]~\\ Although the metamodel can be exported from EA, it is not Ecore conform and code generation will not be possible.
  \item[Fatal Error:]~\\ The metamodel cannot be exported as required information such as names or classifiers of model elements are incorrectly set or missing.
  \item[Eclipse Error:]~\\ Error messages produced by our Eclipse plugin after an unsuccessful attempt to generate code.
  This is currently not actively used.
   %This will be discussed in detail in Sect~\ref{par:validation_in_eclipse}.

\end{description}


\newpage
\genHeader

\section{Get a simple demo running}


\begin{itemize}
\hypertarget{simpleDemo common}{} 
\item[$\blacktriangleright$] Open Eclipse to a clean, fresh workspace. Go to ``Window/Open Perspective/Other\ldots''
\footnote{A path given as ``foo/bar'' indicates how to navigate in a series of menus and toolbars. Items you can open or click on-screen will be given in a
\texttt{different} font, new defintions or concepts will be \emph{italiziced}, and any data you're required to enter will be give as 'foo.'} and choose
\texttt{eMoflon}~(Fig.~\ref{fig_eclipse}).

\begin{figure}[htbp]
	\centering
  \includegraphics[width=\textwidth]{eclipse_firststart}
	\caption{Choose the eMoflon perspective {\bf update}}
	\label{fig_eclipse}
\end{figure} 

\item[$\blacktriangleright$] At either the far right, or center of the toolbar, a new action set should have appeared. Choose ``New Metamodel''
(Fig.~\ref{fig_eclipseNewMetamodelButton}).

\begin{figure}[htbp]
	\centering
  \includegraphics[width= 0.8\textwidth]{eclipse_newMetamodelButton}
	\caption{Invoke the ``New Metamodel'' wizard}
	\label{fig_eclipseNewMetamodelButton}
\end{figure}

%\nxtPage{\thepage}

\item[$\blacktriangleright$] A new dialog box should appear. This is the place where you officially decide which eMoflon syntax you'd like to use
(Fig.~\ref{fig_chooseSyntax}). With the visual syntax, you'll be using Eclipse alongside EA. You'll use EA to specify several kinds of diagrams, export, and
refresh your workspace in Eclipse. With the textual syntax (MOSL\footnote{eMoflon Specification Language}), you'll be working entirely within the Eclipse IDE.
No matter which syntax you choose however, give your project a name and make sure the \texttt{Add Demo Specification} button is selected. This will create all the
files and tests required to ensure you've set up eMoflon correctly.

\vspace{1cm}

\begin{figure}[htbp]
	\centering
  \includegraphics[width=0.9\textwidth]{eclipse_newMetamodelDialog}
	\caption{Choose your syntax}
	\label{fig_chooseSyntax}
\end{figure} 

\vspace{1cm}

% Edit me in the morning? I don't like the way this is written..
\item[$\blacktriangleright$]  Another new action on the toolbar is a button with an ``L'' on it (Fig.~\ref{fig_logger}). A log file is written everytime
you save your file, and appears in the eMoflon console window found below the main editor. This log contains important information for us if anything goes
wrong! You can silence certain loggers, set the level of loggers, and configure other settings\footnote{If you're not sure how to do this, check out a short
Log4j tutorial at \href{http://logging.apache.org/log4j/1.2/manual.html}{http://logging.apache.org/log4j/1.2/manual.html}} by clicking on this button and
modifying the \texttt{log4jConfig.properties} file.

\newpage
\vspace*{3cm}
\begin{figure}[htbp]
	\centering
  \includegraphics[width=0.9\textwidth]{eclipse_logger}
	\caption{The eMoflon logger}
	\label{fig_logger}
\end{figure} 
\end{itemize}

\fancyfoot[R]{ $\triangleright$ \hyperlink{simpleDemo vis}{Next [visual]\hspace{0.2cm}} \\ $\triangleright$ \hyperlink{simpleDemo tex}{Next [textual]}}

\clearpage
\genHeader

\subsection{A first look at EA}

\begin{itemize}
\FloatBarrier
\hypertarget{simpleDemo vis}{}
\item[$\blacktriangleright$] Can you locate the new \texttt{Demo.eap} file in your package explorer? This is the EA project file you'll be
modelling in. Don't worry about any other folders at the moment - all problems will be resolved by the end of this section.

In the meantime, do not rename, move, or delete anything.

\item[$\blacktriangleright$] Double-click \texttt{Demo.eap} to start EA, and choose \texttt{Ultimate} when starting EA for the first time.

\item[$\blacktriangleright$] In EA, select ``Extensions/Add-In Windows'' (Fig.~\ref{ea:validate_dropdown}). This will activate our tool's full
control panel. If nothing happens the installation was probably not successful. Work again through the installation or have a look at the following site:
\newline
\href{https://github.com/eMoflon/emoflon/wiki/Using-EA-With-A-Second-Windows-Account}{https://github.com/eMoflon/emoflon/wiki/Using-EA-With-A-Second-Windows-Account}.

\vspace{0.5cm}

\begin{figure}[htbp]
	\centering
  \includegraphics[width=1\textwidth]{ea_extensionMenu}
	\caption{Export from EA} 
	\label{ea:validate_dropdown} 
\end{figure}

\item[$\blacktriangleright$] This tabbed control panel provides access to all of
eMoflon's functionality. This is where you can validate and export your complete project to Eclipse by pressing \texttt{All} (Fig.~\ref{ea:controlPanel}).

\begin{figure}[htbp]
	\centering
  \includegraphics[width=0.9\textwidth]{ea_controlPanelValidateAll}
	\caption{eMoflon's control panel in EA} 
	\label{ea:controlPanel} 
\end{figure}

\item[$\blacktriangleright$] Now try exploring the EA project browser! Try to navigate to the packages, classes, and diagrams. Don't worry if you don't
understand that much - we'll get to explaining everything in a moment. Just make sure not to change anything!

\item[$\blacktriangleright$] Switch back to Eclipse, choose your metamodel project, and press \texttt{F5} to refresh. 
The export from EA places all required files in a hidden folder (.temp) in the project.
A new, third project named \emph{org.moflon.demo.doublelinkedlist} is now being created.
Do not worry about the problem markers.

\item[$\blacktriangleright$] The three asterisks (Fig.~\ref{eclipse:dirty-project}) signal that the project still needs to be built.

\vspace{0.5cm}

\begin{figure}[htbp]
    \centering
    \includegraphics[width=0.35\textwidth]{eclipse_dirtyProject}
    \caption{Dirty projects are marked with ***} 
    \label{eclipse:dirty-project} 
\end{figure}

\vspace{0.5cm}

\item[$\blacktriangleright$] Now, right-click \emph{org.moflon.demo.doublelinkedlist} and choose ``eMoflon/Build" (or use the shortcut Alt+Shift+E,B\footnote{First press Alt+Shift+E, release, and press B.
By default, most shortcuts eMoflon start with Alt+Shift+E.}).

eMoflon now generates the Java code in your repository project.
You should be able to monitor the progress with the green bar in the lower right corner (Fig.~\ref{eclipse:build}). Pressing the
symbol opens a monitor view that gives more details of the build process. You don't need to worry about any of these details, just remember to \begin{inparaenum}[(i.)]
\item refresh your Eclipse workspace after an export, and
\item rebuild projects that bear a ``dirty" marker (\texttt{***}).
\end{inparaenum}

\begin{figure}[htbp]
    \centering
    \includegraphics[width=1\textwidth]{eclipse_buildingProgress}
    \caption{Eclipse workspace when using visual syntax} 
    \label{eclipse:build} 
\end{figure}

\item[$\blacktriangleright$] If you're ever worried about forgetting to refresh your workspace, or if you just don't want to bother with having to do this,
Eclipse does offer an option to do it for you automatically. To activate this, go to ``Window/Preferences/General/Workspace" and select \texttt{Refresh on
access}.

\end{itemize}

\newpage
\texHeader
\hypertarget{simpleDemo tex}{} 
\subsection{A first look at MOSL}

\begin{itemize}

\item[$\blacktriangleright$] You should immediately have 3 folders available in your project explorer - Your metamodel project,
\texttt{DoubleLinkedListLanguage}, and \\ \texttt{DemoTestSuite} (Fig.~\ref{fig:texErrors}). Initially, you'll see a red exclamation mark indicating there
are errors, but once you give Eclipse a few seconds to refresh, all problems should be solved.

\vspace{1cm}

\begin{figure}[htp]
\begin{center}
  \includegraphics[width=0.5\textwidth]{eclipse_loadedTexDemo}
  \caption{Metamodel file structure}
  \label{fig:texErrors}
\end{center}
\end{figure}

\vspace{1cm}

\item[$\blacktriangleright$] That's it! You're all set up! But, while you're here, feel free to explore the ``Demo'' folder. It has the basic code that implements
this demo, and we recommend you take a brief look to get a feel for the the general syntax.

In the meantime, please do not rename, move, or delete anything.

\end{itemize}




\newpage
\genHeader

\section{Validate your installation with JUnit}
\label{sec:junit}

Did you notice the headings changed? If you did, awesome! If not, we must emphasize viewing this handbook in a 'double paged' format to make changes clear.

\begin{itemize}
% BEING CHANGED? .. we're adding a button, 'Build JUnit Test' and then 'Test Set up Now' (don't refer to .zip file)  
% \item[$\blacktriangleright$] Go to ``File/Import/General/Existing Projects into Workspace'' (Fig.~\ref{fig_eclipseTestsuiteImport}) and choose the Testsuite project that is also in the \texttt{eMoflonTutorial.zip} provided with this tutorial. 

\item[$\blacktriangleright$] Open the ?? Dialouge and select ``Build JUnit Test'' and then ``Test Set up Now ''. If anything goes wrong, try refreshing your 
whole workspace by choosing projects and pressing \texttt{F5} or right-clicking and selecting ``Refresh''.

% POSSIBLY OLD FILE - we're no longer importing the test project?
% \begin{figure}[htbp]
% 	\centering
%   \includegraphics[width=0.55\textwidth]{eclipse_testsuitimport}
% 	\caption{Import our Testsuite as an existing project.}
% 	\label{fig_eclipseTestsuiteImport}
% \end{figure} 
% 
% 
% \item[] At this point, your workspace should resemble Fig.~\ref{fig_eclipsepackageexplorer}.
% 
% 
% \item[$\blacktriangleright$] Right-click on the Testsuite project and select ``Run as/JUnit Test''.
% If anything goes wrong, try refreshing your whole workspace by choosing projects and pressing \texttt{F5} or right-clicking and selecting ``Refresh''.


Congratulations!  If you see a green bar  (Fig.~\ref{fig_eclipsetestsuiterun}), then everything has been set-up correctly and you are now ready to start metamodelling!


\begin{figure}[htbp]
	\centering
  \includegraphics[width=0.5\textwidth]{eclipse_testsuiterun}
	\caption{All's well that ends well\ldots}
	\label{fig_eclipsetestsuiterun}
\end{figure}


\end{itemize}

% --- Temporary white space since the following section needs to be edited
\newpage
\mbox{}
\newpage


\emph{This entire section needs to be rewritten! Right now, it's all in reference to the Visual stuff - is there anything in common with text? If so, we need to have a
'Just EA' section, and then talk about the eclipse interface. (Put the 'JustEA' behind the common text?)}

\section{Project structure and setup}
\% VISUAL

Now that everything is installed and setup properly, let's take a closer look at the different workspaces and our workflow.
Before we continue, please make a few slight adjustments to EA so you can easily compare your current workspace to our screenshots:
\begin{itemize}

\item[$\blacktriangleright$] Select ``Tools/Options/Standard Colors'' in EA, and set your colours to reflect Fig.~\ref{fig_standardColoursEA}.
This is advisable but you're of course free to choose your own colour schema.

\item[$\blacktriangleright$] In the same dialogue, select ``Diagram/Appearance'' and reflect the settings in Fig.~\ref{fig_standardAppearanceEA}.
Again, this is just a suggestion and not mandatory.

\item[$\blacktriangleright$] Last but not least, and still in the same dialogue, select ``Source Code Engineering'' and be sure to choose ``Ecore'' as the default language for code generation (Fig.~\ref{fig_standardSCEEA}). This setting is very important.

\end{itemize}

% \begin{figure}[htbp]
% 	\centering
%   \includegraphics[width=0.4\textwidth]{eclipse_packageexplorer}
% 	\caption{Workspace in Eclipse.}
% 	\label{fig_eclipsepackageexplorer}
% \end{figure}

\begin{figure}[htbp]
  \centering
  \includegraphics[width=0.8\textwidth]{standardColours}
  \caption{Our choice of standard colours for diagrams in EA.}
  \label{fig_standardColoursEA}
\end{figure}


In your EA ``workspace'', actually referred to as an \emph{EA project}\footnote{Words are set in italics when they represent concepts that are introduced or defined  in the corresponding paragraph for the first time.}, take a careful  look at the project structure:  The root node \texttt{Demo}\footnote{Words set  in a \texttt{mono-space font} refer to things that you should find in a tool,  dialogue, figure or code.} is called a \emph{model} in EA lingo and is used as a  container to group a set of related \emph{packages}. 
In our case, \texttt{Demo}  consists of a single package \texttt{DoubleLinkedListLanguage}.
An EA project can however consist of numerous models that in turn group  numerous packages.

\% VISUAL \& TEXTUAL

Now switch to your \emph{Eclipse workspace} and note the two nodes named \texttt{Spe\-ci\-fi\-ca\-tions} and \texttt{Demo}.  These nodes, used to group related \emph{Eclipse projects} in an Eclipse workspace, are called \emph{working sets}.
The working set \texttt{Spe\-ci\-fi\-ca\-tions} contains all \emph{metamodel projects} in a  workspace.

\% VISUAL

A metamodel project contains a single EAP (EA project) file and is used to communicate with EA and initiate code generation by simply pressing F5 or choosing ``refresh'' from the context menu.
In our case, \texttt{Specifications} should contain a single metamodel project \texttt{Demo} containing our EA project file  \texttt{Demo.eap}.

\begin{figure}[htbp]
  \centering
  \includegraphics[width=0.8\textwidth]{standardAppearance}
  \caption{Our choice of the standard appearance for model elements.}
  \label{fig_standardAppearanceEA}
\end{figure}

\% VISUAL

Figure~\ref{fig_fromEAtoEclipse} depicts how the Eclipse working set \texttt{Demo} and its contents were generated from the EA model \texttt{Demo}.
Every model in EA is mapped to a working set in Eclipse with the same name. 
From every package in the EA model, an Eclipse project is generated, also with the same name.

\% VISUAL \& TEXTUAL?

These projects, however, are of a different \emph{nature} than for example metamodel projects or normal Java projects, and are called \emph{repository projects}.  
A nature is Eclipse lingo for ``project type'' and is visually indicated by a corresponding nature icon on the project folder.
Our  metamodel projects sport a spanking little class diagram symbol. 
Repository projects are generated automatically  with a certain project structure according to our conventions.  
The  \texttt{model} subfolder is probably most important, and contains an  \emph{Ecore model}.  
Ecore is a metamodelling language that provides building  blocks like \emph{classes} and \emph{references} for defining the  static structure (concepts and relations between concepts) of a system.  

\% VISUAL

The  export function of our EA plugin generates a valid Ecore model from the  corresponding EA model and persists it as an XML file in the \texttt{model}  subfolder.  
In our concrete example, this is the \texttt{DoubleLinkedListLanguage.ecore} file.  
Go ahead and double-click it to open the file in a simple tree-view editor in Eclipse.  
If you are really interested in the nitty-gritty details or have a masochistic hang, right-click the file and select ``Open With/Text Editor''. 

\begin{figure}[htbp]
  \centering
  \includegraphics[width=0.8\textwidth]{standardCodeEngineering}
  \caption{Make sure you set the standard language to Ecore.}
  \label{fig_standardSCEEA}
\end{figure}

\% VISUAL \& TEXTUAL

This Ecore model is used to drive a code generator that maps the model to Java interfaces and classes.  
The generated Java code that represents the model is often referred to as a \emph{repository} and this is the reason why we refer to such projects as repository projects\footnote{Not to be mixed up with CVS or SVN repositories, although the idea of a source code ``container'' is the same here.}. 
A repository can be viewed as an \emph{adapter} that enables building and manipulating concrete instances of a specific model via a programming language such as Java.  
This is why we indicate repository projects using a cute adapter/plug symbol on the project folder.  

\% VISUAL

Figure~\ref{fig_fromEAtoEclipse} also depicts how the class \texttt{Node} in the EA model is mapped to the Java interface \texttt{Node}.  
Double-click \texttt{Node.java} and take a look at the methods declared in the interface.
These correspond directly to the methods declared in the modelled \texttt{Node} class.  
Indicated by the source folders \texttt{src}, \texttt{injection} and \texttt{gen}, we advocate a clean separation of hand-written (this should go in \texttt{src} and \texttt{injection}) and generated code (lands automatically in \texttt{gen}).  
As we shall see later in the tutorial, hand-written code can be integrated in generated classes vio Injections. 
This is sometimes more elegant for small helper functions or necessary for String manipulation for instance.

\begin{figure}[htbp]
    \centering
  \includegraphics[width=\textwidth]{bothexplorers}
    \caption{From EA to Eclipse}
    \label{fig_fromEAtoEclipse}
\end{figure}

\% VISUAL \& TEXUAL? ({ \small explains Gen, i'm assuming this is still in eclipse })

If you take a careful look at the code structure in \texttt{gen}, you'll find a \texttt{Foo\-Impl.java} for every \texttt{Foo.java}. 
Indeed, the subpackage \texttt{impl} contains Java classes that implement the interfaces in the parent package.  
Although this might strike you as unnecessary (why not merge interface and implementation for simple classes?), this consequent separation in interfaces and implementation allows for a clean and relatively simple mapping of Ecore to Java, even in tricky cases like multiple inheritance (allowed and very common in Ecore models).  
A further package \texttt{util} contains some auxiliary classes such as a factory for creating instances of the model.  

\% VISUAL ({ \small but it still explains the amount of code\ldots? }) \& TEXUAL?

If this is your first time of seeing generated code, you might be shocked at the sheer amount of classes and code generated from our relatively simple EA model.  
You might be thinking: ``hey - if I did this by hand I wouldn't need half of all this stuff!''.  
Well you're right and you're wrong -- the point is that an automatic mapping to Java via a code generator scales quite well.
This means for simple, trivial examples (like our double linked list), it might be possible to come up with a leaner and simpler Java representation.  
For complex, large models with lots of mean pitfalls, however, this becomes a daunting task.  
The code generator provides you with years and years of experience of professional programmers who have thought up clever ways of handling multiple inheritance, an efficient event mechanism, reflection, consistency between bidirectionally linked objects and much more.

A point to note here is that the mapping to Java is obviously not unique. 
Indeed there exist different standards of how to map a modelling language to a general purpose programming language like Java. 

We use a mapping defined and implemented by the Eclipse Modelling Framework (EMF) which tends to favour efficiency and simplicity.

Although getting the \emph{details} of mapping the static structure of our models to Java might be extremely difficult, it is actually straight forward.  
A fantastic productivity boost in any case but (yawn) not exactly exciting.


\% VISUAL

Have you noticed the methods of the \texttt{Node} class in our EA model? 
Now hold on tight -- each method can be \emph{modelled} completely in EA and the corresponding implementation in Java is generated automatically and placed in \texttt{NodeImpl}.  
Just in case you didn't get it: The behavioural or dynamic aspects of a system can be completely modelled in an abstract, platform (programming language) independent fashion using a blend of activity  diagrams and a ``graph pattern'' language called \textit{Story Driven Modelling} (SDM).  
In our EA project, these ``Stories'', ``Story Models'' or simply ``SDMs'' are  placed in SDM Containers named according to the method they implement.  
E.g.  \texttt{$\guillemotleft$SDM Container$\guillemotright$ insertNodeAfter SDM} for the method  \texttt{insertNodeAfter(Node)} as depicted in
Fig.~\ref{fig_fromEAtoEclipse}.  
We'll spend the rest of the tutorial understanding why SDMs are so  {\huge crazily} cool!
 
To recap all we've discussed, let's consider the complete workflow as depicted in Figure~\ref{fig_Overview}. 
We started with a concise model in EA, simple and independent of any platform specific details~(1).  
Our EA model consists not only of static aspects modelled as a class diagram~(2), but also of dynamic
aspects modelled using SDM~(3).  After exporting the model and code generation~(4), we basically switch from \emph{modelling} to \emph{programming} in a specific general purpose programming language (Java).  
On this lower \emph{level of abstraction}, we can flesh out the generated repository~(5) if necessary, and mix as appropriate with hand-written code and libraries.  Our abstract specification of behaviour (methods) in SDM is translated to a series of method calls that form the body of the corresponding Java method~(6).

\vspace{1cm}

\begin{figure}[htbp]
	\centering
  \includegraphics[width=0.9\textwidth]{tafelbild}
	\caption{Overview}
	\label{fig_Overview}
\end{figure}


% ---- END OF PART I  -----------------------------------------------------------------------------------------------------------------
\newpage
\noHeader

End of Handbook Part I

If you feel a bit lost at the moment, don't worry! Be patient - This first chapter has been a lot about installation and tool support, and only aims at giving a very brief glimpse at the big picture of what is actually going on.  

If you enjoyed this section and wish to get started on the key features of eMoflon, Check out Part II! There we will go step-by-step through a hands-on example and cover the core features of Ecore (static structure). %and SDM  (behaviour).

We shall also introduce clear and simple definitions for the most important metamodelling and graph transformation concepts, always referring to the concrete example and providing ample references for further reading.
If you're already familiar with the tool however, feel free to pick and choose chapters that are interesting to you. Check out Story Driven Modeling (SDMs) in Part III, or Triple Graph Transformations in Part IV. 

Cheers!