\newpage
\texHeader

\subsection{Your MOSL workspace}

\hypertarget{projectStructure tex}{}
As you now know, eMoflon is a plug-in for Eclipse. More precisely, eMoflon requires the Eclipse Modeling Framework (EMF) in order to work. EMF uses two separate
models, Genmodel and Ecore, for code generation. The Genmodel contains boring information about code generation such as path, file prefixes, and other
information. We are more interested in the Ecore model, which we specify with MOSL.

When you switched the project explorer from \texttt{Projects} to \texttt{Top Level Elements}, you noticed that a few extra nodes were displayed in the project
browser. Each node you see has different criteria for grouping related Eclipse Projects together, which makes them your project \emph{working sets}.

The \texttt{Specifications} working set contains all \emph{metamodel projects} in a workspace (Fig. ~\ref{fig_modelSpecification}). This means that for every
new metamodel you create in your current workspace, all of the relevant files will be placed here.

 \begin{figure}[htbp]
  \centering
  \includegraphics[width=0.5\textwidth]{eclipse_Specification}
  \caption{Specification working set}
  \label{fig_modelSpecification}
\end{figure}
  
Let's have a look at the two \emph{eclass} files. While you can usually combine several short class declarations in a single file in some languages (like Java),
keep them when using MOSL separated. It'll all make sense in a moment when we discuss \emph{patterns}.

%\nxtPage{\thepage}

Inspect the \texttt{List} eclass, and you'll see it has a just one \emph{EAttribute}. EAttributes are defined by their name, followed by a colon symbol and
type. This class also has one type of \emph{EReference}, a \emph{container reference}. This is represented by the diamond operator in front of an arrow. Switch
to the \texttt{Node} eclass and you can observe the second reference type, a \emph{simple~reference}. It's represented by a plain arrow. EReference names are
immediately followed by their multiplicity and then, just like an attribute, a colon and their datatype.

Let's go back to looking at the code. In the \texttt{Node} class (Fig.~\ref{fig_patternDeleteNode}), a few methods have been declared. You can see that each
function is remarkably small. In fact, the only thing the functions are doing are invoking patterns. These patterns represent structural changes, and their
container functions are used exclusively for control flow (ie., sequences, branches, and loops.).

 \begin{figure}[htbp]
  \centering
  \includegraphics[width=0.6\textwidth]{eclipse_eclassNode}
  \caption{The \texttt{Node} eclass {\bf update (comments)}}
  \label{fig_eclassNode}
\end{figure}

% See how it looks.. may have to remove
%\nxtPage{\thepage}

After many long discussions, it was decided that patterns should always be implemented in separate files. Inspect Fig.~\ref{fig_modelSpecification} again, and
observe the locations where the patterns are placed. You'll notice that there is a folder for the ``Node'' eclass, and a subfolder for each method. There's no
folder for ``List'' because it never calls a pattern.

Check out the \texttt{deleteNode} pattern (Fig ~\ref{fig_patternDeleteNode}).

 \begin{figure}[htbp]
  \centering
  \includegraphics[width=0.6\textwidth]{eclipse_patternDeleteNode}
  \caption{Eclipse: The deleteNode Pattern}
  \label{fig_patternDeleteNode}
\end{figure}

You can see that there is a destroy command on \texttt{@this}. It gets rid of the node, but it doesn't do anything about the previous and next references
defined on it. That's because when the node is removed, everything attached to it will be cleaned and updated. The final command reconnects the next and
previous nodes. All details will be discussed in later parts.
 
So that's a quick overview of the MOSL language but again, how to we generate code from all of this?

%\fancyfoot[R]{  $\triangleright$ \hyperlink{codeGen common}{Next} }

First, eMoflon does not do perform compelete code generation on every change. To improve performance, only the parser is invoked when you save files. This means
that to generate code for the project, you need to invoke the \texttt{Build (and clean)} command from the eMoflon context menu after right clicking your
metamodel project.
