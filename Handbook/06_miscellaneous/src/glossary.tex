\newpage
\phantomsection
\addcontentsline{toc}{section}{Glossary}
\hypertarget{glossary}{}
\genHeader

\vspace{1cm}
{\Huge \bf Glossary}
\vspace{1cm}


\begin{description}

\item[Abstract Syntax] 
Defines the valid static structure of members of a language. 

\item[Activity]
Top-most element of an SDM.

\item[Activity Edge]
A directed connection between activity nodes describing the control flow within an activity.

\item[Activity Node]
Represent atomic steps in the control flow of an SDM. Can be either story patterns or statement nodes.

\item[Assignments]
Used to set attributes of object variables.

\item[Attribute Constraint]
A non-structural contraint that must be satisfied for a story pattern to match. Can be either an assertion or assignment.

\item[Binding State]
Can be either \emph{bound} or \emph{unbound/free}. See \emph{Bound vs Unbound}.

\item[Binding operator]
Determine whether a variable is to be \emph{checked}, \emph{created}, or \emph{destroyed} during pattern matching.

\item[Binding Semantics]
Determines if an object variable \emph{must} exist (\emph{madatory}), may not exist (\emph{negative}; see \emph{NAC}), or is \emph{optional} during
\emph{pattern matching}.

\item[Bound vs Unbound]
Bound variables are completely determined by the current context, whereas unbound (free) variables have to be determined by the \emph{pattern matcher}.
\texttt{this} and parameter values are always bound.

\item[Concrete Syntax]
How members of a language are represented. This is often done textually or visually.

\item[Constraint Language] 
Typically used to specify complex constraints (as part of the static semantics or a language) that cannot be expressed in a metamodel.

\item[Correspondence Types] 
Connect classes of the source and target metamodels.

\item[Dangling Edges]
An edge with no target or source. Graphs with dangling edges are invalid, which is why dangling edges are avoided and automatically deleted by the pattern
matching engine.

\item[Dynamic Semantics] 
Defines the dynamic behavior for members of a language.

\item[EA]
Enterprise Architect; The UML visual modeling tool used as our visual frontend.

\item[Edge Guards]
Refine the control flow in an activity by guarding activity edges with a condition that must be satisfied for the activity edge to be taken.

\item[Grammar] 
A set of rules that can be used to generate a language. 

\item[Graph Grammar] 
A grammar that describes a graph language. This can be used instead of a metamodel or type graph to define the abstract syntax of a language.

\item[Graph Triples] 
Consist of connected source, correspondence, and target components.

\item[Link or correspondence Metamodel] 
Comprised of all correspondence types.

\item[Link Variable]
Placeholders for links between matched objects.

\item[Literal Expression]
Represents literals such as true, false, 7, or ``foo.''

\item[Meta-Language] 
A language that can be used to define another language.

\item[Meta-metamodel] 
A \emph{modeling language} for specifying metamodels.

\item[Metamodel] 
Defines the abstract syntax of a language including some aspects of the static semantics such as multiplicities. 

\item[MethodCallExpression]
Used to invoke any method.

\item[Model] 
Graphs which conform to some metamodel.

\item[Modeling Language] 
Used to specify languages. Typically contains concepts such as classes and connections between classes.

\item[Monotonic] 
In the context of TGGs, a non-deleting characteristic.

\item[NAC]
Negative Application Condition; Used to specify structures that must not be present for a transformation rule to be applied.

\item[Object Variable]
Place holders for actual objects in the current model to be determined during pattern matching.

\item[ObjectVariableExpression]
Used to reference other object variables.

\item[Operationalization] 
The process of deriving step-by-step executable instructions from a declarative specification that just states what the outcome should
be but not how to achieve it.

\item[Parameter Expression]
Used to refer to method parameters.

\item[(Graph) Pattern Matching]
Process of assigning objects and links in a model to the object and link variables in a pattern in a type conform manner. This is also reffered to as finding a
match for the pattern in the given model.

\item[Statement Nodes]
Used to invoke methods as part of the control flow in an activity.

\item[Static Semantics] 
Constraints members of a language must obey in addition to being conform to the abstract syntax of the language.

\item[Story Node]
\emph{Activity nodes} that contain \emph{story pattern}s.

\item[Story Pattern]
Specifies a structure change of the model.

\item[Triple Graph Grammars (TGG)] 
Declarative, rule-based technique of specifying the simultaneous evolution of three connected graphs.

\item[Type Graph] 
The graph that defines all types and relations that form a language. Equivalent to a metamodel but without any static semantics.

\item[TGG Schema] 
The metamodel triple consisting of the source, correspondence (link), and target metamodels.

\item[Unification]  
An extension of the object oriented ``Everything is an object'' principle, where everything is regarded as a model, even the metamodel which defines other
models.

\end{description}
