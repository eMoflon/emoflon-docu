{\bf \huge Part 0:}

\vspace{1cm}

{\bf \Huge Introduction }

\vspace{2cm}



This tutorial has been engineered to be \emph{fun}.

If you work through it and, for some reason, do \emph{not} have a resounding \mbox{``I-Rule''} feeling afterwards, please send us an email and tell us how to improve it at \href{mailto:contact@moflon.org}{contact@moflon.org}

\begin{figure}[htp]
\begin{center}
	% RENAME THIS IMAGE ??
	\includegraphics[height=0.45\textheight]{../introduction_images/i-rule}
	\caption{How you should feel when you're done.}
	\label{i-rule}
\end{center}
\end{figure}
\break
 


To enjoy the experience, you should be fairly comfortable with Java or a comparable object-oriented language, and know how to perform basic tasks in Eclipse.  Although we assume this, we give references to help bring you up to speed as necessary.  
Last but not least, very basic knowledge of common UML notation would be helpful.

Our goal is to give a \emph{hands-on} introduction to metamodelling and graph transformations using our tool \emph{eMoflon}.
The idea is to \emph{learn by doing} and all concepts are introduced while working on a concrete example.
The language and style used throughout is intentionally relaxed and non-academic.
For those of you interested in further details and the mature formalism of graph transformations, we give relevant references throughout the tutorial.

The tutorial has been split into 6 independent parts. Below we offer a few suggestions of what you can read or skip, depending on what you're interested in.
 
% UPDATE/ELABORATE ON THESE AS YOU REWRITE EACH PART; Include approximately how long it should take.
% Also, try to make them average the same amount of words? (So one isn't awkwardly larger than the other)
\begin{description}

\item[Part I: Installation and Set Up]provides a very simple example and a few JUnit tests to test the installation and configuration of eMoflon.
 
After working through this chapter, you should have an installed and tested eMoflon working for a trivial example.
We also explain the general workflow and the different workspaces involved.

This chapter can be considered \emph{mandatory} if you are new to eMoflon and we recommend working through it in any case.
It's also kept as minimal as possible and should only take a few minutes really.

\item[Part II: Ecore] is the main chapter and takes you step-by-step through a more realistic example that showcases most of the features we currently support.

Working through this chapter should serve as a basic introduction to model-driven engineering, and it's reccomended you work through this chapter if you're new to metamodeling in general (Using Ecore/EMF) %, metamodelling and graph transformations.

Approximate Time to Complete: min

Relevant .zip file:

\item[Part III: SDMs] introduces model transformations via graph transformation using Story Driven Modeling.

Approximate Time to Complete: min

Relevant .zip file:

\item[Part IV: TGGs] Even if you're mainly interested in TGGs, we recommend working through at least Part I. Although the example builds up on parts constructed in previous chapters, we provide a ``cheat'' package that you can use to get started directly.

Approximate Time to Complete: min

Relevant .zip file:

\item[Part V: Model To Text Transformations] Description;

Approximate Time to Complete: min

Relevant .zip file:

\item[Part VI: Miscellaenous] Contains some great references files to keep on hand while using ECore and it's partner programs. These will help to avoid mistakes and increase efficiency. %(Grokking EA)
If you're in a hurry, this chapter can be skipped

Approximate Time to Complete: min

Relevant .zip file:
\end{description}

One last thing: at the moment we unfortunately only support Windows. This should hopefully change in future releases.

Well, that's it -- sit back, relax, grab a coffee and enjoy the ride!

% Include some sort of coffee graphic?
