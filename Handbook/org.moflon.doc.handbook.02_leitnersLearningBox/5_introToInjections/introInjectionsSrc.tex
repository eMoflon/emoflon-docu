\newpage
\chapter{Introduction to injections}
\genHeader

This short introduction will show you how to implement small methods by adding handwritten code to classes generated from your model. Injections are inspired by
partial classes in C\#, and are our preferred way of providing a clean separation between generated and handwritten code. 

Let's implement the \texttt{removeCard} method, declared in the \texttt{Partition} EClass. In order to `remove' a card from a partition, all one needs to do is
disable the link between them. Don't forget that (according to the signature) not only does \texttt{removeCard} have to pass in a \texttt{Card}, it must return
one as well.

\begin{stepbystep}

\item From your working set, open ``gen/LearningBoxLanguage.impl/Part\-it\-ionImpl.java'' and enter the following code in the
\texttt{removeCard} declaration, starting at approximately line 347. Do not remove the first comment, which is necessary to indicate that this code is written
by the user and needs to be extracted automatically as an injection. Please also do not copy and paste the following code -- the copying process will most
likely add invisible characters that eMoflon is unable to handle.

\vspace{0.5cm}

\begin{figure}[htbp]
        \centering
        \begin{lstlisting}[language=Java, keywordstyle={\bfseries\color{purple}}, backgroundcolor=\color{white}]
    public Card removeCard(Card toBeRemovedCard) {
		// [user code injected with eMoflon]
		if(toBeRemovedCard != null){
			toBeRemovedCard.setCardContainer(null);
		}
		return toBeRemovedCard;
	}
        \end{lstlisting}
        \caption{Implementation of \texttt{removeCard}}
        \label{code:addToStringRep_impl}
\end{figure}

\vspace{0.5cm}

\item Save the file, then right-click either on the file in the package explorer, or in the editor window, and choose ``eMoflon/
Refresh injection for class'' (Alt+Shift+E,I) from the context menu (\Cref{eclipse:injection_create_injection}).

\begin{figure}[htbp]
    \centering
    \includegraphics[width=\textwidth]{../../org.moflon.doc.handbook.02_leitnersLearningBox/5_introToInjections/iImages/eclipse_createInjection}
    \caption{Create a new injection}
    \label{eclipse:injection_create_injection}
\end{figure}

\item This will create a new file in the ``injection'' folder of your project with the same package and name stucture as the Java class,
but with a new \texttt{.inject} extension (\Cref{eclipse:injection_folder}).

\begin{figure}[htbp]
    \centering
    \includegraphics[width=0.5\textwidth]{../../org.moflon.doc.handbook.02_leitnersLearningBox/5_introToInjections/iImages/eclipse_injectionFolder}
    \caption{Partition injection file}
    \label{eclipse:injection_folder}
\end{figure}

\item Double click to open and view this file. It contains the definition of a \textit{partial class}
(\Cref{eclipse:injection_partialClassPartition}).

\begin{figure}[htbp]
    \centering
    \includegraphics[width=0.8\textwidth]{../../org.moflon.doc.handbook.02_leitnersLearningBox/5_introToInjections/iImages/eclipse_partialClassPartition}
    \caption{Generated injection file for \texttt{PartitionImpl.java}}
    \label{eclipse:injection_partialClassPartition}
\end{figure}

\clearpage

\item As a final step, build your metamodel to check that the code is generated and injected properly.

\item By the way, eMoflon allows you to switch quickly between a Java class and its injection file.
When inside ``PartitionImpl.injection", open the context menu and select ``eMoflon/Class \verb|<->| Injection" (Alt+Shift+E,W).
This brings you to ``PartitionImpl.java".

Repeat this command and you are back in ``PartitionImpl.inject``.


\item That's it! While injecting handwritten code is a remarkably simple process, it is pretty boring and low level to call all those
setters and getters yourself. We'll return to injections for establishing two simple methods in Part III using this strategy, but we'll also learn how to
implement more complex methods using Story Diagrams.
 
\end{stepbystep}

\subsection*{Creating injections automatically with save actions}

Information loss is a typical pitfall that you may encounter when working with injections:
If you forget to create an injection and trigger a rebuild (``eMoflon/Build", Alt+Shift+E,B), the generated code is entirley dropped -- including any unsaved injection code.

To save you from this frustrating experience, eMoflon may automatically save injections whenever you save your Java file.

\begin{stepbystep}

\item 
Open the preferences dialog via ``Window/Preferences" and navigate to the ``Save Actions" (``Java/Editor/Save Actions").

\item
To enable save actions, tick ``Perform the selected actions on save" and ``Additional actions" (\Cref{eclipse:injection_saveActions1}).

\begin{figure}[htbp]
    \centering
    \includegraphics[width=0.75\textwidth]{../../org.moflon.doc.handbook.02_leitnersLearningBox/5_introToInjections/iImages/eclipse_injection_saveActions1.png}
    \caption{Main configuration dialog for save actions}
    \label{eclipse:injection_saveActions1}
\end{figure}

\item
Press ``Configure", switch to the ``eMoflon Injections" tab and tick ``Enable injection extraction on save" (\Cref{eclipse:injection_saveActions2}).

\begin{figure}[htbp]
    \centering
    \includegraphics[width=0.75\textwidth]{../../org.moflon.doc.handbook.02_leitnersLearningBox/5_introToInjections/iImages/eclipse_injection_saveActions2.png}
    \caption{Configuration dialog for additional save actions}
    \label{eclipse:injection_saveActions2}
\end{figure}

\item
After confirming the dialog, the bulleted list should contain the entry ``Create eMoflon injections".

\item
Open up ``PartitionImpl.java", perform a tiny modification on it, and save the file.
The eMoflon console should display a message to confirm that injections have been extracted automatically, e.g.,
\begin{verbatim}
[handlers.CreateInjectionHandler::115] -  Created injection
file for 'PartitionImpl'.
\end{verbatim}

\end{stepbystep}

