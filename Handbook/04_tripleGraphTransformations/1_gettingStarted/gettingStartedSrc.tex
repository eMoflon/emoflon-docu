\newpage
\section{Setting up your workspace}
\genHeader

Unfortunately, before you're able get started with  TGGs, you need to have your two metamodels already prepared. Our source metamodel will be
\texttt{LearningBoxLanguage}, which we built and completed in Parts II and III. We have provided the target metamodel, \texttt{Dictionarylanguage}, in the
download zip file for this part of the handbook. You'll need to properly import this metmodel into your current \texttt{.eap} file. These instructions are
crucial as \emph{both} metamodels must be in the \emph{same} file to specify an integration with TGGs.

If you have just joined us in this part and wish to get started right away with TGGs, complete Section~\ref{sec:loadSourceMeta} to load our running example. If
you've completed Part III however, skip ahead to either \texttt{Section~\ref{sec:multiEAP} (Visual)} or \texttt{Section~\ref{sec:multiMOSL} (Textual)} for
instructions on how to import the target metamodel into your workspace.

\subsection{Loading \texttt{LearningBoxLanguage}}
\label{sec:loadSourceMeta}
\begin{itemize}

\item[$\blacktriangleright$] To get started, press the \texttt{new} button on the toolbar and navigate to ``Examples/eMoflon Handbook Examples/''
(Fig.~\ref{fig:downPartIV})

\begin{figure}[htbp]
\begin{center}
  \includegraphics[width=0.75\textwidth]{eclipse_downloadWizardPartIV}
  \caption{Download a file set to get started}
  \label{fig:downPartIV}
\end{center}
\end{figure}

\item[$\blacktriangleright$] We have created two file packages based on the eMolfon specification type you'd like to learn. Remember, with the visual
syntax, you'll be using an external modeling program to craft your metamodel diagrams, then exporting the data to Eclipse for generation. With textual, you'll
be working entirely within the Eclipse IDE, in the eMolfon perspective. 

\item[$\blacktriangleright$] If your package explorer does not look similar to ours in Fig.~\ref{fig:workingSets} with at least two distinct nodes, select the
small, downward facing arrow in the corner of the module window. Choose ``Working Sets'' as your ``Top Level Elements.'' We use these to structure the
workspace in Eclipse.

\vspace{0.5cm}

\begin{figure}[htbp]
	\centering
  \includegraphics[width=0.9\textwidth]{eclipse_workingSets}
	\caption{Setting your Package Explorer}
	\label{fig:workingSets}
\end{figure}

The ``MyWorkingSet" node contains \texttt{Learn\-ing\-Box\-Lang\-uage}, which in turn contains all the code generated from your metamodel. The metamodel itself
is found under ``Specifications.'' The visual metamodel is a single \texttt{.eap} file, while the textual metamodel is inside an explicit project structure.

\vspace{0.5cm}

These sets are not included under same node due to being different \emph{nature}s, or project types. \texttt{Learning\-Box\-Language} is your
\emph{repository project}, the Eclipse classification of a normal Java project.\footnote{For details on the project setup, review Part I, sections 4 and 5} 

\vspace{0.5cm}

We recommend reading the overview of \texttt{LeitnersLearningBox}, it's purpose and goals, in the introduction to Part II. This will provide the
motivation for creating a TGG with a dictionary language, and may help you understand why certain attributes and links are connected.

\end{itemize}

% Instructions for loading target model
\newpage
\subsection{Working with multiple EAPs}
\visHeader
\label{sec:multiEAP}

\begin{itemize}

\item[$\blacktriangleright$] Included in the \texttt{Part4Download} .zip file is the \texttt{dictionaryLanguage.eap} file. Double click this to open it in
Enterprise Architect (EA).

\vspace{0.5cm}

\item[$\blacktriangleright$] Although you can simply copy and paste single packages between multiple EAPs, packages with dependencies to other packages (i.e.,
those between \texttt{DictionaryCodeAdapter} and \texttt{DictionaryLanguage}) cannot be copied so easily. If you do this, all links will be destroyed!
Therefore, to migrate multiple packages, you have to first export a \emph{complete} root node (the package on the top-most level of the project browser) to an
XMI file.

\vspace{0.5cm}

\item[$\blacktriangleright$] Right click on \texttt{dictionaryLanguageRoot}, and select \texttt{Export Model to XMI\ldots,} as depicted in
Fig.~\ref{fig:export}.

\vspace{0.5cm}

\begin{figure}[htbp]
\begin{center}
  \includegraphics[width=0.5\textwidth]{ea_exportToXMI}
  \caption{Exporting the \emph{target metamodel}}
  \label{fig:export}
\end{center}
\end{figure}

\item[$\blacktriangleright$] Save the file somewhere easily accessible, such as your desktop, and change the export type to \texttt{XMI 2.1}. You should have a
small green bar appear once the action is complete (Fig.~\ref{fig:exportDialogue}).

\vspace{0.5cm}

\begin{figure}[htbp]
\begin{center}
  \includegraphics[width=0.8\textwidth]{ea_exportPackageDialogue}
  \caption{Persisting the export to a file}
  \label{fig:exportDialogue}
\end{center}
\end{figure}

\item[$\blacktriangleright$] Now open your \texttt{LeitnersLearningBox.eap} file, and right-click the root \texttt{MyWorkingSet} node and select \texttt{Import
Model from XMI\ldots}. In the dialogue that appears, find the file you just saved and \texttt{import}. Press \texttt{yes} in each of the confirmation dialogues
that appear after. Your workspace should now resemble Fig.~\ref{fig:postImport}.

\vspace{0.5cm}

\begin{figure}[htbp]
\begin{center}
  \includegraphics[width=0.5\textwidth]{ea_postImport}
  \caption{Project explorer after importing the \emph{target} metamodel}
  \label{fig:postImport}
\end{center}
\end{figure}

\end{itemize}


\newpage
\subsection{Working with multple MOSL projects}
\texHeader
\label{sec:multiMOSL}

(Eclipse import instructions here.)

Your job is easy unlike visual -- we HAD to review how to properly export. theres enoguh documentation for eclipse that all we'll make you do it import. so,
right click and clikc import

Make you root directory the textual source folder that was in the part IV download. Both dicationaryadapter and dictionarlanguage projects will be imported into
your workspace.

