\newpage
\hypertarget{multiEAP}{}
\subsection{Importing and working with multiple EAPs}
\visHeader

Please note that the following instructions on how to properly export and import Enterprise Architect (EA) files are \emph{not} an eMoflon-exclusive feature!
Instead of simply referring you to their instructions, we have included them here as importing projects into the \emph{same} workspace is crucial for
transformations to work.

\begin{itemize}

\item[$\blacktriangleright$] Press the \texttt{new} button on the Eclipse toolbar and navigate to ``Examples/eMoflon Handbook Examples/''
(Fig.~\ref{eclipse:dictionaryDownloadWizard}). Find and select \texttt{Visual Dictionary Language} to copy a new \texttt{Dictionary} metamodel project into your
workspace.

\vspace{0.5cm}

% Image stored in ../1_gettingStarted/textImportImages/
\begin{figure}[htbp]
\begin{center}
  \includegraphics[width=0.8\textwidth]{eclipse_part4DictionaryLanguageDownload}
  \caption{Download the visual Dictionary metamodel}
  \label{eclipse:dictionaryDownloadWizard}
\end{center}
\end{figure}

\item[$\blacktriangleright$] If successful, your workspace should come to resemble Fig.~\ref{eclipse:loadedDictionaryEAP}. Double-click
\texttt{DictionaryLanguage.eap} to open it in EA.

\newpage

\begin{figure}[htbp]
\begin{center}
  \includegraphics[width=0.5\textwidth]{eclipse_loadedDictionaryEAP}
  \caption{\texttt{DictionaryLanguage} successfully copied into the workspace}
  \label{eclipse:loadedDictionaryEAP}
\end{center}
\end{figure}

\vspace{0.5cm}

\item[$\blacktriangleright$] The file's project browser should resemble Fig.~\ref{ea:dictionaryLangStart}. Feel free to inspect the main
\texttt{DictionaryLanguage} diagram until you're familiar with the metamodel. Our work will focus on the \texttt{Dictionary} and \texttt{Entry} classes. You'll
be able to see that dictionaries can be assigned unique \texttt{ESTring title}s, and each entry will have some sort of \texttt{content} matched with one of
three difficulty \texttt{level}s.

\vspace{0.5cm}

\begin{figure}[htbp]
\begin{center}
  \includegraphics[width=0.4\textwidth]{ea_dictLangProBrowser}
  \caption{The \texttt{DictionaryLanguage} metamodel structure}
  \label{ea:dictionaryLangStart}
\end{center}
\end{figure}

\item[$\blacktriangleright$] It should be said that while you are able to copy and paste packages between multiple EAPs (i.e., copy
\texttt{<<EPack\-age>>DictionaryLanguage} into the \texttt{MyWorkingSet} root note of your source metamodel), if any of the copied packages have dependencies on
other packages, it cannot be done so easily. All links would be destroyed! If you tried to copy \texttt{DictionaryLanguage}, for example, the receiving project
would not be able to establish the necessary links to eMoflon's \texttt{Moca Language}.

\clearpage

\item[$\blacktriangleright$] Therefore, to properly migrate the \texttt{DictionaryLanguage} package, right-click on the EPackage root and navigate to
``Import/Export" and select \texttt{Export Model to XMI\ldots} (Fig.~\ref{ea:contextExport}). Alternatively, you can select the root in the project browser and
press \texttt{Ctrl + Alt + E}.

\vspace{0.5cm}

\begin{figure}[htbp]
\begin{center}
  \includegraphics[width=\textwidth]{ea_contextExport}
  \caption{Starting the export process in EA}
  \label{ea:contextExport}
\end{center}
\end{figure}

\item[$\blacktriangleright$] Switch the export type to \texttt{XMI 2.1} in the dialogue and save the file somewhere easily accessible (such as your desktop).
Press export, and close the window once the small green bar appears (Fig.~\ref{ea:export}).

\begin{figure}[htbp]
\begin{center}
  \includegraphics[width=0.9\textwidth]{ea_dialogueExport}
  \caption{Exporting the metamodel to a file}
  \label{ea:export}
\end{center}
\end{figure}

\item[$\blacktriangleright$] Go back to Eclipse and open \texttt{LeitnersLearningBox.eap}. Right-click on \texttt{MyWorkingSet} node and navigate to ``Import
Model from XMI\ldots''

\item[$\blacktriangleright$] Find the \texttt{.xmi} file you just saved and press \texttt{import}. Press \texttt{OK} in the confirmation dialogue; Your project
browser should now resemble Fig.~\ref{ea:importProBrowser}, with both metamodels in the same working set.

\begin{figure}[htbp]
\begin{center}
  \includegraphics[width=0.4\textwidth]{ea_loadedDictionaryMetamodel}
  \caption{The TGG metamodels successfully included in one project}
  \label{ea:importProBrowser}
\end{center}
\end{figure}

\clearpage

\item[$\blacktriangleright$] Confirm the import by validating\footnote{To review the details of how to use the eMoflon control panel, read Section 2.8 from
Part II} (Fig.~\ref{ea:importValidationWindow}) and exporting the dual-metamodel project to Eclipse, refreshing \texttt{LeitnersLearningBox} to rebuild your workspace. 

\begin{figure}[htbp]
\begin{center}
  \includegraphics[width=\textwidth]{ea_importValidationWindow}
  \caption{No validation errors for \texttt{LeitnersLearningBox}}
  \label{ea:importValidationWindow}
\end{center}
\end{figure}

\item[$\blacktriangleright$] That's it! You now have the second piece of your transformation triple, and are ready to start specifying your first
transformation.

\jumpSingle{TGGSchema}

\end{itemize}
