\newpage
\subsection{Importing and working with multiple EAPs}
\visHeader
\label{sec:multiEAP}

Please note that the following instructions are how to properly export and import Enterprise Architect (EA) files -- this is not an
eMoflon-exclusive feature! Instead of referring you to their instructions, however, we wanted to include them here in full as it is critical that you're able to
import a different EAP into the \emph{same} project in order for a TGG Correspondence to work. You may want to bookmark this section for later reference on your
own projects.

\begin{itemize}

\item[$\blacktriangleright$] In the same \texttt{Part4.zip} folder you extracted this document from, open \texttt{DictionaryLanguageSource.eap} in EA.

\item[$\blacktriangleright$] Expand the root node and confirm its resemblance to Fig.~\ref{fig:dictionaryLangStart}. Feel free to inspect the main
\texttt{DictionaryLanguage} diagram until you're familiar with what you'll be working with. We'll primarily be working with the \texttt{Dictionary} and
\texttt{Entry} classes, where dictionaries are unsorted containers for an unlimited number of entries. Dictionaries can be assigned unique \texttt{title}s,
and each entry will have some sort of \texttt{content} to be learned with a difficulty \texttt{level} of ``beginner,'' ``advanced,'' or ``master.''

\vspace{0.5cm}

\begin{figure}[htbp]
\begin{center}
  \includegraphics[width=0.4\textwidth]{ea_dictLangProBrowser}
  \caption{caption}
  \label{fig:dictionaryLangStart}
\end{center}
\end{figure}

\item[$\blacktriangleright$] While you are able to copy and paste packages between multiple EAPs (i.e., copy \texttt{<<EPack\-age>>DictionaryLanguage} into the
\texttt{MyWorkingSet} root note of \texttt{Leit\-nersLearningBox.eap}), if any of the copied packages have dependencies on other packages, it cannot be done so
easily. All links would be destroyed! If you tried to copy \texttt{DictionaryLanguage}, for example, the receiving project would not be able to establish the
neccesary links to eMoflon's \texttt{Moca Language}.

\clearpage

\item[$\blacktriangleright$] Therefore, to migrate packages, you have to export a \emph{complete} node to an XMI file. Right-click on the
\texttt{DictionaryLanguage} EPackage, then navigate to ``Import/Export / Export Model to XMI\ldots'' (Fig.~\ref{fig:contextExport}). Alternatively, select the
root and press \texttt{Ctrl + Alt + E}.

\begin{figure}[htbp]
\begin{center}
  \includegraphics[width=\textwidth]{ea_contextExport}
  \caption{caption}
  \label{fig:contextExport}
\end{center}
\end{figure}

\item[$\blacktriangleright$] Save the file somewhere easily accessible, such as your desktop, and change the export type to \texttt{XMI 2.1}. Press export,
and close the window once the small green bar appears (Fig.~\ref{fig:export}).

\vspace{0.5cm}

\begin{figure}[htbp]
\begin{center}
  \includegraphics[width=0.9\textwidth]{ea_dialogueExport}
  \caption{caption}
  \label{fig:export}
\end{center}
\end{figure}

\item[$\blacktriangleright$] Now open \texttt{LeitnersLearningBox.eap} from Eclipse and right-click anywhere in the project browser. Navigate to ``Import
Model from XMI\ldots''

\item[$\blacktriangleright$] Find the \texttt{.xmi} file you just saved and press \texttt{import}. Press \texttt{OK} in the confirmation dialogue. Your project
browser now resemble Fig.~\ref{fig:importProBrowser}.

\begin{figure}[htbp]
\begin{center}
  \includegraphics[width=0.4\textwidth]{ea_importedProjectBrowser}
  \caption{caption}
  \label{fig:importProBrowser}
\end{center}
\end{figure}

\item[$\blacktriangleright$] The final step is to validate and export both metamodels to Eclipse! First, use the eMolfon control panel to and under
\texttt{Validate}, press \texttt{All}.\footnote{To review the details and how to use the eMoflon control panel, read Section 2.8 from Part II}

\item[$\blacktriangleright$] Switch back to Eclipse and refresh your package explorer. A new \texttt{DictionaryLanguage} project should have appeared under
\texttt{My Working Set}.

\item[$\blacktriangleright$] Thats everything! You're now ready to start using your source and target metamodels with TGGs. If you've just joined us and are
interested in the eMoflon project structure, or curious as to how Java code is generated from your visual metamodel, we invite you to read Part I, Section 4.1.
Otherwise, continue with this part to start developing your TGG Schema.

\jumpSingle{TGGSchema}

\end{itemize}
