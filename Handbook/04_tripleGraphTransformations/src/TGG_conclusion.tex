\section{Conclusion and next steps}
\genHeader

\vspace{0.5cm}

%summarize the key feature of each part here i.e, schema, making rules, integrator does THIS and can use breakpoints, and we have a protocol. Oh, and
%     synchonization. talk about how we were able to meet each of TGG's three goals: Productivity, Maintanability, Traceability

Fantastic work - you've mastered yet another Part of the eMoflon handbook! You've learned the key points about Triple Graph Grammars and transformations,
including their schema and rules. You have also taken an example out for a test run, \emph{just} to make sure that everything was correct, and learned how to
create breakpoints in case things went wrong.

Try moving on and completing the final piece of this handbook, Part V: Model to Text transformations. There, we shall implement some \emph{unidirectional} model
tranformations. Alternatively, if you don't have much time left, skip ahead to Part VI: miscellaneous to learn some quick tips and tricks on using eMoflon
efficiently, as well as seeing an expanded glossary and list of all eMoflon hotkeys.

For detailed descriptions on the upcoming and previous parts of this handbook, please refer to Part 0, which can be found at \dlPartZero.
