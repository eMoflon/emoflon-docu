\newpage
\section{Specifying TGG rules}
\genHeader

After declaring our correspondence types in the TGG schema, we can now specify a set of \emph{TGG rules} to describe the simultaneous evolution of both source,
correspondence and target models.

A TGG rule is quite similar to an SDM storypattern and is also of the form (\emph{precondition, postcondition}). In other words, we have to state:

\begin{itemize}
  \item What pattern must be matched, i.e, under which conditions can the rule be applied (this is the precondition).
  \item The objects and links to be created when the rule is applied to a match (this is deduced from the postcondition).
\end{itemize}

Note that the rules of a TGG only describe the simultaneous \emph{build-up} of the models and do not delete or modify any existing elements, i.e., TGG rules
are \emph{monotonic}. This might seem surprising at first and you might think this is a terrible restriction. The point is that the TGG should only specify a
consistency relation and not directly the forward and backward transformations, which are derived automatically. It turns out that deletion is not necessary on
this level to do this, but will of course be used at the right places in the generated transformations.

\newpage
\hypertarget{rules vis}{}
\subsection{Visual Rules}
\visHeader

Some sort of content so that we don't start right away with a subsection. In essence, this section wants to create a rule normally, and then DERIVE a second
one. In the second rule, we should do 'create new correspondence type' in the window, to show users they don't HAVE to include it in the schema before using
(declare it on the fly.. not have to switch diagram). maybe in this paragraph we can explain how we'll meet the TGG goals (producitivy, maintability, etc etc))

\subsection{BoxToDictionaryRule}
\begin{enumerate}
\item[$\blacktriangleright$] In EA, open the \texttt{Rules} diagram of your TGG project, automatically generated when you first created it. This diagram must
contain any rules relevant to this project.

\item[$\blacktriangleright$] Create your first rule by either holding \texttt{ctrl} while you click in the diagram, or drag-and-drop the \texttt{Rule} item from
the TGG toolbox to the left of the diagram window (Fig.~\ref{fig:create_tgg_rule}). Now press \texttt{alt + enter} to raise its \texttt{Properties} dialogue.
Update its name to \texttt{BoxToDictionaryRule}.

\vspace{0.5cm}

\begin{figure}[htbp]
\begin{center}
  \includegraphics[width=\textwidth]{ea_TGGNewRule}
  \caption{Creating a TGG rule}
  \label{fig:create_tgg_rule}
\end{center}
\end{figure}

\item[$\blacktriangleright$] Double click the element to open the \texttt{BoxToDictionaryRule} diagram. Drag-and-drop the \texttt{Box EClass} from the project
browser into the diagram, choosing to paste the element as an instance from the drop-down menu.\footnote{If the 'Paste Element' dialogue doesn't appear, hold
\texttt{ctrl} and confirm you haven't autosaved the choice as the default move in the \texttt{options} drop-down menu.} The \texttt{name} and \texttt{binding
operator} should already be set to \texttt{box} and \texttt{create}.

\item[$\blacktriangleright$] Repeat the action to create an instance of \texttt{Dictionary}.

\item[$\blacktriangleright$] Quick-link from \texttt{box} to \texttt{dictionary} and create a TGG Correspondence link. To keep things simple, name it
\texttt{boxToDictionary} and select the correspondence type from the drop-down list, which you declared in the schema.

Believe it or not, our rule \emph{already} creates a \texttt{Box}, \texttt{Dictionary}, and correspondence link between them at the same time, as-is!
Unfortunately, this only creates the objects, and doesn't relate any of there values. Why don't we try to connect the \texttt{name} of the \texttt{box} to the \texttt{title}
of the dictionary? I.e., if you have a \texttt{Kitten} LearningBox, you can transform it into a \texttt{Kitten} Dictionary. Luckily, we can once again use
\emph{attribute constraints}!\footnote{These were first defined in Part III, Section 4}. When used with TGG rules, attribute constraints provide a bidirectional
and high-level solution for attribute manipulation. We're looking for a constraint which ensures that \texttt{box.name} and \texttt{dictionary.title} are
consistent

\item[$\blacktriangleright$] Following the same process as a new \texttt{Rule}, either hold \texttt{ctrl} and click in the diagram
(Fig.~\ref{fig:common_toolbox}), or drag-and-drop a \emph{TGG Constraint} from the \texttt{TGGRuleTolboxPage} to create a constraint.

\begin{figure}[htbp]
\begin{center}
  \includegraphics[width=0.3\textwidth]{ea_createTGGConstraint}
  \caption{Constraint from the Toolbox in EA}
  \label{fig:common_toolbox}
\end{center}
\end{figure}

\item[$\blacktriangleright$] Double click the empty box to open its \texttt{TGGConstraint Dialog}. There's a pre-populated list of available constraints; Choose
\texttt{eq} and double click each of the \texttt{Value} fields to specify the \texttt{a} and \texttt{b} values as depicted in
Fig.~\ref{fig:first_tgg_constraint}. Add the constraint and affirm with \texttt{OK}.

\item[$\blacktriangleright$] Your rule should now resemble Fig.~\ref{fig:tgg_rule_with_constraint}, where the new links represent the dependencies between the
  constrain and objects involved

\newpage

\begin{figure}[htbp]
\begin{center}
  \includegraphics[width=\textwidth]{ea_TGGConstraintDialog}
  \caption{Creating a Constraint in EA}
  \label{fig:first_tgg_constraint}
\end{center}
\end{figure}

\begin{figure}[h!]
\begin{center}
  \includegraphics[width=\textwidth]{ea_TGGconstraintDependency}
  \caption{A TGG Rule with a Constraint}
  \label{fig:tgg_rule_with_constraint}
  \end{center}
\end{figure}

\newpage

Our first TGG Rule is not yet complete -- we still need to create the initial structure of learning box. In contrast to the rather
simple dictionary, where \texttt{Dictionary} is a direct container for \texttt{Entry} objects, we have to create a number of connected \texttt{Partitions} that hold
the \texttt{Cards} in the learning box. 

\item[$\blacktriangleright$] Create three \texttt{Partition} object variables, with appropriate link variables that satisfy the LeitnerBoxRules (the
\texttt{next}, \texttt{previous}, and \texttt{box} references). Your TGG rule should then closely resemble Fig.~\ref{fig:boxtodictionaryrule_complete}.


\begin{figure}[htbp]
\begin{center}
  \includegraphics[width=1.1\textwidth]{ea_TGGCompleteRule}
  \caption{Complete TGG rule diagram for \texttt{BoxToDictionaryRule}}
  \label{fig:boxtodictionaryrule_complete}
\end{center}
\end{figure}

\end{enumerate}

% SECOND RULE BEGINS

If you are in hurry, you can jump ahead and proceed to Section~\ref{sect:TGGs_in_Action}: TGGs in Action and transform a box to a dictionary and vice-versa, but
please be aware that your specified TGG (with just one rule) will only be able to cope with empty boxes and dictionaries. Handling additional elements
(i.e., cards in the learning box and entries in the dictionary) requires a second rule. We intend to create this next.

% --------------- Card To Entry ------------------------------------------------------------------------------------------------------------------------
\newpage
\subsection{CardToEntryRule}

Do you remember our \emph{productivity} goal that we hoped to meet with TGGs? Given one rule, we wanted to be able to derive related rules relatively easily.
Luckily, eMoflon is able to do exactly that! To create the rule to take care of \texttt{card}s and \texttt{entry} objects, we can use a cool derivation
feature.

\begin{enumerate}
  
\item[$\blacktriangleright$] First confirm that your eMoflon control panel window is open. If not, activate it by going to ``Extensions/Add-In Windows.''
  
\item[$\blacktriangleright$] Hold \texttt{ctrl} and select each \texttt{box}, \texttt{boxToDictionary}, \texttt{dictionary} and
\texttt{partition0} in the \texttt{Box\-To\-Dictionary\-Rule} diagram.
  
\item[$\blacktriangleright$] Switch to the \texttt{eMoflon TGG Functions} tab on the control panel then press \texttt{Derive} as depicted in
Fig.~\ref{fig:derive_from_tgg_rule}. In the dialogue that appears, enter \texttt{CardToEntryRule} as the name of the rule, and press \texttt{OK.} The new rule
will automatically display in a the editor window.

\begin{figure}[htbp]
\begin{center}
 \includegraphics[width=\textwidth]{ea_selectPreDerivation}
  \caption{Derive from an existing TGG rule}
  \label{fig:derive_from_tgg_rule}
\end{center}
\end{figure}
\FloatBarrier

\item[$\blacktriangleright$] Add instances of \texttt{Card} and \texttt{Entry} to the rule and required links until the diagram closely resembles
Fig.~\ref{fig:cardtoentry_1}.

\item[$\blacktriangleright$] Also: create NEW correspondence type, \texttt{CardToEntry}. (Explain that even though it wasn't originall on schema, you can do
here!)

  \begin{figure}[htbp]
  \begin{center}
    \includegraphics[width=\textwidth]{ea_cardToEntryRule}
    \caption{\texttt{CardToEntryRule} without attribute manipulation}
    \label{fig:cardtoentry_1}
  \end{center}
  \end{figure}

\end{enumerate}

Now we want to create a series of constraints in order to specify how attributes should be handled. Let's define a syntax for every \texttt{Entry} in
\texttt{Dictionary}. \syntax{<word> : <meaning>}. Therefore, syntax for card.back should be Question : <word>, and card.face should be Answer : <meaning>.
Luckily, we have two predefined attribute constraints, \texttt{addPrefix} and \texttt{concat} to help us.

\begin{enumerate}
  \item[$\blacktriangleright$] addPrefix(``Question '', word, card.face)
  \item[$\blacktriangleright$] addPrefix(``Answer'', answer, card.back)
  \item[$\blacktriangleright$] concat(``:'', word, meaning, entry.content)
\end{enumerate}

Your rule should now resemble Fig.~\ref{fig:cardtoentry_2}.

\begin{figure}[htbp]
\begin{center}
  \includegraphics[width=\textwidth]{ea_completedCardToEntry}
  \caption{Attribute manipulation for \texttt{card} and \texttt{entry}}
  \label{fig:cardtoentry_2}
\end{center}
\end{figure}
\FloatBarrier


% Rewrite to make this a bit more clear..
Finally, we have to specify how the partition (into which the new \texttt{card} is to be placed) must be chosen.
We shall implement the following rule: a card in a partition with index 0/1/2 corresponds to an \texttt{Entry} of level beginner/advanced/master.
This time, we must define a unique attribute constraint to handle this mapping. For now, we are just going to declare and use the attribute constraint, which
will be implemented later in Java.

\begin{enumerate}
\item[$\blacktriangleright$] Add one more constraint to your diagram. When you open the dialogue, don't choose a predefined constraint, but instead click
``Add'' below the dropdown menu. Enter the values given in Fig. \ref{fig:create_new_constraint}.

\vspace{0.5cm}

\begin{figure}[htbp]
\begin{center}
  \includegraphics[width=\textwidth]{ea_uniqueConstraint}
  \caption{Create a user defined constraint.}
  \label{fig:create_new_constraint}
\end{center}
\end{figure}
\FloatBarrier

\item[$\blacktriangleright$] Saving this new constraint, then select it from the drop down menu and enter \texttt{partition0.index} as \texttt{Integer} and
\texttt{entry.level} as the \texttt{String}.
\end{enumerate}

After defining the dependencies of the constraint, your complete TGG rule should resemble Fig.~\ref{fig:cardtoentry_complete}.

\newpage

\begin{figure}[htbp]
\begin{center}
  \includegraphics[width=\textwidth]{ea_cardToEntryComplete}
  \caption{\texttt{CardToEntryRule} with complete attribute manipulation}
  \label{fig:cardtoentry_complete}
\end{center}
\end{figure}




\newpage
\hypertarget{rules tex}{}
\subsection{Tex Rules}
\texHeader

Some sort of content so that we don't start right away with a subsection. It should be two lines long.

\subsection{BoxToDictionaryRule}

\begin{itemize}

\item[$\blacktriangleright$] You may have noticed that a \texttt{Rules} folder was created and included in the TGG package when you first created it. Create
your first one by right-clicking on the folder and navigating to ``New/TGG Rule.'' Name it \texttt{BoxToDictionaryRule}, and confirm the file opens in the
editor window.

\item[$\blacktriangleright$] You'll notice that the rule is clearly separated into its three areas -- \texttt{source}, \texttt{correspondence}, and
\texttt{target}. There is a fourth scope, \texttt{constraints}, is where you can can list CSP constraints which manipulate attributes based on the
transformation direction. 

\item[$\blacktriangleright$] Lets first establish the \texttt{target} and \texttt{source} structures. Given that this is the first rule to be applied in a
transformation, we can assume there is no context to work with, so each of our objects will need to be set to `green' (create). In the \texttt{source} scope,
create a \texttt{box} of type \texttt{Box}. Similarily, in the \texttt{target} scope, create a \texttt{dictionary} of type \texttt{Dictionary}. Your rule
should now resemble Fig.~\ref{fig:textSourceRule}.

\vspace{0.5cm}

\begin{figure}[htbp]
\begin{center}
  \includegraphics[width=0.5\textwidth]{eclipse_boxToDictionary_start}
  \caption{start BoxToDictionary}
  \label{fig:textSourceRule}
\end{center}
\end{figure}

\item[$\blacktriangleright$] Now we can create our first TGG Correspondence link! In the \texttt{correspondence} scope, enter 
\syntax{++ box <- boxToDictionary : BoxToDictionary -> dictionary}
Note that the structure of this statement creates \emph{one} link, named \texttt{boxToDictionary}, of correspondence type \texttt{BoxToDictionary} which was
delcared in the schema.

\end{itemize}

If this rule were to be run at this point, as-is, it would be actually successful by creating a single \texttt{Box} and \texttt{Dictionary}! Besides the
correspondence link however, these items have nothing in common. Let's try connecting the \texttt{name} of \texttt{box} to the \texttt{title} of \texttt{dictionary} with an
\emph{attribute constraint}. In TGG rules, attribute constraints provide a bidirectional and high level solution for attribute manipulations. In addition to the
basic math constraints such as addition (add), subtraction (sub), divide, max, multiply, and smallerOrEqual, we have some preexisting string constraints
we can use in this application. These include stringToNumber, concat, addPrefix, addSuffix, and equals (eq).

\begin{itemize}

\item[$\blacktriangleright$] Therefore, under the \texttt{constraints} scope, write:
\syntax{eq(box.name, dictionary.title)}
Your rule should now resemble Fig.~\ref{fig:ruleBasic}.

\vspace{0.5cm}

\begin{figure}[htbp]
\begin{center}
  \includegraphics[width=0.8\textwidth]{eclipse_boxToDictionary_firstElements}
  \caption{first elements}
  \label{fig:ruleBasic}
\end{center}
\end{figure}

\end{itemize}

Switch back to \texttt{BoxToDictionaryRule}. What's missing from our rule? We have created the primary container structures for the \texttt{target} and
\texttt{source}, but \texttt{cards} cannot be stored directly in \texttt{box}! We therefore need to create some \texttt{partition} objects. 

\begin{itemize}

\item[$\blacktriangleright$] Given that there are three difficulty \texttt{level}s for each dictionary \texttt{entry}, create and complete \texttt{partition0},
\texttt{partition1}, and \texttt{partition2} with the appropriate \texttt{containedPartition}, \texttt{next} and \texttt{previous} link variables so that your
rule matches Fig.~\ref{fig:allReferences}.\footnote{Read the introduction to Part II to review the rules and motivation behind our LeitnersBox}

\end{itemize}

\begin{figure}[htbp]
\begin{center}
  \includegraphics[width=0.8\textwidth]{eclipse_boxToDictionary_complete}
  \caption{First rule complete}
  \label{fig:allReferences}
\end{center}
\end{figure}


Great work! Your first TGG rule is complete! This rule is able to transform a \texttt{box} into a \texttt{dictionary} and vice-versa. Unfortunately, it will
only be able to handle completely \emph{empty} boxes and dictionaries -- you can see that we haven't provided additional handling for \texttt{Card} or
\texttt{Entry} items. If you're in a hurry, feel free to jump ahead to Section 4: TGGs in Action to execute this rule. Otherwise, the next rule we create will
integrate itself with \texttt{BoxToDictionaryRule} to take care of this.


% --------------- Card To Entry ------------------------------------------------------------------------------------------------------------------------
\subsection{CardToEntryRule}

\begin{itemize} 

\item[$\blacktriangleright$] Analogously to how you began the previous rule, return to the TGG schema and create a second \emph{Integration Class} called
\texttt{CardToEntry} with a \texttt{Card} source and \texttt{Entry} target. Your updated file should now resemble Fig.~\ref{fig:updatedSchema}.

\vspace{0.5cm}

\begin{figure}[htbp]
\begin{center}
  \includegraphics[width=0.4\textwidth]{eclipse_updatedSchema}
  \caption{udpated schema}
  \label{fig:updatedSchema}
\end{center}
\end{figure}

\item[$\blacktriangleright$] Right click on the \texttt{Rules} folder again, and create a \texttt{CardToEntryRule}.

\end{itemize}

One of the key differences between this rule and the last is that \texttt{CardToEntryRule} will only be invoked within a certain context i.e.,
this will only be used if a preexisting \texttt{partition} has \texttt{card} elements that need to be transformed into entires in an established
\texttt{dictionary}. In terms of MOSL, this means there will be both `black' and `green' elements.

\begin{itemize}

\item[$\blacktriangleright$] To begin, create three object variables in the \texttt{source} scope: \texttt{box}, \texttt{partition0}, and \texttt{card}. Which
ones are already known from the context? Which element still needs to be made? Your rule should come to resemble Fig.~\ref{fig:c2eRuleSource}.

\begin{figure}[htbp]
\begin{center}
  \includegraphics[width=0.45\textwidth]{eclipse_cardToEntry_sourceOVs}
  \caption{source filled}
  \label{fig:c2eRuleSource}
\end{center}
\end{figure}

\item[$\blacktriangleright$] In the \texttt{target} scope, we will know \texttt{dictionary} from the context, but will still need to create a new entry object
via \texttt{++ entry:Entry}.

\vspace{0.5cm}

\item[$\blacktriangleright$] Now we can complete the \texttt{correspondence}! Our contextual \texttt{box} and \texttt{dictionary} objects can be connected via
the same \texttt{boxToDictionary} link as declared in \texttt{BoxToDictionaryRule}, but a second link needs to be created between \texttt{card} and \texttt{entry}.
Use the correspondence type from the updated schema and write: \syntax{++ card <- cardToEntry : CardToEntry -> entry}

% \begin{figure}[htbp]
% \begin{center}
%   \includegraphics[width=0.8\textwidth]{eclipse_cardToEntry_correspondence}
%   \caption{correspondence}
%   \label{fig:c2etargetCorresp}
% \end{center}
% \end{figure}

\vspace{0.5cm}

\item[$\blacktriangleright$] Finally, let's make sure the transformation is able to access the \texttt{card} and \texttt{entry} attributes. Complete each
of your \texttt{box}, \texttt{partition0}, and \texttt{dictionary} object variable scopes until your rule matches Fig.~\ref{fig:c2eAllReferences}.\footnote{Don't
forget that eMoflon's type completion can help you establish references here; Press \texttt{ctrl + space bar} after writing \texttt{->} for a list of available
link variables from the relevant \texttt{eclass}.}

\newpage

\begin{figure}[htb]
\begin{center}
  \includegraphics[width=0.8\textwidth]{eclipse_cardToEntry_objectVariables}
  \caption{all object variables}
  \label{fig:c2eAllReferences}
\end{center}
\end{figure}

\end{itemize}

Finally, let's establish the necessary \texttt{constraints} which can handle the relevant content attributes of \texttt{card} and \texttt{entry}. We'll need to
first decide on some common variables and syntax between \texttt{card.face}, \texttt{card.back}, and \texttt{entry.content} so that we can combine each side of
a \texttt{card} into one attribute, or split each \texttt{entry} into a question and answer. 

\begin{itemize}

\item[$\blacktriangleright$] Except perhaps on a piece of paper so you can keep track, let's define the syntax for \texttt{entry.content} as
\texttt{<word>:<meaning>}, \texttt{card.back} as \texttt{Question:<word>}, and \texttt{card.face} as \texttt{Answer:<meaning>}. 

\vspace{0.5cm}

\item[$\blacktriangleright$] Now, using the preexisting String attribute constraint types, edit your \texttt{constraint} scope until it resembles
Fig.~\ref{fig:contentConstraints}.

\begin{figure}[htbp]
\begin{center}
  \includegraphics[width=0.8\textwidth]{eclipse_cardToEntry_firstConstraints}
  \caption{preexisting constraints}
  \label{fig:contentConstraints}
\end{center}
\end{figure}

\end{itemize}

\newpage

Let's add \emph{one} more constraint. Given that we have three partitions, and three difficulty levels for each \texttt{Entry}, why don't we have the
transformation assign a level based on whatever partition a \texttt{card} is found in? Hard cards, for example, are more likely to be found in the first
partition (due to being shifted backwards from wrong guesses).  As you can imagine, there is no constraint type currently existing in eMoflon to manage this --
we must create our own!

\begin{itemize}

\item[$\blacktriangleright$] Add the following declaration to the \texttt{constraint} scope: \syntax{indexToLevel[BB,BF,FB](EInt, EString)} We will discuss what
each of the options mean in a moment.

\vspace{0.5cm}

\item[$\blacktriangleright$] You can now invoke your rule with \texttt{indexToLevel(partition0.index, entry.level)} immediately below the declaration. Your
completed \texttt{CardToEntryRule} should now resemble Fig.~\ref{fig:c2eDone}.

\begin{figure}[htbp]
\begin{center}
  \includegraphics[width=0.9\textwidth]{eclipse_cardToEntry_complete}
  \caption{COMPLETED rule}
  \label{fig:c2eDone}
\end{center}
\end{figure}

\vspace{0.5cm}

\item[$\blacktriangleright$] Awesome work! If you haven't already, save the file and confirm the MOSL parser hasn't raised any errors. Press \texttt{``Build
(Without Cleaning)''}, and admire your TGG transformation rules. 

\vspace{0.5cm}

\item[$\blacktriangleright$] To see how \texttt{BoxToDictionaryRule} is implemented in the visual syntax, check out Fig.~\ref{fig:boxtodictionaryrule_complete}
from Section 4.1. The \texttt{CardToEntryRule} is depicted in Fig.~\ref{fig:cardtoentry_complete} in Section 4.2.

\end{itemize}

