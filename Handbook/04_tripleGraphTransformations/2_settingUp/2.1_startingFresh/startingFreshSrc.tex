\subsection{Starting Fresh}
\label{sec:loadSourceMeta}
\begin{itemize}

\item[$\blacktriangleright$] Press the \texttt{new} button on the Eclipse toolbar and navigate to ``Examples/eMoflon Handbook Examples/''
(Fig.~\ref{eclipse:downPartIV}). There are two cheat packages for Part IV; one for our visual syntax, the other for our textual syntax. They each contain the
full \texttt{LeitnersLearningBox} metamodel, as well as each method implemented as an SDM, and an exemplary instance of the metamodel. If you need help
deciding which syntax to use, refer to Part I, Section 1.

\begin{figure}[htbp]
\begin{center}
  \includegraphics[width=0.7\textwidth]{eclipse_part4FreshWizardDownload}
  \caption{Initialize your workspace with your preferred syntax}
  \label{eclipse:downPartIV}
\end{center}
\end{figure}

\vspace{0.5cm}

\item[$\blacktriangleright$] After loading, if your package explorer does not resemble ours in Fig.~\ref{eclipse:workingSets} with at least two distinct nodes,
select the small, downward facing arrow in the corner of the package explorer. Choose ``Top Level Elements/Working Sets.'' To review how these nodes are used
to structure our workspace in Eclipse, check out Part I, Section 4.

% Forced placement so it would co-operate
\begin{figure}[htbp]
	\centering
  \includegraphics[width=0.9\textwidth]{eclipse_workingSets}
	\caption{Setting your Package Explorer}
	\label{eclipse:workingSets}
\end{figure}

\vspace{0.5cm}

\item[$\blacktriangleright$] Fantastic -- you now have the source metamodel for your transformation ready to go!

\end{itemize}