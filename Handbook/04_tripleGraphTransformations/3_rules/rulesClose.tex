\newpage
\hypertarget{rules common}{}
\subsection{RulesClose}
\texHeader

Just like the patterns describing \emph{structural} correspondence,  attribute constraints can be automatically \emph{operationalized} as required for  concrete
transformations (forward, backward). Even more interesting, a set of constraints might have to be ordered a bit differently depending on the direction of the
transformation, and some constraints might have to be checked for already set attributes, while others must set values appropriately to fulfill the constraint.

For built-in or \emph{library} constraints such as \emph{eq}, \emph{addPrefix} and \emph{concat}, you do not need to worry about these details and can just
express what should hold -- everything else is handled automatically.

In many cases, however, a constraint might be very problem-specific, such as our \emph{indexToLevel} constraint, and there might not be any fitting combination
of library constraints to express the consistency condition.

In such a case, the new attribute constraint must be declared before its use.

The list of \emph{adornments} in the declaration specifies the cases for which the constraint can be operationalized. Each adornment consists of a \texttt{B}
for bound or an \texttt{F} for free, for each argument of the constraint. This is much simpler than its sounds so lets take a look at our example:

\begin{description}

\item[BB] means that the \texttt{partition.index} and \texttt{entry.level} can both be \emph{bound}, i.e., already have assigned values.
In this case, the \emph{operation} (the operationalized constraint) must check if the assigned values are correct.

\item[BF] means that \texttt{partition.index} is \emph{bound} and \texttt{entry.level} is \emph{free}, i.e., the operation must determine and assign the correct
value to \texttt{entry.level} using \texttt{partition.index}.

\item[FB] means that \texttt{partition.index} is \emph{free} and \texttt{entry.level} is \emph{bound}, i.e., the operation must determine and assign the correct
value to \texttt{parti\-tion.in\-dex} using \texttt{entry.level}.

\end{description}

Note that we decide not to support \textbf{FF} as we would have to generate a consistent pair of index and level.
Although this is possible and might even make sense for some applications, in our case it does not (the pair is not unique\ldots which pair should we take?).

At compile time, the set of constraints (also called \emph{Constraint Satisfaction Problem} (CSP)) for every TGG rule is ``solved'' for each case (forward,
backward) by operationalizing all constraints and determining a feasible (compatible to the declared adornments of each constraint) sequence in which the
operations can be executed. An exception is thrown if this is not possible.