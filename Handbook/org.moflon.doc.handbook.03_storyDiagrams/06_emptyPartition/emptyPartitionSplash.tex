\newpage
\hypertarget{sec:emptyPartition}{}
\section{Emptying a partition of all cards}
\genHeader

This next SDM should \emph{empty} a partition by removing every card contained within it. Since we can assume that there is more than one card in the
partition,\footnote{If there was only one, we would just invoke \texttt{removeCard}} we obviously need some construct for repeatedly deleting each card in the
partition (\Cref{fig:goal_empty}). 

\begin{figure}[htbp]
	\centering
  \includegraphics[width=0.3\textwidth]{goal_partitionEmpty.pdf}
	\caption{Emptying a partition of every card}
	\label{fig:goal_empty}
\end{figure}
\FloatBarrier

In SDM, this \define{For Each} is accomplished via a \emph{for each} story node. It performs the specified actions for \emph{every} match of its
pattern (i.e., every \texttt{Card} that matches the pattern will be deleted). This however, gives us two interesting points to discuss.
Firstly, how would the pattern be interpreted if the story node were a normal, simple control flow node, not a \emph{for each} node?

The pattern would specify that \emph{a} card should be matched and deleted from the current partition - that's it. The \emph{exact} card is not specified,
meaning that the actual choice of the card is \emph{non-deterministic} (random), and it is only done once. This randomness is a common property of graph pattern
matching, and it's something that takes time getting used to.  In general, there are no guarantees concerning the choice and order of valid matches. The
\emph{for each} construct however, ensures that \emph{all} cards will be matched and deleted.

The second point is determining if we actually need to destroy the link between \texttt{this} and \texttt{card}. Would the pattern be interpreted differently if
 we destroyed \texttt{card} and left the link?

The answer is no, the pattern would yield the same result, regardless of whether or not the link is explicitly destroyed! This is due to
the transformation engine eMoflon uses.\footnote{CodeGen2, a part of the Fujaba toolsuite \url{http://www.fujaba.de/}}\define{Dangling Edges} It ensures that
there are never any \emph{dangling edges} in a model. Since deleting just the \texttt{card} would result in a dangling edge attached to \texttt{this}, that
link is deleted as well. Explicitly destroying the links as well is therefore a matter of taste, but \ldots why not be as explicit as possible?

\newpage
\hypertarget{emptyPartition vis}{}
\subsection{Implementing empty}
\visHeader

\begin{itemize}

\item[$\blacktriangleright$] Open a new activity diagram for \texttt{Partition.empty}. To begin building the \emph{for each} pattern, quick create a new story
node and edit its properties. Name it \texttt{deleteCardsInPartition} and change the \texttt{Type} from \texttt{StoryNode} to \texttt{ForEach}. You'll also want
to create the invoking \texttt{Partition} object, so ensure that too, is selected. As you can see, a \emph{for each} node is represented as a stacked node to
indicate the potential for repetition.

\begin{figure}[htbp]
\begin{center}
  \includegraphics[width=0.9\textwidth]{ea_sdmEmptyNew}
  \caption{Creating a looping story node}  
  \label{fig:sdm_foreach}
\end{center}
\end{figure}

\item[$\blacktriangleright$] Now create the neccessary \texttt{card} variable needed to complete this SDM. Unlike \texttt{removeCard}\footnote{See
Fig.~\ref{fig:sdm_complete_control_flow}} however, the goal of \texttt{emptyCards} is not just to remove the link between the selected partition and card, we
want the matched \texttt{card} to be \emph{completely} deleted. This means in the properties tab, after setting the name and binding state, you'll need to set
the \texttt{Binding Operator} to \texttt{Destroy} (Fig.~\ref{fig:sdm_bindingOperator}).

\item[$\blacktriangleright$] Complete the story pattern as indicated in Fig.~\ref{fig:sdm_end}. Notice that the guard that terminates the looping node has an
\texttt{[end]} edge guard. Indeed, a \emph{for each} story node \emph{must} execute an \texttt{end} activity when all matches in the pattern have been
handled. \texttt{empty} is defined as a \texttt{void} method, so don't worry about setting any return values.

\begin{figure}[htbp]
\begin{center}
  \includegraphics[width=0.65\textwidth]{ea_emptyBindingOperator}
  \caption{Editing \texttt{card} so the variable gets destroyed}  
  \label{fig:sdm_bindingOperator}
\end{center}
\end{figure}

\begin{figure}[htbp]
\begin{center}
  \includegraphics[width=0.8\textwidth]{ea_sdmEmptyComplete}
  \caption{Completed \texttt{empty} story pattern}  
  \label{fig:sdm_end}
\end{center}
\end{figure}
\FloatBarrier

\item[$\blacktriangleright$] Done! You've now learnt that in order to create a repeating action, all you need to do is change a standard standard story node
into a \texttt{for each} node, and include \emph{edge guards}. Inspect Fig.~\ref{fig:emptyControlFlow} and Fig.~\ref{fig:emptyPattern} to see how this is
implemented in the textual specifications.

\fancyfoot[R]{ $\triangleright$ \hyperlink{sec:invertCard}{Next}} 

\end{itemize}


