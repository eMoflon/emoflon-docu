\newpage
\chapter{Growing the box}
\genHeader

Ok, back to business. In this SDM, we shall explicitly specify how our learning box is to be built up. We create a specific pattern that will append new
partition elements to the end of a \texttt{Box} that follow our established movement rules (\Cref{fig:membox_depiction}). This means the new partition will
become the \texttt{next} reference of the current last partition, and its \texttt{previous} reference must be connected to the first partition in the box
(\Cref{fig:goal_grow}).

\begin{figure}[htbp]
 	\centering
  	\includegraphics[width=0.7\textwidth]{../../org.moflon.doc.handbook.03_storyDiagrams/08_growBox/growBoxNACGoal.pdf}
	\caption{Growing a box by inserting a new partition}
	\label{fig:goal_grow}
\end{figure}
\FloatBarrier

SDMs provide a declarative means of identifying specific partitions via \emph{Negative Application Conditions}, simply referred to as
\mbox{NAC}s.\footnote{Pronounced $\backslash 'nak \backslash$}\define{NAC} \mbox{NAC}s express structures that are forbidden to exist before applying a
transformation rule. In this SDM, the \mbox{NAC} will be an object variable that must not be assigned a value during pattern matching. In the theory of
algebraic graph transformations \cite{EEPT06}, \mbox{NACs} can be arbitrarily complex graphs that are much more general and powerful than what we currently
support in our implementation,\footnote{To be precise, in CodeGen2 from Fujaba} namely only single negative elements (object or link variables).

As depicted in \Cref{fig:goal_grow}, to create an appropriate \mbox{NAC} that constrains possible matches, we'll need to check to see if the currently
matched pattern can be extended to include the negative elements. Suppose the current potential last partition has a \texttt{nextPartition}. This means it
is \emph{not} the absolute last partition, and so the match becomes invalid. We only want to insert a new partition when the \texttt{nextPartition}
of the current potential last partition is null. Similarly, if the current potential first partition has a \texttt{previousPartition}, the match is invalid. The
complete match is therefore made unique through NACs and thus becomes \emph{deterministic} by construction. In other words, if you \emph{grow} the box with this method, there
will always be exactly one first and one last partition of the box.

Of course, to complete this method we still need to determine the size of the new partition. Since the size must be calculated depending on the
rest of the partitions currently in the box (partitions usually get bigger) we'll need to call a helper method, \texttt{determineNextSize} via a
\emph{MethodCallExpression}\define{MethodCallExpression}. As the name suggests, it is designed to access any method defined in \emph{any} class in the current
project.

Due to the algorithmic and non-structural nature of \texttt{determineNextSize}, it will be easier to implement this method via a Java \emph{injection}, rather
than an SDM. We've already declared this method in our metamodel, so its signature will be available for editing in \texttt{BoxImpl.java}.

\begin{stepbystep}

\item Open ``gen/LearningBoxLanguage.impl/BoxImpl.java.'' Scroll to the method declaration, and replace the contents with the code in
\Cref{code:determineNextSize_impl}. Remember not to remove the first comment, which is necessary to indicate that the code is handwritten and needs to be
extracted automatically as an injection. Please do not copy and paste the following code -- the copying process from your pdf viewer to the Eclipse IDE
will likely add invisible characters to the code that eMoflon is unable to handle.

\begin{figure}[htbp]
        \centering
        \begin{lstlisting}[language=Java, keywordstyle={\bfseries\color{purple}}, backgroundcolor=\color{white}]
    public int determineNextSize() {
    	// [user code injected with eMoflon]
        return getContainedPartition().size()*10;
    }
        \end{lstlisting}
        \caption{Implementation of \texttt{removeCard}}
        \label{code:determineNextSize_impl}
\end{figure}


\item Save the file, then right-click on it, either in the package explorer or in the editor window, and choose ``eMoflon/
Create/Update Injection for class'' from the context menu. 

\item Confirm the update in the new \texttt{BoxImpl.inject} file's partial class. \texttt{determineNextSize} is now ready to be used by
your metamodel!

\end{stepbystep}

\newpage
\hypertarget{growBox vis}{}
\subsection{Implementing grow}
\visHeader

\begin{itemize}
 
\item[$\blacktriangleright$] Start by creating the simple story pattern depicted in Fig.~\ref{fig:sdm_grow_1}. This matches the box, \texttt{this}, with
\emph{any} two partitions.\footnote{Remember, the \emph{pattern matcher} is randomized!}

\vspace{0.5cm}

\begin{figure}[htbp]
\begin{center}
  \includegraphics[width=\textwidth]{ea_elementsGrowBox}
  \caption{Context elements for SDM}  
  \label{fig:sdm_grow_1}
\end{center}
\end{figure}

\item[$\blacktriangleright$] To create an appropriate \mbox{NAC} to constrain the possible matches for \texttt{lastPartitionInBox},  create a new
\texttt{Partition} object variable \texttt{nextPartition} and set\define{Binding\\Semantics}its \emph{binding semantics} to \texttt{negative}
(Fig.~\ref{fig:sdm_grow_2}). The object variable should be visualised as being cancelled or struck out. 
 
\begin{figure}[htbp]
\begin{center}
  \includegraphics[width=0.7\textwidth]{ea_newNac}
  \caption{Adding a negative element}  
  \label{fig:sdm_grow_2}
\end{center}
\end{figure}
 
\item[$\blacktriangleright$] Now, quick link \texttt{nextPartition} to \texttt{lastPartitionInBox}. Be sure choose the link type carefully! The
\texttt{nextPartition} should play the role of \texttt{next} with respect to \texttt{lastPartitionInBox}. This combination (the negative binding and reference)
tells the pattern matcher that if the (assumed) last partition has an element connected to its \texttt{next} reference, the current match is invalid.

\item[$\blacktriangleright$] Great work -- the first NAC is complete! In a similar fashion, create the remaining NAC for \texttt{firstPartitionInBox}. Name the
negative element \texttt{previousPartition}, and again, be sure to double-check the navigation.

\item[$\blacktriangleright$] Finally, complete the pattern rule so that it closely resembles Fig.~\ref{fig:sdm_grow_3}. 

\begin{figure}[htbp]
\begin{center}
  \includegraphics[width=\textwidth]{ea_NACComplete} 
  \caption{Determining the first and last partitions with NACs}  
  \label{fig:sdm_grow_3}
\end{center}
\end{figure}
 
\item[$\blacktriangleright$] Notice how the created partition \texttt{newPartition} is `hung' into the box. It becomes the next partition of the current
\emph{last} partition, and its previous partition is automatically set to the first partition in the box (as dictated by the rules set in
Fig.~\ref{fig:membox_depiction}). In other words, the new partition is appended onto the current order of partitions.

\item[$\blacktriangleright$] In order to complete \texttt{grow}, we need to set the size of the \texttt{newPartition}. Given that the new size is calculated
via the helper function \texttt{det\-er\-mine\-Next\-Size}, we need to invoke a \emph{MethodCallExpression}.\define{MethodCallExpression}A MethodCallExpression
is another specialized EA mechanism. As the name suggests, its designed to access any method defined in any class of the current project. Go ahead and invoke
the corresponding dialogue to activate the assignment (\texttt{:=}) operator, and match your values to Fig.~\ref{fig:sdm_grow_4}
 
\begin{figure}[htbp]
\begin{center}
  \includegraphics[width=0.6\textwidth]{ea_attributeMethodCall.png}
  \caption{Invoking a method via a \texttt{MethodCallExpression}}  
  \label{fig:sdm_grow_4} 
\end{center}
\end{figure}

Since \texttt{determineNextSize} doesn't require any parameters, you can ignore the \texttt{Parameter values} field this time. For future reference however,
parameters can be specified by choosing the appropriate parameter declaration between guillemets (e.g. \texttt{<Box box>}) found in the drop-down menu and typing in the
value (this is basically a literal expression). Don't forget to press the \texttt{Save} button for every parameter, then \texttt{Add} + \texttt{OK} to confirm
and close the dialogue.

\vspace{0.5cm}

\item[$\blacktriangleright$] If you've done everything right, your SDM should now closely resemble Fig.~\ref{fig:growComplete}. As usual, try to export,
generate code, and inspect the method implementation.

\vspace{0.5cm}

\item[$\blacktriangleright$]  That's it - your \texttt{grow} SDM is complete! This was probably the most challenging SDM to build, so give yourself a solid 
pat on the back. If you found it easy, well then \ldots I don't think I'm doing my job correctly. To see how this is done in the texual syntax, review
Fig.~\ref{fig:patternComplete}.\footnote{We do recommend reading the instructions for this one, since NACs can be tricky}

\vspace{0.5cm}

\begin{figure}[htbp]
\begin{center}
  \includegraphics[width=\textwidth]{ea_growFinal}
  \caption{Complete SDM for \texttt{Box::grow}}  
  \label{fig:growComplete}
\end{center}
\end{figure}
\FloatBarrier

\jumpSingle{sec:stringRep}

\end{itemize}

