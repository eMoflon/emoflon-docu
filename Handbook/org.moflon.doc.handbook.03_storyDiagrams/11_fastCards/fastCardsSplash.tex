\newpage
\section{Fast cards!}
\genHeader
\hypertarget{sec:fastCard}{}

Congratulations, you're almost there! This is the last SDM needed before your Leitner's learning box is fully functional.

For very simple cards (i.e., words in a different language that are quite similar), it might be a bit annoying to have to answer these cards again and
again in successive partitions. Such \emph{fast} cards should somehow be marked as such and handled differently. If a fast card is correctly answered once,
it should be immediately moved to the final partition in the box. This way, the card is practiced once, and only tested once more before finally being ejected from the
box.

It makes sense for a \texttt{FastCard} to inherit from \texttt{Card}, so we'll extend the current object in our metamodel by a new \texttt{EClass} for fast cards, 
depicted below with a marker to show it behaves differently (Fig.~\ref{fig:goal_fastCard}).

\begin{figure}[htbp]
	\centering
    \includegraphics[width=0.7\textwidth]{goal_fastCard}
	\caption{Checking a fast card against a guess}
	\label{fig:goal_fastCard}
\end{figure}
\FloatBarrier

In addition to creating a new \texttt{EClass}, we also need to extend the
existing \texttt{check} method to check for this special card type once a guess is determined to be correct. Now \texttt{check} needs to decide, based on the dynamic type\footnote{In a statically typed language like Java, every
object has a static type (determined at compile time) and a dynamic type (that can only be determined at runtime).} of \texttt{card}, if it needs to handle this special fast card. This
can be expressed in SDMs with a \emph{BindingExpression} (or just \emph{Binding})\define{Binding}. A binding can be specified for a \emph{bound} object variable
and is the final case in which an object variable can be marked as being bound.

To refresh your memory, we have already learnt that a bound object variable is either (1) assigned to \texttt{this}, (2) a parameter of the method, or (3) a
value determined in a preceding activity node. Bindings represent a fourth possibility of giving a manual binding for an object variable.

Finally, this new pattern faces a similar challenge as \texttt{grow}. A \texttt{FastCard} can't simply progress to the \texttt{next} partition. It must skip
ahead to the absolute last partition in the box. This means yet another NAC is required to determine the last partition in a \texttt{Box}.
  
\newpage
\subsection{Implementing FastCards}
\visHeader
\hypertarget{fastCard vis}{}

\begin{itemize}

\item[$\blacktriangleright$] To introduce fast cards into your learning box, return to the metamodel diagram and create a new \texttt{eclass},
\texttt{FastCard}. Quick link to \texttt{Card} and choose \texttt{Create Inheritance} from the context menu. We only want to check the dynamic type of a
tested card at runtime, which means we don't need to override anything. Therefore, when the \texttt{Overrides \& Implementations} dialogue appears, make sure
nothing is selected (Fig.~\ref{ea:dialogue_override}). Your metamodel should then resemble Fig.~\ref{ea:metamodel_FastCard}.

\vspace{0.5cm}

% NOTE : NOT ACCURATE: MODIFIED TO REDUCE WHITE SPACE (original screenshot in visFCImages)
\begin{figure}[htp]
\begin{center}
  \includegraphics[width=0.6\textwidth]{ea_overrideDialogueModified}
  \caption{Selecting operations to override}  
  \label{ea:dialogue_override}
\end{center}
\end{figure}

\begin{figure}[htp]
\begin{center}
  \includegraphics[width=0.9\textwidth]{ea_EClassFastCard}
  \caption{Fast cards are a special kind of card}  
  \label{ea:metamodel_FastCard}
\end{center}
\end{figure}

\vspace{0.5cm}

\item[$\blacktriangleright$] Now return to the \texttt{check} SDM (in \texttt{Partition}) and extend the control flow as depicted in
Fig.~\ref{ea:extendCheck}.

 \vspace{0.5cm}
 
\begin{figure}[htbp]
\begin{center}
  \includegraphics[width=\textwidth]{ea_extendCheck}
  \caption{Extend check to handle fast cards.}  
  \label{ea:extendCheck}
\end{center}
\end{figure}
 

\item[$\blacktriangleright$] As you can see, you have created two new story nodes, \texttt{isFastCard}, and \texttt{promoteFastCard}.
 
\item[$\blacktriangleright$] Next, in order to complete the newest conditional, create a bound \texttt{FastCard} object variable, named \texttt{fastcard} in
\texttt{isFastCard} (Fig.~\ref{ea:fastCardBinding}).
 
\item[$\blacktriangleright$] To check the dynamic type, we'll need to create a binding of \texttt{card} (of type \texttt{Card}) to \texttt{fastcard} (of
type \texttt{FastCard}), so edit the \texttt{Binding} tab in the \texttt{Object Variable Properties} dialogue (Fig.~\ref{ea:fastCardBinding}). Please note that
this tab will not allow any changes unless the \texttt{bound} option in \texttt{Object Properties} is selected. As you can see, this set up configures the
pattern matcher to check for types, rather than \texttt{parameters} and \texttt{attributes} as we've previously encountered.

\vspace{0.5cm}

\begin{figure}[htbp]
\begin{center}
  \includegraphics[width=0.9\textwidth]{ea_fastCardBinding}
  \caption{Create a binding for \texttt{fastcard}}  
  \label{ea:fastCardBinding}
\end{center}
\end{figure}

\clearpage

In our case, we could use a \emph{ParameterExpression} or an \emph{ObjectVariableExpression}\define{ObjectVariable\-Expression} as \texttt{card} is indeed a
parameter \emph{and} has already been used in \texttt{checkIfGuessIsCorrect}. We haven't tried the latter yet, so let's use \emph{ObjectVariableExpression}.

\item[$\blacktriangleright$] Update the \texttt{fastcard} binding by switching the expression to 
\texttt{Object\-Vari\-able\-Ex\-pres\-sion}, with \texttt{card} as the target. Note that a binding could also use a \emph{MethodCallExpression} to invoke a
method whose return value would be the bound value. This is very useful as it allows invoking helper methods directly in patterns.

\item[$\blacktriangleright$] To finalize the SDM, extract the \texttt{promoteFastCard} story pattern and build the pattern according to
Fig.~\ref{ea:promoteFastCardPattern}. Compare this pattern to Figs.~\ref{ea:sdm_check_complete_activity_node} and \ref{ea:sdm_check_complete_penalize}, the
original promotion and penalizing card movements. As you can see, they're very similar, except \texttt{fastCard} is transferred from the current partition
(\texttt{this}) immediately to the last partition in \texttt{box}, identified as having no \texttt{nextPartition} with an appropriate NAC.

\begin{figure}[htbp]
\begin{center}
  \includegraphics[width=\textwidth]{ea_promoteFastCardPattern}
  \caption{Story pattern for handling fast cards.}  
  \label{ea:promoteFastCardPattern}
\end{center}
\end{figure}

\item[$\blacktriangleright$] Inspect Fig.~\ref{eclipse:promoFastCardFinal} to see how this is done in the textual syntax.

\item[$\blacktriangleright$] You have now implemented every method using SDMs -- fantastic work! Save, validate, and build your metamodel to see some new code.
Inspect the implementation for \texttt{check}.  Can you find the generated type casts for \texttt{fastcard}?

\item[$\blacktriangleright$] At this point, we encourage you to read each of the textual SDM instructions to try and understand the full scope of eMoflon's
features (which start on page~\hyperlink{page.9}{9}) but you are of course, free to carry on.

\jumpSingle{subsec:fastGUI}

\end{itemize}


\newpage
\hypertarget{subsec:fastGUI}{}
\subsection{FastCards in the GUI}
\genHeader

We hope you haven't forgotten about the GUI! Now that we have a new \texttt{card} type, let's try editing our \texttt{box} instance so we can experiment with
them in our application!

\begin{itemize}
  
\item[$\blacktriangleright$] To review the details of creating instances, read Part II, Section 3 but for now, open \texttt{Box.xmi} and right-click on your
first partition and create a new \texttt{FastCard} child element. Fill in some \texttt{back} and \texttt{face} values to test (Fig.).

\item[$\blacktriangleright$] Save your file, then run the \texttt{check} method on your new card. Experiment with making the wrong and correct guess, and notice
its behavior. If you've done everything right until this point, it should act differently than its standard \texttt{card} counterparts.

\end{itemize}

