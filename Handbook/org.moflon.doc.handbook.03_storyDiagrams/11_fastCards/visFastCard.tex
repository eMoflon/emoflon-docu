\newpage
\subsection{Implementing FastCards}
\genHeader
\hypertarget{fastCard vis}{}

\begin{stepbystep}

\item To introduce fast cards into your learning box, return to the metamodel diagram and create a new \texttt{EClass},
\texttt{FastCard}. Quick link to \texttt{Card} and choose \texttt{Create Inheritance} from the context menu. We only want to check the dynamic type of a
tested card at runtime, which means we don't need to override anything. Therefore, when the \texttt{Overrides \& Implementations} dialogue appears, make sure
nothing is selected (\Cref{ea:dialogue_override}). Your metamodel should then resemble \Cref{ea:metamodel_FastCard}.

\vspace{0.5cm}

% NOTE : NOT ACCURATE: MODIFIED TO REDUCE WHITE SPACE (original screenshot in visFCImages)
\begin{figure}[htp]
\begin{center}
  \includegraphics[width=0.6\textwidth]{../../org.moflon.doc.handbook.03_storyDiagrams/11_fastCards/visFCImages/ea_overrideDialogueModified}
  \caption{Selecting operations to override}  
  \label{ea:dialogue_override}
\end{center}
\end{figure}

\begin{figure}[htp]
\begin{center}
  \includegraphics[width=0.9\textwidth]{../../org.moflon.doc.handbook.03_storyDiagrams/11_fastCards/visFCImages/ea_EClassFastCard}
  \caption{Fast cards are a special kind of card}  
  \label{ea:metamodel_FastCard}
\end{center}
\end{figure}

\vspace{0.5cm}

\item Now return to the \texttt{check} SDM (in \texttt{Partition}) and extend the control flow as depicted in
\Cref{ea:extendCheck}.

 \vspace{0.5cm}
 
\begin{figure}[htbp]
\begin{center}
  \includegraphics[width=\textwidth]{../../org.moflon.doc.handbook.03_storyDiagrams/11_fastCards/visFCImages/ea_extendCheck}
  \caption{Extend check to handle fast cards.}  
  \label{ea:extendCheck}
\end{center}
\end{figure}
 

\item As you can see, you have created two new story nodes, \texttt{isFastCard}, and \texttt{promoteFastCard}.
 
\item Next, in order to complete the newest conditional, create a bound \texttt{FastCard} object variable, named \texttt{fastcard} in
\texttt{isFastCard} (\Cref{ea:fastCardBinding}).
 
\item To check the dynamic type, we'll need to create a binding of \texttt{card} (of type \texttt{Card}) to \texttt{fastcard} (of
type \texttt{FastCard}), so edit the \texttt{Binding} tab in the \texttt{Object Variable Properties} dialogue (\Cref{ea:fastCardBinding}). Please note that
this tab will not allow any changes unless the \texttt{bound} option in \texttt{Object Properties} is selected. As you can see, this set up configures the
pattern matcher to check for types, rather than \texttt{parameters} and \texttt{attributes} as we've previously encountered.

\vspace{0.5cm}

\begin{figure}[htbp]
\begin{center}
  \includegraphics[width=0.9\textwidth]{../../org.moflon.doc.handbook.03_storyDiagrams/11_fastCards/visFCImages/ea_fastCardBinding}
  \caption{Create a binding for \texttt{fastcard}}  
  \label{ea:fastCardBinding}
\end{center}
\end{figure}

\clearpage

In our case, we could use a \emph{ParameterExpression} or an \emph{ObjectVariableExpression}\define{ObjectVariable\-Expression} as \texttt{card} is indeed a
parameter \emph{and} has already been used in \texttt{checkIfGuessIsCorrect}. We haven't tried the latter yet, so let's use \emph{ObjectVariableExpression}.

\item Update the \texttt{fastcard} binding by switching the expression to 
\texttt{Object\-Vari\-able\-Ex\-pres\-sion}, with \texttt{card} as the target. Note that a binding could also use a \emph{MethodCallExpression} to invoke a
method whose return value would be the bound value. This is very useful as it allows invoking helper methods directly in patterns.

\item To finalize the SDM, (i) extract the \texttt{promoteFastCard} story pattern and build the pattern according to
\Cref{ea:promoteFastCardPattern} and (ii) create the parallel \texttt{Success} and \texttt{Failure} edges from this activity to the stop node returning 
\texttt{true} for the same reason as in \texttt{check} earlier.
Compare the pattern in \Cref{ea:promoteFastCardPattern} to \Cref{ea:sdm_check_complete_activity_node}, the original promotion and penalizing card movements.
As you can see, they're very similar, except \texttt{fastCard} is transferred from the current partition
(\texttt{this}) immediately to the last partition in \texttt{box}, identified as having no next \texttt{Partition} with an appropriate NAC.
Note that a second NAC is used to handle the case where \texttt{this} would be the next \texttt{Partition}, which is also not what we want.

\begin{figure}[htbp] 
\begin{center}
  \includegraphics[width=\textwidth]{../../org.moflon.doc.handbook.03_storyDiagrams/11_fastCards/visFCImages/ea_promoteFastCardPattern}
  \caption{Story pattern for handling fast cards.}  
  \label{ea:promoteFastCardPattern}
\end{center}
\end{figure}

\item You have now implemented every method using SDMs -- fantastic work! Save, validate, and build your metamodel to see some new code.
Inspect the implementation for \texttt{check}.  Can you find the generated type casts for \texttt{fastcard}?

\end{stepbystep}
