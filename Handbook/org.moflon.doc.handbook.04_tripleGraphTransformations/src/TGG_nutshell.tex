\chapter{Triple Graph Grammars in a nutshell}
\label{sec:nutshell}
\genHeader

Triple Graph Grammars~\cite{tgg:schuerr_94,sk2008,Klar2010} are a declarative, rule-based technique used to specify the simultaneous evolution of three connected graphs. 
Basically, a TGG is just a bunch of rules. 
Each rule is quite similar to a \emph{story pattern} and describes how a graph structure is to be built-up
via a precondition (LHS) and postcondition (RHS). 
The key difference is that a TGG rule describes how a \emph{graph triple}\define{Graph Triples} evolves, where triples consist of a source, correspondence, and target component. 
This means that executing a sequence of TGG rules will result in source and target graphs connected via nodes in a third (common) correspondence graph.

\vspace{0.25cm}

Please note that the names ``source'' and ``target'' are arbitrarily chosen and do not imply a certain transformation direction. 
Naming the graphs ``left'' and ``right'', or ``foo'' and ``bar'' would also be fine. 
The important thing to remember is that TGGs are \emph{symmetric} in nature.

\vspace{0.25cm}

So far, so good! Except you may be now be asking yourself the following question: ``What on earth does all this have to do with bidirectional model transformation?'' 
There are two main ideas behind TGGs:

\begin{description}

\item[(1) A TGG defines a consistency relation:]% 
Given a TGG (a set of rul\-es), you can inspect a source graph $S$ and a target graph $T$, and say if they are \emph{consistent} with respect to the TGG. 
How? Simply check if a triple ($S\leftarrow C\rightarrow T$) can be created using the rules of the TGG!

\vspace{0.25cm}

If such a triple can be created, then the graphs are consistent, denoted by: $S \Leftrightarrow_{TGG} T$. This consistency relation can be used to check if a given bidirectional transformation (i.e., a pair ($f,b$) of a unidirectional forward transformation $f$ and backward transformation $b$) is correct. 
In summary, a TGG can be viewed as a specification of how the transformations \emph{should behave} ($S \Leftrightarrow_{TGG} f(S)$ and $b(T) \Leftrightarrow_{TGG} T$).
	
\item[(2) The consistency relation can be operationalized:]% 
This is the surprising (and extremely cool) part of TGGs:
correct forward \emph{and} backward transformations (i.e., $f$ and $b$) can be derived automatically from every TGG~\cite{Giese2010,Hermann2011a}! 

\vspace{0.25cm}

In other words, the description of the simultaneous evolution of the source, correspondence, and target graphs is \emph{sufficient} to derive a forward and a backward transformation. 
As these derived rules explicitly state step-by-step how to perform forward and backward transformations, they are called \emph{operational} rules \define{Operationalization}, as opposed to the original TGG \emph{declarative} rules specified by the user. 
This derivation process is therefore also referred to as the \emph{operationalization} of a TGG.
	
\end{description}

Before getting our hands dirty with a concrete example, here are a few extra points for the interested reader:  

\begin{itemize}

\item Many more operational rules can be automatically derived from the $S \Leftrightarrow_{TGG} T$ consistency relation including inverse rules to \emph{undo} a step in a forward/backward transformation~\cite{LAVS_ICGT_2012},\footnote{Note that the TGGs are symmetric and forward or backward can be interchanged freely. 
As it is cumbersome to always write forward/backward, we shall now simply say forward.} and rules that check the consistency of an existing graph triple.

\item You might be wondering why we need the correspondence graph. 
The first reason is that the correspondence graph can be viewed as a set of explicit traceability links, which are often nice to have in any transformation. 
With these you can, e.g., immediately see which elements are related after a forward transformation. 
There's no guessing, no heuristics, and no interpretation or ambiguity.

\vspace{0.25cm}

The second reason is more subtle, and difficult to explain without a concrete TGG, but we'll do our best and come back to this at the end. 
The key idea is that the forward transformation is very often actually \emph{not} injective and cannot be inverted! 
A function can only be inverted if it is \emph{bijective}, meaning it is both \emph{injective} and \emph{surjective}. 
So how can we derive the backward transformation?

\vspace{0.25cm}

eMoflon sort of ``cheats'' when executing the forward transformation and, if a choice had to be made, remembers which target element was chosen. 
In this way, eMoflon \emph{bidirectionalize}s the transformation on-the-fly with correspondence links in the correspondence graph. 
The best part is that if the correspondence graph is somehow lost, there's no reason to worry because the \emph{same} TGG specification that was used to derive your forward transformation
can also be used to reconstruct a possible correspondence model between two existing source and target models.\footnote{We refer to this type of operational rule as \emph{link creation}. 
This turns out to be harder than it appears and support for link creation in eMoflon is currently still work in progress.}

\end{itemize}
This was a lot of information to absorb all at once, so it might make sense to re-read this section after working through the example.
In any case, enough theory! 
Grab your computer (if you're not hugging it already) and get ready to churn out some wicked TGGs!
