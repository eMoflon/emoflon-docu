\newpage
\hypertarget{sec:Rules}{}
\section{Specifying TGG rules}
\genHeader

With our correspondence type defined in the TGG schema, we can now specify a set of \emph{TGG rules} to describe a language of graph
triples.

As discussed in Section~\ref{sec:nutshell}, a TGG rule is quite similar to a SDM story pattern, following a \emph{precondition, postcondition}
format. This means we'll need to state:

\begin{itemize}

\item What must be matched (i.e., under which conditions can a rule be applied; `black' elements)

\item What is to be created when the rule is applied (i.e., which objects and links must exist upon exit; `green' elements)

\end{itemize}

\vspace{0.5cm}

Note that the rules of a TGG only describe the simultaneous \emph{build-up} of source, correspondence, and target models.
Unlike SDMs, they do not delete or modify any existing elements. 
In other words, TGG rules are \emph{monotonic}.\define{Monotonic}
This might seem surprising at first, and you might even think
this is a terrible restriction. 
The intention is that a TGG should only specify a consistency relation, and \emph{not} the forward and backward transformations
directly, which are derived automatically. 
In the end, modifications are not necessary on this level but can, of course, be induced in certain
operationalizations of the TGG.

Let's quickly think about what rules we need in order to successfully transform a learning box into a dictionary. 
We need to first take care of the \texttt{box}
and \texttt{dictionary} structures, where \texttt{box} will need at least one \texttt{partition} to manipulate its \texttt{card}s. If more than one is created, those partitions will need to have appropriate \texttt{next} and \texttt{previous} links. 
Conversely, given that a \texttt{dictionary} is unsorted, there are no
counterparts for partitions. 
A second rule will be needed to transform \texttt{card}s into \texttt{entries}.
More precisely, a one-to-one correspondence must be
established (i.e., one \texttt{card} implies one \texttt{entry}), with suitable
concatenation or splitting of the contents (based on the transformation direction), and some mechanism to assign difficulty levels to each \texttt{entry} or initial position of each \texttt{card}.

\jumpDual{rules vis}{rules tex}

\newpage
\hypertarget{rules vis}{}
\subsection{Visual Rules}
\visHeader

Some sort of content so that we don't start right away with a subsection. In essence, this section wants to create a rule normally, and then DERIVE a second
one. In the second rule, we should do 'create new correspondence type' in the window, to show users they don't HAVE to include it in the schema before using
(declare it on the fly.. not have to switch diagram). maybe in this paragraph we can explain how we'll meet the TGG goals (producitivy, maintability, etc etc))

\subsection{BoxToDictionaryRule}
\begin{enumerate}
\item[$\blacktriangleright$] In EA, open the \texttt{Rules} diagram of your TGG project, automatically generated when you first created it. This diagram must
contain any rules relevant to this project.

\item[$\blacktriangleright$] Create your first rule by either holding \texttt{ctrl} while you click in the diagram, or drag-and-drop the \texttt{Rule} item from
the TGG toolbox to the left of the diagram window (Fig.~\ref{fig:create_tgg_rule}). Now press \texttt{alt + enter} to raise its \texttt{Properties} dialogue.
Update its name to \texttt{BoxToDictionaryRule}.

\vspace{0.5cm}

\begin{figure}[htbp]
\begin{center}
  \includegraphics[width=\textwidth]{ea_TGGNewRule}
  \caption{Creating a TGG rule}
  \label{fig:create_tgg_rule}
\end{center}
\end{figure}

\item[$\blacktriangleright$] Double click the element to open the \texttt{BoxToDictionaryRule} diagram. Drag-and-drop the \texttt{Box EClass} from the project
browser into the diagram, choosing to paste the element as an instance from the drop-down menu.\footnote{If the 'Paste Element' dialogue doesn't appear, hold
\texttt{ctrl} and confirm you haven't autosaved the choice as the default move in the \texttt{options} drop-down menu.} The \texttt{name} and \texttt{binding
operator} should already be set to \texttt{box} and \texttt{create}.

\item[$\blacktriangleright$] Repeat the action to create an instance of \texttt{Dictionary}.

\item[$\blacktriangleright$] Quick-link from \texttt{box} to \texttt{dictionary} and create a TGG Correspondence link. To keep things simple, name it
\texttt{boxToDictionary} and select the correspondence type from the drop-down list, which you declared in the schema.

Believe it or not, our rule \emph{already} creates a \texttt{Box}, \texttt{Dictionary}, and correspondence link between them at the same time, as-is!
Unfortunately, this only creates the objects, and doesn't relate any of there values. Why don't we try to connect the \texttt{name} of the \texttt{box} to the \texttt{title}
of the dictionary? I.e., if you have a \texttt{Kitten} LearningBox, you can transform it into a \texttt{Kitten} Dictionary. Luckily, we can once again use
\emph{attribute constraints}!\footnote{These were first defined in Part III, Section 4}. When used with TGG rules, attribute constraints provide a bidirectional
and high-level solution for attribute manipulation. We're looking for a constraint which ensures that \texttt{box.name} and \texttt{dictionary.title} are
consistent

\item[$\blacktriangleright$] Following the same process as a new \texttt{Rule}, either hold \texttt{ctrl} and click in the diagram
(Fig.~\ref{fig:common_toolbox}), or drag-and-drop a \emph{TGG Constraint} from the \texttt{TGGRuleTolboxPage} to create a constraint.

\begin{figure}[htbp]
\begin{center}
  \includegraphics[width=0.3\textwidth]{ea_createTGGConstraint}
  \caption{Constraint from the Toolbox in EA}
  \label{fig:common_toolbox}
\end{center}
\end{figure}

\item[$\blacktriangleright$] Double click the empty box to open its \texttt{TGGConstraint Dialog}. There's a pre-populated list of available constraints; Choose
\texttt{eq} and double click each of the \texttt{Value} fields to specify the \texttt{a} and \texttt{b} values as depicted in
Fig.~\ref{fig:first_tgg_constraint}. Add the constraint and affirm with \texttt{OK}.

\item[$\blacktriangleright$] Your rule should now resemble Fig.~\ref{fig:tgg_rule_with_constraint}, where the new links represent the dependencies between the
  constrain and objects involved

\newpage

\begin{figure}[htbp]
\begin{center}
  \includegraphics[width=\textwidth]{ea_TGGConstraintDialog}
  \caption{Creating a Constraint in EA}
  \label{fig:first_tgg_constraint}
\end{center}
\end{figure}

\begin{figure}[h!]
\begin{center}
  \includegraphics[width=\textwidth]{ea_TGGconstraintDependency}
  \caption{A TGG Rule with a Constraint}
  \label{fig:tgg_rule_with_constraint}
  \end{center}
\end{figure}

\newpage

Our first TGG Rule is not yet complete -- we still need to create the initial structure of learning box. In contrast to the rather
simple dictionary, where \texttt{Dictionary} is a direct container for \texttt{Entry} objects, we have to create a number of connected \texttt{Partitions} that hold
the \texttt{Cards} in the learning box. 

\item[$\blacktriangleright$] Create three \texttt{Partition} object variables, with appropriate link variables that satisfy the LeitnerBoxRules (the
\texttt{next}, \texttt{previous}, and \texttt{box} references). Your TGG rule should then closely resemble Fig.~\ref{fig:boxtodictionaryrule_complete}.


\begin{figure}[htbp]
\begin{center}
  \includegraphics[width=1.1\textwidth]{ea_TGGCompleteRule}
  \caption{Complete TGG rule diagram for \texttt{BoxToDictionaryRule}}
  \label{fig:boxtodictionaryrule_complete}
\end{center}
\end{figure}

\end{enumerate}

% SECOND RULE BEGINS

If you are in hurry, you can jump ahead and proceed to Section~\ref{sect:TGGs_in_Action}: TGGs in Action and transform a box to a dictionary and vice-versa, but
please be aware that your specified TGG (with just one rule) will only be able to cope with empty boxes and dictionaries. Handling additional elements
(i.e., cards in the learning box and entries in the dictionary) requires a second rule. We intend to create this next.

% --------------- Card To Entry ------------------------------------------------------------------------------------------------------------------------
\newpage
\subsection{CardToEntryRule}

Do you remember our \emph{productivity} goal that we hoped to meet with TGGs? Given one rule, we wanted to be able to derive related rules relatively easily.
Luckily, eMoflon is able to do exactly that! To create the rule to take care of \texttt{card}s and \texttt{entry} objects, we can use a cool derivation
feature.

\begin{enumerate}
  
\item[$\blacktriangleright$] First confirm that your eMoflon control panel window is open. If not, activate it by going to ``Extensions/Add-In Windows.''
  
\item[$\blacktriangleright$] Hold \texttt{ctrl} and select each \texttt{box}, \texttt{boxToDictionary}, \texttt{dictionary} and
\texttt{partition0} in the \texttt{Box\-To\-Dictionary\-Rule} diagram.
  
\item[$\blacktriangleright$] Switch to the \texttt{eMoflon TGG Functions} tab on the control panel then press \texttt{Derive} as depicted in
Fig.~\ref{fig:derive_from_tgg_rule}. In the dialogue that appears, enter \texttt{CardToEntryRule} as the name of the rule, and press \texttt{OK.} The new rule
will automatically display in a the editor window.

\begin{figure}[htbp]
\begin{center}
 \includegraphics[width=\textwidth]{ea_selectPreDerivation}
  \caption{Derive from an existing TGG rule}
  \label{fig:derive_from_tgg_rule}
\end{center}
\end{figure}
\FloatBarrier

\item[$\blacktriangleright$] Add instances of \texttt{Card} and \texttt{Entry} to the rule and required links until the diagram closely resembles
Fig.~\ref{fig:cardtoentry_1}.

\item[$\blacktriangleright$] Also: create NEW correspondence type, \texttt{CardToEntry}. (Explain that even though it wasn't originall on schema, you can do
here!)

  \begin{figure}[htbp]
  \begin{center}
    \includegraphics[width=\textwidth]{ea_cardToEntryRule}
    \caption{\texttt{CardToEntryRule} without attribute manipulation}
    \label{fig:cardtoentry_1}
  \end{center}
  \end{figure}

\end{enumerate}

Now we want to create a series of constraints in order to specify how attributes should be handled. Let's define a syntax for every \texttt{Entry} in
\texttt{Dictionary}. \syntax{<word> : <meaning>}. Therefore, syntax for card.back should be Question : <word>, and card.face should be Answer : <meaning>.
Luckily, we have two predefined attribute constraints, \texttt{addPrefix} and \texttt{concat} to help us.

\begin{enumerate}
  \item[$\blacktriangleright$] addPrefix(``Question '', word, card.face)
  \item[$\blacktriangleright$] addPrefix(``Answer'', answer, card.back)
  \item[$\blacktriangleright$] concat(``:'', word, meaning, entry.content)
\end{enumerate}

Your rule should now resemble Fig.~\ref{fig:cardtoentry_2}.

\begin{figure}[htbp]
\begin{center}
  \includegraphics[width=\textwidth]{ea_completedCardToEntry}
  \caption{Attribute manipulation for \texttt{card} and \texttt{entry}}
  \label{fig:cardtoentry_2}
\end{center}
\end{figure}
\FloatBarrier


% Rewrite to make this a bit more clear..
Finally, we have to specify how the partition (into which the new \texttt{card} is to be placed) must be chosen.
We shall implement the following rule: a card in a partition with index 0/1/2 corresponds to an \texttt{Entry} of level beginner/advanced/master.
This time, we must define a unique attribute constraint to handle this mapping. For now, we are just going to declare and use the attribute constraint, which
will be implemented later in Java.

\begin{enumerate}
\item[$\blacktriangleright$] Add one more constraint to your diagram. When you open the dialogue, don't choose a predefined constraint, but instead click
``Add'' below the dropdown menu. Enter the values given in Fig. \ref{fig:create_new_constraint}.

\vspace{0.5cm}

\begin{figure}[htbp]
\begin{center}
  \includegraphics[width=\textwidth]{ea_uniqueConstraint}
  \caption{Create a user defined constraint.}
  \label{fig:create_new_constraint}
\end{center}
\end{figure}
\FloatBarrier

\item[$\blacktriangleright$] Saving this new constraint, then select it from the drop down menu and enter \texttt{partition0.index} as \texttt{Integer} and
\texttt{entry.level} as the \texttt{String}.
\end{enumerate}

After defining the dependencies of the constraint, your complete TGG rule should resemble Fig.~\ref{fig:cardtoentry_complete}.

\newpage

\begin{figure}[htbp]
\begin{center}
  \includegraphics[width=\textwidth]{ea_cardToEntryComplete}
  \caption{\texttt{CardToEntryRule} with complete attribute manipulation}
  \label{fig:cardtoentry_complete}
\end{center}
\end{figure}




\newpage
\hypertarget{rules tex}{}
\subsection{Tex Rules}
\texHeader

Some sort of content so that we don't start right away with a subsection. It should be two lines long.

\subsection{BoxToDictionaryRule}

\begin{itemize}

\item[$\blacktriangleright$] You may have noticed that a \texttt{Rules} folder was created and included in the TGG package when you first created it. Create
your first one by right-clicking on the folder and navigating to ``New/TGG Rule.'' Name it \texttt{BoxToDictionaryRule}, and confirm the file opens in the
editor window.

\item[$\blacktriangleright$] You'll notice that the rule is clearly separated into its three areas -- \texttt{source}, \texttt{correspondence}, and
\texttt{target}. There is a fourth scope, \texttt{constraints}, is where you can can list CSP constraints which manipulate attributes based on the
transformation direction. 

\item[$\blacktriangleright$] Lets first establish the \texttt{target} and \texttt{source} structures. Given that this is the first rule to be applied in a
transformation, we can assume there is no context to work with, so each of our objects will need to be set to `green' (create). In the \texttt{source} scope,
create a \texttt{box} of type \texttt{Box}. Similarily, in the \texttt{target} scope, create a \texttt{dictionary} of type \texttt{Dictionary}. Your rule
should now resemble Fig.~\ref{fig:textSourceRule}.

\vspace{0.5cm}

\begin{figure}[htbp]
\begin{center}
  \includegraphics[width=0.5\textwidth]{eclipse_boxToDictionary_start}
  \caption{start BoxToDictionary}
  \label{fig:textSourceRule}
\end{center}
\end{figure}

\item[$\blacktriangleright$] Now we can create our first TGG Correspondence link! In the \texttt{correspondence} scope, enter 
\syntax{++ box <- boxToDictionary : BoxToDictionary -> dictionary}
Note that the structure of this statement creates \emph{one} link, named \texttt{boxToDictionary}, of correspondence type \texttt{BoxToDictionary} which was
delcared in the schema.

\end{itemize}

If this rule were to be run at this point, as-is, it would be actually successful by creating a single \texttt{Box} and \texttt{Dictionary}! Besides the
correspondence link however, these items have nothing in common. Let's try connecting the \texttt{name} of \texttt{box} to the \texttt{title} of \texttt{dictionary} with an
\emph{attribute constraint}. In TGG rules, attribute constraints provide a bidirectional and high level solution for attribute manipulations. In addition to the
basic math constraints such as addition (add), subtraction (sub), divide, max, multiply, and smallerOrEqual, we have some preexisting string constraints
we can use in this application. These include stringToNumber, concat, addPrefix, addSuffix, and equals (eq).

\begin{itemize}

\item[$\blacktriangleright$] Therefore, under the \texttt{constraints} scope, write:
\syntax{eq(box.name, dictionary.title)}
Your rule should now resemble Fig.~\ref{fig:ruleBasic}.

\vspace{0.5cm}

\begin{figure}[htbp]
\begin{center}
  \includegraphics[width=0.8\textwidth]{eclipse_boxToDictionary_firstElements}
  \caption{first elements}
  \label{fig:ruleBasic}
\end{center}
\end{figure}

\end{itemize}

Switch back to \texttt{BoxToDictionaryRule}. What's missing from our rule? We have created the primary container structures for the \texttt{target} and
\texttt{source}, but \texttt{cards} cannot be stored directly in \texttt{box}! We therefore need to create some \texttt{partition} objects. 

\begin{itemize}

\item[$\blacktriangleright$] Given that there are three difficulty \texttt{level}s for each dictionary \texttt{entry}, create and complete \texttt{partition0},
\texttt{partition1}, and \texttt{partition2} with the appropriate \texttt{containedPartition}, \texttt{next} and \texttt{previous} link variables so that your
rule matches Fig.~\ref{fig:allReferences}.\footnote{Read the introduction to Part II to review the rules and motivation behind our LeitnersBox}

\end{itemize}

\begin{figure}[htbp]
\begin{center}
  \includegraphics[width=0.8\textwidth]{eclipse_boxToDictionary_complete}
  \caption{First rule complete}
  \label{fig:allReferences}
\end{center}
\end{figure}


Great work! Your first TGG rule is complete! This rule is able to transform a \texttt{box} into a \texttt{dictionary} and vice-versa. Unfortunately, it will
only be able to handle completely \emph{empty} boxes and dictionaries -- you can see that we haven't provided additional handling for \texttt{Card} or
\texttt{Entry} items. If you're in a hurry, feel free to jump ahead to Section 4: TGGs in Action to execute this rule. Otherwise, the next rule we create will
integrate itself with \texttt{BoxToDictionaryRule} to take care of this.


% --------------- Card To Entry ------------------------------------------------------------------------------------------------------------------------
\subsection{CardToEntryRule}

\begin{itemize} 

\item[$\blacktriangleright$] Analogously to how you began the previous rule, return to the TGG schema and create a second \emph{Integration Class} called
\texttt{CardToEntry} with a \texttt{Card} source and \texttt{Entry} target. Your updated file should now resemble Fig.~\ref{fig:updatedSchema}.

\vspace{0.5cm}

\begin{figure}[htbp]
\begin{center}
  \includegraphics[width=0.4\textwidth]{eclipse_updatedSchema}
  \caption{udpated schema}
  \label{fig:updatedSchema}
\end{center}
\end{figure}

\item[$\blacktriangleright$] Right click on the \texttt{Rules} folder again, and create a \texttt{CardToEntryRule}.

\end{itemize}

One of the key differences between this rule and the last is that \texttt{CardToEntryRule} will only be invoked within a certain context i.e.,
this will only be used if a preexisting \texttt{partition} has \texttt{card} elements that need to be transformed into entires in an established
\texttt{dictionary}. In terms of MOSL, this means there will be both `black' and `green' elements.

\begin{itemize}

\item[$\blacktriangleright$] To begin, create three object variables in the \texttt{source} scope: \texttt{box}, \texttt{partition0}, and \texttt{card}. Which
ones are already known from the context? Which element still needs to be made? Your rule should come to resemble Fig.~\ref{fig:c2eRuleSource}.

\begin{figure}[htbp]
\begin{center}
  \includegraphics[width=0.45\textwidth]{eclipse_cardToEntry_sourceOVs}
  \caption{source filled}
  \label{fig:c2eRuleSource}
\end{center}
\end{figure}

\item[$\blacktriangleright$] In the \texttt{target} scope, we will know \texttt{dictionary} from the context, but will still need to create a new entry object
via \texttt{++ entry:Entry}.

\vspace{0.5cm}

\item[$\blacktriangleright$] Now we can complete the \texttt{correspondence}! Our contextual \texttt{box} and \texttt{dictionary} objects can be connected via
the same \texttt{boxToDictionary} link as declared in \texttt{BoxToDictionaryRule}, but a second link needs to be created between \texttt{card} and \texttt{entry}.
Use the correspondence type from the updated schema and write: \syntax{++ card <- cardToEntry : CardToEntry -> entry}

% \begin{figure}[htbp]
% \begin{center}
%   \includegraphics[width=0.8\textwidth]{eclipse_cardToEntry_correspondence}
%   \caption{correspondence}
%   \label{fig:c2etargetCorresp}
% \end{center}
% \end{figure}

\vspace{0.5cm}

\item[$\blacktriangleright$] Finally, let's make sure the transformation is able to access the \texttt{card} and \texttt{entry} attributes. Complete each
of your \texttt{box}, \texttt{partition0}, and \texttt{dictionary} object variable scopes until your rule matches Fig.~\ref{fig:c2eAllReferences}.\footnote{Don't
forget that eMoflon's type completion can help you establish references here; Press \texttt{ctrl + space bar} after writing \texttt{->} for a list of available
link variables from the relevant \texttt{eclass}.}

\newpage

\begin{figure}[htb]
\begin{center}
  \includegraphics[width=0.8\textwidth]{eclipse_cardToEntry_objectVariables}
  \caption{all object variables}
  \label{fig:c2eAllReferences}
\end{center}
\end{figure}

\end{itemize}

Finally, let's establish the necessary \texttt{constraints} which can handle the relevant content attributes of \texttt{card} and \texttt{entry}. We'll need to
first decide on some common variables and syntax between \texttt{card.face}, \texttt{card.back}, and \texttt{entry.content} so that we can combine each side of
a \texttt{card} into one attribute, or split each \texttt{entry} into a question and answer. 

\begin{itemize}

\item[$\blacktriangleright$] Except perhaps on a piece of paper so you can keep track, let's define the syntax for \texttt{entry.content} as
\texttt{<word>:<meaning>}, \texttt{card.back} as \texttt{Question:<word>}, and \texttt{card.face} as \texttt{Answer:<meaning>}. 

\vspace{0.5cm}

\item[$\blacktriangleright$] Now, using the preexisting String attribute constraint types, edit your \texttt{constraint} scope until it resembles
Fig.~\ref{fig:contentConstraints}.

\begin{figure}[htbp]
\begin{center}
  \includegraphics[width=0.8\textwidth]{eclipse_cardToEntry_firstConstraints}
  \caption{preexisting constraints}
  \label{fig:contentConstraints}
\end{center}
\end{figure}

\end{itemize}

\newpage

Let's add \emph{one} more constraint. Given that we have three partitions, and three difficulty levels for each \texttt{Entry}, why don't we have the
transformation assign a level based on whatever partition a \texttt{card} is found in? Hard cards, for example, are more likely to be found in the first
partition (due to being shifted backwards from wrong guesses).  As you can imagine, there is no constraint type currently existing in eMoflon to manage this --
we must create our own!

\begin{itemize}

\item[$\blacktriangleright$] Add the following declaration to the \texttt{constraint} scope: \syntax{indexToLevel[BB,BF,FB](EInt, EString)} We will discuss what
each of the options mean in a moment.

\vspace{0.5cm}

\item[$\blacktriangleright$] You can now invoke your rule with \texttt{indexToLevel(partition0.index, entry.level)} immediately below the declaration. Your
completed \texttt{CardToEntryRule} should now resemble Fig.~\ref{fig:c2eDone}.

\begin{figure}[htbp]
\begin{center}
  \includegraphics[width=0.9\textwidth]{eclipse_cardToEntry_complete}
  \caption{COMPLETED rule}
  \label{fig:c2eDone}
\end{center}
\end{figure}

\vspace{0.5cm}

\item[$\blacktriangleright$] Awesome work! If you haven't already, save the file and confirm the MOSL parser hasn't raised any errors. Press \texttt{``Build
(Without Cleaning)''}, and admire your TGG transformation rules. 

\vspace{0.5cm}

\item[$\blacktriangleright$] To see how \texttt{BoxToDictionaryRule} is implemented in the visual syntax, check out Fig.~\ref{fig:boxtodictionaryrule_complete}
from Section 4.1. The \texttt{CardToEntryRule} is depicted in Fig.~\ref{fig:cardtoentry_complete} in Section 4.2.

\end{itemize}


\hypertarget{subsec:IndexToLevel}{}
\subsection{Implementing IndexToLevel}
\genHeader

If everything has been done correctly up to this point, your project should save and build (hit the hammer symbol in the eMolfon task bar).
The generated code will have some compilation errors (Step 1 in Fig.~\ref{eclipse:tggGenerated}) as Eclipse does not know where to access the generated code for the imported source and target ecore files (these could also be supplied from jars or installed plugins).
In our case the generated code is in the respective source and target projects so let's communicate this to Eclipse.

\begin{enumerate}

\item[$\blacktriangleright$] Open the \texttt{MANIFEST.MF} file (Step 2 in Fig.~\ref{eclipse:tggGenerated}) and choose the \texttt{De\-pen\-den\-cies} tab (Step 3).

\item[$\blacktriangleright$] Choose both source and target projects (Step 4) as dependencies and click \texttt{OK}.
All compilations errors should now be resolved.
\end{enumerate}

\begin{figure}[htb]
\begin{center}
  \includegraphics[width=0.8\textwidth]{eclipse_generatedTGG}
  \caption{Add source and target projects as dependencies}
  \label{eclipse:tggGenerated}
\end{center}
\end{figure}

Our TGG still isn't yet complete. 
While we've declared and actually used our custom \texttt{indexTolevel} attribute condition, we haven't actually implemented it yet. 
Let's quickly review the purpose of attribute conditions.

Just like patterns describing \emph{structural} correspondence, \emph{attribute conditions} can be automatically \emph{operationalized} as required, e.g., for a forward transformations. 
Even more interesting, a set of attribute conditions might have to be ordered in a specific way depending on the direction of the transformation.
Enforcing the conditions might involve checking existing attribute values, or setting these values appropriately.

For built-in \emph{library} attribute conditions such as \emph{eq}, \emph{addPrefix} and \emph{concat}, you do not need to worry about these details and can just focus
on expressing what should hold. 
Everything else is handled automatically.

In some cases however, a required attribute condition might be problem-specific, such as our \emph{indexToLevel}. 
There might not be any fitting combination of library attribute conditions to express the consistency condition, so a new attribute condition type must be declared and implemented.

There is a list of \emph{adornments} in the declaration which specify the cases for which the attribute condition can be operationalized. 
Each adornment consists of a \texttt{B} (bound) or \texttt{F} (free) variable setting for each argument of the attribute condition. 
This might sound a bit complex, but it's really quite simple, especially in the context of our example:

\begin{description}

\item[BB] indicates that the \texttt{partition.index} and \texttt{entry.level} are both \emph{bound}, i.e., they already have assigned values.
In this case, the \emph{operation} must check if the assigned values are valid and correct.

\item[BF] indicates that \texttt{partition.index} is \emph{bound} and \texttt{entry.level} is \emph{free}, i.e., the operation must determine and assign the correct value to \texttt{entry.level} using \texttt{partition.index}.

\item[FB] would indicate that \texttt{partition.index} is \emph{free} and \texttt{entry.level} is \emph{bound}, i.e., the operation must determine and assign the correct value to \texttt{parti\-tion.in\-dex} using \texttt{entry.level}.

\item[FF] would indicate that both \texttt{partition.index} and \texttt{entry.level} are \emph{free} and we have to somehow generate consistent values out of thin air.

\end{description}

As \texttt{partition} is a context element in the rule (the partition is always bound in whatever direction), \textbf{FF} and \textbf{FB} are irrelevant cases and we do not need to declare or implement what they mean.
For the record, note that adornments can be declared as either \texttt{\#gen} or \texttt{\#sync}.
The reason is that it might make sense to restrict some adornments (typically \textbf{FF} cases) to only when generating models.
Using \textbf{FF} cases for synchronisation might possibly makes sense, but most of the time it would be weird to generate random values during a forward or backward synchronisation.  

At compile time, the set of attribute conditions for every TGG rule is ``solved'' for each case by
operationalizing all constraints and determining a feasible sequence in which the operations can be executed, compatible to the declared adornments of each attribute condition. 
If the set of attribute conditions cannot be solved, an exception is thrown at compile time.

Now that we have a better understanding behind the construction of attribute conditions, let's implement \texttt{indexToLevel}.

\begin{itemize}
\item[$\blacktriangleright$] Locate and open \texttt{IndexToLevel.java} under ``src/csp.constraints'' in \texttt{LearningBoxToDictionaryIntegration}.

\item[$\blacktriangleright$] As you can see, some code has been generated in order to handle the current unimplemented state of \texttt{IndexToLevel}. 
Use the code depicted in Fig.~\ref{code:indexToLevel} to replace this default implementation.\footnote{Depending of course on your pdf viewer, copy and pasting this code should work.}

\begin{figure}[htb]
\begin{verbatim}
package csp.constraints;

import java.util.Arrays;
import java.util.List;
import org.moflon.tgg.language.csp.Variable;
import org.moflon.tgg.language.csp.impl.TGGConstraintImpl;

public class IndexToLevel extends TGGConstraintImpl {

    private static List<String> levels = 
      Arrays.asList(new String[] {"master","advanced","beginner"});

    public void solve(Variable var_0, Variable var_1) {
        int index = ((Integer) var_0.getValue()).intValue();
        int normalisedIndex = Math.min(Math.max(0, index), 2);
        String bindingStates = getBindingStates(var_0, var_1);
        
        switch (bindingStates) {
        case "BB":
            String level = (String) var_1.getValue();
            setSatisfied(levels.get(normalisedIndex).equals(level));
            break;
        case "BF":
            var_1.bindToValue(levels.get(normalisedIndex));
            setSatisfied(true);
            break;
        }}}
\end{verbatim}
  \caption{Implementation of our custom \texttt{IndexToLevel} constraint}
  \label{code:indexToLevel}
\end{figure}

\end{itemize}

To briefly explain, the \texttt{levels} list contains difficulty level at positions 0, 1, or 2 in the list, which correspond to our three partitions in the learning box. 
You'll notice that instead of setting ``master'' to 2, it has rather been set to match the first 0th partition. 
Unlike an \texttt{entry} in \texttt{dictionary}, the position of each \texttt{card} in \texttt{box} is \emph{not} based on difficulty, but simply how it has been moved as a result of the user's correct and incorrect guesses. 
Easy cards are more likely to be in the final partition (due to moving through the box quickly) while challenging cards are most likely to have been returned to (and currently to be at) the starting position, i.e., the 0th partition.

In the \texttt{solve} method, the index of the matched partition in the rule is first of all normalised (negative values do not make sense, and we handle all partitions  after partition 2 in the same way).
A switch statement is then used, based on whichever adornment is currently the case, to enforce or check the condition. 

For \texttt{BB} we check if the normalised index of the partition corresponds to the difficulty level of the card.
For \texttt{BF}, the normalised index is used to set the appropriate difficulty level of the card.

%%% Local Variables: 
%%% mode: latex
%%% TeX-master: "../src/TGG_mainFile"
%%% End: 

