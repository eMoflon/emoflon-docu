\newpage
\section{Reviewing eMoflon's expressions}
\genHeader

As you've discovered while making SDMs, eMoflon employs a simple context-sensitive expression language for specifying  values. We have intentionally avoided
creating a full-blown sub-language, and limit expressions to a few simple types. The philosophy here is to keep things simple and concentrate on what SDMs are good for -- expressing structural
changes. Our approach is to provide a clear and type-safe interface to a general purpose language (Java) and support a simple \emph{fallback} via calls to
injected methods as soon as things get too low-level and difficult to express structurally as a pattern.

The alternative approach to eMoflon would be to support arbitrary expressions, for example, in a script language like JavaScript or in an appropriate
DSL\footnote{A DSL is a Domain Specific Language: a language designed for a specific task which is usually simpler than a general purpose language like Java and
more suitable for the exact task.} designed for this purpose. 

We've encountered several different expression types throughout our SDMs so far, and all of them can be used for binding expressions. Since each syntax has
used at least three of these once, let's consider what each type would mean:

\begin{description}
  
  \item[LiteralExpression:]~\\ 
  As usual this can be anything and is literally copied with a surrounding typecast 
  into the generated code.  Using \emph{LiteralExpression}s too often is usually a sign 
  for not thinking in a pattern oriented manner and is considered a \emph{bad smell}.
  
  \vspace{0.5cm}
  
  \item[MethodCallExpression:]~\\ 
  This would allow invoking a method and binding its return value to the object variable.  
  This is how non-primitive return values of methods can be used safely in SDMs.
  
  \vspace{0.5cm}
  
  \item[ParameterExpression:]~\\ 
  This could be used to bind the object variable to a parameter of the method.  
  If the object variable is of a different type than the parameter (i.e., a subtype), 
  this represents basically a successful typecast if the pattern matches.
  
  \newpage
  
  \item[ObjectVariableExpression:]~\\ 
  This can be used to refer to other object variables in preceding story nodes.  
  Just like \emph{ParameterExpression}s, this represents a simple typecast if the 
  types of the \texttt{target} and the object variable with the binding are different.

\end{description}

For those using the textual syntax, we have provided a \hyperlink{quickRef}{quick-reference guide} at the end of this part listing the syntax of each of these
expressions. We have also included the structure of several other commands used in the example so far. Part VI: Miscellaneous, however contains a complete
detailed reference of all MOSL commands.
