\newpage
\hypertarget{emptyPartition tex}{}
\subsection{Implementing empty}
\texHeader

\begin{itemize}
 
\item[$\blacktriangleright$] To initialize your new control flow, you can once again take advantage of eMolfon's type completion here. Inside the
\texttt{empty} declaration, press  \texttt{ctrl + space} and select \texttt{forEach} from the menu (Fig.~\ref{fig:typeCompletion}).

\vspace{1cm}

\begin{figure}[htpb]
\begin{center}
  \includegraphics[width=0.7\textwidth]{eclipse_emptyTypeCompletion}
  \caption{eMoflon's auto completion}
  \label{fig:typeCompletion}
\end{center}
\end{figure}

\vspace{1cm}

\item[$\blacktriangleright$] Create a single pattern, \texttt{deleteCardsInPartition}. Remove the suggested second pattern as you only need to complete the
deletion -- no extra steps required!

\item[$\blacktriangleright$] Your activity should now resemble Fig.~\ref{fig:emptyControlFlow}.

\clearpage

\begin{figure}[htpb]
\begin{center}
  \includegraphics[width=0.5\textwidth]{eclipse_emptyControlFlow}
  \caption{Control flow for \texttt{partition.empty()S}}
  \label{fig:emptyControlFlow}
\end{center}
\end{figure}

\item[$\blacktriangleright$] While similar to \texttt{removeCard}, this new pattern goes one step further by requesting a full destruction of card, instead of
just detaching the link. This means, in addition to destroying the link in between the partition and card, we need a destructive object variable.

\item[$\blacktriangleright$] Create a \texttt{@this}, variable, and set its rule as \texttt{-~- -> card:card}. Then create \texttt{card}, prefacing its name
with an \texttt{-~-} operator.

\vspace{0.5cm}

\item[$\blacktriangleright$] Your pattern should now resemble Fig.~\ref{fig:emptyPattern}.

\vspace{0.5cm}

\begin{figure}[htpb]
\begin{center}
  \includegraphics[width=0.6\textwidth]{eclipse_emptyPattern}
  \caption{Destroying an entire card}
  \label{fig:emptyPattern}
\end{center}
\end{figure}

\item[$\blacktriangleright$] That's it! Look at you go, you're just speeding through these SDMs now! To see how \texttt{empty} is done in the visual syntax,
review Fig.~\ref{fig:sdm_end} from the previous section.

\end{itemize}
