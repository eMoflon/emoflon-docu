\newpage
\subsection{Implementing FastCards}
\texHeader
\hypertarget{fastCard tex}{}

\begin{itemize}
  
\item[$\blacktriangleright$] Slight problem. We haven't defined a \texttt{FastCard} before. Create a new eclass under ``LearningBoxLanguage" named
\texttt{FastCard} which extends \texttt{Card}. It doesn't need any new attributes, so leave its declaration empty. It should resemble Fig.~\ref{fig:fastClass}.

\vspace{0.5cm}

\begin{figure}[htp]
\begin{center}
  \includegraphics[width=0.45\textwidth]{eclipse_fastCardClass}
  \caption{A new Data Type}
  \label{fig:fastClass}
\end{center}
\end{figure}

\item[$\blacktriangleright$] Open \texttt{Partition.eclass} once again, and find \texttt{check(card, guess)}. Edit the activity by adding a second
\texttt{if/else} construct. Call the new assertion pattern \texttt{isFastCard}, and the action pattern \texttt{promoteFastCard}. Move the original
\texttt{[promoteCard]} pattern into the \texttt{else} statement. Your workspace should then resemble Fig.~\ref{fig:isFastCard}.

\vspace{0.5cm}

\begin{figure}[htp]
\begin{center}
  \includegraphics[width=0.7\textwidth]{eclipse_isFastCardFlow}
  \caption{Checking for \texttt{FastCard}}
  \label{fig:isFastCard}
\end{center}
\end{figure}

\item[$\blacktriangleright$] \texttt{isFastCard} is a simple, one line statement pattern. You simply need to create an assignment constraint to check a bounded
object, of type \texttt{FastCard}, against the type of \texttt{card} that was passed in through the parameter. Remember, to access parameter values, preface
the name with a `\$' symbol. Your workspace should now resemble Fig.~\ref{fig:isFastCardPattern}.

\begin{figure}[htp]
\begin{center}
  \includegraphics[width=0.5\textwidth]{eclipse_isFastCardPattern}
  \caption{A \texttt{FastCard} attribute constraint}
  \label{fig:isFastCardPattern}
\end{center}
\end{figure}

\item[$\blacktriangleright$] To establish \texttt{promoteFastCard}, first create the four main object variables - \texttt{@fastCard}, \texttt{@this},
\texttt{lastPartition}, and \texttt{box}. Immediately under \texttt{lastPartition}, also create a \texttt{next} NAC.
Your workspace should then resemble Fig.~\ref{fig:objVarFastCard}.

\vspace{0.5cm}

\begin{figure}[htp]
\begin{center}
  \includegraphics[width=0.6\textwidth]{eclipse_promoteFastCardObjVars}
  \caption{Object variables for \texttt{promoteFastCard}}
  \label{fig:objVarFastCard}
\end{center}
\end{figure}

\item[$\blacktriangleright$] When creating the necessary references, remember - this is the pattern that will be invoked when the fast card status has
already been confirmed! This means that, in the appropriate rules, you'll want to:
(1) link the partition to the current box.
(2) remove \texttt{fastCard} from its current partition, and insert it into \texttt{lastPartition}
(3) confirm \texttt{lastPartition} is in a box, then check to see if it has a \texttt{next} value.\footnote{If you need help remembering how NACs work, review
section 7}

\vspace{0.5cm}

\item[$\blacktriangleright$] Your final workspace should resemble Fig.~\ref{fig:promoFastCardFinal}. As you can see, this pattern is remarkably similar to the
original movement patterns, \texttt{promoteCard} and \texttt{penalizeCard} (Fig.~\ref{fig:completedPatterns}). This of course makes sense - it only
needed a brief NAC to check the \texttt{card} type.

\newpage

\vspace*{1cm}

\begin{figure}[htp]
\begin{center}
  \includegraphics[width=0.6\textwidth]{eclipse_promoFastCardFinal}
  \caption{The completed fast card promotion pattern}
  \label{fig:promoFastCardFinal}
\end{center}
\end{figure}

\vspace{0.5cm}

\item[$\blacktriangleright$] You have now completed \emph{every} method signature from your abstract syntax using SDMs - fantastic work! Build your project to
confirm there aren't any errors, and review Fig.~\ref{fig:promoteFastCardPattern} to see how \texttt{FastCard}s are implemented in the visual syntax.

\item[$\blacktriangleright$] You are encouraged to read the visual SDM sections on each method to understand the full scope of eMoflon's features (which start
on page~\hyperlink{page.9}{9}). 
  
\end{itemize}
