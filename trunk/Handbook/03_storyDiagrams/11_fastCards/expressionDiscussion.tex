\newpage

\subsection{Expression Discussion}
\genHeader

As usual, all of our expression types can be used for the binding expression. Since you have encountered the first three of these at least once over the course
of building the complete metamodel, let's consider what each type would mean in this context:
\begin{description}
  \item[MethodCallExpression:]~\\ This would allow invoking a method and binding
  its return value to the object variable.  This is how non-primitive return
  values of methods can be used safely in SDMs.
  
  \item[ParameterExpression:]~\\ This could be used to bind the object variable
  to a parameter of the method.  If the object variable is of a different type
  than the parameter (e.g. a subtype) this represents basically a successful
  typecast if the pattern matches.
  
  \item[LiteralExpression:]~\\ As usual this can be anything and is literally
  copied with a surrounding typecast into the generated code.  Using
  LiteralExpressions too often is usually a sign for not thinking in a
  \emph{pattern oriented} manner and is considered a \emph{bad smell}.

  
  \item[ObjectVariableExpression:]~\\ This can be used to refer to other object
  variables in preceding story nodes.  Just like for ParameterExpressions, this
  represents a simple typecast if the types of the \texttt{target} and the
  object variable with the binding are different.
\end{description}
