\newpage
\subsection{Checking Cards (Textual)}
\texHeader
\hypertarget{checkCard tex}{}
 
 \begin{itemize}
   
\item[$\blacktriangleright$] This time, we need to create three separate patterns - an assertion pattern, a pattern to promote a card, and another to return.
But, given that this is an assertion, we need to generate proper control flow; we need to create an \texttt{if/else}

\item[$\blacktriangleright$] in the \texttt{check} declaration, enter the following code: (Fig) \\ 
if \texttt{checkCard}  { \\ \texttt{promoteCard} } else \\ \texttt{ [penalizeCard] } 

\item[$\blacktriangleright$] Using the ``Quick Fix'' wizard, generate the pattern files for all three. Your package explorer should now resemble Fig.

\item[$\blacktriangleright$] In order to validate the guess, go to the \texttt{checkCard} pattern, and create a rule bound to the passed-in \texttt{card} (Fig).
\texttt{@card : Card{ } } \\ that's it!

\item[$\blacktriangleright$] Now that we have the card to be checked, we need to compare the user's guess to the value on the back of the card. To do this, we
need to specify an \emph{Attribute Constraint}. This is a non-structural condition that must be satisfied for a story pattern to match.

\item[$\blacktriangleright$] enter \texttt{card.back == $guess$} in the \texttt{card} rule

\item[$\blacktriangleright$] Promote card: three rules. create a bounded \texttt{card : Card} and  \texttt{this : Partition}, and a free  \texttt{next :
Partition} variable.

\item[$\blacktriangleright$] \texttt{-- -> cardContainer : this \\ ++ -> cardContainer : next}. Because a link was set to update a card anytime a partition
deletes or creates a reference to it, it would be redundant to put any commands under \texttt{this} or  \texttt{next}.

\item[$\blacktriangleright$] You also need to guarantee the \texttt{previous} and \texttt{next} references stay intact. In the \texttt{@this} rule, enter
\texttt{-> next : next} in \texttt{promoteCard} and \texttt{-> previous : previous} . To set the references to the partitions

\item[$\blacktriangleright$] Be sure to check each of your eclasses to make sure the reference names are correct, or else nothing will be generated. Your
workspace should now resemble (Fig.)

\item[$\blacktriangleright$] \texttt{penalizeCard} has the Same variable/rule set up as \texttt{promoteCard}, except the reference \texttt{previous} is used
instead of next. This should look like (Fig.)

\item[$\blacktriangleright$] The final thing we need to do is return a final boolean statement depending on how the card was moved. If the card was promoted,
that means it was a success. It penalized, fail. (Fig)

\item[$\blacktriangleright$] Build project. View the changes in ``PartitionImpl.Java'' (theres changes in setNext, etc etc? ). New generated code in check.

\item[$\blacktriangleright$] Have now just implemented a ctrl flow, checking, where you created three story patterns to represent actions, with two unique stop
returns. To see how this is repped visually..

 \end{itemize}
 