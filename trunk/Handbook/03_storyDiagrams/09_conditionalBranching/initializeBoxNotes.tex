\newpage
\hypertarget{initialize notes}{}
\subsection{A short note on the \texttt{initializeBox} method}
\genHeader

Pretend you've just updated the control flow in your metamodel, and haven't specified \texttt{initializeBox} yet. After saving and building, you will be able
to see the changes in \texttt{BoxImpl.java}, the source file of all your generated code. In fact, open this file now and navigate to \texttt{grow}, which
starts at (approximately) line 207. Scan the comments until you find \texttt{``//statement node `initialize' ''} (Fig.~\ref{fig:initBoxImpl}). This is the generated
\emph{statement node} code and as you can see, all it does it invoke your method and branches based on its result. The method isn't actually defined here.

\begin{figure}[htp]
\begin{center}
  \includegraphics[width=0.5\textwidth]{eclipse_boxImplStatementNode}
  \caption{Code generated for branching with a statement node}
  \label{fig:initBoxImpl}
\end{center}
\end{figure}

Instead, hold \texttt{ctrl} while clicking on \texttt{initializeBox()} to automatically jump to its declaration in the file. If you didn't complete the SDM, it
would look like Fig.~\ref{fig:initBoxDecl}.

\begin{figure}[htp]
\begin{center}
  \includegraphics[width=\textwidth]{eclipse_initializeBoxDeclaration}
  \caption{The \texttt{initializeBox} declaration}
  \label{fig:initBoxDecl}
\end{center}
\end{figure}

Every method you declare can be found this way, and you will always have the choice of either implementing the method by hand here
in Java as an injection, or you can return to the metamodel and implement it there as an SDM. The statement node will work just fine in both cases. 

Using Java and injections makes sense if the method is non-structural, but seeing as we must check to see if there is a single partition, then create the
first two partitions of the box if it succeeds, \texttt{initializeBox} is actually quite structural and can be described beautifully as a pattern. Such is why
we opted to specify it as an SDM.

