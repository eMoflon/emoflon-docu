\newpage
\hypertarget{conBran tex}{}
\subsection{Branching with statement nodes}
\texHeader

\begin{itemize}

\item[$\blacktriangleright$] Before doing anything else, let's declare the method that will insert two new partitions into \texttt{box} when the original
pattern match fails. Open \texttt{Box.eclass} and add the following signature anywhere in the file: 
\syntax{initializeBox() : EBoolean}

\vspace{0.5cm}

\item[$\blacktriangleright$] Now modify \texttt{Box.grow()} by adding a nested \emph{if/Else} construct, with \texttt{[addNewPartitionBox]} as the
first conditional, and execute a \emph{statement node} if it fails. \texttt{grow} should now resemble Fig.~\ref{fig:updateGrow}.

\vspace{0.5cm}

\begin{figure}[htp]
\begin{center}
  \includegraphics[width=0.5\textwidth]{eclipse_updateGrow}
  \caption{Extending \texttt{Box} with a \emph{statement node}}
  \label{fig:updateGrow}
\end{center}
\end{figure}

\vspace{0.5cm}

\item[$\blacktriangleright$] Next, we want to specify our newest method. Create a new pattern called \texttt{buildPartitions} in its scope. Complete
the pattern as illustrated in Fig.~\ref{fig:pattBuildParts}.

\item[$\blacktriangleright$] As you can see, we have created a NAC that can only be fulfilled if the box has no lonely partition. In turn, this means that if a
box is completely empty, it will be intialized for the first time with two partitions, and guaranteed to remain in a valid state if growing continues.

\clearpage

\vspace*{2cm}

\begin{figure}[htp]
\begin{center}
  \includegraphics[width=0.7\textwidth]{eclipse_buildPartitionsPattern}
  \caption{NAC initalizing an empty box}
  \label{fig:pattBuildParts}
\end{center}
\end{figure}

\item[$\blacktriangleright$] That's it! Save and build your metamodel to make sure no errors exist. To see how this is depicted in the visual syntax, check out
Fig.~\ref{fig:newGrowControl} and Fig.~\ref{fig:buildPartitions}.

\end{itemize}
