\newpage
\section{Conditional branching}
\genHeader
\hypertarget{sec:conBran}{}

{\bf \large This is appendix content}

\vspace{1cm}

When working with SDMs, you often need to choose between two different patterns based on the return value of an arbitrary (black box) operation.
This [conditional branching] is like the normal \texttt{Success}/\texttt{Failure} construction but, instead of depending on a \emph{pattern match}, the decision
can be implemented with as another SDM or standard Java method. This feature is a further means (besides \texttt{MethodCallExpressions} for attribute values and
\texttt{Bindings}) of integrating hand-written Java code in SDMs.\footnote{It can lead to 'spaghetti' SDMs however, so please use with caution!}

As an example, consider a class \texttt{A} with an operation:
\begin{quote}
 \mbox{\texttt{doSomeCheck(p$_1$,\ldots,p$_n$ :EClass) :EBoolean}}
\end{quote}

This method could be implemented in hand-written Java code or be specified by another SDM specification.

Although dummy boolean attributes \emph{could} be used to achieve the same effect, it is much simpler to branch the control flow based on the result of a \texttt{StatementNode}.
If the method returns an \texttt{EBoolean}, \texttt{Success} and \texttt{Failure} correspond to \texttt{true} and \texttt{false}, respectively. If the method returns anything else, then \texttt{Failure} corresponds to \texttt{null}. Void methods \emph{cannot} be used to branch and an exception is thrown during code generation.

Figure~\ref{fig:cond_branch_on_op} depicts the class \texttt{A}, and shows how \texttt{doSomeCheck} is used to branch in an SDM.

\newpage

\vspace*{3cm}

\begin{figure}[htp]
\begin{center}
  \includegraphics[width=1.2\textwidth]{SDM_with_branch}
  \caption{Conditional branching based on the result of an operation}
  \label{fig:cond_branch_on_op}
\end{center}
\end{figure}

\fancyfoot[R]{ $\triangleright$ \hyperlink{conBran vis}{Next [visual]\hspace{0.2cm} } \\ $\triangleright$ \hyperlink{conBran tex}{Next [textual]} } 

\clearpage
\hypertarget{conBran vis}{}
\subsection{Branching with statement nodes}
\visHeader

\begin{itemize}

\item[$\blacktriangleright$] Currently, there is no method to help us initialize \texttt{box} from its pristine state (no partitions). Create one by editing
your metamodel (the \texttt{LearningBoxLanguage} diagram) and invoking the \texttt{Operations} dialogue by first selecting \texttt{Box}, then pressing
\texttt{F10}.\footnote{To review creating new operations, review Section 2.6 of Part II}

\item[$\blacktriangleright$] Name the new method \texttt{initializeBox} and, recalling the one rule of conditional branching, set its return type to
\texttt{EBoolean}.

\item[$\blacktriangleright$] Save and close the dialogue, then re-open the \texttt{grow} SDM and \emph{Quick Create} a new activity node from
\texttt{addNewPartition}.

\item[$\blacktriangleright$] This will be the node we'll use to invoke our helper method. Double click the node to invoke its properties editor and switch the
\texttt{Type} to a \texttt{StatementNode}. Name it \texttt{initialize} (Fig.~\ref{fig:newStatementNode}).

\item[$\blacktriangleright$] Before closing the dialogue, switch to the \texttt{Statement} tab, and create a \texttt{MethodCallExpression} to invoke your newest
method (Fig.~\ref{fig:statementMCE}). We want to access the \texttt{Box} object (\texttt{this}) and its \texttt{initalizeBox} method. It doesn't require any
parameters, so leave the values field empty. 

\begin{figure}[htbp]
   \centering
      \subfloat[Create a new \emph{StatementNode}]{
        \includegraphics[width=0.5\textwidth]{ea_newStatementNode}
        \label{fig:newStatementNode}
      }
      \subfloat[Edit the \texttt{MethodCallExpression} ]{
        \includegraphics[width=0.5\textwidth]{ea_statementMCExpression}
        \label{fig:statementMCE}
      }
      \caption{}
\end{figure}
\FloatBarrier

\clearpage

\item[$\blacktriangleright$] Now we need to update the edge guards stemming from \texttt{add\-New\-Part\-ition\-In\-Box}. Given that we only want to call
\texttt{initializeBox} if the pattern fails, change the edge guard leading to your statement node to \texttt{Failure}. Similarly, update the edge guard
returning \texttt{true} to \texttt{Success}.

\item[$\blacktriangleright$] Finally, attach two stop nodes -- \texttt{true} and \texttt{false} -- along with their appropriate edge guards from
\texttt{initialize}. These indicate that if the method execution worked, the box could be initialized. If it failed however, \texttt{box} was
in an invalid state (by e.g., having only one partition) and returns \texttt{false}. Overall, the new additions to \texttt{box.grow()} should resemble
Fig.~\ref{fig:newGrowControl}.

\vspace{0.5cm}

\begin{figure}[htp]
\begin{center}
  \includegraphics[width=\textwidth]{ea_growAdditions}
  \caption{Extending \texttt{grow} with a \emph{MethodCallExpression}}
  \label{fig:newGrowControl}
\end{center}
\end{figure}

\item[$\blacktriangleright$] To review our work up to this point, we have declared \texttt{initializeBox} and invoked it from a statement node. We have yet
to actually specify the method however. Double-click the anchor to return to the main diagram and create a new SDM
for \texttt{initializeBox}.

\item[$\blacktriangleright$] Create a normal activity node named \texttt{buildPartitions} with the pattern depicted in Fig.~\ref{fig:buildPartitions}.

\newpage
 
\begin{figure}[htp]
\begin{center}
  \includegraphics[width=\textwidth]{eclipse_buildPartitions}
  \caption{Complete SDM}
  \label{fig:buildPartitions}
\end{center}
\end{figure}
 
\item[$\blacktriangleright$] The NAC used here is only fulfilled if the box has absolutely no partitions, i.e., is in a pristine state and can be
initialized. In other words, if \texttt{grow} is used for an empty box, it initializes the box for the first time and grows it after that, ensuring that the box
is always in a valid state.
 
\item[$\blacktriangleright$] You're finished! Save, validate, and build your metamodel, then check out how this is done in the textual syntax in
Fig.~\ref{fig:updateGrow} and Fig.~\ref{fig:pattBuildParts}.

\jumpSingle{initialize notes}

\end{itemize}


\newpage
\hypertarget{conBran tex}{}
\subsection{Branching via}
\texHeader

\begin{itemize}

\item[$\blacktriangleright$] Before doing anything else, lets declare the method that will insert two new partitions into \texttt{box} when the original pattern
match fails. Open \texttt{Box.eclass} and write the following signature: 
\syntax{initializeBox () : EBoolean}

\vspace{0.5cm}

\item[$\blacktriangleright$] Now we need to modify \texttt{Box.grow()} with nested \emph{if/else} constructs, where \texttt{[addNewPartitionBox]} is the first
conditional, and a statement node is the second. Your EClass should now include Fig.~\ref{fig:updateGrow}

\vspace{0.5cm}

\begin{figure}[htp]
\begin{center}
  \includegraphics[width=0.5\textwidth]{eclipse_updateGrow}
  \caption{Extending \texttt{box.grow()}}
  \label{fig:updateGrow}
\end{center}
\end{figure}

\vspace{0.5cm}

\item[$\blacktriangleright$] Easy! Now, we want to define our newest method. As we mentioned before, you now have a choice -- you can either write the Java
implementation yourself in \texttt{Box.impl}, or continue as we have construct a pattern. Given that this uses an NAC, let's do it the way we already know by
creating a \texttt{[buildPartitions]} pattern.

\vspace{0.5cm}

\item[$\blacktriangleright$] Open and complete \texttt{[buildPartitions]} as illustrated in Fig.~\ref{fig:pattBuildParts}. As you can see, we have a bounded
box to check and see if a connection to \texttt{onePartition} exists. If none exists, the pattern will proceed to create a \texttt{firstPartition} and
\texttt{lastPartition}, and set up their references accordingly.

\clearpage

\vspace*{2cm}

\begin{figure}[htp]
\begin{center}
  \includegraphics[width=0.7\textwidth]{eclipse_buildPartitionsPattern}
  \caption{Pattern to check for only \texttt{one} partition.}
  \label{fig:pattBuildParts}
\end{center}
\end{figure}

\item[$\blacktriangleright$] That's everything! Save and build your metamodel to make sure no errors exist.

\end{itemize}

