\genHeader
\section{Conclusion and next steps}
\hypertarget{conclusion}{}

\vspace{0.5cm}

Congratulations - you've reached the end of yet another feature! In this part, you learned how to implement unidirectional model transformations through SDMs.
Declared inside method \emph{activities}, SDMs are comprised of different \emph{pattern}s, which themselves are made up of \emph{Object} and \emph{Link
Variables}. These variables can be given different \emph{binding states}, depending on their purpose. You have also learned how to build \emph{NACs}, which make
pattern matches \emph{deterministic} by construction.

\vspace{0.5cm}

To test your skills of these sections, challenge yourself to try:
\begin{itemize}
\item Creating a method/pattern to eject the card from the final partition (to signal it's been learnt)
\item Edit \texttt{Partition.empty} by including a method call to \texttt{removeCard}, and other commands so it achieves the same functionality
\item Modifying the GUI source files to execute the other methods we've established here in the view
\end{itemize}

\vspace{0.5cm}
	
If you had difficulties, comments, or suggestions for this part, feel free to drop us a line at \href{mailto:contact@moflon.org}{contact@moflon.org}.
Otherwise, if you enjoyed this section, continue to Part IV to learn about Triple Graph transformations, or onto Part V for Model-to-Text transformations. The
final part of this handbook is Part VI, Miscellaneous, which goes over tips and trick for using eMoflon, EA, subversion, and much more. As always, for a detailed description on each part of our handbook, refer to Part 0.

\vspace{0.5cm}


Cheers!
