\newpage
\section{Growing the box}
\genHeader
\hypertarget{sec:growBox}{}

% I'm not so sure I like this.
In this SDM, we shall specify how our learning box is built up by examining how the contained partitions are connected together, and how cards are able to
move back and forth in the box. While different behaviors can be implemented, we will use the classical rules as depicted in Fig.~\ref{fig:membox_depiction}.

No matter what, we know that our box will always be able to define a \texttt{first} and \texttt{last} partition. As explained in
\hyperlink{sec:emptyPartition}{section 5}, the current story pattern will determine these partitions non-deterministically.

SDMs provide a declarative means of identifying the first and last partition via \emph{Negative Application Conditions}, also simply referred to as
\mbox{NACs}\footnote{Pronounced $\backslash 'nak \backslash$}\note{\mbox{NACs}}. \mbox{NAC}s express structures that are forbidden before or after applying a
rule. In this SDM, the \mbox{NAC} will be a negative partition element that should \emph{not} be present in a valid pattern match. In the theory of algebraic
graph transformations \cite{EEPT06}, \mbox{NACs} can be complex graphs that are much more general and powerful but in our implementation\footnote{To be more
precise CodeGen2 from Fujaba.}, however, we only support single negative elements (object or link variables).

To create an appropriate \mbox{NAC} that that constrains the possible matches for \texttt{lastPartitionInBox} to \emph{exactly} the last partition in the box,
we'll need to create a new object variable, \texttt{nextPartition}, and link it to the last. It can then be completed by connecting it to both \texttt{Box} and
\texttt{firstPartitionInBox}. Overall, the NAC will be able to be interpreted as follows:
The first/last partitions in the box are arbitrary partitions in the box with no previous/next partitions defined. The valid matches are made unique and thus
deterministic by construction. In other words, if you \emph{grow} the box via this method, there will always be exactly one first and one last partition.

\begin{figure}[htbp]
 	\centering
  \includegraphics[width=0.6\textwidth]{goal_growBox.pdf}
	\caption{Growing a box by inserting a new partition}
	\label{fig:goal_grow}
\end{figure}
\FloatBarrier

\vfill
\newpage

Of course, it's not just a matter of NACs or how to link/insert a new partition into the box. We still need to figure how \emph{many} partitions to add! Since
the new size must be calculated depending on the rest of the partitions currently in the box (partitions usually get bigger) we'll need to call the
\texttt{determineNextSize()} method that we previously declared\footnote{Inspect your model, and review Part II, section \#} in the same class. Of course, we 
won't need to implement this until the end.
% Image? show the new boxes and new links to one another (given the size.. )?

Whoo, that was a lot of reading! To summarize our task, we will first create a basic control flow between the \texttt{Box}, and the first and the last
partitions which were determined at random by \emph{pattern matching}. We will then need to create some NACs as rules to insert new partitions, and then 
finally determine the new size.

Don't worry if you're a bit confused by all this theory (especially that part about \mbox{NAC}s). Once implemented, it should make more sense.
% If not, the interested reader can always refer to.. 

\fancyfoot[R]{ $\triangleright$ \hyperlink{growBox vis}{Next [visual]\hspace{0.2cm} } \\ $\triangleright$ \hyperlink{growBox tex}{Next [textual]} }

\newpage
\hypertarget{growBox vis}{}
\subsection{Implementing grow}
\visHeader

\begin{itemize}
 
\item[$\blacktriangleright$] Start by creating the simple story pattern depicted in Fig.~\ref{fig:sdm_grow_1}. This matches the box, \texttt{this}, with
\emph{any} two partitions.\footnote{Remember, the \emph{pattern matcher} is randomized!}

\vspace{0.5cm}

\begin{figure}[htbp]
\begin{center}
  \includegraphics[width=\textwidth]{ea_elementsGrowBox}
  \caption{Context elements for SDM}  
  \label{fig:sdm_grow_1}
\end{center}
\end{figure}

\item[$\blacktriangleright$] To create an appropriate \mbox{NAC} to constrain the possible matches for \texttt{lastPartitionInBox},  create a new
\texttt{Partition} object variable \texttt{nextPartition} and set\define{Binding\\Semantics}its \emph{binding semantics} to \texttt{negative}
(Fig.~\ref{fig:sdm_grow_2}). The object variable should be visualised as being cancelled or struck out. 
 
\begin{figure}[htbp]
\begin{center}
  \includegraphics[width=0.7\textwidth]{ea_newNac}
  \caption{Adding a negative element}  
  \label{fig:sdm_grow_2}
\end{center}
\end{figure}
 
\item[$\blacktriangleright$] Now, quick link \texttt{nextPartition} to \texttt{lastPartitionInBox}. Be sure choose the link type carefully! The
\texttt{nextPartition} should play the role of \texttt{next} with respect to \texttt{lastPartitionInBox}. This combination (the negative binding and reference)
tells the pattern matcher that if the (assumed) last partition has an element connected to its \texttt{next} reference, the current match is invalid.

\item[$\blacktriangleright$] Great work -- the first NAC is complete! In a similar fashion, create the remaining NAC for \texttt{firstPartitionInBox}. Name the
negative element \texttt{previousPartition}, and again, be sure to double-check the navigation.

\item[$\blacktriangleright$] Finally, complete the pattern rule so that it closely resembles Fig.~\ref{fig:sdm_grow_3}. 

\begin{figure}[htbp]
\begin{center}
  \includegraphics[width=\textwidth]{ea_NACComplete} 
  \caption{Determining the first and last partitions with NACs}  
  \label{fig:sdm_grow_3}
\end{center}
\end{figure}
 
\item[$\blacktriangleright$] Notice how the created partition \texttt{newPartition} is `hung' into the box. It becomes the next partition of the current
\emph{last} partition, and its previous partition is automatically set to the first partition in the box (as dictated by the rules set in
Fig.~\ref{fig:membox_depiction}). In other words, the new partition is appended onto the current order of partitions.

\item[$\blacktriangleright$] In order to complete \texttt{grow}, we need to set the size of the \texttt{newPartition}. Given that the new size is calculated
via the helper function \texttt{det\-er\-mine\-Next\-Size}, we need to invoke a \emph{MethodCallExpression}.\define{MethodCallExpression}A MethodCallExpression
is another specialized EA mechanism. As the name suggests, its designed to access any method defined in any class of the current project. Go ahead and invoke
the corresponding dialogue to activate the assignment (\texttt{:=}) operator, and match your values to Fig.~\ref{fig:sdm_grow_4}
 
\begin{figure}[htbp]
\begin{center}
  \includegraphics[width=0.6\textwidth]{ea_attributeMethodCall.png}
  \caption{Invoking a method via a \texttt{MethodCallExpression}}  
  \label{fig:sdm_grow_4} 
\end{center}
\end{figure}

Since \texttt{determineNextSize} doesn't require any parameters, you can ignore the \texttt{Parameter values} field this time. For future reference however,
parameters can be specified by choosing the appropriate parameter declaration between guillemets (e.g. \texttt{<Box box>}) found in the drop-down menu and typing in the
value (this is basically a literal expression). Don't forget to press the \texttt{Save} button for every parameter, then \texttt{Add} + \texttt{OK} to confirm
and close the dialogue.

\vspace{0.5cm}

\item[$\blacktriangleright$] If you've done everything right, your SDM should now closely resemble Fig.~\ref{fig:growComplete}. As usual, try to export,
generate code, and inspect the method implementation.

\vspace{0.5cm}

\item[$\blacktriangleright$]  That's it - your \texttt{grow} SDM is complete! This was probably the most challenging SDM to build, so give yourself a solid 
pat on the back. If you found it easy, well then \ldots I don't think I'm doing my job correctly. To see how this is done in the texual syntax, review
Fig.~\ref{fig:patternComplete}.\footnote{We do recommend reading the instructions for this one, since NACs can be tricky}

\vspace{0.5cm}

\begin{figure}[htbp]
\begin{center}
  \includegraphics[width=\textwidth]{ea_growFinal}
  \caption{Complete SDM for \texttt{Box::grow}}  
  \label{fig:growComplete}
\end{center}
\end{figure}
\FloatBarrier

\jumpSingle{sec:stringRep}

\end{itemize}


\clearpage
\hypertarget{growBox tex}{}
\subsection{Implementing grow}
\texHeader

\vspace*{0.5cm}

\begin{itemize}

\item[$\blacktriangleright$] In \texttt{grow}, create a simple control flow with one story pattern. You'll want this pattern to match the invocation box with
\emph{any} two partitions, so create a bound \texttt{this} box, and free \texttt{firstPartitionInBox} and \texttt{lastPartitionInBox} partitions.
You'll also need a variable set to create to represent the new partition. The skeleton of your pattern should now resemble (Fig.~\ref{fig:growPattSkel}).

\vspace{0.5cm}

\begin{figure}[htbp]
\begin{center}
  \includegraphics[width=0.5\textwidth]{eclipse_growPatternSkeleton}
  \caption{addPartitionsBox.pattern skeleton}
  \label{fig:growPattSkel}
\end{center}
\end{figure}

\item[$\blacktriangleright$] Next, we need to create an appropriate \emph{NAC} which will constrain the possible choices for \texttt{lastPartitionInBox}.
Create a \texttt{nextPartition} variable with an `!' operator (negation) preceding it immediately below \texttt{last\-Part\-it\-ion\-In\-Box}.

\vspace{0.5cm}

\item[$\blacktriangleright$] Now add \texttt{-> next : nextPartition} the the \texttt{lastPartition} scope. This
command will attempt to establish a \texttt{next} link from the last partition. Next, add \texttt{++ -> next: newPartition} (Fig.~\ref{fig:firstNAC}).
This line will only be reachable if the NAC fails! It establishes the \texttt{newPartition} as the final partition.

\begin{figure}[htbp]
\begin{center}
  \includegraphics[width=0.5\textwidth]{eclipse_growLastNAC}
  \caption{Creating the first NAC}
  \label{fig:firstNAC}
\end{center}
\end{figure}

\vspace{0.5cm}

\item[$\blacktriangleright$] In a similar fashion, create a second NAC, \texttt{previousPartition}, for \texttt{firstPartitionInBox}. No new references have to
be created here, so all you need to establish is the link connecting \texttt{firstPartitionInBox} to the negative element, \texttt{previousPartition}.

\begin{figure}[htp]
\begin{center}
  \includegraphics[width=0.55\textwidth]{eclipse_growFirstNAC}
  \caption{Pattern now with both NACs}
  \label{fig:growPatt}
\end{center}
\end{figure}

\clearpage

\item[$\blacktriangleright$] Connect \texttt{@this} with appropriate links to the first and last partitions, then establish the \texttt{box} and
\texttt{previous} references in \texttt{newPartition} (Fig~\ref{fig:growAllLinks}).

\vspace{0.5cm}

\begin{figure}[htp]
\begin{center}
  \includegraphics[width=0.65\textwidth]{eclipse_growLinks}
  \caption{A complete \emph{deterministic} pattern match}
  \label{fig:growAllLinks}
\end{center}
\end{figure}

\item[$\blacktriangleright$] \update explain the assumed `++'

\item[$\blacktriangleright$] We're not \emph{quite} done yet - our newest partition doesn't yet have a size. This means that not only do we need to make
another attribute contsraint, but \texttt{newPartition} needs to invoke a method in order to get the correct value. Call the method as you would in Java. 
Your workspace should then resemble Fig.~\ref{fig:patternComplete}.

\begin{figure}[htp]
\begin{center}
  \includegraphics[width=0.9\textwidth]{eclipse_growFinished}
  \caption{Complete pattern for adding a new \texttt{partition} to \texttt{Box}}
  \label{fig:patternComplete}
\end{center}
\end{figure}

\vspace{0.5cm}

\item[$\blacktriangleright$] That's all! While NACs may be difficult to understand at first, as you can see, they're not hard to implement, and
can be used in a wide variety of applications. To see how this method is implemented in the visual syntax, check out Fig~\ref{fig:growComplete} from the previous
section.

\item[$\blacktriangleright$] \update	move branching content here (to complete grow)

\end{itemize}
