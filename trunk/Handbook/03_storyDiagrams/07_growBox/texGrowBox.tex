\clearpage
\subsection{Textual; Growing a Box}
\texHeader
\hypertarget{growBox tex}{}

\begin{itemize}
  
\item[$\blacktriangleright$] Create a simple control flow. Match the box (this) with ANY two partitions in the box (Fig)

\item[$\blacktriangleright$] To create the correct NACs.. first fill in box. establish two references between the matched references as the first and last.

\item[$\blacktriangleright$] Next, in last partition, set next to the NAC, and create a brand new one to new partition.

\item[$\blacktriangleright$] the exclamation mark sets it as an NAC

\item[$\blacktriangleright$] in first partition, reinforce the previous link, and below the declaration, create the second Nac

\item[$\blacktriangleright$] Lastly, declare a creation variable and set the remaining two links: box and previous
  
\item[$\blacktriangleright$] Overall, this inserts a new partition between two arbitrarily selected first and last partitions.

\item[$\blacktriangleright$] We're not quite done yet - our newest partition doesn't yet have a size. that means not only do we need to make another attribute
contsraint, but we need to do a method call in order to set the value. %
% Move to splash
If you remember, Box has the method determineNextSize. This was implemented using injections. we need to call this now using a methodExpression

\item[$\blacktriangleright$] Above your reference declarations in newPartition, add the following: \texttt{newPartition.partitionSize :=
@this.determineNextSize()}

\item[$\blacktriangleright$] That's it - this was a difficult one to wrap your head around but, as you can see, not hard to implement. If you're worried you
messed something up, build!

\end{itemize}