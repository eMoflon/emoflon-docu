\newpage
\hypertarget{growBox vis}{}
\subsection{Implementing grow}
\visHeader

\begin{itemize}
 
\item[$\blacktriangleright$] Start by creating the simple story pattern depicted in Fig.~\ref{fig:sdm_grow_1}. This matches the box, \texttt{this}, with
\emph{any} two partitions.\footnote{Remember, the \emph{pattern matcher} is randomized!}

\vspace{0.5cm}

\begin{figure}[htbp]
\begin{center}
  \includegraphics[width=\textwidth]{ea_elementsGrowBox}
  \caption{Context elements for SDM}  
  \label{fig:sdm_grow_1}
\end{center}
\end{figure}

\item[$\blacktriangleright$] To create an appropriate \mbox{NAC} to constrain the possible matches for \texttt{lastPartitionInBox},  create a new
\texttt{Partition} object variable \texttt{nextPartition} and set\define{Binding\\Semantics}its \emph{binding semantics} to \texttt{negative}
(Fig.~\ref{fig:sdm_grow_2}). The object variable should be visualised as being cancelled or struck out. 
 
\begin{figure}[htbp]
\begin{center}
  \includegraphics[width=0.7\textwidth]{ea_newNac}
  \caption{Adding a negative element}  
  \label{fig:sdm_grow_2}
\end{center}
\end{figure}
 
\item[$\blacktriangleright$] Now, quick link \texttt{nextPartition} to \texttt{lastPartitionInBox}. Be sure choose the link type carefully! The
\texttt{nextPartition} should play the role of \texttt{next} with respect to \texttt{lastPartitionInBox}. This combination (the negative binding and reference)
tells the pattern matcher that if the (assumed) last partition has an element connected to its \texttt{next} reference, the current match is invalid.

\item[$\blacktriangleright$] Great work -- the first NAC is complete! In a similar fashion, create the remaining NAC for \texttt{firstPartitionInBox}. Name the
negative element \texttt{previousPartition}, and again, be sure to double-check the navigation.

\item[$\blacktriangleright$] Finally, complete the pattern rule so that it closely resembles Fig.~\ref{fig:sdm_grow_3}. 

\begin{figure}[htbp]
\begin{center}
  \includegraphics[width=\textwidth]{ea_NACComplete} 
  \caption{Determining the first and last partitions with NACs}  
  \label{fig:sdm_grow_3}
\end{center}
\end{figure}
 
\item[$\blacktriangleright$] Notice how the created partition \texttt{newPartition} is `hung' into the box. It becomes the next partition of the current
\emph{last} partition, and its previous partition is automatically set to the first partition in the box (as dictated by the rules set in
Fig.~\ref{fig:membox_depiction}). In other words, the new partition is appended onto the current order of partitions.

\item[$\blacktriangleright$] In order to complete \texttt{grow}, we need to set the size of the \texttt{newPartition}. Given that the new size is calculated
via the helper function \texttt{det\-er\-mine\-Next\-Size}, we need to invoke a \emph{MethodCallExpression}.\define{MethodCallExpression}A MethodCallExpression
is another specialized EA mechanism. As the name suggests, its designed to access any method defined in any class of the current project. Go ahead and invoke
the corresponding dialogue to activate the assignment (\texttt{:=}) operator, and match your values to Fig.~\ref{fig:sdm_grow_4}
 
\begin{figure}[htbp]
\begin{center}
  \includegraphics[width=0.6\textwidth]{ea_attributeMethodCall.png}
  \caption{Invoking a method via a \texttt{MethodCallExpression}}  
  \label{fig:sdm_grow_4} 
\end{center}
\end{figure}

Since \texttt{determineNextSize} doesn't require any parameters, you can ignore the \texttt{Parameter values} field this time. For future reference however,
parameters can be specified by choosing the appropriate parameter declaration between guillemets (e.g. \texttt{<Box box>}) found in the drop-down menu and typing in the
value (this is basically a literal expression). Don't forget to press the \texttt{Save} button for every parameter, then \texttt{Add} + \texttt{OK} to confirm
and close the dialogue.

\vspace{0.5cm}

\item[$\blacktriangleright$] If you've done everything right, your SDM should now closely resemble Fig.~\ref{fig:growComplete}. As usual, try to export,
generate code, and inspect the method implementation.

\vspace{0.5cm}

\item[$\blacktriangleright$]  That's it - your \texttt{grow} SDM is complete! This was probably the most challenging SDM to build, so give yourself a solid 
pat on the back. If you found it easy, well then \ldots I don't think I'm doing my job correctly. To see how this is done in the texual syntax, review
Fig.~\ref{fig:patternComplete}.\footnote{We do recommend reading the instructions for this one, since NACs can be tricky}

\vspace{0.5cm}

\begin{figure}[htbp]
\begin{center}
  \includegraphics[width=\textwidth]{ea_growFinal}
  \caption{Complete SDM for \texttt{Box::grow}}  
  \label{fig:growComplete}
\end{center}
\end{figure}
\FloatBarrier

\jumpSingle{sec:stringRep}

\end{itemize}
