\newpage
\subsection{Box::grow() - The visual syntax}
\visHeader
\hypertarget{growBox vis}{}

\begin{itemize}
 
\item[$\blacktriangleright$] Start by creating a simple control flow via the story pattern depicted in Fig.~\ref{fig:sdm_grow_1}. This matches the box
(\texttt{this}), with \emph{any} two partitions in the box\footnote{Remember, this is for the \emph{pattern matcher}}.

\begin{figure}[htbp]
\begin{center}
  \includegraphics[width=\textwidth]{ea_elementsGrowBox.pdf}
  \caption{Context elements for SDM}  
  \label{fig:sdm_grow_1}
\end{center}
\end{figure}

\item[$\blacktriangleright$] To create the appropriate \mbox{NAC} (to constrain the possible matches for \texttt{lastPartitionInBox}),  create the new object
variable \texttt{nextPartition}, of type \texttt{Partition}, and set \note{Binding Semantics} its \emph{Binding Semantics} to \texttt{negative}
(Fig.~\ref{fig:sdm_grow_2}). The object variable should be visualised as being cancelled or struck out. % why?
 
\begin{figure}[htbp]
\begin{center}
  \includegraphics[width=0.9\textwidth]{ea_negElement}
  \caption{Adding a negative element}  
  \label{fig:sdm_grow_2}
\end{center}
\end{figure}
 
\item[$\blacktriangleright$] Now, quick link \texttt{nextPartition} to \texttt{lastPartitionInBox}, but be sure choose the link type carefully! The
\texttt{nextPartition} should play the role of \texttt{next} with respect to \texttt{lastPartitionInBox}.

\item[$\blacktriangleright$] Complete the story pattern so that it closely resembles Fig.~\ref{fig:sdm_grow_3}. To review, \mbox{NACs} can be interpreted as
follows:  The first/last partition in the box are partitions in the box with no previous/next partitions. The valid matches are made unique and thus
deterministic by construction, i.e., if you \emph{grow} the box via this method, there will always be exactly one first and one last partition.

\begin{figure}[htbp]
\begin{center}
  \includegraphics[width=\textwidth]{ea_NACFirstLast.pdf} 
  \caption{Determining the first and last partition with NACs}  
  \label{fig:sdm_grow_3}
\end{center}
\end{figure}
 
\item[$\blacktriangleright$] Notice how the created partition \texttt{newPartition} is `hung' into the box. It becomes the next partition of the current
\emph{last} partition, and has its previous partition set to the first partition in the box (as depicted with the arrows in Fig.~\ref{fig:membox_depiction}).
  
\item[$\blacktriangleright$] The only step missing to complete this SDM is an assignment to set the size of the new partition. We know from the last SDM that an
assignment is an attribute constraint defined by a \texttt{:=} operator, so go ahead and invoke the corresponding dialogue. 

% DOES THIS APPLY TO VISUAL ONLY? If not, move to splash!
Since the new size must be calculated
depending on the rest of the partitions in the box (partitions usually get bigger) we call a helper function \note{MethodCallExpression} via a
\emph{MethodCallExpression}. A MethodCallExpression is used to invoke a method that is defined in a class in the current EA project. Enter the values in
Fig.~\ref{fig:sdm_grow_4} choosing the argument \texttt{this} as target and \texttt{determineNextSize} as the method to be invoked.
Parameters could be specified by just choosing the appropriate parameter declaration between guillemets (e.g. \texttt{<Box box>}) via the drop-down menu and
typing in the value (this is basically a literal expression). Don't forget to press the \texttt{Save} button for every parameter, and \texttt{Add} + \texttt{OK}
to confirm and close the dialogue. Since \texttt{determineNextSize} does not require any parameters, you can ignore the \texttt{Parameters} field this time.
 
 %UPDATE not under 'target' but 'objects'
\begin{figure}[htbp]
\begin{center}
  \includegraphics[width=0.5\textwidth]{ea_methodCall.png}
  \caption{Invoking a method via a \texttt{MethodCallExpression} {\bf UPDATE}}  
  \label{fig:sdm_grow_4} 
\end{center}
\end{figure}
\FloatBarrier

\item[$\blacktriangleright$]  If you've done everything right, your SDM should now closely resemble Fig.~\ref{fig:sdm_grow_5}. 

% As usual, try to export, generate code, inspect the
% method implementation and write a JUnit test.  This time around you also have to
% implement the helper method \texttt{determineNextSize} directly in the
% generated code
% (\texttt{gen/\-LearningBoxLanguage/\-facade/\-impl/\-LearningBoxUtilImpl}).
% Don't forget to add \texttt{@generated NOT} to the Java doc comment of the
% method so the code generator preserves your code in future runs.
% When testing (which you will \emph{of course} do right?), note that you can only grow a ``minimal'' box that has at least a first and last partition, i.e., a box with 
%no partitions at all cannot be grown using our specified SDM. 

\begin{figure}[htbp]
\begin{center}
  \includegraphics[width=0.9\textwidth]{ea_completeActivityGrowBox.pdf}
  \caption{Complete SDM for \texttt{Box::grow}}  
  \label{fig:sdm_grow_5}
\end{center}
\end{figure}

\item[$\blacktriangleright$]  That's it - your \texttt{grow} SDM is complete! This was probably the most challenging SDM to build, so give yourself a solid 
pat on the back. If you found it easy then \ldots gee whiz, I don't think i'm doing my job correctly. Curse you, easy-to-use software!

\end{itemize}
