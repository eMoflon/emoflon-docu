% Custom Heading; Change ??

{\bf \huge Part 0:}

\vspace{1cm}

{\bf \Huge Introduction }

\vspace{2cm}


This tutorial has been engineered to be \emph{fun}.

If you work through it and, for some reason, do \emph{not} have a resounding \mbox{``I-Rule''} feeling afterwards, please send us an email and tell us how to improve it at \href{mailto:contact@moflon.org}{contact@moflon.org}

\begin{figure}[htp]
\begin{center}
	% RENAME THIS IMAGE ??
	\includegraphics[height=0.45\textheight]{../introduction_images/i-rule}
	\caption{How you should feel when you're done.}
	\label{i-rule}
\end{center}
\end{figure}
\break
 

To enjoy the experience, you should be fairly comfortable with Java or a comparable object-oriented language, and know how to perform basic tasks in Eclipse.  Although we assume this, we give references to help bring you up to speed as necessary. Last but not least, very basic knowledge of common UML notation would be helpful. Please note that this entire tutorial is best viewed double-sided.

Our goal is to give a \emph{hands-on} introduction to metamodelling and graph transformations using our tool \emph{eMoflon}. The idea is to \emph{learn by doing} and all concepts are introduced while working on a concrete example. The language and style used throughout is intentionally relaxed and non-academic.

{\bf \large So, what is eMoflon?}

EMoflon is a metamodelling and graph transformation tool. It can be used two different ways, depending entirely on your preference. It can be done visually or textually, and we provide side-by-side tutorials for both in this handbook.
% Should we discuss here the benefits and pitfalls of each?

{\bf \large How does \emph{that} work?}

It was a challenge, figuring out how to present lessons for both methods in one handbook, but here's how we did it. You'll notice right away, starting in Part I, that there are red header lines on only the right hand pages, and blue header lines only on the left. Sometimes, both pages have a black header! This is because we've split the instructions into three different categories - visual, textual, and both. Visual explanations and screenshots will always be under the red headings, and Textual will be under blue. Black means that the following paragraphs apply to both methods. This is why the tutorial is best viewed two pages at a time - we want you to be able to compare and contrast the differences in the steps and realize the full potential of our tool.

{\bf \large What does this tutorial cover?}

We've described below each of the 6 parts that make up this tutorial. You can work through them sequentially and become an \emph{official}\footnote{Certificate not guaranteed} eMoflon master or, depending on your interests, you can decide what you'd like to read and what you'd like to skip. We provide links to each of their .zip files so you can jump right in without having to complete the previous parts. For those of you interested in further details and the mature formalism of graph transformations, we give relevant references throughout the tutorial.

\begin{description}

\item[Part I: Installation and Set Up]provides a very simple example and a few JUnit tests to test the installation and configuration of eMoflon.
 
After working through this chapter, you should have an installed and tested eMoflon working for a trivial example.
We also explain the general workflow and the different workspaces involved.

This chapter can be considered \emph{mandatory} if you are new to eMoflon, but recommend working through it anyway.
It's also kept as minimal as possible and should only take a few minutes, really.

\vspace{0.5cm}

\item[Part II: Ecore] is the main chapter and takes you step-by-step through a more realistic example that showcases most of the features we currently support.
Working through this chapter should serve as a basic introduction to model-driven engineering, and it's reccomended you work through this chapter if you're new to metamodeling in general (Using Ecore/EMF) %, metamodelling and graph transformations.

{\small Approximate Time to Complete: min

Relevant .zip file: \url{http://www.emoflon.org} }

\item[Part III: SDMs] introduces model transformations via graph transformation using Story Driven Modeling.

{\small Approximate Time to Complete: min

Relevant .zip file: \url{http://www.emoflon.org} }

\item[Part IV: TGGs] Even if you're mainly interested in TGGs, we recommend working through at least Part I. Although the example builds up on parts constructed in previous chapters, we provide a ``cheat'' package that you can use to get started directly.

{\small Approximate Time to Complete: min

Relevant .zip file: \url{http://www.emoflon.org} }

\item[Part V: Model To Text Transformations] Description;

{\small Approximate Time to Complete: min

Relevant .zip file: \url{http://www.emoflon.org} }

\item[Part VI: Miscellaenous] Contains some great references files to keep on hand while using ECore and it's partner programs. These will help to avoid mistakes and increase efficiency. %(Grokking EA)
If you're in a hurry, this chapter can be skipped

{\small Approximate Time to Complete: min

Relevant .zip file: \url{http://www.emoflon.org} }

\end{description}

One last thing: at the moment we unfortunately only support Windows. This should hopefully change in future releases.

Well, that's it -- sit back, relax, grab a coffee and enjoy the ride!

% Include some sort of coffee graphic?
