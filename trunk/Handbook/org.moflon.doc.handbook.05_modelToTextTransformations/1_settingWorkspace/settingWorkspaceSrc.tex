\newpage
\section{Setting up your workspace}
\genHeader

Nowadays, \emph{no one} writes a complex parser completely by hand. Although this is sometimes still necessary for syntactically challenging languages, most
parsers can be quickly whipped up using context-free \emph{string grammars}\footnote{For simple cases, \emph{regular expressions} can also be used} that are
typically written in Extended Backus-Naur Form (EBNF)\define{EBNF}. ANTLR is a tool that can generate a parser from this compact specification for
a host of target programming languages, including Java. Although ANTLR might not be the most efficient or powerful parser generator, it's open-source, well
documented and supported, and allows for a pragmatic and elegant fallback to Java if things get nasty and we have to resort to some dirty tricks to get the job
done.

To set up your workspace for the model-to-text transformation, you have two options: (1) Import a cheat package with everything already
prepared (useful if you're just joining us), or (2) if you've worked through the previous part, continue with your existing workspace. 
Both options should work, but we have only tested and updated all screenshots for Option (1) and thus highly recommend this.

As some of you are just reading this handbook without actually getting your hands dirty with an implementation (beware: no pain, no gain!), we have included a
screenshot of the dictionary metamodel that you get with both options in our visual (Fig.~\ref{ea:dictLang}) and textual (Fig.~\ref{eclipse:dictLangMetamodel})
concrete syntax.

\vspace{0.5cm}

% --- Dictionary metamodels --
\begin{figure}[htbp]
\begin{centering}
  \includegraphics[width=\textwidth]{ea_dictionaryMetamodel}
  \caption{Metamodel for dictionaries (visual concrete syntax)}
  \label{ea:dictLang}
  \end{centering}
\end{figure}

\newpage

\begin{figure}[h!]
  \hspace{-1.5cm}
  \includegraphics[width=1.2\textwidth]{eclipse_dictionaryMetamodel}
  \caption{Metamodel for dictionaries (textual concrete syntax)}
  \label{eclipse:dictLangMetamodel}
\end{figure}

\vspace{0.5cm}

% --- Option descriptions --
\begin{description}

\item[Option 1: Import a complete cheat package]

\item[$\blacktriangleright$] \hspace{0.3cm} Import the Part V `cheat package' by selecting ``New" in the toolbar, and the cheat package in the concrete syntax
of your choice (Fig.~\ref{eclipse_cheatPackage}).

\begin{figure}[htbp]
\begin{center}
  \includegraphics[width=0.65\textwidth]{eclipse_loadDictionaryProject}
  \caption{Load the cheat package for Part V into your workspace}
  \label{eclipse_cheatPackage}
\end{center}
\end{figure}

\vspace{0.5cm}

% -- Export/Import ---
\item[Option 2: Continue with the workspace from Part IV]


\item[$\blacktriangleright$] \hspace{0.3cm} Use the same metamodel for \texttt{Dictionary} as completed in Part IV. Just make sure you haven't radically changed
the dictionary metamodel (i.e., it still closely resembles the metamodel in either Fig.~\ref{ea:dictLang} or Fig.~\ref{eclipse:dictLangMetamodel}). Everything
else should work fine using the exact same workspace but remember, your screen may look different than our screenshots.

\end{description}


We recommend reviewing the dictionary metamodel until you feel comfortable with what you'll be working with. 

\newpage

\texttt{DictionaryLanguage} is only one of two metamodels that we'll be using to specify the TGG transformation. After all, TGGs typically require separate
source and target metamodels. The second metamodel involved in the transformation will be eMoflon's standard \texttt{MocaTree} language.\footnote{MOCA stands
for Moflon Code Adapter (not coffee, sorry.)} It basically combines concepts from a filesystem (folders and files), XML (text-only nodes and attributes), and a
general indexed containment hierarchy. It is provided by our Eclipse plugin and is automatically added to the build path, so it won't actually appear
anywhere in your Eclipse workspace.

Figure~\ref{mocaTreeMetamodel} is a visual depiction of this MocaTree model.\footnote{If you are using the visual syntax, feel free to view a detailed metamodel
by opening \texttt{dictionary.eap}, navigating to the \texttt{MocaTree} EPackage, and opening its diagram.} As you can see, the most important element is
\texttt{Node}. Note that a single \texttt{Node} can store any number of \texttt{Attribute} or \texttt{Text} elements (subnodes), but only belongs to one
\texttt{File}. If you look closer at \texttt{File}, you'll also notice that it belongs to a single \texttt{Folder}. \texttt{Folder} is able to store any number
of \texttt{File}s or subfolders.

\newpage

You can see that all elements inherit an \texttt{index} and \texttt{name} attribute. \texttt{Index} can be used to demand a certain \emph{order}
of nodes in a tree, otherwise not guaranteed by default (i.e., to enforce a hierarchy), while \texttt{name} can be any arbitrary string value. 

\vspace{1cm}

\begin{figure}[htbp]
  \begin{centering}
  \includegraphics[width=\textwidth]{MocaTreeMetamodel}
  \caption{Visual depiction of the MocaTree metamodel}
  \label{mocaTreeMetamodel}
  \end{centering}
\end{figure}

\vspace{1cm}

Enough chatting -- let's begin by creating the TGG project that will implement our model-to-text transformation.

\jumpDual{initialize vis}{initialize tex}

\newpage

\newpage
\hypertarget{initialize vis}{}
\subsection{Establishing the visual TGG}
\visHeader

\begin{itemize}

\item[$\blacktriangleright$] From Eclipse, open \texttt{Dict\-ion\-ary.eap} in Enterprise Architect (EA). The project broswer should resemble
Fig.~\ref{ea:mocaTagged}. As you can see, the project is already populated with the source \texttt{MocaTree} specification for a generic tree structure.

\vspace{0.5cm}

\begin{figure}[htpb]
\begin{center}
  \includegraphics[width=0.4\textwidth]{ea_mocaTaggedValues}
  \caption{Preventing \texttt{MocaTree} from exporting to Eclipse}
  \label{ea:mocaTagged}
\end{center}
\end{figure}

\end{itemize}
\vspace{-0.5cm}
If you inspect the tagged values\footnote{The ``Tagged Values'' window can be opened by going to ``View/Tagged Values''} for each language, you'll notice that
the \texttt{MocaTree} package has the \texttt{Moflon::Export} value set to \texttt{false}. This ensures that the package is \emph{ignored} when exporting. As
with all standard metamodels (e.g., Ecore or the SDM metamodel) the \texttt{MocaTree} package in EA should be regarded as read-only, required only in the
EA project so that SDMs can refer to the classes defined in the package. As discussed, the Java code is provided and added automatically by our Eclipse plugin.

\begin{itemize}

\item[$\blacktriangleright$] Go ahead and inspect the \texttt{MocaTree} diagram (Fig.~\ref{ea:mocaTree}). Make sure you understand which attributes and
references each element contains until you feel comfortable with what you'll be working with.

\item[$\blacktriangleright$] Given that TGGs can only succeed when the involved metamodels are contained in the same working set, add a new package to
\texttt{My Working Set} named \texttt{Dict\-ion\-ary\-Code\-Adap\-ter}.

\newpage

\begin{figure}[htpb]
\begin{center}
  \includegraphics[width=\textwidth]{ea_metamodelMocaTree}
  \caption{The MocaTree Metamodel}
  \label{ea:mocaTree}
\end{center}
\end{figure}

\vspace{1cm}

\item[$\blacktriangleright$] Add a new TGG schema diagram as depicted in Fig.~\ref{ea:newTGGDiagram}. In the next dialogue that appears, set the source project
as \texttt{MocaTree}, and the target project as \texttt{Dict\-ion\-ary\-Lang\-uage}.

\vspace{1cm}

\begin{figure}[htpb]
\begin{center}
  \includegraphics[width=0.8\textwidth]{ea_adapterTGGDiagram}
  \caption{Create a new TGG Diagram}
  \label{ea:newTGGDiagram}
\end{center}
\end{figure}

\end{itemize}

\clearpage

\begin{itemize}

\item[$\blacktriangleright$] To ensure the package exports correctly to the Eclipse workspace as a TGG project, add a single correspondence type to your new
diagram (the \texttt{schema}) between \texttt{Folder} and \texttt{Library}. Remember -- you can get the classes by drag-and-dropping each element into the
diagram, then quick-linking a new \texttt{TGG Correspondence Type} between them.\footnote{For details on how to do this, refer to Part IV, Section \update.}
Your diagram should come to resemble Fig.~\ref{ea:firstCorrType}.

\vspace{0.5cm}

\begin{figure}[htpb]
\begin{center}
  \includegraphics[width=\textwidth]{ea_firstAdapterCorrespondence}
  \caption{The transformation's first correspondence type}
  \label{ea:firstCorrType}
\end{center}
\end{figure}

\vspace{0.5cm}
\item[$\blacktriangleright$] Your project broswer should also now resemble Fig.~\ref{ea:TGGProjBrow}, where \texttt{Dict\-ion\-ary\-Code\-Adap\-ter} has
transformed into a \texttt{TGGSchemaPackage}.

\begin{figure}[htpb]
\begin{center}
  \includegraphics[width=0.5\textwidth]{ea_TGGProjectBrowser}
  \caption{TGG Project prepared with both metamodels}
  \label{ea:TGGProjBrow}
\end{center}
\end{figure}

\item[$\blacktriangleright$] Save and validate the project via the eMoflon control panel.\footnote{Introducted in Part \update, Section \update.} Switch back
to Eclipse and refresh the package explorer. A new \texttt{Dict\-ion\-ary\-Code\-Adap\-ter} folder should appear under \texttt{My Working Set}; Your workspace
is nearly complete!

\jumpSingle{subSec:setupParser}

\end{itemize}


\newpage
\hypertarget{initialize tex}{}
\subsection{Initializing the project}
\texHeader



After confirming your \texttt{Dictionary} package explorer resembles Fig.~\ref{eclipse:dictLang}, open \texttt{\_imports.mconf}
(Fig.~\ref{eclipse:standardImports}). You'll notice that it's already accessing the \texttt{MocaTree} metamodel, but where is it?

\begin{figure}[htbp]
\begin{center}
  \includegraphics[width=0.4\textwidth]{eclipse_importsFile}
  \caption{\texttt{DictionaryLanguage}'s imports file}
  \label{eclipse:standardImports}
\end{center}
\end{figure}

\texttt{MocaTree} is classified as a standard language, so every eMoflon project includes this specification as a hidden file. Therefore, you're not able to
inspect your target domain directly in Eclipse. Instead, we recommend reviewing Fig.~\ref{ea:mocaTree} which depicts \texttt{MocaTree} in the visual syntax.
Make sure you understand the classes and references before continuing.

\begin{enumerate}

\item[$\blacktriangleright$] Let's establish the TGG we'll use to transform between  \texttt{MocaTree} and\texttt{Dictionary}. Right-click on
\texttt{MyWorkingSet}, and navigate to ``New/ TGG.''

\item[$\blacktriangleright$] Name the package \texttt{DictionaryCodeAdapter}, setting the source as \texttt{MocaTree} and
target as \texttt{DictionaryLanguage} (Fig.~\ref{eclipse:newTGGProject}).

\begin{figure}[htbp]
\begin{center}
  \includegraphics[width=0.9\textwidth]{eclipse_dictionaryCodeAdapterTGGProject}
  \caption{Settings for our TGG}
  \label{eclipse:newTGGProject}
\end{center}
\end{figure}

\item[$\blacktriangleright$] A \texttt{schema.sch} file should have automatically opened in the editor. To ensure this TGG package is successfully generated as
a TGG, let's add something to this by creating our first correspondence type. Specify one between a tree's \texttt{Folder} instance, and a dictionary's
\texttt{Library} as depicted in Fig~\ref{eclipse:firstSchema}.\footnote{For details on this correspondence metamodel and how to build types
between classes, refer to Part IV, Section 3.} Don't forget -- you can use eMolfon's auto-completion feature here!

\begin{figure}[htbp]
\begin{center}
  \includegraphics[width=0.5\textwidth]{eclipse_schemaStart}
  \caption{A first correspondence type between domains}
  \label{eclipse:firstSchema}
\end{center}
\end{figure}

\item[$\blacktriangleright$] Save and build your project! Confirm the generated project has a solid black hexagon symbol, not a plug icon
overlaying the folder. This shape indicates the \texttt{Dictionary} is a TGG Package, and is not a standard ECore project (the default generation type).

\end{enumerate}


\newpage
\hypertarget{subSec:setupParser}{}
\subsection{Setting up the Parser}
\genHeader

The final workspace requirement that needs to be met is the creation of our transformation's ANTLR parser/unparser.
\begin{itemize}

\item[$\blacktriangleright$] Right-click on the \texttt{DictionaryCodeAdapter} folder and navigate to ``eMolfon/ Add Parser/Unparser''
(Fig~\ref{eclipse:contextParser}).

\vspace{0.5cm}

\begin{figure}[htpb]
\begin{center}
  \includegraphics[width=0.8\textwidth]{eclipse_contextAddParserUnparser}
  \caption{figureCaption}
  \label{eclipse:contextParser}
\end{center}
\end{figure}


\item[$\blacktriangleright$] In the parser settings window, enter \texttt{dictionary} as the \texttt{File extension}, and confirm the \texttt{Create Parser},
\texttt{Create Unparser}, and \texttt{ANTLR} options are selected as the corresponding technology for each case (Fig~\ref{eclipse:wizardParser}). Affirm by
pressing \texttt{Finish}.

\begin{figure}[htpb]
\begin{center}
  \includegraphics[width=0.8\textwidth]{eclipse_wizardParser}
  \caption{figureCaption}
  \label{eclipse:wizardParser}
\end{center}
\end{figure}

\vspace{0.5cm}

\item[$\blacktriangleright$] If everything was installed and completed without error, parser and unparser stubs should be generated under
\texttt{DictionaryCodeAdapter}, where \texttt{ANTLR} automatically built the corresponding Java packages as depicted in Fig.~\ref{eclipse:generatedParser}.

\begin{figure}[htpb]
\begin{center}
  \includegraphics[width=0.4\textwidth]{eclipse_generatedParser}
  \caption{figureCaption}
  \label{eclipse:generatedParser}
\end{center}
\end{figure}

\vspace{0.5cm}

\item[$\blacktriangleright$] Your workspace is now fully prepared, and you're ready to start transforming a textual source into a \texttt{Dictionary}
model!

\end{itemize}

