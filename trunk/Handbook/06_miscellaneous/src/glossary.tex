% \newpage
% \phantomsection
% \addcontentsline{toc}{section}{Glossary}
% \hypertarget{glossary}{}
% \genHeader
% 
% \vspace{1cm}
% {\Huge \bf Glossary}
% \vspace{1cm}
% 
% \normalsize{}

\newpage
\genHeader
\hypertarget{glossary}{}
\section{Glossary}

% TODO: Any updates here must be reflected in their original Part.

\begin{description}

\item[Abstract Syntax] % Part 2
Defines the valid static structure of members of a language. 

\item[Activity] % Part 3
Top-most element of an SDM.

\item[Activity Edge] % Part 3
A directed connection between activity nodes describing the control flow within an activity.

\item[Activity Node] % Part 3
Represents atomic steps in the control flow of an SDM. Can be either a story node or statement node.

\item[Assignments] % Part 3
Used to set attributes of object variables.

\item[Attribute Constraint] % Part 3
A non-structural constraint that must be satisfied for a story pattern to match. Can be either an assertion or assignment.

\item[Bidirectional Model Transformation] % Part 5
Consists of two unidirectional mo\-del transformations, which are consistent to each other. This requirement of
consistency can be defined in many ways, including using a TGG.

\item[Binding State] % Part 3
Can be either \emph{bound} or \emph{unbound/free}. See \emph{Bound vs Unbound}.

\item[Binding operator] % Part 3
Determine whether a variable is to be \emph{checked}, \emph{created}, or \emph{destroyed} during pattern matching.

\item[Binding Semantics] % Part 3
Determines if an object variable \emph{must} exist (\emph{mandatory}), may not exist (\emph{negative}; see \emph{NAC}), or is \emph{optional} during
\emph{pattern matching}.

\item[Bound vs Unbound] % Part 3
Bound variables are completely determined by the current context, whereas unbound (free) variables have to be determined by the \emph{pattern matcher}.
\texttt{this} and parameter values are always bound.

\item[Concrete Syntax] % Part 2
How members of a language are represented. This is often done textually or visually.

\item[Constraint Language] % Part 2
Typically used to specify complex constraints (as part of the static semantics of a language) that cannot be expressed in a metamodel.

\item[Correspondence Types] % Part 4
Connect classes of the source and target metamodels.

\item[Dangling Edges] % Part 3
An edge with no target or source. Graphs with dangling edges are invalid, which is why dangling edges are avoided and automatically deleted by the pattern
matching engine.

\item[Dynamic Semantics] % Part 2
Defines the dynamic behaviour for members of a language.

\item[EA] % Part 3
Enterprise Architect; The UML visual modeling tool used as our visual frontend.

\item[EBNF] % Part 5
Extended Backus-Naur Form; Concrete syntax for specifying con\-text-free string grammars, used to describe the context-free syntax of a string
language.

\item[Edge Guards] % Part 3
Refine the control flow in an activity by guarding activity edges with a condition that must be satisfied for the activity edge to be taken.

\item[Endogenous] % Part 5
Transformations between models in the same language (i.e., same input/output metamodel). 
 
\item[Exogenous] % Part 5
Transformations between models in different languages (i.e., unique metamodel instances). 

\item[Grammar] % Part 2
A set of rules that can be used to generate a language. 

\item[Graph Grammar] % Part 2
A grammar that describes a graph language. This can be used instead of a metamodel or type graph to define the abstract syntax of a language.

\item[Graph Triples] % Part 4
Consist of connected source, correspondence, and target components.

\item[In-place Transformation] 
Performs destructive changes directly to the input model, thus transforming it into the output model. Typically \emph{endogenous}.

\item[Link or correspondence Metamodel] % Part 4
Comprised of all correspondence types.

\item[Link Variable] % Part 3
Placeholders for links between matched objects.

\item[Literal Expression] % Part 3
Represents literals such as true, false, 7, or ``foo.''

\item[Meta-Language] % Part 2
A language that can be used to define another language.

\item[Meta-metamodel] % Part 2
A \emph{modeling language} for specifying metamodels.

\item[Metamodel] % Part 2
Defines the abstract syntax of a language including some aspects of the static semantics such as multiplicities. 

\item[MethodCallExpression] % Part 3
Used to invoke any method.

\item[Model] % Part 2
Graphs which conform to some metamodel.

\item[Modelling Language] % Part 2
Used to specify languages. Typically contains concepts such as classes and connections between classes.

\item[Monotonic] % Part 4
In the context of TGGs, a non-deleting characteristic.

\item[NAC] % Part 3
Negative Application Condition; Used to specify structures that must not be present for a transformation rule to be applied.

\item[Object Variable] % Part 3
Place holders for actual objects in the current model to be determined during pattern matching.

\item[ObjectVariableExpression] % Part 3
Used to reference other object variables.

\item[Operationalization] % Part 4
The process of deriving step-by-step executable instructions from a declarative specification that just states what the outcome should
be but not how to achieve it.

\item[Out-place Transformation] % Part 5
Source model is left intact by the transformation which creates the output model. Can be \emph{endogenous} or \emph{exogenous}.

\item[Parameter Expression]  % Part 3
Used to refer to method parameters.

\item[(Graph) Pattern Matching] % Part 3
Process of assigning objects and links in a model to the object and link variables in a pattern in a type conform manner. This is also referred to as finding a
match for the pattern in the given model.

\item[Statement Nodes] % Part 3
Used to invoke methods as part of the control flow in an activity.

\item[Static Semantics] % Part 2
Constraints members of a language must obey in addition to being conform to the abstract syntax of the language.

\item[Story Node] % Part 3
\emph{Activity nodes} that contain \emph{story pattern}s.

\item[Story Pattern] % Part 3
Specifies a structural change of the model.

\item[Triple Graph Grammars (TGG)] % Part 4
Declarative, rule-based technique of specifying the simultaneous evolution of three connected graphs.

\item[Type Graph] % Part 2
The graph that defines all types and relations that form a language. Equivalent to a metamodel but without any static semantics.

\item[TGG Schema] % Part 4
The metamodel triple consisting of the source, correspondence (link), and target metamodels.

\item[Unification]  % Part 2/3
An extension of the object oriented ``Everything is an object'' principle, where everything is regarded as a model, even the metamodel which defines other
models.

\end{description}
