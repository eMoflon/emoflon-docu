\newpage
\subsubsection{Creating EReferences with MOSL}
\texHeader
\hypertarget{static:references tex}{}

In MOSL, the declaration of a reference is simple - you simply set each property according to the following syntax:

{ \begin{quote} \small
[$<>$] -$>$ role\_name ( multiplicity ) :  target\_type \\
\\
role\_name := STRING \\
multiplicity := $|$ 0..1 $|$ 0..* $|$ 1 $|$ \ldots \\
target\_type := STRING \\
\end{quote} }

The source type is determined by the class in which the reference is placed. You can signal an aggregation reference by including the sideways diamond before
the arrow symbol. Don't worry - you don't have to remember this syntax! Our type completion provides a \texttt{Link Variable} template when you activate the hot
keys.

\begin{itemize}

\item[$\blacktriangleright$] Open \texttt{Box.eclass} in the editor and add a \emph{container reference} named \texttt{containedPartition} with a
multiplicity of zero to infinity, from \texttt{Box} to \texttt{Partition} (Fig.~\ref{fig:cpartitionReference}). This reference means a \texttt{Box}
can hold an infinite number of partitions.

\vspace{0.5cm}

\begin{figure}[htbp]
	\centering
  \includegraphics[width=0.6\textwidth]{eclass_box}
	\caption{Creating a \emph{contained reference} in \texttt{Box}}
	\label{fig:cpartitionReference}
\end{figure} 

\vspace{0.5cm}

\item[$\blacktriangleright$] Now add a \emph{simple reference} to \texttt{Partition}. Name it \texttt{box}, and allow it to hold up to one \texttt{Box}
(Fig.~\ref{fig:boxReference}). This means a single partition can belong to either zero, or one \texttt{Box}, and that's it. It can't belong to two different
boxes at the same time.

\item[$\blacktriangleright$] Congratulations, you have just built your first pair of EReferences! To see how this is depicted visually, check out
Fig.~\ref{fig:ereference_completed} from the previous subsection.

\newpage

\vspace{0.5cm}

\begin{figure}[htbp]
	\centering
  \includegraphics[width=0.6\textwidth]{eclass_partition}
	\caption{Creating a \emph{simple reference} in \texttt{Partition}}
	\label{fig:boxReference}
\end{figure} 

\vspace{0.5cm}

\item[$\blacktriangleright$] Now, lets create another pair of EReferences between \texttt{Partition} and \texttt{Card}. If you think about it, it's really not
all that different from the relation between \texttt{Box} and \texttt{Partition}. A \texttt{Partition} should be able to hold an unlimited amount of
\texttt{Card}s, but a \texttt{Card} should only be allowed to belong to zero or one \texttt{Partition}s. Name the two new references
\texttt{containedPartition}, and \texttt{box}. Your classes should now closely resemble Fig.~\ref{fig:almostAllReferences}.

\vspace{0.5cm}

\begin{figure}[htbp]
	\centering
  \includegraphics[width=0.65\textwidth]{eclipse_workspaceReferences}
	\caption{The Completed Bidirectional EReferences}
	\label{fig:almostAllReferences}
\end{figure} 

\item[$\blacktriangleright$] The next step is to construct two connections between \texttt{Partition}s and \texttt{Card}, so cards can be moved between its
previous and next partitions in the box. Create two new simple references, named \texttt{previous}, and \texttt{next}, each with a \texttt{0..1} multiplicity.
Allow them to have a maximum of 1 link each.

\vspace{0.5cm}

\item[$\blacktriangleright$] If you have done everything correctly, your classes should now resemble Fig.~\ref{fig:allReferences}. 

\vspace{0.5cm}

\begin{figure}[htbp]
	\centering
  \includegraphics[width=0.6\textwidth]{eclipse_allReferences}
	\caption{All references for our Leitner's Learning Box}
	\label{fig:allReferences}
\end{figure} 

\clearpage

At this point, all of your references have been created! The problem is, suppose you set the \texttt{containedPartition} reference in a particular \texttt{Box}.
That's great, you would now have the box containing one partition. However, if you went and examined that partition independently, its \texttt{box} reference
would still be null. We still need to set up the link between these references so that when one is updated, the other will be too.

\vspace{0.5cm}

\item[$\blacktriangleright$] Navigate to the ``\_constraints.mconf'' file. You can see it has a single \texttt{opposites} scope that's currently empty.
Constraints follow the syntax structure below: 
{ \begin{quote} \small
	opposite\_constraint : reference `$<$ - $>$' reference \\
	reference : reference\_name `:' source\_type \\
	\\
	reference\_name : STRING \\
	source\_type : STRING \\
\end{quote} }

This statement sets the two references to be opposites of one another, i.e., the connection between will classes be bidirectional. To start, enter the following
text:

{\begin{quote} \small
	containedPartition : Box $<$ - $>$ : Partition
\end{quote}}

\item[$\blacktriangleright$] Reviewing the \texttt{Partition} class, its easy to see that \texttt{previous} and \texttt{next} are certainly not
opposites,\footnote{Review the rules depicted in (Fig.~\ref{fig:membox_illustration})} but we do need to establish an opposing link between
\texttt{card} and its \texttt{cardContainer}. Follow the same steps until your constraint file resembles Fig.~\ref{fig:bothConstraints}.

\begin{figure}[htbp]
	\centering
  \includegraphics[width=0.6\textwidth]{eclipse_workspaceBothConstraints}
	\caption{The completed constraints file}
	\label{fig:bothConstraints}
\end{figure} 

\item[$\blacktriangleright$] Now the references for your Leitners learning box are truly complete! To see how each of the classes, attributes, and references
are depicted in the visual syntax, check out Fig.~\ref{fig:ereferences_all} in \hyperlink{sec:static vis}{Section 2.1}. Otherwise, build your project to
make sure there are no errors, and continue to the next section to finalize the declaration of your classes.

\fancyfoot[R]{$\triangleright$ \hyperlink{static:methods tex}{Next}}

\end{itemize}
