\newpage
\subsection{Connecting your classes}
\genHeader
\hypertarget{static:references splash}{}

\texttt{}
\emph{}

At this point, you've declared your types and attributes, but what good are those if none of them can communicate with each other? We need to create some
\emph{EReferences}!

There are 3 properties that must be set in order to create an EReference: \emph{Navigation}, \emph{Aggregation}, and
\emph{Multiplicity}. As you can probably guess, they're declared differently in each syntax, but let's review the concepts since they're the same.

Firstly, \texttt{Navigable} ends are mapped to class attributes with getters and setters in Java, and therefore \emph{must} have a specified name and
multiplicity for successful code generation. Corresponding values for \texttt{Non-Navigable} ends can  be regarded as additional documentation, and do not have
to be specified.

Secondly, the \texttt{Multiplicity} refers to that of a reference\footnote{Don't get multiplicity of a reference confused with the multiplicity of an element!
Multiplicity of an element defines the permitted range of individual instances, rather than types.}. It controls if the relation is mapped to a
Java Collection (\texttt{*},~\texttt{1..*},~\texttt{0..*}), or to a single valued class attribute (\texttt{1}, \texttt{0..1}). We'll explain this setting in
detail when you encounter them in your syntax instructions.

Lastly, in Ecore, the \texttt{Aggregration} values of a reference can be \texttt{none}, \texttt{shared}, or \texttt{com\-po\-site}. Composite means that the
current role is that of a \emph{container} for the opposite role. You'll see in our example that \texttt{Box} is a container for several \texttt{Partition}s.
This has a series of consequences: (1) every element must have a container, (2) an element cannot be in more than one container at the same time, and (3) a
container's contents are deleted together with the container. Conversely, non-composite (\texttt{none}) means that the current role is not that of a container,
and the rules for containment do not hold (in other words, the reference is set a simple `pointer'). The \texttt{shared} setting is beyond the scope of this
handbook.


\fancyfoot[RO]{ $\triangleright$ \hyperlink{static:references vis}{Next [visual]\hspace{0.2cm}} \\ $\triangleright$ \hyperlink{static:references tex}{Next
[textual]}}

\newpage
\subsubsection{Connecting your classes in EA}
\visHeader
\hypertarget{static:references vis}{}

\begin{itemize}

\item[$\blacktriangleright$] A fundamental gesture in EA is \emph{Quick Link}. Quick Link is used to create links between elements in a context-sensitive
manner. To use Quick Link, choose an element and note the little black arrow in its top-right corner (Fig.~\ref{fig:quicklink}).

\begin{figure}[htbp]
	\centering
  \includegraphics[width=0.4\textwidth]{ea_quickLink}
	\caption{Quick Link is a central gesture in EA}
	\label{fig:quicklink}
\end{figure}
\FloatBarrier

Click this black arrow and `pull' to the element you wish to link to. To start, quick-link from \texttt{Box} to \texttt{Partition}. In the context menu
that appears, select ``Create Bidirectional EReference'' (Fig.~\ref{fig:ereference}).

\begin{figure}[htbp]
	\centering
  \includegraphics[width=0.6\textwidth]{ea_eReferenceBidirectional}
	\caption{Create a reference via Quick Link}
	\label{fig:ereference}
\end{figure}
\FloatBarrier

\item[$\blacktriangleright$] Double click the reference to invoke a dialogue. Here you can change the reference direction and enter a name. Feel free to leave
the \texttt{Name} value blank - this property is only used for documentation purposes, and not relevant to code generation.

\item[$\blacktriangleright$] In the same dialogue, go to ``Target Role'' and enter the values in Fig.~\ref{fig:reference_ends} to set the properties for the
`target' end of the reference (the \texttt{Box} role). As you can see, the default target is set to the class you linked \emph{from}, and the default source is
the class you linked \emph{to}. 

\vspace{0.5cm}

\item[$\blacktriangleright$] If you decided to ignore the previous instructions, and went from \texttt{Partition} to \texttt{Box}, the only difference between
the two references are their titles. The following information will still be the same! In this window, it's important not to forget to confirm and modify the
\texttt{Role}, \texttt{Navigability}, \texttt{Multiplicity}, and \texttt{Aggregation} settings for the target.  Repeat the process for the \texttt{Source Role}.

\vspace{1cm}

\begin{figure}[htbp]
	\centering
	  \includegraphics[width=0.73\textwidth]{ea_assocPropsTarget}\\
	\caption{Properties for the target role of a reference}
	\label{fig:reference_ends}
\end{figure}
\FloatBarrier

\begin{figure}[htbp]
	\centering
    \includegraphics[width=0.73\textwidth]{ea_assocPropsSource}
	\caption{Properties for the source role of a reference}
	\label{fig:sketch_roles}
\end{figure}
\FloatBarrier

\end{itemize}

To explain, the first value you completed was the navigation name. The \texttt{Navigation} should have been automatically set to \texttt{Navigable}. Without these,
the correct code cannot be generated. 

Next, you set the \texttt{Multiplicity} value. In your target role (\texttt{Box}), you have allowed the creation of up to
one target (\texttt{box}) reference for every connected source (\texttt{Partition}). This means you could not have a single source connected to two targets
(ie., one partition that belongs to two unique boxes). In the source (\texttt{Partition}) role, you have specified that any target (in our case, \texttt{box})
can have any positive-sized number of sources. Figure~\ref{fig:sketch_roles} sketches this idea\footnote{Remember, if you quick-linked in the opposite
direction, the concept is the same, but the role titles are reversed.}. 

Finally, you set the \texttt{Aggregation} value. In this case, \texttt{box} is a
container for \texttt{Partition}s, and \texttt{containedPartition} doesn't need to adhere to any rules.

\begin{figure}[htbp]
	\centering
    \includegraphics[width=0.6\textwidth]{sketch_multiplicities.pdf}
	\caption{The target and source roles of Leitner's Learning Box}
	\label{fig:sketch_roles}
\end{figure}
\FloatBarrier

\begin{itemize}

\item[$\blacktriangleright$] Take a moment to review how the \texttt{Aggregration} settings extend the \texttt{Multiplicity}. If you've done everything right,
your workspace should now resemble Fig.~\ref{fig:ereference_completed}, with a single bidirectional EReference between \texttt{Box} and \texttt{Partition}.

\vspace{1cm}

\begin{figure}[htbp]
	\centering
  \includegraphics[width=0.4\textwidth]{ea_relationBoxPartition.pdf}
	\caption{\texttt{Box} contains \texttt{Partition}s}
	\label{fig:ereference_completed}
\end{figure}
\FloatBarrier

\clearpage
\item[$\blacktriangleright$] Create another bidirectional EReference\footnote{To be precise, \emph{all} references in Ecore are actually unidirectional.
A ``bidirectional'' reference in our metamodel is really two mapped \texttt{EReferences} that are opposites of each other.
We however, believe it is simpler to handle these pairs as single references, and prefer this concise concrete syntax.} between \texttt{Partition} and
\texttt{Card}, then two unidirectional self-references for \texttt{Partition}. Your final workspace should resemble Fig.~\ref{fig:ereferences_all}.

\vspace{1cm}

\begin{figure}[htbp]
	\centering
  \includegraphics[width=0.5\textwidth]{ea_relationsAll.pdf}
	\caption{All relations in our metamodel}
	\label{fig:ereferences_all}
\end{figure}

\FloatBarrier

\item[$\blacktriangleright$] All the attributes and references required for your learning box have now been set up. We encourage you to \emph{not} click the link below, and
see how these are declared in the textual syntax, in the next section. In particular, check out Fig.~\ref{fig:allReferences}!

\fancyfoot[R]{$\triangleright$ \hyperlink{static:methods vis}{Next}}

\end{itemize}

\newpage
\subsubsection{Creating EReferences with MOSL}
\texHeader
\hypertarget{static:references tex}{}

In MOSL, the declaration of an EReference is simple - you set each property according to the following syntax (specified in simple EBNF, if you know what that
is):

\syntax{ [ `<>' ] `-' role\_name `(' multiplicity `)' `->'  target\_type \\
\\
With:\\
role\_name := STRING \\
multiplicity := `0..1' $|$ `0..*' $|$ `1' $|$ \ldots \\
target\_type := STRING \\
}

The source type is determined by the EClass in which the EReference is placed. You can signal an aggregation EReference by including the sideways diamond before
the arrow symbol. Don't worry - you don't have to remember this syntax. Our type completion provides a \texttt{reference} template when you activate the hot
keys. Try it out!

\begin{itemize}

\item[$\blacktriangleright$] Open \texttt{Box.eclass} in the editor and add a \emph{container reference} named \texttt{containedPartition} with a
multiplicity of zero to infinity, from \texttt{Box} to \texttt{Partition} (Listing~\ref{eclipse:cpartitionReference}, Line \ref{mosl:containedPartition}). This EReference means a \texttt{Box}
can hold an infinite number of partitions.

\vspace{0.5cm}


\item[$\blacktriangleright$] Now add a \emph{simple reference} to \texttt{Partition}. Name it \texttt{box}, and allow it to hold up to one \texttt{Box}
(Listing~\ref{eclipse:moslBoxReference}, Line~\ref{mosl:referencedBox}). This means a single partition can belong to either zero, or one \texttt{Box}, and that's it. It can't belong to two different
boxes at the same time.

\vspace{0.5cm}

\item[$\blacktriangleright$] Congratulations, you have just built your first pair of EReferences! To see how this is depicted visually, check out
Fig.~\ref{ea:ereference_completed} from the previous subsection.

\vspace{0.5cm}

\item[$\blacktriangleright$] Now, lets create another pair of EReferences between \texttt{Partition} and \texttt{Card}. If you think about it, it's really not
all that different from the relation between \texttt{Box} and \texttt{Partition}. A \texttt{Partition} should be able to hold an unlimited amount of
\texttt{Card}s, but a \texttt{Card} should only be allowed to belong to zero or one \texttt{Partition}s. Name the two new EReferences
\texttt{card}, and \texttt{cardContainer} (Listing~\ref{eclipse:moslBoxReference}, Line~\ref{mosl:containedCard} and Listing~\ref{eclipse:moslCardReference}, Line~\ref{mosl:cardContainer}).

\vspace{0.5cm}

\item[$\blacktriangleright$] The next step is to construct two connections between \texttt{Partition}s so cards can be moved between their previous and next
partitions in the box. Create two new simple references, named \texttt{previous}, and \texttt{next}, each with a \texttt{0..1} multiplicity (Listing~\ref{eclipse:moslBoxReference}, Line~\ref{mosl:nextPartition} and~\ref{mosl:previousPartition}).

\vspace{0.5cm}

\item[$\blacktriangleright$] If you have done everything correctly, your EClasses should now resemble Listings~\ref{eclipse:cpartitionReference},~\ref{eclipse:moslBoxReference} and~\ref{eclipse:moslCardReference}. 

\vspace{0.5cm}

\lstinputlisting[style=eclass, label=eclipse:cpartitionReference ,caption={Creating a \emph{container reference} in \texttt{Box}}] {../2_staticSemantics/3_connectingClasses/ccTexCode/Box.txt} 


\vspace{0.5cm}

\lstinputlisting[style=eclass, label=eclipse:moslBoxReference,caption={Creating  \emph{references} in \texttt{Partition}}] {../2_staticSemantics/3_connectingClasses/ccTexCode/Partition.txt} 


\vspace{0.5cm}

\lstinputlisting[style=eclass, label=eclipse:moslCardReference,caption={Creating a \emph{simple references} in \texttt{Card}}] {../2_staticSemantics/3_connectingClasses/ccTexCode/Card.txt} 


\clearpage

At this point, all of your EReferences have been created! The problem is, suppose you set the \texttt{containedPartition} EReference in a particular
\texttt{Box}. That's great, you would now have the box containing one partition. However, if you went and examined that partition independently, its
\texttt{box} EReference would still be null. We still need to set up the link between these EReferences so that when one is updated, the other will be too.

\vspace{0.5cm}

\item[$\blacktriangleright$] Navigate to the \texttt{\_ModelConfiguration.mconf} file. You can see it has a single \texttt{opposites} scope that's currently empty.
Constraints follow the syntax below: 

\syntax{reference `<->' reference \\
\\
With:\\
reference := reference\_name `:' source\_type \\
reference\_name := STRING \\
source\_type := STRING \\}

This statement sets the two EReferences to be opposites of one another, i.e., the connection between EClasses will be bidirectional. As you can see, syntax here
is slightly different than that of a standard EReference. Instead of the reference type trailing the colon operator, it has switched to become the source type.

\vspace{0.5cm}

\item[$\blacktriangleright$] To begin, press \texttt{Ctrl + Space} and complete
the template with the following:

\syntax{containedPartition : Box <-> box : Partition}

\item[$\blacktriangleright$] Reviewing the \texttt{Partition} EClass, its easy to see that \texttt{previous} and \texttt{next} are certainly not
opposites,\footnote{Review the rules depicted in Fig.~\ref{fig:membox_illustration}} but we do need to establish an opposing link between
\texttt{card} and its \texttt{cardContainer}. Follow the same steps until your constraint file resembles Listing~\ref{eclipse:bothConstraints}.

\vspace{5mm}

\lstinputlisting[style=mconf, label=eclipse:bothConstraints,caption={The completed constraints file}] {../2_staticSemantics/3_connectingClasses/ccTexCode/_ModelConfiguration.txt} 

\newpage

\item[$\blacktriangleright$] Now the EReferences for your learning box are complete! To see how each of the classes, attributes, and references
are depicted in the visual syntax, check out Fig.~\ref{ea:ereferences_all} from \hyperlink{sec:static vis}{section 2.1}. Otherwise, build your project to
make sure there are no errors, and continue to the next section to finalize the declaration of your EClasses.

\jumpSingle{static:methods tex}

\end{itemize}

