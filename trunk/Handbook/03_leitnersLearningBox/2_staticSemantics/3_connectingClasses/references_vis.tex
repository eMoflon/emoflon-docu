\newpage
\subsubsection{Connecting your classes in EA}
\visHeader
\hypertarget{static:references vis}{}

\begin{itemize}

\item[$\blacktriangleright$] A fundamental gesture in EA is \emph{Quick Link}. Quick Link is used to create links between elements in a context-sensitive
manner. To use Quick Link, choose an element and note the little black arrow in its top-right corner (Fig.~\ref{fig:quicklink}).

\begin{figure}[htbp]
	\centering
  \includegraphics[width=0.4\textwidth]{ea_quickLink}
	\caption{Quick Link is a central gesture in EA}
	\label{fig:quicklink}
\end{figure}
\FloatBarrier

Click this black arrow and `pull' to the element you wish to link to. To start, quick-link from \texttt{Box} to \texttt{Partition}. In the context menu
that appears, select ``Create Bidirectional EReference'' (Fig.~\ref{fig:ereference}).

\begin{figure}[htbp]
	\centering
  \includegraphics[width=0.6\textwidth]{ea_eReferenceBidirectional}
	\caption{Create a reference via Quick Link}
	\label{fig:ereference}
\end{figure}
\FloatBarrier

\item[$\blacktriangleright$] Double click the reference to invoke a dialogue. Here you can change the reference direction and enter a name. Feel free to leave
the \texttt{Name} value blank - this property is only used for documentation purposes, and not relevant to code generation.

\item[$\blacktriangleright$] In the same dialogue, go to ``Target Role'' and enter the values in Fig.~\ref{fig:reference_ends} to set the properties for the
`target' end of the reference (the \texttt{Box} role). As you can see, the default target is set to the class you linked \emph{from}, and the default source is
the class you linked \emph{to}. 

\vspace{0.5cm}

\item[$\blacktriangleright$] If you decided to ignore the previous instructions, and went from \texttt{Partition} to \texttt{Box}, the only difference between
the two references are their titles. The following information will still be the same! In this window, it's important not to forget to confirm and modify the
\texttt{Role}, \texttt{Navigability}, \texttt{Multiplicity}, and \texttt{Aggregation} settings for the target.  Repeat the process for the \texttt{Source Role}.

\vspace{1cm}

\begin{figure}[htbp]
	\centering
	  \includegraphics[width=0.73\textwidth]{ea_assocPropsTarget}\\
	\caption{Properties for the target role of a reference}
	\label{fig:reference_ends}
\end{figure}
\FloatBarrier

\begin{figure}[htbp]
	\centering
    \includegraphics[width=0.73\textwidth]{ea_assocPropsSource}
	\caption{Properties for the source role of a reference}
	\label{fig:sketch_roles}
\end{figure}
\FloatBarrier

\end{itemize}

To explain, the first value you completed was the navigation name. The \texttt{Navigation} should have been automatically set to \texttt{Navigable}. Without these,
the correct code cannot be generated. 

Next, you set the \texttt{Multiplicity} value. In your target role (\texttt{Box}), you have allowed the creation of up to
one target (\texttt{box}) reference for every connected source (\texttt{Partition}). This means you could not have a single source connected to two targets
(ie., one partition that belongs to two unique boxes). In the source (\texttt{Partition}) role, you have specified that any target (in our case, \texttt{box})
can have any positive-sized number of sources. Figure~\ref{fig:sketch_roles} sketches this idea\footnote{Remember, if you quick-linked in the opposite
direction, the concept is the same, but the role titles are reversed.}. 

Finally, you set the \texttt{Aggregation} value. In this case, \texttt{box} is a
container for \texttt{Partition}s, and \texttt{containedPartition} doesn't need to adhere to any rules.

\begin{figure}[htbp]
	\centering
    \includegraphics[width=0.6\textwidth]{sketch_multiplicities.pdf}
	\caption{The target and source roles of Leitner's Learning Box}
	\label{fig:sketch_roles}
\end{figure}
\FloatBarrier

\begin{itemize}

\item[$\blacktriangleright$] Take a moment to review how the \texttt{Aggregration} settings extend the \texttt{Multiplicity}. If you've done everything right,
your workspace should now resemble Fig.~\ref{fig:ereference_completed}, with a single bidirectional EReference between \texttt{Box} and \texttt{Partition}.

\vspace{1cm}

\begin{figure}[htbp]
	\centering
  \includegraphics[width=0.4\textwidth]{ea_relationBoxPartition.pdf}
	\caption{\texttt{Box} contains \texttt{Partition}s}
	\label{fig:ereference_completed}
\end{figure}
\FloatBarrier

\clearpage
\item[$\blacktriangleright$] Create another bidirectional EReference\footnote{To be precise, \emph{all} references in Ecore are actually unidirectional.
A ``bidirectional'' reference in our metamodel is really two mapped \texttt{EReferences} that are opposites of each other.
We however, believe it is simpler to handle these pairs as single references, and prefer this concise concrete syntax.} between \texttt{Partition} and
\texttt{Card}, then two unidirectional self-references for \texttt{Partition}. Your final workspace should resemble Fig.~\ref{fig:ereferences_all}.

\vspace{1cm}

\begin{figure}[htbp]
	\centering
  \includegraphics[width=0.5\textwidth]{ea_relationsAll.pdf}
	\caption{All relations in our metamodel}
	\label{fig:ereferences_all}
\end{figure}

\FloatBarrier

\item[$\blacktriangleright$] All the attributes and references required for your learning box have now been set up. We encourage you to \emph{not} click the link below, and
see how these are declared in the textual syntax, in the next section. In particular, check out Fig.~\ref{fig:allReferences}!

\fancyfoot[R]{$\triangleright$ \hyperlink{static:methods vis}{Next}}

\end{itemize}
