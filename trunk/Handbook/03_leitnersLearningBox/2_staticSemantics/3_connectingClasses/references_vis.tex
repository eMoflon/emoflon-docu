\newpage
\subsubsection{Creating EReferences in EA}
\visHeader
\hypertarget{static:references vis}{}

\begin{itemize}

\item[$\blacktriangleright$] A fundamental gesture in EA is \emph{Quick Link}. Quick Link is used to create references between elements in a context-sensitive
manner. To use quick link, choose an element and note the little black arrow in its top-right corner (Fig.~\ref{fig:quicklink}).

\begin{figure}[htbp]
	\centering
  \includegraphics[width=0.4\textwidth]{ea_quickLink}
	\caption{Quick Link is a central gesture in EA}
	\label{fig:quicklink}
\end{figure}
\FloatBarrier

\item[$\blacktriangleright$] Click this black arrow and `pull' to the element you wish to link to. To start, quick-link from \texttt{Box} to \texttt{Partition}.
In the context menu that appears, select ``Create Bidirectional EReference'' (Fig.~\ref{fig:ereference}).

\begin{figure}[htbp]
	\centering
  \includegraphics[width=0.6\textwidth]{ea_eReferenceBidirectional}
	\caption{Create a reference via Quick Link}
	\label{fig:ereference}
\end{figure}
\FloatBarrier

\item[$\blacktriangleright$] Double click the reference to invoke a dialogue. In this window you can adjust all relevant settings. Feel free to leave the
\texttt{Name} value blank - this property is only used for documentation purposes, and is not relevant to code generation.

\item[$\blacktriangleright$] Within this dialogue, go to ``Target Role'', and compare the relevant values in Fig.~\ref{fig:role_target} for the \emph{Target}
end of the reference (the \texttt{Box} role). As you can see, the default target is set to the class you linked \emph{from}, and the default \emph{Source} is
the class you linked \emph{to}. In this window, do not forget to confirm and modify the \texttt{Role}, \texttt{Navigability},
\texttt{Multiplicity}, and \texttt{Aggregation} settings for the target are required.  Repeat the process for the \texttt{Source Role}
(Fig.~\ref{fig:role_source}).

\vspace{0.5cm}

\begin{figure}[htbp]
	\centering
	  \includegraphics[width=0.9\textwidth]{ea_assocPropsTarget}
	\caption{Properties for the target role of a reference}
	\label{fig:role_target}
\end{figure}
\FloatBarrier

\begin{figure}[htbp]
	\centering
    \includegraphics[width=0.9\textwidth]{ea_assocPropsSource}
	\caption{Properties for the source role of a reference}
	\label{fig:role_source}
\end{figure}
\FloatBarrier

\end{itemize}

To review these properties, the first value you edited was the role name. The \texttt{Navigation} value should have been automatically set to
\texttt{Na\-vi\-ga\-ble}. Without these two settings, getter and setter methods will not be generated.

\vspace{0.5cm}

Next, you set the \texttt{Multiplicity} value. In your target role (\texttt{Box}), you have allowed the creation of up to one target (\texttt{box}) reference
for every connected source (\texttt{Partition}). This means you could not have a single source connected to two targets (ie., one partition that belongs to two
boxes). In the source (\texttt{Partition}) role, you have specified that any target (in our case, \texttt{box}) can have any positive-sized number of sources.
Figure~\ref{fig:sketch_roles} sketches this idea.

\vspace{0.5cm}

\begin{figure}[htbp]
	\centering
    \includegraphics[width=0.6\textwidth]{sketch_multiplicities.pdf}
	\caption{The target and source roles of Leitner's Learning Box}
	\label{fig:sketch_roles}
\end{figure}
\FloatBarrier

Finally, you set the \texttt{Aggregation} value. In this case, \texttt{box} is a container for \texttt{Partition}s, and \texttt{containedPartition} is
consquently not.

\begin{itemize}
\item[$\blacktriangleright$] Take a moment to review how the \texttt{Aggregration} settings extend the \texttt{Multiplicity} rules. If you've done everything
right, your workspace should now resemble Fig.~\ref{fig:ereference_completed}, with a single \emph{bidirectional EReference} between \texttt{Box} and
\texttt{Partition}.

\vspace{1cm}

\begin{figure}[htbp]
	\centering
  \includegraphics[width=0.35\textwidth]{ea_relationBoxPartition.pdf}
	\caption{\texttt{Box} contains \texttt{Partition}s}
	\label{fig:ereference_completed}
\end{figure}
\FloatBarrier

\item[$\blacktriangleright$] Following the same process, create two unidirectional self-EReferences for \texttt{Partition}, and then a second bidirectional
EReference\footnote{To be precise, \emph{all} references in Ecore are actually unidirectional. A ``bidirectional'' reference in our metamodel is really two
mapped \texttt{EReferences} that are opposites of each other. We however, believe it is simpler to handle these pairs as single references, and prefer this
concise concrete syntax.} between \texttt{Partition} and \texttt{Card} (Fig.~\ref{fig:ereferences_all}). 

\vspace{1cm}

\begin{figure}[htbp]
	\centering
  \includegraphics[width=0.6\textwidth]{ea_classAttributes}
	\caption{All relations in our metamodel}
	\label{fig:ereferences_all}
\end{figure}
\FloatBarrier

\vspace{1cm}

\item[$\blacktriangleright$] You'll notice that the connection \texttt{card} and \texttt{Partition} is similar to that between \texttt{Partition} and
\texttt{Box}. This makes sense as a partition should be able to hold an unlimited amount of cards, but a card can only belong to one partition at a time.

\vspace{1cm}

\item[$\blacktriangleright$] Export your diagram to Eclipse and refresh your workspace. Your Ecore file should now resemble Fig~\ref{fig:model_allClasses}.

\vspace{1cm}

\begin{figure}[htbp]
	\centering
  \includegraphics[width=0.7\textwidth]{eclipse_modelDeclaredClasses}
	\caption{Your Ecore file}
	\label{fig:model_allClasses}
\end{figure}

\vspace{1cm}

\item[$\blacktriangleright$] All the required attributes and references for your learning box have now been set up. We encourage you to see how these are
declared in the textual syntax, starting on the immediate next page. In particular, check out Fig.~\ref{fig:allReferences}, where the classes are fully
declared, and Fig.~\ref{fig:bothConstraints}, where bidirectionality is explicitly specified as a constraint.

\fancyfoot[R]{$\triangleright$ \hyperlink{static:methods vis}{Next}}

\end{itemize}
