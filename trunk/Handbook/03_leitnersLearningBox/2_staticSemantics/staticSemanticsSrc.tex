\newpage
\section{Abstract syntax and static semantics}
\genHeader
\label{sec: staticSemantics}

The first step in creating any metamodel is defining the abstract syntax, also known as the type graph. This involves defining each class, its attributes,
references, and method signatures.

If you completed the demo in Part I, your Eclipse workspace will look slightly different than ours depicted in the screenshots. In an effort to keep things as
clear as possible, we have removed those files from our package explorer, but still recommend keeping them for future reference. 

Additionally, if you're continuing from the visual syntax, you can begin modeling this project in two different ways. You can either develop your metamodel in
the same workspace as the demo, or create a new metamodel. If you choose to do this, please note that the steps are exactly the same, but our package explorers
may not exactly match. This handbook has assumed you prefer the latter.

\fancyfoot[RO]{ $\triangleright$ \hyperlink{static:starting vis}{Next [visual]\hspace{0.2cm}} \\ $\triangleright$ \hyperlink{static:starting tex}{Next
[textual]}}

\newpage
\hypertarget{static:starting vis}{}
\subsection{Getting started in EA}
\visHeader
  
{\bf Note:} To continue with Part I's Demo files, open the \texttt{demo.eap} file in EA and ignore the first instruction below.

\begin{itemize}

\item[$\blacktriangleright$]  To begin, navigate to ``New Metamodel Project,'' and start a new visual project (this time without the demo specifications)
named \texttt{Leit\-ners\-Learn\-ing\-Box} (Fig.~\ref{fig:new_visModel}). Open the empty \texttt{.eap} file in EA.

\vspace{0.5cm}

\begin{figure}[htbp]
	\centering
  \includegraphics[width=0.8\textwidth]{eclipse_visNewMetamodelPlain}
	\caption{Starting a new visual project}
	\label{fig:new_visModel}
\end{figure}

\vspace{0.5cm}

\item[$\blacktriangleright$] In EA, select your working set and press the ``Add a Package'' button (Fig.~\ref{fig:new_package}). 

\begin{figure}[htbp]
	\centering
  \includegraphics[width=0.5\textwidth]{ea_addPackage}
	\caption{Add a new package to \texttt{MyWorkingSet}}
	\label{fig:new_package}
	\vspace{0.5cm}
\end{figure}

\clearpage

\item[$\blacktriangleright$] In the dialogue that pops up (Fig.~\ref{fig:new_package_name}), enter \texttt{LearningBoxLanguage} as the name of the new
package. Make sure \texttt{Class View} is selected, and click \texttt{OK}.

\vspace{0.5cm}

\begin{figure}[htbp]
	\centering
    \includegraphics[width=0.33\textwidth]{ea_nameEPackage.png}
	\caption{Enter the name of the new package}
	\label{fig:new_package_name}
\end{figure}
\FloatBarrier

\vspace{0.5cm}

\item[$\blacktriangleright$] Your \texttt{Project Browser} should now resemble Fig.~\ref{fig:new_package_completed}.

\vspace{0.5cm}

\begin{figure}[htbp]
	\centering
  \includegraphics[width=0.5\textwidth]{ea_newPackage}
	\caption{State after creating the new package.}
	\label{fig:new_package_completed}
\end{figure}
\FloatBarrier


\vspace{0.5cm}

\item[$\blacktriangleright$] Now select your new package and create a ``New Diagram'' (Fig.~\ref{fig:diagram}).

\vspace{0.5cm}

\begin{figure}[htbp]
	\centering
  \includegraphics[width=0.5\textwidth]{ea_addDiagram}
	\caption{Add a diagram.}
	\label{fig:diagram}
\end{figure}
\FloatBarrier

\clearpage

\item[$\blacktriangleright$] In the dialogue that appears (Fig.~\ref{fig:diagram_type}), choose \texttt{eMoflon Ecore Diagrams} and press \texttt{OK}. 

\begin{figure}[htbp]
	\centering
  \includegraphics[width=0.8\textwidth]{ea_chooseDiagramType}
	\caption{Select the ecore diagram type}
	\label{fig:diagram_type}
\end{figure}
\FloatBarrier

 
\item[$\blacktriangleright$] After creating the new diagram, your  \texttt{Project Browser} should now resemble Fig.~\ref{fig:diagram_completed}. You'll notice
that your \texttt{LearningBoxLanguage} package has been transformed into an EPackage.

\begin{figure}[htbp]
	\centering
  \includegraphics[width=0.5\textwidth]{ea_afterDiagramState}
	\caption{State after creating diagram}
	\label{fig:diagram_completed}
\end{figure}
\FloatBarrier

\item[$\blacktriangleright$] You can now already export your project to Eclipse,\footnote{If unsure how to perform this step, please refer to Part I, section
2.1} then refresh your \texttt{Package Explorer}. A new node, \texttt{My Working Set}\footnote{If you do not have the two distinct nodes, ensure your ``Top
Level Elements'' are set to \texttt{Working Sets}} should have appeared containing your EPackage (Fig.~\ref{fig:init_export}). You can see that a
\texttt{LearningBoxLanguage.ecore} file has been generated, and placed in ``model.'' This is your metamodel that will contain all future types you create in
your diagrams.

\clearpage

\vspace*{2cm}

\begin{figure}[htbp]
	\centering
  \includegraphics[width=0.55\textwidth]{eclipse_visInitExport}
	\caption{Inital export to Eclipse}
	\label{fig:init_export}
\end{figure}

\vspace{1cm}

\item[$\blacktriangleright$] If you're interested in reviewing the overall project structure, the purposes of certain files and folders, read section 4.1 from
Part~I. Otherwise, continue to the next to section learn how to declare classes and attributes.

\fancyfoot[R]{$\triangleright$ \hyperlink{static:classes vis}{Next}}
\end{itemize}

\clearpage
\subsection{Getting started with MOSL}
\texHeader
\hypertarget{static:starting tex}{}

\begin{itemize}

\item[$\blacktriangleright$] \hypertarget{static tex}{Begin a} new metamodel project from eclipse by navigating to the \texttt{New Metamodel} button on the
toolbar. In the dialog that appears, enter `LeitnersLearningBox' as the project name, and select \texttt{Textual (MOSL)}  (Fig.~\ref{fig:new_project}).

\begin{figure}[htbp]
	\centering
  \includegraphics[width=0.7\textwidth]{eclipse_newMetamodelTextPlain}
	\caption{Create a new metamodeling project}
	\label{fig:new_project}
\end{figure}

\item[$\blacktriangleright$] Expand the project as deep as it goes. Your Package explorer may look different than ours (Fig.~\ref{fig:expanded_folders}),
depending on whether or not you completed the Demo in Part I. In an effort to keep things clear as possible, we have removed them from our workspaces, but
still recommend keeping them for future reference.

\begin{figure}[htbp]
	\centering
  \includegraphics[width=0.5\textwidth]{eclipse_foldersExpanded}
	\caption{Expanded project files}
	\label{fig:expanded_folders}
\end{figure} 

\clearpage

\item[$\blacktriangleright$] You should be able to see two folders, \texttt{.temp}, and \texttt{MOSL}, and one subfolder, \texttt{MyWorkingSet}\footnote{if you
can't, make sure your `Top Level Elements' are set to \texttt{Working Sets}.}. We're most insterested in \texttt{MyWorkingSet}, which will store the collection
of files and \emph{unifies} the things we need for our Leitner's Box\footnote{for detailed review on the explorer structure, review Part I.}.

\vspace{0.5cm}

\item[$\blacktriangleright$] Right click on your current workspace folder, \texttt{MyWorkingSet}, and create a new subfolder. Name it
`LearningBoxLanguage'. This is now the container for all your modelling files. In essence, this is the location all your generated
files will be derived from.

\vspace{0.5cm}

\begin{figure}[htbp]
	\centering
  \includegraphics[width=0.8\textwidth]{eclipse_preBuild}
	\caption{}
	\label{fig:all_files}
\end{figure} 

\vspace{0.5cm}

\item[$\blacktriangleright$] To finalize initalization of your metamodel, navigate to ``Build (Without cleaning),'' found beside ``New Metamodel'' on the
toolbar (Fig.~\ref{fig:all_files}).

\item[$\blacktriangleright$] Navigate to ``LearningBoxLanguage/model.'' There, you'll be able to see a \texttt{LearningBoxLanguage.ecore} file. This is the
concrete model for your metamodel, which will adhere to any types and contraints you define within
``LeitnersLearningBox/MOSL/MyWorkingSet/LearningBoxLanguage.''

\item[$\blacktriangleright$] You've just finished creation process of a new metamodel project! If you would like to review specific details on the textual 
project structure, read Section 4.2 in Part I\footnote{\downLink}.

\fancyfoot[R]{$\triangleright$ \hyperlink{static:classes tex}{Next}}

\end{itemize}


\newpage
\subsection{Declaring classes and attributes}
\genHeader
\hypertarget{static:classes vis}{}

\begin{stepbystep}

\item Return to EA, and double-click your \texttt{LearningBoxLanguage} diagram to ensure it's open.

\vspace{0.5cm}

\item There are two ways for you to create your first \texttt{EClass}. First, to the left of the workbench, a \emph{Toolbox} containing
the Ecore types available for metamodelling should have appeared (\Cref{ea:eclass}).\footnote{If not, choose ``Diagram/Diagram Toolbox'' to show the
current toolbox (Alt+ 5)} Click on the \texttt{EClass} icon then somewhere in the diagram to create a new object. Alternatively, you can click in the diagram and press
\texttt{space} to invoke the toolbox context menu, then select \texttt{EClass}.

\vspace{0.5cm}

\begin{figure}[htbp]
	\centering
  \includegraphics[width=0.7\textwidth]{../../org.moflon.doc.handbook.02_leitnersLearningBox/2_staticSemantics/2_definingClasses/dcVisImages/ea_createEClass}
	\caption{Create an EClass}
	\label{ea:eclass}
\end{figure}

\vspace{0.5cm}

\item In the dialogue that pops-up, set \texttt{Box} as the name and click \texttt{OK} (\Cref{ea:eclass_properties}).
This dialogue can always be invoked again by double-clicking the EClass, or by pressing \texttt{Alt} and single-clicking. It contains many other properties that we'll investigate later in the handbook. In general, a similar properties dialogue can be opened in the same fashion for almost every element in EA.

\clearpage

\begin{figure}[ht]
	\centering
  \includegraphics[width=0.9\textwidth]{../../org.moflon.doc.handbook.02_leitnersLearningBox/2_staticSemantics/2_definingClasses/dcVisImages/ea_propertiesEClass}
	\caption{Edit the properties of an EClass}
	\label{ea:eclass_properties}
\end{figure}

\item After creating \texttt{Box}, your EA workspace should resemble \Cref{ea:eclass_completed}.

\vspace{0.5cm}

\begin{figure}[htbp]
	\centering
  \includegraphics[width=1\textwidth]{../../org.moflon.doc.handbook.02_leitnersLearningBox/2_staticSemantics/2_definingClasses/dcVisImages/ea_afterBoxCreation}
	\caption{State after creating \texttt{Box}}
	\label{ea:eclass_completed}
\end{figure}

\item Now create the \texttt{Partition} and \texttt{Card} EClasses the same way, until your workspace resembles
\Cref{ea:all_eclasses}. These are the main classes of your learning box metamodel.

\vspace{0.5cm}

\item Lets add some attributes! Either right-click on \texttt{Box} to activate the context menu and choose ``Features \&
Properties/Attributes..'' (\Cref{ea:attribute}), or press \texttt{F9} to open the editing dialogue.

\begin{figure}[htbp]
	\centering
  \includegraphics[width=\textwidth]{../../org.moflon.doc.handbook.02_leitnersLearningBox/2_staticSemantics/2_definingClasses/dcVisImages/ea_createPartitionCard}
	\caption{All EClasses for the metamodel}
	\label{ea:all_eclasses}
\end{figure}

\begin{figure}[htbp]
	\centering
  \includegraphics[width=\textwidth]{../../org.moflon.doc.handbook.02_leitnersLearningBox/2_staticSemantics/2_definingClasses/dcVisImages/ea_contextAddAttribute}
	\caption{Context menu for an EClass}
	\label{ea:attribute}
\end{figure}
\FloatBarrier

\item Enter \texttt{name} as the name of the attribute, select \texttt{EString} as its type from the drop-down menu, and press
\texttt{Close} (\Cref{ea:attribute_properties}). New attributes for the same EClass can be added by clicking on \texttt{New Attribute}...\,.

\vspace{1.0cm}

\begin{figure}[htbp]
	\centering
  \includegraphics[width=0.9\textwidth]{../../org.moflon.doc.handbook.02_leitnersLearningBox/2_staticSemantics/2_definingClasses/dcVisImages/ea_addAttributesDialogue}
	\caption{Adding attributes to an EClass}
	\label{ea:attribute_properties}
\end{figure}

\vspace{0.5cm}

\item Add the remaining attributes analogously to each EClass until your workspace resembles \Cref{ea:attribute_completed}.

\vspace{0.5cm}

\item Save and export to Eclipse. After refreshing your workspace, your \texttt{.ecore} model can now be expanded as it includes
every class and attribute from your metamodel. So far, so good!

\newpage

\vspace*{3cm}

\begin{figure}[htbp]
	\centering
  \includegraphics[width=0.40\textwidth]{../../org.moflon.doc.handbook.02_leitnersLearningBox/2_staticSemantics/2_definingClasses/dcVisImages/ea_allAttributes}
	\caption{Main EClasses declared with their attributes}
	\label{ea:attribute_completed}
\end{figure}
\FloatBarrier

\end{stepbystep}

\newpage
\subsection{Declaring classes and attributes}
\texHeader
\hypertarget{static:classes tex}{}

\begin{itemize}

\item[$\blacktriangleright$] Right click your \texttt{LearningBoxLanguage} model and create your first EClass by navigating to ``New/EClass.'' Name it
\texttt{Box}.

\item[$\blacktriangleright$] The class editor should automatically open. Let's add the first two EAttributes of our program, \texttt{name} and
\texttt{stringRep}. eMoflon offers type completion to help you with this task. Go to an empty line and press \texttt{cntrl + space}.

\item[$\blacktriangleright$] Given that your EClass is empty, you'll be presented with a short list of declaration template suggestions
(Fig.~\ref{fig:typeComp_Main}). The first four items are relevant for method signatures, so select \texttt{attribute} near the bottom, and create \texttt{name}
of type \texttt{EString}.

\begin{figure}[htbp]
	\centering
  \includegraphics[width=0.5\textwidth]{eclipse_typeCompletion_main}
	\caption{eMoflon's type completion}
	\label{fig:typeComp_Main}
\end{figure} 

\item[$\blacktriangleright$] Type completion also supports you by providing a list of types. Start to create a second attribute, \texttt{stringRep}, but after
the \texttt{``:''} operator, press the same hotkeys. The new list provides all the types currently available - any further EClasses you create will also appear
in this list. Begin to type \texttt{EString} until the option is highlighted and press \texttt{enter} ({\bf get image}). Your workspace should now resemble
(Fig.~\ref{fig:boxDeclaration}).

\begin{figure}[h!]
	\centering
  \includegraphics[width=0.5\textwidth]{eclipse_classBoxDeclaration}
	\caption{Newly created \texttt{Box} EClass}
	\label{fig:boxDeclaration}
\end{figure} 
\FloatBarrier

\newpage

\item[$\blacktriangleright$] Now create two empty EClasses in your model, \texttt{Partition} and \texttt{Card}.

\item[$\blacktriangleright$] In \texttt{Partition}, add two \texttt{EInt} attributes, \texttt{index} and \texttt{partitionSize}.

\item[$\blacktriangleright$] In \texttt{Card}, create three \texttt{EString} attributes, \texttt{back}, \texttt{face} , and \texttt{partitionHistory}.

\item[$\blacktriangleright$] If you've done everything correctly, your workspace should now resemble Fig.~\ref{fig:workspaceClassAttributes}.

\vspace{0.5cm}

\begin{figure}[htbp]
	\centering
  \includegraphics[width=1.0\textwidth]{eclipse_workspaceTexClassAttributes}
	\caption{Declaration of all EClasses and attributes}
	\label{fig:workspaceClassAttributes}
\end{figure} 

\vspace{0.5cm}

\item[$\blacktriangleright$] That's it for declaring class attributes! Feel free to build your project again and view the changes in the \texttt{.ecore}
mode, and the generated files in ``gen" and ``src." On a final note, while some languages (such as Java) allow the declaration of several small classes (such as
these three) in the same file, when tooling with eMolfon, we keep them separated. Don't worry - we'll explain this later in the handbook. As for now, continue
to the next section to start creating references between these EClasses.

% \fancyfoot[R]{$\triangleright$ \hyperlink{static:references splash}{Next}}

\end{itemize}


\newpage
\subsection{Connecting your classes}
\genHeader
\hypertarget{static:references splash}{}

\texttt{}
\emph{}

At this point, you've declared your types and attributes, but what good are those if none of them can communicate with each other? We need to create some
\emph{EReferences}!

There are 3 properties that must be set in order to create an EReference: \emph{Navigation}, \emph{Aggregation}, and
\emph{Multiplicity}. As you can probably guess, they're declared differently in each syntax, but let's review the concepts since they're the same.

Firstly, \texttt{Navigable} ends are mapped to class attributes with getters and setters in Java, and therefore \emph{must} have a specified name and
multiplicity for successful code generation. Corresponding values for \texttt{Non-Navigable} ends can  be regarded as additional documentation, and do not have
to be specified.

Secondly, the \texttt{Multiplicity} refers to that of a reference\footnote{Don't get multiplicity of a reference confused with the multiplicity of an element!
Multiplicity of an element defines the permitted range of individual instances, rather than types.}. It controls if the relation is mapped to a
Java Collection (\texttt{*},~\texttt{1..*},~\texttt{0..*}), or to a single valued class attribute (\texttt{1}, \texttt{0..1}). We'll explain this setting in
detail when you encounter them in your syntax instructions.

Lastly, in Ecore, the \texttt{Aggregration} values of a reference can be \texttt{none}, \texttt{shared}, or \texttt{com\-po\-site}. Composite means that the
current role is that of a \emph{container} for the opposite role. You'll see in our example that \texttt{Box} is a container for several \texttt{Partition}s.
This has a series of consequences: (1) every element must have a container, (2) an element cannot be in more than one container at the same time, and (3) a
container's contents are deleted together with the container. Conversely, non-composite (\texttt{none}) means that the current role is not that of a container,
and the rules for containment do not hold (in other words, the reference is set a simple `pointer'). The \texttt{shared} setting is beyond the scope of this
handbook.


\fancyfoot[RO]{ $\triangleright$ \hyperlink{static:references vis}{Next [visual]\hspace{0.2cm}} \\ $\triangleright$ \hyperlink{static:references tex}{Next
[textual]}}

\newpage
\subsubsection{Connecting your classes in EA}
\visHeader
\hypertarget{static:references vis}{}

\begin{itemize}

\item[$\blacktriangleright$] A fundamental gesture in EA is \emph{Quick Link}. Quick Link is used to create links between elements in a context-sensitive
manner. To use Quick Link, choose an element and note the little black arrow in its top-right corner (Fig.~\ref{fig:quicklink}).

\begin{figure}[htbp]
	\centering
  \includegraphics[width=0.4\textwidth]{ea_quickLink}
	\caption{Quick Link is a central gesture in EA}
	\label{fig:quicklink}
\end{figure}
\FloatBarrier

Click this black arrow and `pull' to the element you wish to link to. To start, quick-link from \texttt{Box} to \texttt{Partition}. In the context menu
that appears, select ``Create Bidirectional EReference'' (Fig.~\ref{fig:ereference}).

\begin{figure}[htbp]
	\centering
  \includegraphics[width=0.6\textwidth]{ea_eReferenceBidirectional}
	\caption{Create a reference via Quick Link}
	\label{fig:ereference}
\end{figure}
\FloatBarrier

\item[$\blacktriangleright$] Double click the reference to invoke a dialogue. Here you can change the reference direction and enter a name. Feel free to leave
the \texttt{Name} value blank - this property is only used for documentation purposes, and not relevant to code generation.

\item[$\blacktriangleright$] In the same dialogue, go to ``Target Role'' and enter the values in Fig.~\ref{fig:reference_ends} to set the properties for the
`target' end of the reference (the \texttt{Box} role). As you can see, the default target is set to the class you linked \emph{from}, and the default source is
the class you linked \emph{to}. 

\vspace{0.5cm}

\item[$\blacktriangleright$] If you decided to ignore the previous instructions, and went from \texttt{Partition} to \texttt{Box}, the only difference between
the two references are their titles. The following information will still be the same! In this window, it's important not to forget to confirm and modify the
\texttt{Role}, \texttt{Navigability}, \texttt{Multiplicity}, and \texttt{Aggregation} settings for the target.  Repeat the process for the \texttt{Source Role}.

\vspace{1cm}

\begin{figure}[htbp]
	\centering
	  \includegraphics[width=0.73\textwidth]{ea_assocPropsTarget}\\
	\caption{Properties for the target role of a reference}
	\label{fig:reference_ends}
\end{figure}
\FloatBarrier

\begin{figure}[htbp]
	\centering
    \includegraphics[width=0.73\textwidth]{ea_assocPropsSource}
	\caption{Properties for the source role of a reference}
	\label{fig:sketch_roles}
\end{figure}
\FloatBarrier

\end{itemize}

To explain, the first value you completed was the navigation name. The \texttt{Navigation} should have been automatically set to \texttt{Navigable}. Without these,
the correct code cannot be generated. 

Next, you set the \texttt{Multiplicity} value. In your target role (\texttt{Box}), you have allowed the creation of up to
one target (\texttt{box}) reference for every connected source (\texttt{Partition}). This means you could not have a single source connected to two targets
(ie., one partition that belongs to two unique boxes). In the source (\texttt{Partition}) role, you have specified that any target (in our case, \texttt{box})
can have any positive-sized number of sources. Figure~\ref{fig:sketch_roles} sketches this idea\footnote{Remember, if you quick-linked in the opposite
direction, the concept is the same, but the role titles are reversed.}. 

Finally, you set the \texttt{Aggregation} value. In this case, \texttt{box} is a
container for \texttt{Partition}s, and \texttt{containedPartition} doesn't need to adhere to any rules.

\begin{figure}[htbp]
	\centering
    \includegraphics[width=0.6\textwidth]{sketch_multiplicities.pdf}
	\caption{The target and source roles of Leitner's Learning Box}
	\label{fig:sketch_roles}
\end{figure}
\FloatBarrier

\begin{itemize}

\item[$\blacktriangleright$] Take a moment to review how the \texttt{Aggregration} settings extend the \texttt{Multiplicity}. If you've done everything right,
your workspace should now resemble Fig.~\ref{fig:ereference_completed}, with a single bidirectional EReference between \texttt{Box} and \texttt{Partition}.

\vspace{1cm}

\begin{figure}[htbp]
	\centering
  \includegraphics[width=0.4\textwidth]{ea_relationBoxPartition.pdf}
	\caption{\texttt{Box} contains \texttt{Partition}s}
	\label{fig:ereference_completed}
\end{figure}
\FloatBarrier

\clearpage
\item[$\blacktriangleright$] Create another bidirectional EReference\footnote{To be precise, \emph{all} references in Ecore are actually unidirectional.
A ``bidirectional'' reference in our metamodel is really two mapped \texttt{EReferences} that are opposites of each other.
We however, believe it is simpler to handle these pairs as single references, and prefer this concise concrete syntax.} between \texttt{Partition} and
\texttt{Card}, then two unidirectional self-references for \texttt{Partition}. Your final workspace should resemble Fig.~\ref{fig:ereferences_all}.

\vspace{1cm}

\begin{figure}[htbp]
	\centering
  \includegraphics[width=0.5\textwidth]{ea_relationsAll.pdf}
	\caption{All relations in our metamodel}
	\label{fig:ereferences_all}
\end{figure}

\FloatBarrier

\item[$\blacktriangleright$] All the attributes and references required for your learning box have now been set up. We encourage you to \emph{not} click the link below, and
see how these are declared in the textual syntax, in the next section. In particular, check out Fig.~\ref{fig:allReferences}!

\fancyfoot[R]{$\triangleright$ \hyperlink{static:methods vis}{Next}}

\end{itemize}

\newpage
\subsubsection{Creating EReferences with MOSL}
\texHeader
\hypertarget{static:references tex}{}

In MOSL, the declaration of an EReference is simple - you set each property according to the following syntax (specified in simple EBNF, if you know what that
is):

\syntax{ [ `<>' ] `-' role\_name `(' multiplicity `)' `->'  target\_type \\
\\
With:\\
role\_name := STRING \\
multiplicity := `0..1' $|$ `0..*' $|$ `1' $|$ \ldots \\
target\_type := STRING \\
}

The source type is determined by the EClass in which the EReference is placed. You can signal an aggregation EReference by including the sideways diamond before
the arrow symbol. Don't worry - you don't have to remember this syntax. Our type completion provides a \texttt{reference} template when you activate the hot
keys. Try it out!

\begin{itemize}

\item[$\blacktriangleright$] Open \texttt{Box.eclass} in the editor and add a \emph{container reference} named \texttt{containedPartition} with a
multiplicity of zero to infinity, from \texttt{Box} to \texttt{Partition} (Listing~\ref{eclipse:cpartitionReference}, Line \ref{mosl:containedPartition}). This EReference means a \texttt{Box}
can hold an infinite number of partitions.

\vspace{0.5cm}


\item[$\blacktriangleright$] Now add a \emph{simple reference} to \texttt{Partition}. Name it \texttt{box}, and allow it to hold up to one \texttt{Box}
(Listing~\ref{eclipse:moslBoxReference}, Line~\ref{mosl:referencedBox}). This means a single partition can belong to either zero, or one \texttt{Box}, and that's it. It can't belong to two different
boxes at the same time.

\vspace{0.5cm}

\item[$\blacktriangleright$] Congratulations, you have just built your first pair of EReferences! To see how this is depicted visually, check out
Fig.~\ref{ea:ereference_completed} from the previous subsection.

\vspace{0.5cm}

\item[$\blacktriangleright$] Now, lets create another pair of EReferences between \texttt{Partition} and \texttt{Card}. If you think about it, it's really not
all that different from the relation between \texttt{Box} and \texttt{Partition}. A \texttt{Partition} should be able to hold an unlimited amount of
\texttt{Card}s, but a \texttt{Card} should only be allowed to belong to zero or one \texttt{Partition}s. Name the two new EReferences
\texttt{card}, and \texttt{cardContainer} (Listing~\ref{eclipse:moslBoxReference}, Line~\ref{mosl:containedCard} and Listing~\ref{eclipse:moslCardReference}, Line~\ref{mosl:cardContainer}).

\vspace{0.5cm}

\item[$\blacktriangleright$] The next step is to construct two connections between \texttt{Partition}s so cards can be moved between their previous and next
partitions in the box. Create two new simple references, named \texttt{previous}, and \texttt{next}, each with a \texttt{0..1} multiplicity (Listing~\ref{eclipse:moslBoxReference}, Line~\ref{mosl:nextPartition} and~\ref{mosl:previousPartition}).

\vspace{0.5cm}

\item[$\blacktriangleright$] If you have done everything correctly, your EClasses should now resemble Listings~\ref{eclipse:cpartitionReference},~\ref{eclipse:moslBoxReference} and~\ref{eclipse:moslCardReference}. 

\vspace{0.5cm}

\lstinputlisting[style=eclass, label=eclipse:cpartitionReference ,caption={Creating a \emph{container reference} in \texttt{Box}}] {../2_staticSemantics/3_connectingClasses/ccTexCode/Box.txt} 


\vspace{0.5cm}

\lstinputlisting[style=eclass, label=eclipse:moslBoxReference,caption={Creating  \emph{references} in \texttt{Partition}}] {../2_staticSemantics/3_connectingClasses/ccTexCode/Partition.txt} 


\vspace{0.5cm}

\lstinputlisting[style=eclass, label=eclipse:moslCardReference,caption={Creating a \emph{simple references} in \texttt{Card}}] {../2_staticSemantics/3_connectingClasses/ccTexCode/Card.txt} 


\clearpage

At this point, all of your EReferences have been created! The problem is, suppose you set the \texttt{containedPartition} EReference in a particular
\texttt{Box}. That's great, you would now have the box containing one partition. However, if you went and examined that partition independently, its
\texttt{box} EReference would still be null. We still need to set up the link between these EReferences so that when one is updated, the other will be too.

\vspace{0.5cm}

\item[$\blacktriangleright$] Navigate to the \texttt{\_ModelConfiguration.mconf} file. You can see it has a single \texttt{opposites} scope that's currently empty.
Constraints follow the syntax below: 

\syntax{reference `<->' reference \\
\\
With:\\
reference := reference\_name `:' source\_type \\
reference\_name := STRING \\
source\_type := STRING \\}

This statement sets the two EReferences to be opposites of one another, i.e., the connection between EClasses will be bidirectional. As you can see, syntax here
is slightly different than that of a standard EReference. Instead of the reference type trailing the colon operator, it has switched to become the source type.

\vspace{0.5cm}

\item[$\blacktriangleright$] To begin, press \texttt{Ctrl + Space} and complete
the template with the following:

\syntax{containedPartition : Box <-> box : Partition}

\item[$\blacktriangleright$] Reviewing the \texttt{Partition} EClass, its easy to see that \texttt{previous} and \texttt{next} are certainly not
opposites,\footnote{Review the rules depicted in Fig.~\ref{fig:membox_illustration}} but we do need to establish an opposing link between
\texttt{card} and its \texttt{cardContainer}. Follow the same steps until your constraint file resembles Listing~\ref{eclipse:bothConstraints}.

\vspace{5mm}

\lstinputlisting[style=mconf, label=eclipse:bothConstraints,caption={The completed constraints file}] {../2_staticSemantics/3_connectingClasses/ccTexCode/_ModelConfiguration.txt} 

\newpage

\item[$\blacktriangleright$] Now the EReferences for your learning box are complete! To see how each of the classes, attributes, and references
are depicted in the visual syntax, check out Fig.~\ref{ea:ereferences_all} from \hyperlink{sec:static vis}{section 2.1}. Otherwise, build your project to
make sure there are no errors, and continue to the next section to finalize the declaration of your EClasses.

\jumpSingle{static:methods tex}

\end{itemize}



\newpage
\subsection{Method Signatures}
\visHeader
\hypertarget{static:methods vis}{}

To finish defining our types, lets define the \emph{signatures} of some operations they'll eventually support.

\begin{enumerate}
  
\item[$\blacktriangleright$] Right-click \texttt{Partition} to invoke the context-menu as depicted in Fig.~\ref{fig:add_operation}

\item[$\blacktriangleright$] Select \texttt{Partition} and either right-click to invoke the context-menu (Fig.~\ref{fig:add_operation})  and choose ``Features \&
Properties/Operations..'' or simply press \texttt{F10}.

\begin{figure}[htbp]
	\centering
  \includegraphics[width=0.8\textwidth]{ea_contextAddOperation}
	\caption{Add an operation}
	\label{fig:add_operation}
\end{figure}
\FloatBarrier

\item[$\blacktriangleright$] In the dialogue that pops-up (Fig.~\ref{fig:operation_properties}), enter \texttt{empty} as the \texttt{Name} of the operation,
leave the \texttt{Return Type} as \texttt{void}, and press \texttt{Save}. 

\vspace{0.5cm}

\item[$\blacktriangleright$] In the same dialogue, press \texttt{New} to add a second operation, \texttt{removeCard}, and edit the values as seen in 
Fig.~\ref{fig:operation_parameters}. Notice that the \texttt{Return Type} can be chosen by either the drop-down menu for
primitives (e.g. \texttt{EBoolean}), or via the `\texttt{\ldots}' button (highlighted in Fig.~\ref{fig:operation_properties}) for types you've established in
the metamodel (e.g. \texttt{Card}).
\vspace{-.3cm}
\begin{quote}
{ \small
$\textbf{Please note:}$ Non-primitive types \emph{must} be chosen via the `\texttt{\ldots}' button. It allows you to browse for the corresponding elements in
your project. Simply typing them won't work!
}
\end{quote}

\begin{figure}[htbp]
	\centering
  	\includegraphics[width=0.8\textwidth]{ea_operationEmpty}
	\caption{EClass properties editor}
	\label{fig:operation_properties}
\end{figure}

\begin{figure}[htbp]
	\centering
  \includegraphics[width=0.85\textwidth]{ea_operationRemoveCard}
	\caption{Parameters and return type}
	\label{fig:operation_parameters}
\end{figure}

\item[$\blacktriangleright$] Parameters can be added by pressing \texttt{Edit}\footnote{You must save the operation before this option will become active} and
completing the dialouge. Please remember that you must also use either the drop-down menu, or the `\texttt{\ldots}' button to select the type or else validation
will fail.

\item[$\blacktriangleright$] Repeat this process for the \texttt{check} operation (with two parameters), as depicted in Fig.~\ref{fig:operation_partition}. 

\item[$\blacktriangleright$] If you've done everything right, your dialogue should now contain three methods - \texttt{check}, \texttt{empty}, and
\texttt{removeCard} - each with the corresponding parameters and return types in Fig.~\ref{fig:operation_partition}.

\begin{figure}[htbp]
	\centering
  \includegraphics[width=0.9\textwidth]{ea_operationCheck}
	\caption{All operations in \texttt{Partition}}
	\label{fig:operation_partition}
\end{figure}

\item[$\blacktriangleright$] Add all operations analogously for \texttt{Box} and \texttt{Card} until your metamodel closely resembles
Fig.~\ref{fig:metamodel_complete}.\footnote{Please note that names of parameters may not be displayed by default in EA}

\item[$\blacktriangleright$] To finish, export the metamodel for code generation in Eclipse, and examine the model once again. Each signature should have
appeared in their respective EClass.

\begin{figure}[htbp]
	\centering
  \includegraphics[width=0.63\textwidth]{ea_metamodelComplete.png}
\caption[Complete metamodel for our learning box.]{Complete metamodel for our learning box}
	\label{fig:metamodel_complete}
\end{figure}

\item[$\blacktriangleright$] To see how this complete metamodel is represented in the textual syntax, examine Fig.~\ref{fig:workspaceMethods} in the following
section. 

\end{enumerate}

\fancyfoot[R]{$\triangleright$ \hyperlink{validation vis}{Next}}

\newpage
\subsection{Method Signatures}
\texHeader
\hypertarget{static:methods tex}{}

\begin{itemize}

\item[$\blacktriangleright$] We're nearing the end of our model creation! One of the last things we need to do is to make the model \emph{do} something. After
all, a model that only stores attributes and references is a bit boring, right?

\item[$\blacktriangleright$] Let's set up the operations we want each EClass to do by declaring their \emph{signatures} which follow the syntax below:
\syntax{name `(' argument* `)' `:' return\_type \\
\\
With:\\
name, arguments, return\_type := STRING}

\item[$\blacktriangleright$] Starting with the \texttt{Partition} EClass, we want a partition to be able to do three things: compare the answer on a
\texttt{Card} with a guess and return a true/false response, remove a specific card from the partition, or empty itself of all cards.

\item[$\blacktriangleright$] Start with the \texttt{empty} method. It won't need any parameters, and it doesn't return anything. Declare this via:
\syntax{empty() : void}

\item[$\blacktriangleright$] Create two more functions for \texttt{Partition} the same way. We'll need a \texttt{removeCard} method that accepts and returns a
\texttt{Card}, as well as a EBoolean \texttt{check} method that accepts a \texttt{Card} and an \texttt{EString} guess. 

\item[$\blacktriangleright$] Your \texttt{Partition} EClass should now resemble Fig.~\ref{fig:partitionMethods}.

\vspace{0.5cm}

\begin{figure}[htbp]
	\centering
  \includegraphics[width=0.6\textwidth]{eclipse_partitionMethods}
	\caption{The completed \texttt{Partition} EClass}
	\label{fig:partitionMethods}
\end{figure}

\vspace{0.5cm}

\item[$\blacktriangleright$] What needs to be done in the \texttt{Card} EClass? Well, in order to check the card, we'll need to be able to look at the flip
side. We'll also need to print whatever is on the current side. Create two paramater-less void functions, \texttt{invert} and \texttt{printCard}. 

\item[$\blacktriangleright$] Finally, what do we want to do with \texttt{Box}? In summary, we want a \texttt{Box} to:

\begin{description}
  \item[\texttt{determineNextSize():EInt}] Calculate how large a new partition in the box should be
  \item[\texttt{grow():void}] Increase the box by adding a new partition \update
  \item[\texttt{toString():EString}] Produce a string representation of the box with all its contents
  \item[\texttt{addToStringRep(card:Card):void}] Update the current string representation to include \texttt{card}
\end{description}

\item[$\blacktriangleright$] Implement the above signatures, and your entire workspace should now resemble Fig.~\ref{fig:workspaceMethods}.

\item[$\blacktriangleright$] Congratulations! You have now created a metamodel for our Learning Box using eMoflon's textual syntax! To see how
this looks in the visual syntax, check out Fig.~\ref{fig:metamodel_complete} from the previous section. As a final step, make sure you build the project and
wait for the package explorer to refresh. 

\newpage

\fancyfoot[R]{ $\triangleright$ \hyperlink{validation tex}{Next}}

\begin{figure}[htbp]
	\centering
  \includegraphics[width=0.7\textwidth]{eclipse_classesFullyDeclared}
	\caption{Completed method signatures}
	\label{fig:workspaceMethods}
\end{figure}
\FloatBarrier

\end{itemize}

\visHeader

\hypertarget{validation vis}{Our} EA extension provides rudimentary support for validating both the static semantics (Ecore) and dynamic semantics (SDM) of
metamodels. Validation results are displayed and, in some cases, even ``quick fixes'' to automatically solve the problems are offered. In addition to reviewing
your model, the validation option automatically exports the current model to your eclipse workspace.

\subsection{eMoflon validation support in EA}

\begin{enumerate}
\item[$\blacktriangleright$] If not already active, make the eMoflon control panel visible in EA by choosing ``Extensions/\-Add-In Windows''. This should
display a new output window, as depicted in Fig.~\ref{fig:validation_output}. Many users prefer this interface, as it provides quick access to all of eMoflon's
features, as opposed to the drop down menu under ``Exensions/MOFLON::Ecore Addin" which only offers limited functionality.

\begin{figure}[htbp]
	\centering
  \includegraphics[width=0.9\textwidth]{ea_controlPanel}
	\caption{Activating the validation output window}
	\label{fig:validation_output}
\end{figure}
\FloatBarrier

\clearpage
\item[$\blacktriangleright$] To start the validation, choose ``Validate all'' in the ``Validate" section of the control panel
(Fig.~\ref{fig:validation_menu}). If you haven't made any mistakes while modelling your \texttt{LearningBoxLanguage} so far, the validation results window
should remain empty, indicating your metamodels are free of errors.

\begin{figure}[htbp]
	\centering
  \includegraphics[width=1.0\textwidth]{ea_startValidation}
	\caption{Starting the validation}
	\label{fig:validation_menu}
\end{figure}
\FloatBarrier
\end{enumerate}

If an error did appear, the validation system would try to suggest a ``Quick Fix.'' Why don't we try to get familiar and examine the validation and quick fix
features in detail? Let's add two small modelling errors in \texttt{LearningBoxLanguage}.

\begin{enumerate}
\item[$\blacktriangleright$] Create a new Eclass in the \texttt{Learning\-Box\-Language} diagram. You can retain the default name \texttt{EClass1}. Let's
assume, you wish to delete this class from your metamodel.

\item[$\blacktriangleright$] Select the rouge class in the diagram and press the \texttt{Delete} button on your keyboard. Note that \texttt{EClass1} still
exists in the project browser (and thus in your metamodel).

\item[$\blacktriangleright$] Run the validation test, and notice the new \texttt{Information} message in the validation output
(Fig.~\ref{fig:validation_information}).

\begin{figure}[htbp]
	\centering
  \includegraphics[width=1.0\textwidth]{EA_validationDeleteElement}
	\caption{Validation information error: element still exists}
	\label{fig:validation_information}
\end{figure}

This message informs you that \texttt{EClass1} is not on any diagram, and seeing as it is still in the model, that this \emph{could} be a mistake. As you
can see, just pressing the \texttt{Delete} button is not the proper way of removing a class from a metamodel - It only removes it from the current
diagram!\footnote{Deleting elements properly and other EA specific aspects are discussed in detail in Part VI: Micellaneous}

\item[$\blacktriangleright$] Suppose you were inpecting a different diagram, and were not on the current screen. To navigate to the problematic element in the
\texttt{Project Browser}, click \emph{once} on the information message.

\item[$\blacktriangleright$] To check to see if there are any quick fixes available, \emph{double} click the information message to invoke the ``QuickFix''
dialogue. In this case, there is one quick fix which suggests (properly) deleting the element from the model (Fig.~\ref{fig:quick-fix1}). Since this was indeed
the intent, click \texttt{Ok}.

\begin{figure}[htbp]
	\centering
  \includegraphics[width=0.55\textwidth]{ea_quickFixElements}
	\caption{Quick fix for elements that are not on any diagram}
	\label{fig:quick-fix1}
\end{figure}
\FloatBarrier

\item[$\blacktriangleright$] \texttt{EClass1} should now be correctly removed from your metamodel. Your metamodel should now be error-free again as indicated by
the validation output window.

\item[$\blacktriangleright$] To make an error that leads to a more critical message than ``information,'' double click the navigable reference end
\texttt{previous} of the class \texttt{Partition}, and delete its role name as depicted in Fig.~\ref{fig:delete-role-name}. Affirm with \texttt{OK}.

\begin{figure}[htbp]
    \centering
  \includegraphics[width=1.0\textwidth]{EA_validationDeleteRoleName}
    \caption{Deleting a navigable role name of a reference}
    \label{fig:delete-role-name}
\end{figure}

\item[$\blacktriangleright$] You should now see a new \texttt{Fatal Error} in the validation output, stating that a navigable end \emph{must} have a role name.
Double click the error to view the quick fix menu (Fig.~\ref{fig:fatal-error}). As navigable references are mapped to data members in a Java class, omitting the
name of a navigable reference makes code generation impossible (data members (i.e., class variables) must have a name).

\begin{figure}[htbp]
	\centering
  \includegraphics[width=0.6\textwidth]{EA_quickFixFatal}
	\caption{Fatal error after deleting a navigable role name}
	\label{fig:fatal-error}
\end{figure}

\item[$\blacktriangleright$] Given there are no automatic solutions, correct your metamodel manually by setting the name of the navigable reference back to
\texttt{previous}.

\item[$\blacktriangleright$] Ensure that your metamodel closely resembles Fig.~\ref{fig:metamodel_complete} again, and that there are no error messages before
you proceed with the rest of this handbook.
\end{enumerate}

\vspace*{1cm}

As you may have already noticed, eMoflon distinguishes between five different types of validation messages:
\begin{description}
  \item[Information:]~\\
  This is only a hint for the user and can be safely ignored if you know what you're doing.
  Export and code generation should be possible, but certain naming/modelling conventions are violated, or a problematic situation has been detected.
  
  \item[Warning:]~\\ Export and code generation is possible, but only with defaults and automatic corrections applied by the code generator.
  As this might not be what the user wants, such cases are flagged as warnings (e.g., omitting the multiplicity at references which is automatically set by the
  code generator to 1).
  Being as explicit as possible is often better than relying on defaults.
  
  \item[Error:]~\\ Although the metamodel can be exported from EA, it is not Ecore conform, and code generation will not be possible.
 
  \item[Fatal Error:]~\\ The metamodel cannot be exported as required information, such as names or classifiers of model elements, are incorrectly set or
  missing.
  
  \item[Eclipse Error:]~\\ Display error messages produced by our Eclipse plugin after an unsuccessful attempt to generate code. This is currently not actively
  used.

\end{description}

\fancyfoot[R]{ $\triangleright$ \hyperlink{sec:creatingInstance common}{Next task} }
\newpage
\texHeader

\subsection{ eMoflon support with the MOSL Builder }

\hypertarget{validation tex}{} Our MOSL language is accompanied with its own builder that proves support for validating the static semantics of your metamodel.
Integrated with the Eclipse IDE, if there's an error in your files when you save, a message will appear in the console. Lets try to make an error, just to
confirm it works.

Go to your \texttt{Partition} class and change the parameter type in \texttt{removeCard} from \texttt{Card} to \texttt{card}. Press save. An error should
immediately appear below the editor to inform you of your terrible mistake. Change your file back to the way it was, and the message should disappear.

In summary, MOSL extends the EClipse IDE to let you know if there are any problems with your metamodel. MOSL also lets you know whether or not your project has
recently been built. You may have noticed this feature in the previous sections. Similar to how Eclipse informs you your file has changed by placing a
\texttt{*} beside the name in the editor tab, MOSL places a \texttt{***} symbol beside your metamodel folder title in the package explorer to remind you to
build your project.

% \fancyfoot[R]{ $\triangleright$ \hyperlink{sec:creatingInstance common}{Next step} }


\newpage 
\genHeader
\subsection{Reviewing your metamodel}
\hypertarget{static review}{}

Before moving on, lets take a step back and review what we have accomplished. We have modelled a \texttt{Box} that can contain an arbitrary amount of
\texttt{Partition}s. A \texttt{Partition} in the \texttt{Box} has a \texttt{next} and \texttt{previous} \texttt{Partition} that can be set. Finally,
\texttt{Partition}s contain \texttt{Card}s.

A \texttt{Box} has a \texttt{name}, and can be extended by calling \texttt{grow}. A \texttt{Box} can print out its contents via the \texttt{toString} method.

The main method of the learning box is \texttt{Partition::check}, which takes a \texttt{Card} and the user's \texttt{EString} guess, and returns a \texttt{true}
or \texttt{false} value.

A \texttt{Partition} can also \texttt{remove} a specific \texttt{Card}, or empty itself of all  existing \texttt{Cards}. Last but not least, a
\texttt{Partition} has a \texttt{partitionSize} to indicate how many cards it should hold. Too many cards in the first partition could indicate that not
enough time has been dedicated to learning the terms. Too many near the end of the box could show that the vocabulary set is too easy, and probably
already mastered.

A \texttt{Card} contains the actual content to be learned as a question on the card's \texttt{face} and the answer on the card's \texttt{back}. \texttt{Card}s
also maintain a \texttt{partition\-History}, which can be used to keep track of how often a \texttt{Card} has been answered incorrectly.
This may indicate how difficult the \texttt{Card} is for a specific user, and remind them to spend more time on it. When learning a language, it makes
sense to be able to swap the target and source language and this is supported by \texttt{Card} via \texttt{invert} (turns the card around).

Examine the generated files in ``gen'', especially the default implementation for those methods that currently just throw an \texttt{OperationNotSupported}
exception. We shall see in later parts of this handbook how our code generator supports injecting hand-written implementation of methods into generated methods
and classes. With eMoflon however, we can actually model a large part of the dynamic semantics with ease, and only need to implement small helper methods (such
as those for string manipulation) by hand.

On a final note, we encourage you to review how the metamodel was constructed in the alternative syntax to the one you primarily used. Compare the
differences between modelling classes and references visually using a separate program (Enterprise Architect), then exporting to Eclipse to build, versus
modelling textually entirely within Eclipse. Which do you find easier to work with?

If you have had problems with this section, and, despite firmly believing everything is correct, things \emph{still} don't work, feel free to contact us at:\\ 
\eMoflonContact.

 
