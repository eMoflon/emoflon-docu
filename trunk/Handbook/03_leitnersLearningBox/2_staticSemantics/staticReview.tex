\newpage 
\genHeader
\subsection{Reviewing your metamodel}
\hypertarget{static review}{}

Before moving on, lets take a step back and review what we have accomplished. We have modelled a \texttt{Box} that can contain an arbitrary amount of
\texttt{Partition}s. A \texttt{Partition} in the \texttt{Box} has a \texttt{next} and \texttt{previous} \texttt{Partition} that can be set. Finally,
\texttt{Partition}s contain \texttt{Card}s.

A \texttt{Box} has a \texttt{name}, and can be extended by calling \texttt{grow}. A \texttt{Box} can print out its contents via the \texttt{toString} method.

The main method of the learning box is \texttt{Partition::check}, which takes a \texttt{Card} and the user's \texttt{EString} guess, and returns a \texttt{true}
or \texttt{false} value.

A \texttt{Partition} can also \texttt{remove} a specific \texttt{Card}, or empty itself of all  existing \texttt{Cards}. Last but not least, a
\texttt{Partition} has a \texttt{partitionSize} to indicate how many cards it should hold. Too many cards in the first partition could indicate that not
enough time has been dedicated to learning the terms. Too many near the end of the box could show that the vocabulary set is too easy, and probably
already mastered.

A \texttt{Card} contains the actual content to be learned as a question on the card's \texttt{face} and the answer on the card's \texttt{back}. \texttt{Card}s
also maintain a \texttt{partition\-History}, which can be used to keep track of how often a \texttt{Card} has been answered incorrectly.
This may indicate how difficult the \texttt{Card} is for a specific user, and remind them to spend more time on it. When learning a language, it makes
sense to be able to swap the target and source language and this is supported by \texttt{Card} via \texttt{invert} (turns the card around).

Examine the generated files in ``gen'', especially the default implementation for those methods that currently just throw an \texttt{OperationNotSupported}
exception. We shall see in later parts of this handbook how our code generator supports injecting hand-written implementation of methods into generated methods
and classes. With eMoflon however, we can actually model a large part of the dynamic semantics with ease, and only need to implement small helper methods (such
as those for string manipulation) by hand.

On a final note, we encourage you to review how the metamodel was constructed in the alternative syntax to the one you primarily used. Compare the
differences between modelling classes and references visually using a separate program (Enterprise Architect), then exporting to Eclipse to build, versus
modelling textually entirely within Eclipse. Which do you find easier to work with?

If you have had problems with this section, and, despite firmly believing everything is correct, things \emph{still} don't work, feel free to contact us at:\\ 
\eMoflonContact.

