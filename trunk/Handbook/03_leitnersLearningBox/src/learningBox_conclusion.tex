\genHeader
\section{Conclusion and Next Steps}

\hypertarget{conclusion}{Great job!} You finished Part II! This part is a key skill for eMoflon, as we learned how to create our static models using either syntax! They are each complete with all the required classes, attributes, references, and methods we need for a working Leitner Box.

We would like to now introduce you to the \texttt{Leitner Box Gui}, a small java application that is generated from your code! Within it, you'll be able to view the unique instances you created in Section 3. Go to the top left of your toolbar to the ``New" wizard %includ image?

Go to ``Examples/eMoflon Handbook Examples/Leitner Box GUI.'' This will load a new project into your workspace. 

Right click it to bring up the contex menu, and select ``Run as/Java application.'' Its default settings are to have three partitions with 4 cards each. Click on any card in any of the partition and look at what the console says. It tells you the back attribute of the current card. Right now, they're the same as their  name. Try experimenting with the .ecore editor by adding, removing, or renaming some attributes, and observe how they're reflected your program set up.

If you enjoyed this section and wish to develop each of the methods we declared to make the program actually \emph{do} something, carry on to Part III - Story Diagrams! Don't worry if you added or removed too many items from the code you just developed by experimenting with instances - We provide a download to let you jump right in with a fresh start. 

Of course, you're always free to pick another handbook if you feel like skipping ahead, and checking out some of the other features eMolfon has to offer. Check out Triple Graph Transformations (TGGs)  in Part IV, or Model-to-Text Transformations in Part V. We'll provide instructions on how to easily download all the required resources so you can start without having to complete any of the previous sections. 

For a more detailed description of each part, refer to Part O.

Cheers!