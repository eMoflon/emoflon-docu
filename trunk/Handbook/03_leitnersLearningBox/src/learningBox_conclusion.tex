\newpage
\section{Conclusion and next steps}
\genHeader
\hypertarget{conclusion}{}

\vspace{0.5cm}

Whoo, that was a busy handbook. Great job, you've finished Part II! This part contained some key skills for eMoflon, as we learned how to create complete
static models through an abstract syntax! Both the visual and textual syntaxes are now complete with all the classes, attributes, references, and
method signatures that make up the type graph for a working learning box. Additionally, we also learned how to insert a small injection method into generated
code, and tested all our work in an interactive GUI.

If you enjoyed this section and wish to fully develop \emph{all} the methods we just declared, we invite you carry on to Part III, Story
Diagrams! Story Diagrams are a powerful feature of eMoflon as we can model a large part of any metamodel's dynamic semantics. In this handbook, we'll show you
how to implement complicated methods that would otherwise be limited by injections.

Don't worry about the \texttt{.ecore} model file, if you added or deleted too many items. If you want to start fresh, we provide a streamlined download folder
in EClipse to help you jump right in. Otherwise, all the code you just wrote should work perfectly.

Of course, you're always free to pick another handbook if you feel like skipping ahead and checking out some of the other features eMolfon has to offer. Check
out Triple Graph Grammars (TGGs)  in Part IV, or Model-to-Text Transformations in Part V. We'll provide instructions on how to easily download all the
required resources so you can start without having to complete the previous sections.

For a detailed description of all parts, please refer to the Part 0 handbook, which can be found at \href{http://www.moflon.org/ }{http://www.moflon.org/}.

\vspace{1.0cm}

Cheers!