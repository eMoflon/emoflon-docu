\genHeader
\section{Conclusion and Next Steps}

\hypertarget{conclusion}{Great job -} you've finished Part II! This part was a key skill for eMoflon, as we learned how to create static models through an abstract syntax! Both the visual and textual syntaxes are complete with all the required classes, attributes, references, and methods needed for a working Leitner's Box.

We would like to now introduce you to the \texttt{Leitner Box Gui}, a small java application generated from your code! Within it, you'll be able to view the unique instances you created in Section 3. Later, this application will also include the implmentations of each method. Navigate to the top left of your toolbar and select ``New.''

Load ``Examples/eMoflon Handbook Examples/Leitner Box GUI.'' This will load a new project into your workspace. Right click it to bring up the context menu and select ``Run as/Java application.'' Direct the program to open the \texttt{Box.xmi} file you saved in your workspace \texttt{instances} folder, then view your work! Keep experimenting in .ecore editor by adding, removing, or renaming more attributes, and observe how they're reflected in the GUI.

If you enjoyed this section and wish to develop and implement the methods we just declared, carry on to Part III, Story Diagrams! Don't worry about the ecore model, and if you added or deleted too many items. If you want to start fresh, we provide a download to help you jump right in. Otherwise, all the code you just wrote should work perfectly.

Of course, you're always free to pick another handbook if you feel like skipping ahead and checking out some of the other features eMolfon has to offer. Check out Triple Graph Transformations (TGGs)  in Part IV, or Model-to-Text Transformations in Part V. We'll provide instructions on how to easily download all the required resources so you can start without having to complete any of the previous sections. 

For a detailed description of each part, refer to Part 0.

Cheers!