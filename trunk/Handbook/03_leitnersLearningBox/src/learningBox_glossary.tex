\newpage
\phantomsection
\addcontentsline{toc}{section}{Glossary}

\vspace{1cm}
{\Huge \bf Glossary}
\vspace{1cm}

%TODO: Update 06_miscellaneous/src/glossary.tex with any changes made here.

\begin{description}

\item[\bf Abstract Syntax] 
Defines the valid static structure of members of a language. 

\item[\bf Concrete Syntax]
How members of a language are represented. This is often done textually or visually.

\item[\bf Constraint Language] 
Typically used to specify complex constraints (as part of the static semantics or a language) that cannot be expressed in a metamodel.

\item[\bf Dynamic Semantics] 
Defines the dynamic behavior for members of a language.

\item[\bf Grammar] 
A set of rules that can be used to generate a language. 

\item[\bf Graph Grammar] 
A grammar that describes a graph language. This can be used instead of a metamodel or type graph to define the abstract syntax of a language.

\item[\bf Meta-Language] 
A language that can be used to define another language.

\item[\bf Meta-metamodel] 
A \emph{modeling language} for specifying metamodels.

\item[\bf Metamodel] 
Defines the abstract syntax of a language including some aspects of the static semantics such as multiplicities. 

\item[\bf Model] 
Graphs which conform to some metamodel.

\item[\bf Modeling Language] 
Used to specify languages. Typically contains concepts such as classes and connections between classes.

\item[\bf Static Semantics] 
Constraints members of a language must obey in addition to being conform to the abstract syntax of the language.

\item[\bf Type Graph] 
The graph that defines all types and relations that form a language. Equivalent to a metamodel but without any static semantics.

\item[\bf Unification]  
An extension of the object oriented ``Everything is an object'' principle, where everything is regarded as a model, even the metamodel which defines other
models.

\end{description}
