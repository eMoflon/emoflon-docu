\newpage
\phantomsection
\addcontentsline{toc}{section}{Glossary}

\vspace{1cm}
{\Huge \bf Glossary}
\vspace{1cm}

\begin{description}

\item[\bf Concrete Syntax]
The implementation syntax of a model. (ie: Visual or Textual)

\item[\bf Grammar] 
Syntax description; A set of rules or constructs a language must follow.

\item[\bf Graph Grammar] 
A collection of the language grammar, presented as a graph with a left-hand argument, and a right-hand description.

\item[\bf Type Graph] 
The graph that defines all types, relations, and objects that form a language.

\item[\bf Abstract Syntax] 
Defines the structure for a program. It must strictly conform to the type graph for the specified language, and the types and relations defined here must exist
without errors.

\item[\bf Static Semantics] 
An additional list of rules and constraints a language must obey.

\item[\bf Metamodel] 
Defines the abstract syntax of a language and includes some static semantics by which a model can be derived.

\item[\bf Constraint Language] 
Complex constraints (static semantics) that cannot be expressed in a metamodel.

\item[\bf Model] 
Graphs which conform to some previously-defined metamodel.

\item[\bf Unification]  
An extension of the Object Oriented ``Everything is a object'' principle, where everything is classified as a model, even the metamodel which defines other models.

\item[\bf Meta-metamodel] 
A model that defines a \emph{modeling language} for metamodels.

\item[\bf Meta-Language] 
A language used to define another language through a consistent set of rules. 

\item[\bf Modeling Language] 
See Meta-language; Used to formulage models, where graphical (visual) languages use diagrammed techniques, while textual languages use standardized keywords.

\item[\bf Dynamic Semantics] 
Set of rules that define a system's behavior and reactions to external stimulus.

\end{description}