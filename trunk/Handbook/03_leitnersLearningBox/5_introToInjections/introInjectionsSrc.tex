\newpage
\section{Introduction to injections}
\genHeader

This short introducion will show you how to implement small methods by adding handwritten code to classes created from your model. Injections are inspired by
partial classes in C\#, and are our preferred way of providing a clean separation between generated and handwritten code. 

Let's implement the \texttt{removeCard} method, declared in the \texttt{Partition} EClass. In order to `remove' a card from a partition, all one needs to do is
disable the link between them. Don't forget that (according to our signature) not only does \texttt{removeCard} have to pass in a \texttt{Card}, it must return
one as well.

\begin{itemize}


\item[$\blacktriangleright$] From your working set, open ``gen/LearningBoxLanguage.impl/Part\-it\-ionImpl.java'' and enter the following code\footnote{To avoid
errors, you are able to copy/paste this text} in the \texttt{removeCard} declaration, starting at approximately line 358. Do not remove the first comment,
which is necessary to indicate that this code is written by the user and needs to be extracted automatically into our injection file.

\begin{figure}[htbp]
        \centering
        \begin{lstlisting}[language=Java, keywordstyle={\bfseries\color{purple}}, backgroundcolor=\color{white}]
    public Card removeCard(Card card) {
        // [user code injected with eMoflon]
        
        card.setCardContainer(null);
        return card;
    }
        \end{lstlisting}
        \caption{Implementation of \texttt{removeCard}}
        \label{fig:addToStringRep_impl}
\end{figure}

\item[$\blacktriangleright$] Save the file, then right-click either on the file in the package explorer, or in editor window, and choose ``eMoflon/
Create/Update Injection for class'' from the context menu (Fig.~\ref{fig:injection_create_injection}).

\begin{figure}[htbp]
    \centering
    \includegraphics[width=\textwidth]{eclipse_createInjection}
    \caption{Create a new injection}
    \label{fig:injection_create_injection}
\end{figure}
    
\item[$\blacktriangleright$] This will create a new file in the ``injection'' folder of your project with the same package and name stucture as the Java class,
but with a new \texttt{.inject} extension (Fig.~\ref{fig:injection_folder}).

\begin{figure}[htbp]
    \centering
    \includegraphics[width=0.5\textwidth]{eclipse_injectionFolder}
    \caption{Injection location}
    \label{fig:injection_folder}
\end{figure}

\item[$\blacktriangleright$] Double click to open and view this file. It contains the definition of a \textit{partial class}
(Fig.~\ref{fig:injection_partialClassPartition}).

\begin{figure}[htbp]
    \centering
    \includegraphics[width=0.9\textwidth]{eclipse_partialClass}
    \caption{Generated injection file for \texttt{PartitionImpl.java}}
    \label{fig:injection_partialClassPartition}
\end{figure}

\item[$\blacktriangleright$] That's it! While injecting handwritten code is a remarkably simple process, it is pretty boring and low level to call all those
setters and getters yourself. While we'll still establish two simple methods in Part III using this strategy, we'll also learn how to implement more complex
methods using Story Diagrams in Part III.
 
\end{itemize}
