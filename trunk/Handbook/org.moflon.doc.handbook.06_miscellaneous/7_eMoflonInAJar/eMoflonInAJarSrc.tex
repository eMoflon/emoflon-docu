\newpage
\section{eMoflon in a Jar}
\genHeader

This section describes how to package code generated with eMoflon into
runnable Jar files, which is useful if you want to build applications for end-users.

We distinguish between repository, i.e., SDM-based, and integration,
i.e., TGG-based, projects.

\subsection{Packaging SDM projects into a Jar file}

The following explanations use the demo specification that is shipped with
eMoflon to explain the workflow of building a runnable Jar file.

\begin{itemize}
    
\item[$\blacktriangleright$] 
Open a fresh workspace and add to it the eMoflon Demo specification (visual or textual syntax) by selecting ``File/New/Other..." and then ``eMoflon/New Metamodel Wizard".
Do not forget to to tick the ``Add demo specification" checkbox!
    
\item[$\blacktriangleright$]
Generate code for the demo and verify the result by running the test cases in \emph{DemoTestSuite}.

\item[$\blacktriangleright$]
Add a suitable main method to \texttt{NodeTest}, for instance:
\begin{lstlisting}
public static void main(String[] args) {
    System.out.println("Begin of test runs");
    new NodeTest().testDeleteNode();
    new NodeTest().testInsertNodeAfter();
    new NodeTest().testInsertNodeBefore();
    System.out.println("End of test runs");
}
\end{lstlisting}

\item[$\blacktriangleright$]
Run \texttt{NodeTest} as ``Java Application" (\emph{not} as ``JUnit Test").
Now you have a new launch configuration named ``NodeTest".

\item[$\blacktriangleright$]
Now, select the repository project (containing the generated code) and the project \emph{DemoTestSuite}.
You do not need to add the project containing the EA project.
Right-click and select ``Export...".
Choose ``Runnable JAR file".
  
\item[$\blacktriangleright$]
On the next page, select the launch configuration you just created by running \texttt{NodeTest} and an appropriate target location for your Jar file.
The libraries should be packaged or extracted into the generated Jar file.

\item[$\blacktriangleright$]
Afterwards, open up a console in the folder containing the generated Jar file and execute it as follows:
\begin{lstlisting}
java -jar [GeneratedJarFile.jar]
\end{lstlisting}

    
\end{itemize}

\subsection{Packaging TGG projects into a Jar file}
\emph{yet to come\dots}

\begin{comment}
In the following, you will create a runnable Jar from a TGG specification.
We assume that you have some existing TGG implementation and that you want to execute the \texttt{main} method in the \texttt{org.moflon.tie.[YourIntegrationProject]Trafo}.
\begin{itemize}
    
\item[$\blacktriangleright$] 
Ensure that your TGG rules from within Eclipse.
For simplicity, we assume that your main method currently resembles the following snippet:
\begin{lstlisting}
public static void main(String[] args) throws IOException {
    // Set up logging
    BasicConfigurator.configure();

    // Forward Transformation
    [YourIntegrationProject]Trafo helper = 
        new [YourIntegrationProject]Trafo();
    helper.performForward("instances/fwd.src.xmi");
}
\end{lstlisting}
The default transformation helper should print a short success message when the forward transformation has finished.

\item[$\blacktriangleright$]
Before packaging your project, one more modification is necessary:
The method for loading the TGG rules (in the consructor of \texttt{[Your\-Integration\-Project]\-Trafo}) looks as follows in the default generated code:
\begin{lstlisting}
loadRulesFromProject("..");
\end{lstlisting}
Replace this line with the following to allow for the TGG rules to be loaded from within a Jar file:
\begin{lstlisting}
\end{lstlisting}
\end{itemize} 
\end{comment}



   
