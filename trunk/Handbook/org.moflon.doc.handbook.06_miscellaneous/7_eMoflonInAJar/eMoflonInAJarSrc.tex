\newpage
\section{eMoflon in a Jar}
\genHeader

This section describes how to package code generated with eMoflon into
runnable Jar files, which is useful if you want to build applications for end-users.

We distinguish between repository, i.e., SDM-based, and integration,
i.e., TGG-based, projects.

\subsection{Packaging SDM projects}

The following explanations use the demo specification that is shipped with
eMoflon to explain the workflow of building a runnable Jar file.

\begin{itemize}
    
\item[$\blacktriangleright$] 
Open a fresh workspace and add to it the eMoflon Demo specification (visual or textual syntax) by selecting ``File/New/Other..." and then ``eMoflon/New Metamodel Wizard".
Do not forget to to tick the ``Add demo specification" checkbox!
    
\item[$\blacktriangleright$]
Generate code for the demo and verify the result by running the test cases in \emph{DemoTestSuite}.

\item[$\blacktriangleright$]
Add a suitable main method to \texttt{NodeTest}, for instance:
\begin{lstlisting}
public static void main(String[] args) {
    System.out.println("Begin of test runs");
    new NodeTest().testDeleteNode();
    new NodeTest().testInsertNodeAfter();
    new NodeTest().testInsertNodeBefore();
    System.out.println("End of test runs");
}
\end{lstlisting}

\item[$\blacktriangleright$]
Run \texttt{NodeTest} as ``Java Application" (\emph{not} as ``JUnit Test").
Now you have a new launch configuration named ``NodeTest".

\item[$\blacktriangleright$]
Now, select the repository project (containing the generated code) and the project \emph{DemoTestSuite}.
Right-click and select ``Export...".
Choose ``Runnable JAR file".
  
\item[$\blacktriangleright$]
On the next page, select the launch configuration you just created by running \texttt{NodeTest} and an appropriate target location for your Jar file.
The libraries should be packaged or extracted into the generated Jar file.

\item[$\blacktriangleright$]
Afterwards, open up a console in the folder containing the generated Jar file and execute it as follows:
\begin{lstlisting}
java -jar [GeneratedJarFile.jar]
\end{lstlisting}

    
\end{itemize}

