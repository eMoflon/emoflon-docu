\newpage
\visHeader

\begin{figure}[htbp]
    \centering
  \includegraphics[width=\textwidth]{bothexplorers}
    \caption{From EA to Eclipse}
    \label{fig_fromEAtoEclipse}
\end{figure}

Figure~\ref{fig_fromEAtoEclipse} also depicts how the class \texttt{Node} in the EA model is mapped to the Java interface \texttt{Node}.  
Double-click \texttt{Node.java} and take a look at the methods declared in the interface.
These correspond directly to the methods declared in the modelled \texttt{Node} class.  
Indicated by the source folders \texttt{src}, \texttt{injection} and \texttt{gen}, we advocate a clean separation of hand-written (this should go in \texttt{src} and \texttt{injection}) and generated code (lands automatically in \texttt{gen}).  
As we shall see later in the tutorial, hand-written code can be integrated in generated classes vio Injections. 
This is sometimes more elegant for small helper functions or necessary for String manipulation for instance.

If you take a careful look at the code structure in \texttt{gen}, you'll find a \texttt{Foo\-Impl.java} for every \texttt{Foo.java}. 
Indeed, the subpackage \texttt{impl} contains Java classes that implement the interfaces in the parent package.  
Although this might strike you as unnecessary (why not merge interface and implementation for simple classes?), this consequent separation in interfaces and implementation allows for a clean and relatively simple mapping of Ecore to Java, even in tricky cases like multiple inheritance (allowed and very common in Ecore models).  
A further package \texttt{util} contains some auxiliary classes such as a factory for creating instances of the model.  

If this is your first time of seeing generated code, you might be shocked at the sheer amount of classes and code generated from our relatively simple EA model.  
You might be thinking: ``hey - if I did this by hand I wouldn't need half of all this stuff!''.  
Well you're right and you're wrong -- the point is that an automatic mapping to Java via a code generator scales quite well.