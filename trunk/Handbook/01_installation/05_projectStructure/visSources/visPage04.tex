\newpage
\visHeader

\begin{figure}[htbp]
    \centering
  \includegraphics[width=\textwidth]{bothexplorers}
    \caption{From EA to Eclipse}
    \label{fig_fromEAtoEclipse}
\end{figure}

Have you noticed the methods of the \texttt{Node} class in our EA model? 
Now hold on tight -- each method can be \emph{modelled} completely in EA and the corresponding implementation in Java is generated automatically and placed in \texttt{NodeImpl}.  
Just in case you didn't get it: The behavioural or dynamic aspects of a system can be completely modelled in an abstract, platform (programming language) independent fashion using a blend of activity  diagrams and a ``graph pattern'' language called \textit{Story Driven Modelling} (SDM).  
In our EA project, these ``Stories'', ``Story Models'' or simply ``SDMs'' are  placed in SDM Containers named according to the method they implement.  
E.g.  \texttt{$\guillemotleft$SDM Container$\guillemotright$ insertNodeAfter SDM} for the method  \texttt{insertNodeAfter(Node)} as depicted in
Fig.~\ref{fig_fromEAtoEclipse}.  
We'll spend the rest of the tutorial understanding why SDMs are so  {\huge crazily} cool!
 
To recap all we've discussed, let's consider the complete workflow as depicted in Figure~\ref{fig_Overview}. 
We started with a concise model in EA, simple and independent of any platform specific details~(1).  
Our EA model consists not only of static aspects modelled as a class diagram~(2), but also of dynamic
aspects modelled using SDM~(3).  After exporting the model and code generation~(4), we basically switch from \emph{modelling} to \emph{programming} in a specific general purpose programming language (Java).  
On this lower \emph{level of abstraction}, we can flesh out the generated repository~(5) if necessary, and mix as appropriate with hand-written code and libraries.  Our abstract specification of behaviour (methods) in SDM is translated to a series of method calls that form the body of the corresponding Java method~(6).