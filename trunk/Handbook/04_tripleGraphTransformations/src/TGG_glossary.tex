\newpage
\phantomsection
\addcontentsline{toc}{section}{Glossary}
\hypertarget{glossary}{}

\vspace{1cm}
{\Huge \bf Glossary}
\vspace{1cm}

\begin{description}

\item[Bijective function] A unique one-to-one  correspondence between the elements of two sets (i.e., graphs or metamodels). A function can only be bijective if
it is both \emph{injective} and \emph{surjective}.

\item[Correspondence Types] Connect classes between the source and target metamodels in a \emph{graph triple}

\item[Graph Triple] Consist of source, correspondence, and target components.

\item[Injective function] A one-to-one function that preserves distinction i.e., an element in x can only correspond to one element in y (and vice versa).

\item[Link Metamodel] The correspondence component of a \emph{graph triple}, made up of \emph{correspondence types} between the source and target metamodels in
a triple.

\item[Monotonic] Sameness; An element that never increases or decreases its properties (see \emph{TGG Rules}).

\item[Operationalization] The process of defining or deriving something (i.e., a transformation rule) given a set of defintions or instructions.

\item[Surjective function] A correspondence from one set `onto' another. I.e., every element in y \emph{must} have at least one corresponding element in x.

\item[Triple Graph Grammars (TGG)] Declarative, rule-based technique of specifying the simultaneous evolution of three connected graphs.

\item[TGG Rules] Describe the simultaneous construction, or build-up, of each model in a \emph{graph tiple}.

\item[TGG Schema] The overall metamodel consisting of the source, correspondence (link), and target metamodels.

\end{description}