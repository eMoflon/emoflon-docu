\newpage
\section{Specifying a TGG schema}
\genHeader

As you might assume, the first step to a completing a TGG is to model the correspondence component of your triples. The correspondence, or \emph{link metamodel}
consists, of \emph{correspondence types} which connect classes between the source and target metamodels. In our case, our source metamodel is our
current running example, \texttt{LearningBoxLanguage}, and our target is \texttt{DictionaryLanguage}, which we'll have you import into your project in the next
few steps. Remember, a \emph{correspondence type} can be considered as a traceability link. While the correspondence metamodel is really just a standard
metamodel, we use a slightly different concrete syntax to present it, and adhere to certain naming conventions.\define{TGG Schema}The overall metamodel triple
consisting of the relevant parts of the source, correspondence, and target metamodels is called the \emph{TGG schema}.

A TGG schema can be viewed as the metamodel triple to which all created triples must conform. In less technical lingo, it gives an abstract view on the
relationship or \emph{correspondence} between two metamodels or domains. Just by looking at the TGG schema, a domain expert should understand \emph{why} the
connected elements are related, irrespective of \emph{how} the relationship is actually established by TGG rules (coming up next). Note that all other
information in the source and target metamodels such as attributes, methods and references between classes are also visualized per default in the TGG schema --
just in case you were wondering where all the attributes and references suddenly came from.


\newpage
\hypertarget{schema vis}{}
\subsection{Visual Schema}
\visHeader

\begin{itemize}

\item[$\blacktriangleright$] Open \texttt{LeitnersLearningBox} in EA, and add a new package to your model root, \texttt{My Working Set}. Name it
\texttt{Learning\-Box\-To\-Dictionary\-Integration} (Fig.~\ref{fig:intgPackage}).

\vspace{0.5cm}

\begin{figure}[htbp]
\begin{center}
  \includegraphics[width=0.5\textwidth]{ea_integrationPackage}
  \caption{Create a new package}  
  \label{fig:intgPackage}
\end{center}
\end{figure}

\item[$\blacktriangleright$] Create a diagram in the new package, selecting \texttt{TGG Schema} as diagram type (Fig.~\ref{fig:tgg_diagram_type}). The diagram
type indicates to EA that the new package is a TGG Project.

\vspace{0.5cm}

\begin{figure}[htbp]
\begin{center}
  \includegraphics[width=0.9\textwidth]{ea_newTGGSchema}
  \caption{Choose \texttt{TGG Schema} as your diagram type}  
  \label{fig:tgg_diagram_type}
\end{center}
\end{figure}

\item[$\blacktriangleright$] A dialogue should pop up asking you for the source and target projects of your TGG project.  Choose
\texttt{Learning\-Box\-Language} as source and \texttt{Dictionary\-Language} as target project and affirm with \texttt{OK} (Fig.~\ref{fig:select_source_target}).

\vspace{0.5cm}

\begin{figure}[htbp]
\begin{center}
  \includegraphics[width=0.55\textwidth]{ea_TGGSourceTarget}
  \caption{Select source and target projects for the TGG project}  
  \label{fig:select_source_target}
\end{center}
\end{figure}

\item[$\blacktriangleright$] The structure of your TGG project should now resemble Fig.~\ref{fig:new_tgg_project}. Please note that a subpackage \texttt{Rules}
and an underlying diagram with the same name are also generated. \update -- started from the wrong EAP

\begin{figure}[htbp]
\begin{center}
  \includegraphics[width=0.5\textwidth]{ea_browserPostDiagram}
  \caption{Initial structure of a new TGG project}  
  \label{fig:new_tgg_project}
\end{center}
\end{figure}
\end{itemize}
\clearpage

Now it's time to insert classes from our source and target projects into our TGG project and declare our first \emph{correspondence type} between them.
The classes \texttt{Box} and \texttt{Dictionary} are to be related to each other.

\begin{itemize}
\item[$\blacktriangleright$] Hold \texttt{ctrl}, then drag-and-drop the \texttt{Box} class from \texttt{Learning\-Box\-Language} into the newly
created TGG schema diagram, which should have automatically opened in the editor when you made it. Ensure that the class is pasted \texttt{as a simple link} into the
diagram (Fig.~\ref{fig:TGGdragDrop})

\begin{figure}[htbp]
\begin{center}
  \includegraphics[width=0.6\textwidth]{ea_TGGDragDrop}
  \caption{Copying an element as a simple link} 
  \label{fig:TGGdragDrop}
\end{center}
\end{figure}

\item[$\blacktriangleright$] Note that you are able to set \texttt{Autosave Selection as default}. In fact, we did this in the previous Part! We'll need to
switch drag types several times during this part however, so it's best to leave this unchecked if you do not want to hold \texttt{ctrl} each time you use this
gesture.

\item[$\blacktriangleright$] With a class from both source and target metamodels, we can now create a correspondence type between them! Quick-link from
\texttt{Box} to \texttt{Dictionary}, and select \texttt{Create TGG Corres\-pon\-dence Type} (Fig.~\ref{fig:create_correspondence}).

\newpage

\begin{figure}[htbp]
\begin{center}
  \includegraphics[width=\textwidth]{ea_TGGCorrespType}
  \caption{Creating a TGG correspondence type} 
  \label{fig:create_correspondence}
\end{center}
\end{figure}

\item[$\blacktriangleright$] As you can see, a correspondence type has been created, visualised via a hexagonal shape. It is automatically named
\texttt{BoxToDiction\-ary}, and the relevant references have been generated.

% Can this be omitted? Users create a new one on the fly?
% \item[$\blacktriangleright$] To finish our TGG schema, declare a second correspondence type in the same file between \texttt{Card} and \texttt{Entry}. You'll
% notice that the reference between \texttt{Dictionary} and \texttt{Entry} was automatically created! Your completed TGG Schema should resemble
% Fig.~\ref{fig:complete_tgg_schema}.

% \begin{figure}[htbp]
% \begin{center}
%   \includegraphics[width=\textwidth]{ea_completeTGGSchema}
%   \caption{Complete TGG schema for our example}
%   \label{fig:complete_tgg_schema}
% \end{center}
% \end{figure}

\end{itemize}



\newpage
\hypertarget{schema tex}{}
\subsection{Textual TGG Schema}
\texHeader

\begin{itemize}

\item[$\blacktriangleright$] Right-click on the \texttt{MyWorkingSet} folder, and navigate to ``New / TGG'' (Fig.~\ref{eclipse:contextTGG}).

\vspace{0.5cm}

\begin{figure}[htbp]
\begin{center}
  \includegraphics[width=0.8\textwidth]{eclipse_contextNewTGG}
  \caption{Creating a new TGG schema}
  \label{eclipse:contextTGG}
\end{center}
\end{figure}

\item[$\blacktriangleright$] Name the TGG \texttt{LearningBoxToDictionaryIntegration}, setting the source as \texttt{LearningBoxLanguage}, and target as
\texttt{DictionaryLanguage} (Fig.~\ref{eclipse:newTGG}).

\item[$\blacktriangleright$] A new TGG \texttt{schema} file should now be active in the editor! This is the \emph{TGG Schema} which declares each
\emph{correspondence type} as an \texttt{integration class}. Press \texttt{ctrl + space bar} and use the auto completion to generate a new integration class.

\newpage

\begin{figure}[htbp]
\begin{center}
  \includegraphics[width=0.8\textwidth]{eclipse_newTGG}
  \caption{Setting your \texttt{source} and \texttt{target} metamodels}
  \label{eclipse:newTGG}
\end{center}
\end{figure}

\item[$\blacktriangleright$] Note that when using a template, you can press \texttt{tab} to cycle through each element. Name the class
\texttt{BoxToDictionary}, and list the source as \texttt{Box} and target as \texttt{Dictionary} (Fig.~\ref{eclipse:firstCorrType}). 

\begin{figure}[htbp]
\begin{center}
  \includegraphics[width=0.4\textwidth]{eclipse_schemaFirstClass}
  \caption{Creating a correspondence type}
  \label{eclipse:firstCorrType}
\end{center}
\end{figure}

\item[$\blacktriangleright$] Believe it or not, that's all you need for your first correspondence type! Your schema is now complete with connections to your
\texttt{source} and \texttt{target} metamodels via a \emph{link} metamodel. To see the equivalent structure in the visual syntax, check out
Fig.~\ref{ea:firstCorrType} from the previous section.

\end{itemize}

