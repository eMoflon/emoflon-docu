\newpage
\hypertarget{schema vis}{}
\subsection{Visual Schema}
\visHeader

((But user is working from THEIR metamodel, LeitnersLearningBox.eap))

\begin{itemize}

\item[$\blacktriangleright$] Open \texttt{LearningBox2Dictionary.eap} in EA, and add a new package to \texttt{MyWorkingSet} (your model root) with
\texttt{Learning\-Box\-To\-Dictionary\-Integration} as its name (Fig.~\ref{fig:intgPackage}). 

\vspace{0.5cm}

\begin{figure}[htbp]
\begin{center}
  \includegraphics[width=0.4\textwidth]{ea_integrationPackage}
  \caption{Create a new package}  
  \label{fig:intgPackage}
\end{center}
\end{figure}


\item[$\blacktriangleright$] Create a diagram in the new package, selecting \texttt{TGGSchema} as diagram type (Fig.~\ref{fig:tgg_diagram_type}). The diagram
type indicates to EA that the new package is a TGG Project.

\vspace{0.5cm}

\begin{figure}[htbp]
\begin{center}
  \includegraphics[width=0.9\textwidth]{ea_newTGGSchema}
  \caption{Choose \texttt{TGGSchema} as your diagram type}  
  \label{fig:tgg_diagram_type}
\end{center}
\end{figure}

\item[$\blacktriangleright$] After choosing \texttt{TGGSchema} as diagram type, a new dialogue should pop up asking you for the source and target projects of your TGG project. 
Choose \texttt{Learning\-Box\-Language} as source and \texttt{Dictionary\-Language} as target project and affirm with \texttt{OK} (Fig.~\ref{fig:select_source_target}).

\vspace{0.5cm}

\begin{figure}[htbp]
\begin{center}
  \includegraphics[width=0.55\textwidth]{ea_TGGSourceTarget}
  \caption{Select source and target projects for the TGG project}  
  \label{fig:select_source_target}
\end{center}
\end{figure}

\item[$\blacktriangleright$] The structure of your TGG project should now resemble Fig.~\ref{fig:new_tgg_project}. Please note that a subpackage \texttt{Rules}
and an underlying diagram with the same name are also generated. \update -- started from the wrong EAP

\begin{figure}[htbp]
\begin{center}
  \includegraphics[width=0.5\textwidth]{ea_browserPostDiagram}
  \caption{Initial structure of a new TGG project}  
  \label{fig:new_tgg_project}
\end{center}
\end{figure}
\end{itemize}
\clearpage

Now it's time to insert classes from our source and target projects into our TGG project and declare our first \emph{correspondence type} between them.
The classes \texttt{Box} and \texttt{Dictionary} are to be related to each other.

\begin{itemize}
\item[$\blacktriangleright$] Hold \texttt{ctrl}, then drag-and-drop the \texttt{Box} class from \texttt{Learning\-Box\-Language} into the newly created TGG
schema diagram, which should have automatically opened in the editor when you made it. Ensure that the class is pasted \texttt{as a simple link} into the
diagram (Fig.)

\begin{figure}[htbp]
\begin{center}
  \includegraphics[width=0.6\textwidth]{ea_TGGDragDrop}
  \caption{Copying an element as a simple link} 
  \label{fig:TGGdragDrop}
\end{center}
\end{figure}

\item[$\blacktriangleright$] If you selected the \texttt{Autosave Selection as default} option during the previous drag and drop, you no longer need to hold
\texttt{ctrl} to bring the \texttt{Dictionary} class from \texttt{Dictionary\-Language} into the TGG schema.
\end{itemize}

With a class from both source and target projects, we can now create a correspondence type between them.

\begin{itemize}
\item[$\blacktriangleright$] Quick-link from \texttt{Box} to \texttt{Dictionary} and select \texttt{Create TGG Corres\-pon\-dence Type} as depicted in
Fig.~\ref{fig:create_correspondence}.
\end{itemize}

\begin{figure}[htbp]
\begin{center}
  \includegraphics[width=\textwidth]{ea_TGGCorrespType}
  \caption{Creating a TGG correspondence type} 
  \label{fig:create_correspondence}
\end{center}
\end{figure}

A hexagon-shaped correspondence type named \texttt{BoxToDiction\-ary}, and references to each source and target element should have been generated.
You can rename the type as you wish, but please leave the references as they are (multiplicity and naming conventions are satisfied automatically).

To finish our TGG schema, declare a second correspondence type in the same file between \texttt{Card} and \texttt{Entry}. You'll notice that the
reference between \texttt{Dictionary} and \texttt{Entry} was automatically created! Your completed TGG Schema should resemble
Fig.~\ref{fig:complete_tgg_schema}.

\begin{figure}[htbp]
\begin{center}
  \includegraphics[width=\textwidth]{ea_completeTGGSchema}
  \caption{Complete TGG schema for our example}
  \label{fig:complete_tgg_schema}
\end{center}
\end{figure}

