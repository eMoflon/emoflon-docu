\newpage
\section{Setting up your workspace}
\genHeader

Before starting any TGG transformation, you'll need to have the separate source and target metamodels already prepared. In our case, these will be
\texttt{LeitnersLearningBox} and a new \texttt{DictionaryLanguage}. Don't worry - we have created this for you to import into your workspace. If you haven't
completed the previous parts in the handbook series, work through Section~\ref{sec:loadSourceMeta} to load a pre prepared \texttt{LeitnersLearningBox}
metamodel into your workspace. If your source metamodel is ready to go however, skip ahead to  either \texttt{\hyperlink{sec:multiEAP}{Section 2.2 (Visual)}}
or \texttt{\hyperlink{sec:multiMOSL}{Section 2.3 (Textual)}}.


% --- NEW USER INSTRUCTIONS
\subsection{Starting Fresh}
\label{sec:loadSourceMeta}
\begin{itemize}

\item[$\blacktriangleright$] Press the \texttt{new} button on the toolbar and navigate to ``Examples/eMoflon Handbook Examples/''
(Fig.~\ref{fig:downPartIV}).  We have created two cheat packages based on eMolfon's two specification types. For a brief discussion on the differences between
the two, refer to Part I, Section I. They are both completed with all objects, references, and implemented SDMs.

\begin{figure}[htbp]
\begin{center}
  \includegraphics[width=0.8\textwidth]{eclipse_downloadWizardPartIV}
  \caption{Use the wizard to download your cheat package}
  \label{fig:downPartIV}
\end{center}
\end{figure}

\item[$\blacktriangleright$] After loading, if your package explorer does not resemble ours in Fig.~\ref{fig:workingSets} with at least two
distinct nodes, select the small, downward facing arrow in the corner of the module window. Choose ``Working Sets/Top Level Elements.'' To review how these
nodes are used to structure the workspace in Eclipse, check out Part I, Section 4.

\vspace{0.5cm}

% Forced placement so it would co-operate
\begin{figure}[h!]
	\centering
  \includegraphics[width=0.9\textwidth]{eclipse_workingSets}
	\caption{Setting your Package Explorer}
	\label{fig:workingSets}
\end{figure}

\vspace{0.5cm}

At this point we recommend reading the introduction to Part II for a detailed overview of \texttt{LeitnersLearningBox}, its purpose and goals, and how
\texttt{card}s and \texttt{FastCards} are moved between each \texttt{partition}. It should only take a minute or two, and the background information may help
you better understand the reasoning behind some steps in this part as we continue to build on the idea.

\end{itemize}

\jumpDual{multiEAP}{multiMOSL}

% Instructions for loading target model
\newpage
\subsection{Working with multiple EAPs}
\visHeader
\label{sec:multiEAP}

\begin{itemize}

\item[$\blacktriangleright$] Included in the \texttt{Part4Download} .zip file is the \texttt{dictionaryLanguage.eap} file. Double click this to open it in
Enterprise Architect (EA).

\vspace{0.5cm}

\item[$\blacktriangleright$] Although you can simply copy and paste single packages between multiple EAPs, packages with dependencies to other packages (i.e.,
those between \texttt{DictionaryCodeAdapter} and \texttt{DictionaryLanguage}) cannot be copied so easily. If you do this, all links will be destroyed!
Therefore, to migrate multiple packages, you have to first export a \emph{complete} root node (the package on the top-most level of the project browser) to an
XMI file.

\vspace{0.5cm}

\item[$\blacktriangleright$] Right click on \texttt{dictionaryLanguageRoot}, and select \texttt{Export Model to XMI\ldots,} as depicted in
Fig.~\ref{fig:export}.

\vspace{0.5cm}

\begin{figure}[htbp]
\begin{center}
  \includegraphics[width=0.5\textwidth]{ea_exportToXMI}
  \caption{Exporting the \emph{target metamodel}}
  \label{fig:export}
\end{center}
\end{figure}

\item[$\blacktriangleright$] Save the file somewhere easily accessible, such as your desktop, and change the export type to \texttt{XMI 2.1}. You should have a
small green bar appear once the action is complete (Fig.~\ref{fig:exportDialogue}).

\vspace{0.5cm}

\begin{figure}[htbp]
\begin{center}
  \includegraphics[width=0.8\textwidth]{ea_exportPackageDialogue}
  \caption{Persisting the export to a file}
  \label{fig:exportDialogue}
\end{center}
\end{figure}

\item[$\blacktriangleright$] Now open your \texttt{LeitnersLearningBox.eap} file, and right-click the root \texttt{MyWorkingSet} node and select \texttt{Import
Model from XMI\ldots}. In the dialogue that appears, find the file you just saved and \texttt{import}. Press \texttt{yes} in each of the confirmation dialogues
that appear after. Your workspace should now resemble Fig.~\ref{fig:postImport}.

\vspace{0.5cm}

\begin{figure}[htbp]
\begin{center}
  \includegraphics[width=0.5\textwidth]{ea_postImport}
  \caption{Project explorer after importing the \emph{target} metamodel}
  \label{fig:postImport}
\end{center}
\end{figure}

\end{itemize}


\newpage
\subsection{Working with multple MOSL projects}
\texHeader
\label{sec:multiMOSL}

(Eclipse import instructions here.)

Your job is easy unlike visual -- we HAD to review how to properly export. theres enoguh documentation for eclipse that all we'll make you do it import. so,
right click and clikc import

Make you root directory the textual source folder that was in the part IV download. Both dicationaryadapter and dictionarlanguage projects will be imported into
your workspace.

