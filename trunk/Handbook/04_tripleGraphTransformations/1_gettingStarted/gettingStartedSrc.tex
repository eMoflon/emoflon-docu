\newpage
\section{Setting up your workspace}
\genHeader

Before you're able start any TGG transformation, you need to have two metamodels -- a source and target -- already prepared. Our source will be Leitner's
Learning Box, a partitioned container populated with cards. Each card has a keyword on its back, and a definition on its face. It is specified in
\texttt{LearningBoxLanguage}, fully built and completed in Parts II and II. Conversely, our target metamodel will be a Dictionary, a single container object
able to hold an unlimited number of entries. The goal is to have one dictionary per topic, where each entry would have a main keyword, and some relevant content. 

So how do we hope to transform a learning box into a dictionary, and vice versa? The first goal, from \texttt{BoxToDictionary}, is to combine each side of the
card as a single entry's content, then insert it anywhere into a dictionary. In the other direction, we want to be able to sort a series of entires based on
their difficulty level, in addition to splitting up their content into a useful keyword or question, and answer.

If you've just joined us in the handbook, complete Section~\ref{sec:loadSourceMeta} to load the preprepared \texttt{LeitnersLearningBox} metamodel into your
workspace. If your source metamodel is ready to go, skip ahead to  either \texttt{\hyperlink{sec:multiEAP}{Section 2.2 (Visual)}}
or~\texttt{\hyperlink{sec:multiMOSL}{Section 2.3 (Textual)}} to import \texttt{DictionaryLanguage} into your workspace. Please note that these instructions for
properly importing the metamodel into the same project are crucial as both metamodels must be in the same project in order to properly specify an integration
with TGGs.

\subsection{Starting Fresh}
\label{sec:loadSourceMeta}
\begin{itemize}

\item[$\blacktriangleright$] To get started, press the \texttt{new} button on the toolbar and navigate to ``Examples/eMoflon Handbook Examples/''
(Fig.~\ref{fig:downPartIV}).  We have created two `cheat' packages based on eMolfon's specification types --  pick the one you would prefer to use.\footnote{For
a brief discussion on the differences between the two, refer to Part I, Section I} They both contain the finished metamodel, complete with all objects,
references, and implemented SDMs as completed in the previous parts.

\item[$\blacktriangleright$] After loading, if your package explorer does not look similar to ours in Fig.~\ref{fig:workingSets} with at least two distinct
nodes, select the small, downward facing arrow in the corner of the module window. Choose ``Working Sets/Top Level Elements.'' We use these to structure the
workspace in Eclipse.

\end{itemize}

At this point we recommend reading the introduction to Part II for an overview of \texttt{LeitnersLearningBox}, it's purpose and goals, and how \texttt{card}s
are moved between each \texttt{partition}. Hopefully background will help you better understand the steps in \emph{this} part, as we're continuing to build on
that idea. 

\begin{figure}[htbp]
\begin{center}
  \includegraphics[width=0.8\textwidth]{eclipse_downloadWizardPartIV}
  \caption{Download a file set to get started}
  \label{fig:downPartIV}
\end{center}
\end{figure}

\vspace{0.5cm}

% Forced placement so it would co-operate
\begin{figure}[h!]
	\centering
  \includegraphics[width=0.9\textwidth]{eclipse_workingSets}
	\caption{Setting your Package Explorer}
	\label{fig:workingSets}
\end{figure}

\jumpDual{sec:multiEAP}{sec:multiMOSL}

% Instructions for loading target model
\newpage
\subsection{Working with multiple EAPs}
\visHeader
\label{sec:multiEAP}

\begin{itemize}

\item[$\blacktriangleright$] Included in the \texttt{Part4Download} .zip file is the \texttt{dictionaryLanguage.eap} file. Double click this to open it in
Enterprise Architect (EA).

\vspace{0.5cm}

\item[$\blacktriangleright$] Although you can simply copy and paste single packages between multiple EAPs, packages with dependencies to other packages (i.e.,
those between \texttt{DictionaryCodeAdapter} and \texttt{DictionaryLanguage}) cannot be copied so easily. If you do this, all links will be destroyed!
Therefore, to migrate multiple packages, you have to first export a \emph{complete} root node (the package on the top-most level of the project browser) to an
XMI file.

\vspace{0.5cm}

\item[$\blacktriangleright$] Right click on \texttt{dictionaryLanguageRoot}, and select \texttt{Export Model to XMI\ldots,} as depicted in
Fig.~\ref{fig:export}.

\vspace{0.5cm}

\begin{figure}[htbp]
\begin{center}
  \includegraphics[width=0.5\textwidth]{ea_exportToXMI}
  \caption{Exporting the \emph{target metamodel}}
  \label{fig:export}
\end{center}
\end{figure}

\item[$\blacktriangleright$] Save the file somewhere easily accessible, such as your desktop, and change the export type to \texttt{XMI 2.1}. You should have a
small green bar appear once the action is complete (Fig.~\ref{fig:exportDialogue}).

\vspace{0.5cm}

\begin{figure}[htbp]
\begin{center}
  \includegraphics[width=0.8\textwidth]{ea_exportPackageDialogue}
  \caption{Persisting the export to a file}
  \label{fig:exportDialogue}
\end{center}
\end{figure}

\item[$\blacktriangleright$] Now open your \texttt{LeitnersLearningBox.eap} file, and right-click the root \texttt{MyWorkingSet} node and select \texttt{Import
Model from XMI\ldots}. In the dialogue that appears, find the file you just saved and \texttt{import}. Press \texttt{yes} in each of the confirmation dialogues
that appear after. Your workspace should now resemble Fig.~\ref{fig:postImport}.

\vspace{0.5cm}

\begin{figure}[htbp]
\begin{center}
  \includegraphics[width=0.5\textwidth]{ea_postImport}
  \caption{Project explorer after importing the \emph{target} metamodel}
  \label{fig:postImport}
\end{center}
\end{figure}

\end{itemize}


\newpage
\subsection{Working with multple MOSL projects}
\texHeader
\label{sec:multiMOSL}

(Eclipse import instructions here.)

Your job is easy unlike visual -- we HAD to review how to properly export. theres enoguh documentation for eclipse that all we'll make you do it import. so,
right click and clikc import

Make you root directory the textual source folder that was in the part IV download. Both dicationaryadapter and dictionarlanguage projects will be imported into
your workspace.

