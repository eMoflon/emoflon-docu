\newpage
\section{Using the Integrator}
\genHeader
\label{sec:app_integrator}

Let's check out another cool visualization feature of eMolfon, the integrator. The graph viewer allows you to ``see" the structure of a model, but the
integrator allows you to actually trace the transformation process. In other words, the integrator works as an ``offline'' debugger.

\begin{itemize}

\item[$\blacktriangleright$] Right-click on \texttt{corr\_BWD.xmi} and choose ``eMoflon $\rightarrow$ Start Integrator'' which will open the window depicted in
Fig.~\ref{fig:integrator_start}. This first window depicts your TGG Triple, showing how the \texttt{source}, and \texttt{target} metamodels are connected via
the \texttt{link} (correspondence) metamodel.

\begin{figure}[htbp]
\begin{center}
  \includegraphics[width=0.7\textwidth]{eclipse_integratorStart}
  \caption{Default view of the integrator}
  \label{fig:integrator_start}
\end{center}
\end{figure}

\item[$\blacktriangleright$] Drag and drop \texttt{protocol\_BWD.xmi} into the main window.\footnote{If the integrator window minimizes, press \texttt{alt+tab}
to re-activate it} You will be able to see the controls for navigating through the transformation process explained in the lower part of the window.

\item[$\blacktriangleright$]  Click on \texttt{Box} to begin the process there and use \texttt{Alt+Right} to navigate forwards through the
transformation, and \texttt{Alt+Left} to go backwards. A small description of each step will appear in the small window below.

\item[$\blacktriangleright$] As you progress through, the window will state what node is currently active, and list the possible actions (rule candidates) that
may be applied to that object. The integrator will then attempt and give the rule of any operation it performs. In our example, the transformation begins and
process a \texttt{partition}, then continues to establish its \texttt{card} element. Each time a node is successful, a connecting edge appears leading to its
result in the target model. 

\vspace{0.5cm}

\begin{figure}[h!]
\begin{center}
  \includegraphics[width=0.75\textwidth]{eclipse_integratorPerformed}
  \caption{Integrator after protocol insertion.}
  \label{fig:integrator_after_protocol}
\end{center}
\end{figure} 

\clearpage

\item[$\blacktriangleright$] While stepping through the transformation, you'll notice there are bright colours for each node based on their state. They
represent how the element is currently being processed, and have the following definitions:

\begin{description}
  \item[Blue] The element is currently being ``looked at," and is about to be processed.
  
  \item[Yellow] The element cannot be transformed right now and has been queued for later transformation (i.e., when transforming the first
  \texttt{Entry} into a \texttt{Card}, the \texttt{Box} with each \texttt{partition} to store \texttt{card} must be translated first); ``Paused.''
  
  \item[Green] The object has been created.

\end{description}

\end{itemize}
