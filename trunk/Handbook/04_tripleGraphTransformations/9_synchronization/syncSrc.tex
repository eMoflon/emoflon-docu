\newpage
\section{Model Synchronization}
\genHeader

At this point you have successfully created a trio of rules that can transform a \texttt{box} (with an undetermined number of \texttt{partition}s which can
contain an unknown number of \texttt{card}s) \emph{forwards} into a single \texttt{dictionary} able to store an unlimited number of \texttt{entry} elements.
Unfortunately, there would be unavoidable data loss if you attempted transform data \emph{backwards} from \texttt{dictionary}.

Suppose you first ran the TGG on a \texttt{source.xmi} graph with five \texttt{partition}s, each containing three \texttt{Card}s. According to our
\texttt{IndexToLevel} implementation, the \texttt{card}s would be transformed as (1) three ``beginner'' entries, (2) three ``advanced'' entries, and (3)
nine ``master'' entries. If you were to now run the TGG on the generated \texttt{target} graph and inspect the output, you would have a \texttt{box} with
only three partitions of size nine, three, and three, respectively. There is no possible way to achieve the original construction of the other two partitions!

Just to show how synch works, in the forward direction, from a Box to a Dictionary, let's add a fourth card to partition2. We don't want to run the ENTIRE
transformation again, we only want to do a minor update.

\begin{itemize}

\item example to find out/explain/resolve model synchronization
  
\item[$\blacktriangleright$] (in performForward) Run on the original (load source)
  
\item[$\blacktriangleright$] Load correspondence (corr FWD; to grab original structure)
    
\item[$\blacktriangleright$] load existing target (source.xmi\_FWD.xmi)

\item[$\blacktriangleright$] then, integrate forward (w.e. that means)

\item[$\blacktriangleright$] save src (since you're modying it, different locations [it's always good to compare])

\item[$\blacktriangleright$] save target (that's where the end result will be)

\item[$\blacktriangleright$] save the synchronization correspondence

\item[$\blacktriangleright$] Finall, save the protocol. Will be smaller than the original FWD (just like the corr) because it focuses EXCLUSIVELY on one thing,
i.e., only makes the minor change

\item[$\blacktriangleright$] in SYNCH, open source\_FWD. You'll notice the fourth card! Compared to the original source xmi, partition 2 only had one.
Similarly, open target\_FWD. It has the fourth card! They match!

\item[$\blacktriangleright$] In fact, run the integrator on the correspondence. You'll notice that you can't go very far. in fact, the ONLY thing the corr will
show is the creation of this single card. It doesn't run through the entire transformation because the only thing that was modified or changed was this.

\item[$\blacktriangleright$] Now, lets update the backwards direction, from dictionary to Box. In dictionary, the current target found in LOCATION has three
entries, not four! This is because we separated the original and the new. Let's remove `One: Eins' in index 0, and add a new `Five: fuenf'

\item[$\blacktriangleright$] (In performBackwards) load original target

\item[$\blacktriangleright$] load correspondence BWD

\item[$\blacktriangleright$] load src (will save to)

\item[$\blacktriangleright$] integrate back

\item[$\blacktriangleright$] Can copy/paste at this point, but save target (since that's what you changed)

\item[$\blacktriangleright$] save src (results)

\item[$\blacktriangleright$] save updated/minor correspondence and finally it's paired protocol.

\item[$\blacktriangleright$] Run the integrator. Notice how it doesn't show the deletion of the first card? As mentioned before (LOCATION), TGGs are only used
to build up or construct models - a separate compiler/construction/thing entirely makes destructive changes. So the card is essentially `gone,' not even
appearing in the correspondence.

\end{itemize}
