\newpage
\hypertarget{allCards tex}{}
\subsection{Text new rule}
\texHeader

First re-open your schema and make a new integration class (rule) named AllOtherCards. Source: Box, target: Dictionary

Go back to the Rules folder of LearningBoxToDictionaryIntegration and make a new one : AllOtherCardsRule.

In source:

pre condition: box:Box, partition0, partitionN:Partition. Create operator: partitionNN. Set equals attirbute operator on partition0, and for partitionN, give
it index >= 2. Leave the third blank. (Fig.)

Next set up the containedPartition references in box. Remember, the first two will be black, the one to partitionNN must be green.

Knowing that partitionN+1 will be our latest addition, point the next reference of partitionN to it, and set the previous reference to partition 0. we don't
need a next on partition0 since we dont know if it will be the next value.

no need to set up a NEW target - we're working exclusively within box to include this extra card in the pattern and assuming dictionary already exists. By
consequence, we don't need to set up a NEW correspondence to Dictionary, just box <- allOtherCards : AllOtherCards -> dictionary
But that doesn't mean we're done - we still need to establish a constraint that
will add one to the index value of partitionN to make partitionN+1 truly the NEXT partition. Luckily, this attribute type is established by eMoflon, and so we don't need to create our own again. 

in the constraints scope, enter:
add(partitionN.index, 1, partitionNN)

it should be explained that the syntax for this is x+y=z, where add(x,y,z).

What are some other ways you could have constructed this same rule? i.e., does each reference you just declared HAVE to be in their current position?

Alles gut! Save and build. Make sure no errors in the generated files. (then move on)
