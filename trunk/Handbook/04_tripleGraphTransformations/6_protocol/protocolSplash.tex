\newpage
\section{Extending the transformation}
\genHeader

% Remember, this is relevant for BOTH specifications.

% \emph{Introduce the problem: four partitions. Integrator doesn't help, what do? Protocol! Explain what went wrong, new rule! Run again, show protocol round
% two, and look! it runs fine.}

% Introduce problem: we want four partitions. Add one - woah, it doesn't work! Let's try the integrator. Doesn't help. Chat about the protocol. Lets implement a
% new rule!
% protocol Splash

% vis rule

% tex rule

% Run it again! It works! BRILLIANT. Show in protocol.

At this point, we now have a working TGG triple to transform a \texttt{Dictionary} into a \texttt{Box} with three \texttt{partition}s, and a \texttt{Box} with
exactly three \texttt{Partition}s into \texttt{Dictionary}. The only potential problem is that a learning box with only three partitions may not be the most
useful studying tool. After all, the more partitions you have, the more practice you'll have with the cards by being quizzed again and again.

Let's try adding a fourth partition to \texttt{source.xmi} and run the TGG again. Given that we have a rule to transform three partitions, it should at least
complete a partial transformation, right? 

\begin{itemize}

\item[$\blacktriangleright$] Add a \texttt{partition3} with one \texttt{card} to your \texttt{source.xmi} so that it resembles
Fig.~\ref{fig:fourthPartitionStart}. Don't set or update the partition's \texttt{previous} attribute yet -- let's first run \texttt{TGGMain} again and find out
if the TGG can handle the extra element.

\begin{figure}[htbp]
\begin{center}
  \includegraphics[width=0.7\textwidth]{eclipse_fillFourthPartition}
  \caption{caption}
  \label{fig:fourthPartitionStart}
\end{center}
\end{figure}

\item[$\blacktriangleright$] An error should appear in the eMolfon console window stating that there was a problem while translating your new partition, but the
forward transformation was still able to finish. In fact, if you open \texttt{source.xmi\_FWD.xmi}, you'll be able to confirm the \texttt{English Numbers
Dictionary} was created and includes the newest card! Let's run the integrator on \texttt{corr\_fwd.xmi} to find out exactly what happened.

\item[$\blacktriangleright$] Starting with the initial \texttt{Box}, you'll notice that if you continue pressing \texttt{alt + right} until the end of the
transformation, the TGG detected it would endlessly continue peforming the same \texttt{BoxToDictionaryRule} candidate check. To avoid creating the infinite
cycle trying to resolve it, it exited the action but still went ahead to \texttt{Question Four : card} and successfully created the \texttt{dictionary} entry
(Fig.~\ref{fig:integrator_debugSuccess}).

\begin{figure}[htb]
\begin{center}
  \includegraphics[width=0.7\textwidth]{eclipse_integratorDebug}
  \caption{caption}
  \label{fig:integrator_debugSuccess}
\end{center}
\end{figure}

\item[$\blacktriangleright$] Let's now try and properly connect \texttt{partition3} to \texttt{box} by updating its \texttt{previous} value to
\texttt{partition0}, and changing \texttt{partition2.next} to \texttt{partition3}. Run \texttt{TGGMain} one more time.

\item[$\blacktriangleright$] It didn't work! The integrator won't help us here, so let's try inspecting the protocol file directly for more details. This will
always be produced in \emph{every} transformation. Therefore, if the integrator ever doesn't work, you will always have THIS file for reference.

\item[$\blacktriangleright$] Screenshot of protocol, explain what went wrong (Dangling edge)

\item[$\blacktriangleright$] To help explain, check out figure~\ref{fig:partition3comparison}. The first transformation (without references) still worked
because \ldots The second one failed however, because the MOSL look-ahead feature detected that there was no way it would be able to resolve the dangling
\texttt{partition2.next} reference, which fujaba does not allow.

\newpage

\begin{figure}[htbp]
 	\centering
   \includegraphics[width=0.7\textwidth]{partition3_noConnections}
   \\ \vspace{1cm}
    \includegraphics[width=0.7\textwidth]{partition3_connections}
 	\caption{Successful and Unsuccessful TGG}
 	\label{fig:partition3comparison}
\end{figure}
\FloatBarrier

\end{itemize}

Let's try adding a new rule to handle this additional structure. While we could keep things simple by extending the existing
\texttt{BoxToDictionaryRule} by connecting a fourth partition, what if we wanted a fith one? A sixth? As you can see, this obviously won't work -- there will
always be the potential for an additional \texttt{n+1}th card in an \texttt{n}-sized box.

It should be noted that while we addressed any partition \texttt{index} greater than 2 in our \texttt{IndexToLevel} implementation
code (anything above 2 would simple be assigned as index 2), that only took care of the attribute, not the actual object. The main goal of this rule will be to
extend the structure of a \texttt{Box} during the forward transformation so that the \texttt{CardToEntry} rule
can be applied to any \texttt{card} found in a fourth, fifth, sixth, or greater \texttt{partition}. We therefore don't need to create any new \emph{correspondence
types}  as the changes will only affect the link variables and \texttt{index} attributes of a few \texttt{partition}s.

\jumpDual{allCards vis}{allCards tex}

\newpage
\hypertarget{allCards vis}{}
\subsection{AllOtherPartitionsRule}
\genHeader

\begin{itemize}

\item[$\blacktriangleright$] Create a new rule \texttt{AllOtherPartitionsRule}, and complete it according to Fig.~\ref{fig:ea_AllOtherPartitionsRuleComplete}.


\begin{figure}[htbp]
\begin{center}
  \includegraphics[width=\textwidth]{ea_AllOtherPartitionsRule}
  \caption{The completed \texttt{AllOtherPartitionsRule}}
  \label{fig:ea_AllOtherPartitionsRuleComplete}
\end{center}
\end{figure}

\item[$\blacktriangleright$] As you can see, this rule doesn't assume to know the final \texttt{partition} in the transformation. 
It matches the \texttt{n}th partition as the partition without any next partition, then connects a new \texttt{n+1}th partition to \texttt{n} and \texttt{partition0} (clear as every partitions previous is \texttt{partition0}).
Note that TGG transformations assume that the models are valid, i.e., have the expected structure (in our case meaning that the learning box is correctly ``wired'').\footnote{This should actually be formalised with a set of metamodel constraints that must be checked before a transformation is run, but we've omitted this here to simplify things.}  
Remember that ``blue'' means ``negative''.

\item[$\blacktriangleright$] Generate code for your improved TGG and re-run the transformation. 
It should work now without any error message.
Inspect the protocol to understand what happened.

\item[$\blacktriangleright$] Go ahead and add as many \texttt{partition}s and \texttt{card}s as you like to your model instance.
Your TGG is now also able to handle a \texttt{box} with any number of \texttt{partition}s beautifully.
For five partitions all with cards, the protocol gets quite interesting and is no longer a flat tree.
Try it out! 

\end{itemize}



%%% Local Variables: 
%%% mode: latex
%%% TeX-master: "../src/TGG_mainFile"
%%% End: 


\newpage
\hypertarget{allCards tex}{}
\subsection{AllOtherCardsRule}
\texHeader

\begin{itemize}

\item[$\blacktriangleright$] Right click on the \texttt{Rules} folder again and create \texttt{AllOtherCardsRule}. Complete each scope until your file resembles
Fig.~\ref{eclipse:allOtherCardsRuleComplete}.

\vspace{0.5cm}

\begin{figure}[htbp]
\begin{center}
  \includegraphics[width=0.6\textwidth]{eclipse_AllOtherCardsRule}
  \caption{A complete \texttt{AllOtherCardsRule}}
  \label{eclipse:AllOtherPartitionsRuleComplete}
\end{center}
\end{figure}

\item[$\blacktriangleright$] You'll notice that \texttt{box} and \texttt{partition0} have been established as `black' objects. This is so the rule may only be
evaluated when these objects are already translated, so we can use their values from the context of the transformation.

\vspace{0.5cm}

\item[$\blacktriangleright$] A second partition, \texttt{partitionN}, has also been established as part of the context. It represents the \texttt{n}th, or last
translated partition in a \texttt{box} (with an index of 2 or higher), whose \texttt{next} reference will also be translated in order to provide an access
link to the new \texttt{partitionNN} element.

\newpage

\item[$\blacktriangleright$] Given that the syntax of \texttt{add(a,b,c)} is \texttt{a+b=c}, the sole constraint of this rule sets the \texttt{index}
of the \texttt{n+1}th partition so that the \texttt{partition}s are still listed in order. Note that the correspondence and target scopes are empty, which is typical
for such ignore rules.

\vspace{0.5cm}

\item[$\blacktriangleright$] That's it! Save and build, then run the TGG again with the `extra' \texttt{partition} to confirm it worked!
If so, you are now free to add as many \texttt{partition}s and \texttt{card}s to \texttt{source.xmi} -- the transformation is now able to elegantly ignore them all.

\vspace{0.5cm}

\item[$\blacktriangleright$] Be sure to check out how this rule is implemented in eMoflon's visual syntax in Fig.~\ref{fig:ea_AllOtherPartitionsRuleComplete} from
the previous section.

\end{itemize}


% Not needed? remind users to check out and read the protocol in their syyntaxes for detail or run the integrator.
% \newpage
\section{The Protocol Explained}
\genHeader

% files are saved and built/refereshed in the Eclipse workspace.

Run TGGMain again. It should work without error this time. (make a note somewhere that if the dictionary is populated with wntries, but the properties have not
been updated to include stuff, the TGG will not work - it cannot operate with nulls).

Lets check out the protocol

{\bf protocol stuff here; it checks EVERY possibility}

Confirm/Explain

