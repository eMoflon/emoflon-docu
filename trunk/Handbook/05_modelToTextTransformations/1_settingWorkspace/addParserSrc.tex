\newpage
\hypertarget{subSec:setupParser}{}
\subsection{Setting up the Parser}
\genHeader

Before moving on, it should be quickly mentioned that our conventions and workflow state that the \emph{code adapter} package we established in the
previous step is a package that contains tree-to-model transformation logic. This logic \emph{could} be integrated directly in the corresponding metamodel
(\texttt{Dic\-tion\-ary\-Language}), but a separation makes sense here as there could be \emph{different} code adapters for the \emph{same} language.

With our graph grammar finally defined, we can finish setting up our transformation project by generating the ANTLR parser/unparser.

\begin{itemize}

\item[$\blacktriangleright$] Right-click on \texttt{DictionaryCodeAdapter} and navigate to ``eMoflon/ Add Parser/Unparser''
(Fig~\ref{eclipse:contextParser}).\footnote{For presentation purposes, this context menu screenshot has been edited}

\vspace{0.5cm}

\begin{figure}[htpb]
\begin{center}
  \includegraphics[width=0.8\textwidth]{eclipse_contextAddParserUnparser}
  \caption{Adding a new parser to a project}
  \label{eclipse:contextParser}
\end{center}
\end{figure}


\item[$\blacktriangleright$] In the parser settings window, enter \texttt{dictionary} as the \texttt{File extension}, and confirm the \texttt{Create Parser},
\texttt{Create Unparser}, and \texttt{ANTLR} options are selected as the corresponding technology for each case (Fig~\ref{eclipse:wizardParser}). Affirm by
pressing \texttt{Finish}.

\begin{figure}[htpb]
\begin{center}
  \includegraphics[width=0.8\textwidth]{eclipse_wizardParser}
  \caption{Project parser settings}
  \label{eclipse:wizardParser}
\end{center}
\end{figure}


\item[$\blacktriangleright$] If everything executed without error, parser and unparser stubs should generate and appear in the \texttt{src}
package, where \texttt{ANTLR} automatically built the corresponding Java packages (Fig.~\ref{eclipse:generatedParser}). In addition, a new
\texttt{in} folder should appear under \texttt{instances}. We'll explain later why this will be the source of any text files but for now, your setup
is complete!

\begin{figure}[htpb]
\begin{center}
  \includegraphics[width=0.4\textwidth]{eclipse_generatedParser}
  \caption{Generated parser package and stubs}
  \label{eclipse:generatedParser}
\end{center}
\end{figure}

\end{itemize}
