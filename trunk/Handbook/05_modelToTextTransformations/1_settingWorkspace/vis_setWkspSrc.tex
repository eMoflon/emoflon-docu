\newpage
\hypertarget{initialize vis}{}
\subsection{Establishing the visual TGG}
\visHeader

\begin{itemize}

\item[$\blacktriangleright$] From Eclipse, open \texttt{Dict\-ion\-ary.eap} in Enterprise Architect (EA). The project broswer should resemble
Fig.~\ref{ea:mocaTagged}. As you can see, the project is already populated with the source \texttt{MocaTree} specification for a generic tree structure.

\vspace{0.5cm}

\begin{figure}[htpb]
\begin{center}
  \includegraphics[width=0.4\textwidth]{ea_mocaTaggedValues}
  \caption{Preventing \texttt{MocaTree} from exporting to Eclipse}
  \label{ea:mocaTagged}
\end{center}
\end{figure}

\end{itemize}
\vspace{-0.5cm}
If you inspect the tagged values\footnote{The ``Tagged Values'' window can be opened by going to ``View/Tagged Values''} for each language, you'll notice that
the \texttt{MocaTree} package has the \texttt{Moflon::Export} value set to \texttt{false}. This ensures that the package is \emph{ignored} when exporting. As
with all standard metamodels (e.g., Ecore or the SDM metamodel) the \texttt{MocaTree} package in EA should be regarded as read-only, required only in the
EA project so that SDMs can refer to the classes defined in the package. As discussed, the Java code is provided and added automatically by our Eclipse plugin.

\begin{itemize}

\item[$\blacktriangleright$] Go ahead and inspect the \texttt{MocaTree} diagram (Fig.~\ref{ea:mocaTree}). Make sure you understand which attributes and
references each element contains until you feel comfortable with what you'll be working with.

\item[$\blacktriangleright$] Given that TGGs can only succeed when the involved metamodels are contained in the same working set, add a new package to
\texttt{My Working Set} named \texttt{Dict\-ion\-ary\-Code\-Adap\-ter}.

\newpage

\begin{figure}[htpb]
\begin{center}
  \includegraphics[width=\textwidth]{ea_metamodelMocaTree}
  \caption{The MocaTree Metamodel}
  \label{ea:mocaTree}
\end{center}
\end{figure}

\vspace{1cm}

\item[$\blacktriangleright$] Add a new TGG schema diagram as depicted in Fig.~\ref{ea:newTGGDiagram}. In the next dialogue that appears, set the source project
as \texttt{MocaTree}, and the target project as \texttt{Dict\-ion\-ary\-Lang\-uage}.

\vspace{1cm}

\begin{figure}[htpb]
\begin{center}
  \includegraphics[width=0.8\textwidth]{ea_adapterTGGDiagram}
  \caption{Create a new TGG Diagram}
  \label{ea:newTGGDiagram}
\end{center}
\end{figure}

\end{itemize}

\clearpage

\begin{itemize}

\item[$\blacktriangleright$] To ensure the package exports correctly to the Eclipse workspace as a TGG project, add a single correspondence type to your new
diagram (the \texttt{schema}) between \texttt{Folder} and \texttt{Library}. Remember -- you can get the classes by drag-and-dropping each element into the
diagram, then quick-linking a new \texttt{TGG Correspondence Type} between them.\footnote{For details on how to do this, refer to Part IV, Section \update.}
Your diagram should come to resemble Fig.~\ref{ea:firstCorrType}.

\vspace{0.5cm}

\begin{figure}[htpb]
\begin{center}
  \includegraphics[width=\textwidth]{ea_firstAdapterCorrespondence}
  \caption{The transformation's first correspondence type}
  \label{ea:firstCorrType}
\end{center}
\end{figure}

\vspace{0.5cm}
\item[$\blacktriangleright$] Your project broswer should also now resemble Fig.~\ref{ea:TGGProjBrow}, where \texttt{Dict\-ion\-ary\-Code\-Adap\-ter} has
transformed into a \texttt{TGGSchemaPackage}.

\begin{figure}[htpb]
\begin{center}
  \includegraphics[width=0.5\textwidth]{ea_TGGProjectBrowser}
  \caption{TGG Project prepared with both metamodels}
  \label{ea:TGGProjBrow}
\end{center}
\end{figure}

\item[$\blacktriangleright$] Save and validate the project via the eMoflon control panel.\footnote{Introducted in Part \update, Section \update.} Switch back
to Eclipse and refresh the package explorer. A new \texttt{Dict\-ion\-ary\-Code\-Adap\-ter} folder should appear under \texttt{My Working Set}; Your workspace
is nearly complete!

\jumpSingle{subSec:setupParser}

\end{itemize}
