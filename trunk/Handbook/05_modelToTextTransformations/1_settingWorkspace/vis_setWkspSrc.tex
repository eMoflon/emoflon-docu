\newpage
\hypertarget{initialize vis}{}
\subsection{Establishing visual TGGs}
\visHeader

\begin{itemize}

\item[$\blacktriangleright$] From Eclipse, open \texttt{Dict\-ion\-ary.eap} in Enterprise Architect (EA). The project browser should already resemble
Fig.~\ref{ea:mocaTagged}. As you can see, the project is populated with the source \texttt{MocaTree} specification for a generic tree structure in the
\texttt{eMoflon Languages} working set.

\vspace{0.5cm}

\begin{figure}[htpb]
\begin{center}
  \includegraphics[width=0.4\textwidth]{ea_mocaTaggedValues}
  \caption{Preventing \texttt{MocaTree} from exporting to Eclipse}
  \label{ea:mocaTagged}
\end{center}
\end{figure}

\end{itemize}
\vspace{-0.5cm}
If you inspect the tagged values\footnote{The ``Tagged Values'' window can be opened by going to ``View/Tagged Values'' or by hovering over the \texttt{Tagged
Values} tab immediately to the right of the project browser.} for each language, you'll notice that the \texttt{MocaTree} package has the
\texttt{Moflon::Export} value set to \texttt{false}. This ensures that the package is \emph{ignored} when exporting. As with all standard metamodels (e.g.,
Ecore or the SDM metamodel) the \texttt{MocaTree} package in EA should be regarded as read-only, required only in the EA project so that SDMs can refer to the
classes defined in the package.

\begin{itemize}

\item[$\blacktriangleright$] Go ahead and inspect the \texttt{MocaTree} diagram (Fig.~\ref{ea:mocaTree}). Make sure you understand which attributes and
references each element contains.

\item[$\blacktriangleright$] Despite \texttt{DictionaryLanguage} being contained in a different node than \texttt{MocaTree}, it's clear these two metamodels are
contained within the same project. Therefore, you can start a new TGG by adding a package to \texttt{My Working Set}. Name it
\texttt{Dict\-ion\-ary\-Code\-Adap\-ter}.

\newpage

\begin{figure}[htpb]
  \hspace{-3cm}
  \includegraphics[width=1.5\textwidth]{ea_metamodelMocaTree}
  \caption{The \texttt{MocaTree} metamodel}
  \label{ea:mocaTree}
\end{figure}

\item[$\blacktriangleright$] Select the package and add a new TGG schema diagram as depicted in Fig.~\ref{ea:newTGGDiagram}. In the next dialogue,
set the source project as \texttt{MocaTree}, and the target project as \texttt{Dict\-ion\-ary\-Lang\-uage}.

\begin{figure}[h!]
\begin{center}
  \includegraphics[width=0.9\textwidth]{ea_adapterTGGDiagram}
  \caption{Create a new TGG diagram}
  \label{ea:newTGGDiagram}
\end{center}
\end{figure}

\end{itemize}

\clearpage

\begin{itemize}

\item[$\blacktriangleright$] To ensure the package exports correctly to Eclipse as a TGG project, add a single correspondence type to your new
diagram (the \texttt{schema}) between \texttt{Folder} and \texttt{Library}. Remember -- you can get the classes by drag-and-dropping each element into the
diagram, then quick-linking a new \texttt{TGG Correspondence Type} between them.\footnote{For details on this correspondence metamodel and how to build types
between classes, refer to Part IV, Section 3.} Your diagram should come to resemble Fig.~\ref{ea:firstCorrType}.

\vspace{0.5cm}

\begin{figure}[htpb]
\begin{center}
  \includegraphics[width=\textwidth]{ea_firstAdapterCorrespondence}
  \caption{The transformation's first correspondence type}
  \label{ea:firstCorrType}
\end{center}
\end{figure}

\item[$\blacktriangleright$] Your project browser should now resemble Fig.~\ref{ea:TGGProjBrow}, where \texttt{Dict\-ion\-ary\-Code\-Adap\-ter} is
explicitly listed as a \texttt{TGGSchemaPackage}.

\vspace{0.5cm}

\begin{figure}[htpb]
\begin{center}
  \includegraphics[width=0.5\textwidth]{ea_TGGProjectBrowser}
  \caption{A fully prepared TGG project}
  \label{ea:TGGProjBrow}
\end{center}
\end{figure}

\item[$\blacktriangleright$] Save and validate your file via the eMoflon control panel,\footnote{Actviate via ``Extensions/Add-in Windows''} then switch
back to Eclipse and refresh the package explorer. A new \texttt{Dict\-ion\-ary\-Code\-Adap\-ter} folder should appear under \texttt{My Working Set}; Your TGG setup
is nearly complete!

\jumpSingle{subSec:setupParser}

\end{itemize}
