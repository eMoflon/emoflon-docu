\newpage
\section{Setting up the Model-to-Text workspace}
\genHeader

Nowadays, \emph{no one} really writes a complex parser completely by hand. Although this is sometimes still necessary,\footnote{Some languages are syntactically
challenging} most parsers can be whipped up pretty quickly using context-free \emph{string grammars}\footnote{For simple cases, \emph{regular
expressions} can also be used.} typically in EBNF.\footnote{Extended Backus-Naur Form}
ANTLR~\cite{ANTLR} is a tool that can generate a parser from a compact EBNF specification for a host of target programming languages, including Java.
Although ANTLR might not be the most efficient or powerful parser generator out there, it is open-source, well documented and supported, and allows for a
pragmatic and quite elegant \emph{fallback} to Java when things get nasty and we have to resort to some dirty tricks.


% --------------------------------------------------------------------------------------------------------------------------------------------------------
% 	Does there need to be a 'fresh start section?' i.e, do we need to briefly explain/tell where to look for info? Does user need to download anything from
% previous parts?
% \subsection{A fresh start}
% --------------------------------------------------------------------------------------------------------------------------------------------------------

For the next step, there are a couple of different thing you can do to build the actual \texttt{Dictionary} we wish to transform. You can\ldots
\begin{description}

% -- Metamodel Screenshots ---

\item[Option 1:] Start a new metamodel project (Fig.~\ref{fig:startDictLang}), and build it yourself until it matches either Fig.~\ref{ea_dictLang} or
Fig.~\ref{eclipse_dictLang} using what you learned in Part II.	

\begin{figure}[htbp]
\begin{center}
  \includegraphics[width=0.75\textwidth]{eclipse_startDictionary}
  \caption{Begin parser project}
  \label{fig:startDictLang}
\end{center}
\end{figure}

\newpage

\vspace*{1cm}

\begin{figure}[htb]
\begin{center}
  \includegraphics[width=\textwidth]{ea_dictionaryMetamodel}
  \caption{Build the \texttt{Dictionary} visual metamodel}
  \label{ea_dictLang}
\end{center}
\end{figure}

\vspace{1cm}

\begin{figure}[htb]
\begin{center}
  \includegraphics[width=0.8\textwidth]{eclipse_dictionaryMetamodel}
  \caption{Build \texttt{Dictionary} in Eclipse}
  \label{eclipse_dictLang}
\end{center}
\end{figure}

\newpage

% -- Export/Import ----

\item[Option 2:] Export, then import\footnote{To review how to import projects into a specific workspace, review Part IV, Section 2 for instructions.} the
\texttt{DictionaryLanguage} metamodel used in Part IV: Triple Graph Transformations into a new metamodel project (Fig.~\ref{ea_expImp} or
Fig.~\ref{eclipse_expImp})

\vspace{0.5cm}

\begin{figure}[htbp]
\begin{center}
  \includegraphics[width=0.65\textwidth]{ea_exportImport}
  \caption{Exporting your \texttt{DictionaryLanguage}}
  \label{ea_expImp}
\end{center}
\end{figure}

\begin{figure}[htbp]
\begin{center}
  \includegraphics[width=0.4\textwidth]{eclipse_exportImport}
  \caption{Exporting your \texttt{DictionaryLanguage}}
  \label{eclipse_expImp}
\end{center}
\end{figure}

\newpage

% -- Cheat Package(s)
\item[Option 3:] Download our cheat package using the Eclipse ``New" wizard to load a project with \texttt{DictionaryLanguage} fully prepared and ready to
go (Fig.~\ref{eclipse_cheatPackage}).

\vspace{0.5cm}

\begin{figure}[htbp]
\begin{center}
  \includegraphics[width=0.8\textwidth]{eclipse_loadDictionaryProject}
  \caption{Load \texttt{Dictionary} into your workspace}
  \label{eclipse_cheatPackage}
\end{center}
\end{figure}

\end{description} 

Once your dictionary is loaded, be sure to save, validate, and build before moving on to creating your parser.

\jumpDual{M2TSettngUp vis}{M2TSettngUp vis}

\newpage
\hypertarget{initialize vis}{}
\subsection{Establishing the visual TGG}
\visHeader

\begin{itemize}

\item[$\blacktriangleright$] From Eclipse, open \texttt{Dict\-ion\-ary.eap} in Enterprise Architect (EA). The project broswer should resemble
Fig.~\ref{ea:mocaTagged}. As you can see, the project is already populated with the source \texttt{MocaTree} specification for a generic tree structure.

\vspace{0.5cm}

\begin{figure}[htpb]
\begin{center}
  \includegraphics[width=0.4\textwidth]{ea_mocaTaggedValues}
  \caption{Preventing \texttt{MocaTree} from exporting to Eclipse}
  \label{ea:mocaTagged}
\end{center}
\end{figure}

\end{itemize}
\vspace{-0.5cm}
If you inspect the tagged values\footnote{The ``Tagged Values'' window can be opened by going to ``View/Tagged Values''} for each language, you'll notice that
the \texttt{MocaTree} package has the \texttt{Moflon::Export} value set to \texttt{false}. This ensures that the package is \emph{ignored} when exporting. As
with all standard metamodels (e.g., Ecore or the SDM metamodel) the \texttt{MocaTree} package in EA should be regarded as read-only, required only in the
EA project so that SDMs can refer to the classes defined in the package. As discussed, the Java code is provided and added automatically by our Eclipse plugin.

\begin{itemize}

\item[$\blacktriangleright$] Go ahead and inspect the \texttt{MocaTree} diagram (Fig.~\ref{ea:mocaTree}). Make sure you understand which attributes and
references each element contains until you feel comfortable with what you'll be working with.

\item[$\blacktriangleright$] Given that TGGs can only succeed when the involved metamodels are contained in the same working set, add a new package to
\texttt{My Working Set} named \texttt{Dict\-ion\-ary\-Code\-Adap\-ter}.

\newpage

\begin{figure}[htpb]
\begin{center}
  \includegraphics[width=\textwidth]{ea_metamodelMocaTree}
  \caption{The MocaTree Metamodel}
  \label{ea:mocaTree}
\end{center}
\end{figure}

\vspace{1cm}

\item[$\blacktriangleright$] Add a new TGG schema diagram as depicted in Fig.~\ref{ea:newTGGDiagram}. In the next dialogue that appears, set the source project
as \texttt{MocaTree}, and the target project as \texttt{Dict\-ion\-ary\-Lang\-uage}.

\vspace{1cm}

\begin{figure}[htpb]
\begin{center}
  \includegraphics[width=0.8\textwidth]{ea_adapterTGGDiagram}
  \caption{Create a new TGG Diagram}
  \label{ea:newTGGDiagram}
\end{center}
\end{figure}

\end{itemize}

\clearpage

\begin{itemize}

\item[$\blacktriangleright$] To ensure the package exports correctly to the Eclipse workspace as a TGG project, add a single correspondence type to your new
diagram (the \texttt{schema}) between \texttt{Folder} and \texttt{Library}. Remember -- you can get the classes by drag-and-dropping each element into the
diagram, then quick-linking a new \texttt{TGG Correspondence Type} between them.\footnote{For details on how to do this, refer to Part IV, Section \update.}
Your diagram should come to resemble Fig.~\ref{ea:firstCorrType}.

\vspace{0.5cm}

\begin{figure}[htpb]
\begin{center}
  \includegraphics[width=\textwidth]{ea_firstAdapterCorrespondence}
  \caption{The transformation's first correspondence type}
  \label{ea:firstCorrType}
\end{center}
\end{figure}

\vspace{0.5cm}
\item[$\blacktriangleright$] Your project broswer should also now resemble Fig.~\ref{ea:TGGProjBrow}, where \texttt{Dict\-ion\-ary\-Code\-Adap\-ter} has
transformed into a \texttt{TGGSchemaPackage}.

\begin{figure}[htpb]
\begin{center}
  \includegraphics[width=0.5\textwidth]{ea_TGGProjectBrowser}
  \caption{TGG Project prepared with both metamodels}
  \label{ea:TGGProjBrow}
\end{center}
\end{figure}

\item[$\blacktriangleright$] Save and validate the project via the eMoflon control panel.\footnote{Introducted in Part \update, Section \update.} Switch back
to Eclipse and refresh the package explorer. A new \texttt{Dict\-ion\-ary\-Code\-Adap\-ter} folder should appear under \texttt{My Working Set}; Your workspace
is nearly complete!

\jumpSingle{subSec:setupParser}

\end{itemize}


\newpage
\hypertarget{initialize tex}{}
\subsection{Initializing the project}
\texHeader



After confirming your \texttt{Dictionary} package explorer resembles Fig.~\ref{eclipse:dictLang}, open \texttt{\_imports.mconf}
(Fig.~\ref{eclipse:standardImports}). You'll notice that it's already accessing the \texttt{MocaTree} metamodel, but where is it?

\begin{figure}[htbp]
\begin{center}
  \includegraphics[width=0.4\textwidth]{eclipse_importsFile}
  \caption{\texttt{DictionaryLanguage}'s imports file}
  \label{eclipse:standardImports}
\end{center}
\end{figure}

\texttt{MocaTree} is classified as a standard language, so every eMoflon project includes this specification as a hidden file. Therefore, you're not able to
inspect your target domain directly in Eclipse. Instead, we recommend reviewing Fig.~\ref{ea:mocaTree} which depicts \texttt{MocaTree} in the visual syntax.
Make sure you understand the classes and references before continuing.

\begin{enumerate}

\item[$\blacktriangleright$] Let's establish the TGG we'll use to transform between  \texttt{MocaTree} and\texttt{Dictionary}. Right-click on
\texttt{MyWorkingSet}, and navigate to ``New/ TGG.''

\item[$\blacktriangleright$] Name the package \texttt{DictionaryCodeAdapter}, setting the source as \texttt{MocaTree} and
target as \texttt{DictionaryLanguage} (Fig.~\ref{eclipse:newTGGProject}).

\begin{figure}[htbp]
\begin{center}
  \includegraphics[width=0.9\textwidth]{eclipse_dictionaryCodeAdapterTGGProject}
  \caption{Settings for our TGG}
  \label{eclipse:newTGGProject}
\end{center}
\end{figure}

\item[$\blacktriangleright$] A \texttt{schema.sch} file should have automatically opened in the editor. To ensure this TGG package is successfully generated as
a TGG, let's add something to this by creating our first correspondence type. Specify one between a tree's \texttt{Folder} instance, and a dictionary's
\texttt{Library} as depicted in Fig~\ref{eclipse:firstSchema}.\footnote{For details on this correspondence metamodel and how to build types
between classes, refer to Part IV, Section 3.} Don't forget -- you can use eMolfon's auto-completion feature here!

\begin{figure}[htbp]
\begin{center}
  \includegraphics[width=0.5\textwidth]{eclipse_schemaStart}
  \caption{A first correspondence type between domains}
  \label{eclipse:firstSchema}
\end{center}
\end{figure}

\item[$\blacktriangleright$] Save and build your project! Confirm the generated project has a solid black hexagon symbol, not a plug icon
overlaying the folder. This shape indicates the \texttt{Dictionary} is a TGG Package, and is not a standard ECore project (the default generation type).

\end{enumerate}

