\newpage
\section{Setting up the workspace}
\genHeader

Nowadays, \emph{no one} writes a complex parser completely by hand. Although this is sometimes still necessary for syntactically challenging languages, most
parsers can be quickly whipped up using context-free \emph{string grammars}.\footnote{For simple cases, \emph{regular expressions} can also be used} These are
typically written in Extended Backus-Naur Form (EBNF)\define{EBNF}. ANTLR~\cite{ANTLR} is a tool that can generate a parser from this compact specification for
a host of target programming languages, including Java. Although ANTLR might not be the most efficient or powerful parser generator, it is open-source, well
documented and supported, and allows for a pragmatic and elegant fallback to Java when things get nasty and we have to resort to some dirty tricks to get our
job done.

A parser, of course, is based on whatever model it must work with. Given that we want to parse to and from a dictionary, we need to have the
\texttt{DictionaryLanguage} metamodel defined before starting. The dictionary is a relatively simple device, so you have three options to loading this project
into your workspace:

\begin{description}

% -- Develop Metamodel ---
\item[Option 1: Develop] You can start a new metamodel project called \texttt{Dict\-ion\-aryLanguage} (Fig.~\ref{eclipse:startMetamodel}) and complete a
\texttt{Dictionary} model until it matches either Fig.~\ref{ea:dictLang} or Fig.~\ref{eclipse:dictLang}.\footnote{Review metamodel construction by
working through Part II: Ecore}

\begin{figure}[htbp]
\begin{center}
  \includegraphics[width=0.65\textwidth]{eclipse_startDictionary}
  \caption{Begin parser project}
  \label{eclipse:startMetamodel}
\end{center}
\end{figure}

\newpage % ---

\vspace*{1cm}

\begin{figure}[htb]
\begin{center}
  \includegraphics[width=\textwidth]{ea_dictionaryMetamodel}
  \caption{Create the package in \texttt{My Working Set} with an eMoflon Ecore diagram of the same name.}
  \label{ea:dictLang}
\end{center}
\end{figure}

\vspace{1cm}

\begin{figure}[htb]
\begin{center}
  \includegraphics[width=0.8\textwidth]{eclipse_dictionaryMetamodel}
  \caption{\texttt{Dictionary} constructed in Eclipse \update (constraints)}
  \label{eclipse:dictLang}
\end{center}
\end{figure}

\newpage % ---

% -- Download Cheat Package ---
\item[Option 2: Download] You can download our cheat package and load a \texttt{Dictionary} specification into your workspace using
the Eclipse ``New" wizard (Fig.~\ref{eclipse_cheatPackage}).

\vspace{0.5cm}

\begin{figure}[htbp]
\begin{center}
  \includegraphics[width=0.65\textwidth]{eclipse_loadDictionaryProject}
  \caption{Load a \texttt{Dictionary} project into your workspace}
  \label{eclipse_cheatPackage}
\end{center}
\end{figure}

\item[$\blacktriangleright$] Pressing \texttt{Finish} will load either a MOSL directory or \texttt{Dict\-ion\-ary\-Lang\-uage.eap} file into a working set,
depending on your syntax choice (Fig.~\ref{ea:cheatLoaded} or Fig.~\ref{eclipse:cheatLoaded}).

\begin{figure}[htbp]
   \centering
      \subfloat[comment 1 \update]{
        \includegraphics[width=0.45\textwidth]{eclipse_loadedCheatPackageVisual}
        \label{ea:cheatLoaded}
      }
      \subfloat[Bugger. Comment 2]{
        \includegraphics[width=0.45\textwidth]{eclipse_loadedCheatPackageVisual}
        \label{eclipse:cheatLoaded}
      }
      \caption{}
\end{figure}

\clearpage % ---

% -- Export/Import ---
\item[Option 3: Import] You can simply use the same files for \texttt{Dictionary} as you used in Part IV. Export the specification files, then import them
into a new project in a fresh workspace. Read Part IV, Section 2 for a detailed review on how to do this in either syntax.

\end{description} 
% -- End

\vspace{0.5cm}

This metamodel is one of two that we'll be using to specify the transformation. After all, TGGs \emph{always} require a source and target graph, and this will
represent our target. If you haven't already, we recommend inspecting \texttt{DictionaryLanguage} until you feel comfortable with what you'll be working with.
Its important to understand which classes are connected.

Our second metamodel will be eMoflon's standard \texttt{MocaTree} language. It basically combines concepts from a filesystem (folders and files), XML concepts
(text-only nodes and attributes), and a general indexed containment hierarchy.\footnote{This is done with an index attribute, which can be used to demand a
certain \emph{order} of nodes in an SDM which are otherwise not guaranteed by default.} It models a generic directory structure with a \texttt{Folder} able to
connect to other \texttt{Folders}, and may contain an unlimited number of \texttt{File} elements. Its Java code is provided by the Eclipse plugin and
automatically added to the build path.  We'll go into more detail about this later, so let's begin the next task: initialize the TGG project that will drive our
model-to-text transformation.

\jumpDual{initialize vis}{initialize tex}

\newpage
\hypertarget{initialize vis}{}
\subsection{Establishing the visual TGG}
\visHeader

\begin{itemize}

\item[$\blacktriangleright$] From Eclipse, open \texttt{Dict\-ion\-ary.eap} in Enterprise Architect (EA). The project broswer should resemble
Fig.~\ref{ea:mocaTagged}. As you can see, the project is already populated with the source \texttt{MocaTree} specification for a generic tree structure.

\vspace{0.5cm}

\begin{figure}[htpb]
\begin{center}
  \includegraphics[width=0.4\textwidth]{ea_mocaTaggedValues}
  \caption{Preventing \texttt{MocaTree} from exporting to Eclipse}
  \label{ea:mocaTagged}
\end{center}
\end{figure}

\end{itemize}
\vspace{-0.5cm}
If you inspect the tagged values\footnote{The ``Tagged Values'' window can be opened by going to ``View/Tagged Values''} for each language, you'll notice that
the \texttt{MocaTree} package has the \texttt{Moflon::Export} value set to \texttt{false}. This ensures that the package is \emph{ignored} when exporting. As
with all standard metamodels (e.g., Ecore or the SDM metamodel) the \texttt{MocaTree} package in EA should be regarded as read-only, required only in the
EA project so that SDMs can refer to the classes defined in the package. As discussed, the Java code is provided and added automatically by our Eclipse plugin.

\begin{itemize}

\item[$\blacktriangleright$] Go ahead and inspect the \texttt{MocaTree} diagram (Fig.~\ref{ea:mocaTree}). Make sure you understand which attributes and
references each element contains until you feel comfortable with what you'll be working with.

\item[$\blacktriangleright$] Given that TGGs can only succeed when the involved metamodels are contained in the same working set, add a new package to
\texttt{My Working Set} named \texttt{Dict\-ion\-ary\-Code\-Adap\-ter}.

\newpage

\begin{figure}[htpb]
\begin{center}
  \includegraphics[width=\textwidth]{ea_metamodelMocaTree}
  \caption{The MocaTree Metamodel}
  \label{ea:mocaTree}
\end{center}
\end{figure}

\vspace{1cm}

\item[$\blacktriangleright$] Add a new TGG schema diagram as depicted in Fig.~\ref{ea:newTGGDiagram}. In the next dialogue that appears, set the source project
as \texttt{MocaTree}, and the target project as \texttt{Dict\-ion\-ary\-Lang\-uage}.

\vspace{1cm}

\begin{figure}[htpb]
\begin{center}
  \includegraphics[width=0.8\textwidth]{ea_adapterTGGDiagram}
  \caption{Create a new TGG Diagram}
  \label{ea:newTGGDiagram}
\end{center}
\end{figure}

\end{itemize}

\clearpage

\begin{itemize}

\item[$\blacktriangleright$] To ensure the package exports correctly to the Eclipse workspace as a TGG project, add a single correspondence type to your new
diagram (the \texttt{schema}) between \texttt{Folder} and \texttt{Library}. Remember -- you can get the classes by drag-and-dropping each element into the
diagram, then quick-linking a new \texttt{TGG Correspondence Type} between them.\footnote{For details on how to do this, refer to Part IV, Section \update.}
Your diagram should come to resemble Fig.~\ref{ea:firstCorrType}.

\vspace{0.5cm}

\begin{figure}[htpb]
\begin{center}
  \includegraphics[width=\textwidth]{ea_firstAdapterCorrespondence}
  \caption{The transformation's first correspondence type}
  \label{ea:firstCorrType}
\end{center}
\end{figure}

\vspace{0.5cm}
\item[$\blacktriangleright$] Your project broswer should also now resemble Fig.~\ref{ea:TGGProjBrow}, where \texttt{Dict\-ion\-ary\-Code\-Adap\-ter} has
transformed into a \texttt{TGGSchemaPackage}.

\begin{figure}[htpb]
\begin{center}
  \includegraphics[width=0.5\textwidth]{ea_TGGProjectBrowser}
  \caption{TGG Project prepared with both metamodels}
  \label{ea:TGGProjBrow}
\end{center}
\end{figure}

\item[$\blacktriangleright$] Save and validate the project via the eMoflon control panel.\footnote{Introducted in Part \update, Section \update.} Switch back
to Eclipse and refresh the package explorer. A new \texttt{Dict\-ion\-ary\-Code\-Adap\-ter} folder should appear under \texttt{My Working Set}; Your workspace
is nearly complete!

\jumpSingle{subSec:setupParser}

\end{itemize}


\newpage
\hypertarget{initialize tex}{}
\subsection{Initializing the project}
\texHeader



After confirming your \texttt{Dictionary} package explorer resembles Fig.~\ref{eclipse:dictLang}, open \texttt{\_imports.mconf}
(Fig.~\ref{eclipse:standardImports}). You'll notice that it's already accessing the \texttt{MocaTree} metamodel, but where is it?

\begin{figure}[htbp]
\begin{center}
  \includegraphics[width=0.4\textwidth]{eclipse_importsFile}
  \caption{\texttt{DictionaryLanguage}'s imports file}
  \label{eclipse:standardImports}
\end{center}
\end{figure}

\texttt{MocaTree} is classified as a standard language, so every eMoflon project includes this specification as a hidden file. Therefore, you're not able to
inspect your target domain directly in Eclipse. Instead, we recommend reviewing Fig.~\ref{ea:mocaTree} which depicts \texttt{MocaTree} in the visual syntax.
Make sure you understand the classes and references before continuing.

\begin{enumerate}

\item[$\blacktriangleright$] Let's establish the TGG we'll use to transform between  \texttt{MocaTree} and\texttt{Dictionary}. Right-click on
\texttt{MyWorkingSet}, and navigate to ``New/ TGG.''

\item[$\blacktriangleright$] Name the package \texttt{DictionaryCodeAdapter}, setting the source as \texttt{MocaTree} and
target as \texttt{DictionaryLanguage} (Fig.~\ref{eclipse:newTGGProject}).

\begin{figure}[htbp]
\begin{center}
  \includegraphics[width=0.9\textwidth]{eclipse_dictionaryCodeAdapterTGGProject}
  \caption{Settings for our TGG}
  \label{eclipse:newTGGProject}
\end{center}
\end{figure}

\item[$\blacktriangleright$] A \texttt{schema.sch} file should have automatically opened in the editor. To ensure this TGG package is successfully generated as
a TGG, let's add something to this by creating our first correspondence type. Specify one between a tree's \texttt{Folder} instance, and a dictionary's
\texttt{Library} as depicted in Fig~\ref{eclipse:firstSchema}.\footnote{For details on this correspondence metamodel and how to build types
between classes, refer to Part IV, Section 3.} Don't forget -- you can use eMolfon's auto-completion feature here!

\begin{figure}[htbp]
\begin{center}
  \includegraphics[width=0.5\textwidth]{eclipse_schemaStart}
  \caption{A first correspondence type between domains}
  \label{eclipse:firstSchema}
\end{center}
\end{figure}

\item[$\blacktriangleright$] Save and build your project! Confirm the generated project has a solid black hexagon symbol, not a plug icon
overlaying the folder. This shape indicates the \texttt{Dictionary} is a TGG Package, and is not a standard ECore project (the default generation type).

\end{enumerate}


\newpage
\hypertarget{subSec:setupParser}{}
\subsection{Setting up the Parser}
\genHeader

The final workspace requirement that needs to be met is the creation of our transformation's ANTLR parser/unparser.
\begin{itemize}

\item[$\blacktriangleright$] Right-click on the \texttt{DictionaryCodeAdapter} folder and navigate to ``eMolfon/ Add Parser/Unparser''
(Fig~\ref{eclipse:contextParser}).

\vspace{0.5cm}

\begin{figure}[htpb]
\begin{center}
  \includegraphics[width=0.8\textwidth]{eclipse_contextAddParserUnparser}
  \caption{figureCaption}
  \label{eclipse:contextParser}
\end{center}
\end{figure}


\item[$\blacktriangleright$] In the parser settings window, enter \texttt{dictionary} as the \texttt{File extension}, and confirm the \texttt{Create Parser},
\texttt{Create Unparser}, and \texttt{ANTLR} options are selected as the corresponding technology for each case (Fig~\ref{eclipse:wizardParser}). Affirm by
pressing \texttt{Finish}.

\begin{figure}[htpb]
\begin{center}
  \includegraphics[width=0.8\textwidth]{eclipse_wizardParser}
  \caption{figureCaption}
  \label{eclipse:wizardParser}
\end{center}
\end{figure}

\vspace{0.5cm}

\item[$\blacktriangleright$] If everything was installed and completed without error, parser and unparser stubs should be generated under
\texttt{DictionaryCodeAdapter}, where \texttt{ANTLR} automatically built the corresponding Java packages as depicted in Fig.~\ref{eclipse:generatedParser}.

\begin{figure}[htpb]
\begin{center}
  \includegraphics[width=0.4\textwidth]{eclipse_generatedParser}
  \caption{figureCaption}
  \label{eclipse:generatedParser}
\end{center}
\end{figure}

\vspace{0.5cm}

\item[$\blacktriangleright$] Your workspace is now fully prepared, and you're ready to start transforming a textual source into a \texttt{Dictionary}
model!

\end{itemize}

