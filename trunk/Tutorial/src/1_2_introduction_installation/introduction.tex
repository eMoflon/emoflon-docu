\chapter{Introduction}
\label{chap:introduction}

This tutorial has been engineered to be \emph{fun}.

If you work through it and, for some reason, do \emph{not} have a resounding \mbox{``I-Rule''} feeling afterwards, please send us an email and tell us how to improve it: \href{mailto:contact@moflon.org}{contact@moflon.org}

%\usepackage{graphics} is needed for \includegraphics
\begin{figure}[htp]
\begin{center}
  \includegraphics[height=0.45\textheight]{pics/installationAndSetup/i-rule}
  \caption{How you should feel when you're done.}
  \label{i-rule}
\end{center}
\end{figure}

To enjoy the experience, you should be fairly comfortable with Java or a comparable object-oriented langauge, and know how to perform basic tasks in Eclipse.  Although we assume this, we give references to help bring you up to speed as necessary.  
Last but not least, very basic knowledge of common UML notation would be helpful.

Our goal is to give a \emph{hands-on} introduction to metamodelling and graph transformations using our tool \emph{eMoflon}.
The idea is to \emph{learn by doing} and all concepts are introduced while working on a concrete example.
The language and style used throughout is intentionally relaxed and non-academic.
For those of you interested in further details and the mature formalism of graph transformations, we give relevant references throughout the tutorial.

As the tutorial is quite a few pages long, here are a few suggestions of what you must read or can skip depending on what you're interested in:

\begin{description}
\item[Chapter~\ref{chap:installation}] provides a very simple example and a few JUnit tests to test the installation and configuration of eMoflon.
 
After working through this chapter, you should have an installed and tested eMoflon working for a trivial example.
We also explain the general workflow and the different workspaces involved.

This chapter can be considered \emph{mandatory} if you are new to eMoflon and we recommend working through it in any case.
It's also kept as minimal as possible and should only take a few minutes really.

\item[Chapter \ref{chap:membox}] is the main chapter and takes you step-by-step through a more realistic example that showcases most of the features we currently support.

Working through this chapter should serve as a basic introduction to model-driven engineering, metamodelling and graph transformations.

You should definitely work through this chapter if you're new to metamodelling in general (using Ecore/EMF) and if you're interested in model transformation via graph transformation using \emph{Story Driven Modelling} (SDM).

\item[Chapter \ref{chap:Tips and Tricks}] gives a few tips and tricks for using our frontend Enterprise Architect (EA) effectively and avoiding typical mistakes.

If you're in a hurry this chapter can be skipped and used as a reference when you get stuck with EA or can't figure out how to do something.

\item[Chapter \ref{chap:A-Dictionary-Language}] treats \emph{model-to-text} transformation and how eMoflon can be used together with parsers and template languages.

If you're only interested in model-to-model transformation then you can safely skip this chapter and come back to it only when you need to generate code or extract your models from XML or some other textual format.
 
\item[Chapter \ref{chap:Learning-Box-to-Dictionary-and-Back-Again}] introduces \emph{Triple Graph Grammars} (TGGs) used for bidirectional model transformation.

Feel free to jump directly here after working through at least Chap.~\ref{chap:installation} if you're mainly interested in TGGs.
Although the example builds up on parts constructed in previous chapters, we provide a ``cheat'' package that you can use to get started directly.
 
\end{description}

One last thing: at the moment we unfortunately only support Windows. 
This should hopefully change in future releases.

That's it -- sit back, relax, grab a coffee and enjoy the ride!


