\section{Setup your M2T Workspace}

The next step is preparing an Eclipse workspace according to our suggested workflow.

\begin{enumerate}
\item[$\blacktriangleright$] In Eclipse (we recommend an empty workspace), switch to the Moflon perspective.
\item[$\blacktriangleright$] Invoke the \texttt{New Metamodel} wizard (Fig.~\ref{fig:moca-1-NewMetamodelWizard}). 

%\usepackage{graphics} is needed for \includegraphics
\begin{figure}[!htbp]
\begin{center}
 \includegraphics[width=0.3\textwidth]{pics/moca/1DictionaryMetaModel/1-NewMetamodelWizard}
  \caption{Invoking the \texttt{New Metamodel} wizard.}
  \label{fig:moca-1-NewMetamodelWizard}
\end{center}
\end{figure}

\item[$\blacktriangleright$] Choose ``Dictionary'' as project name and select \texttt{Add Moca Support} as depicted in Fig.~\ref{fig:moca-2-AddMocaSupport-ProjectName}. 

%\usepackage{graphics} is needed for \includegraphics
\begin{figure}[!htbp]
\begin{center}
 \includegraphics[width=0.7\textwidth]{pics/moca/1DictionaryMetaModel/2-AddMocaSupport-ProjectName}
  \caption{Add Metamodel project with MOCA support}
  \label{fig:moca-2-AddMocaSupport-ProjectName}
\end{center}
\end{figure}

\clearpage

\item[$\blacktriangleright$] After the project is created as usual (Fig. \ref{fig:moca-3-WizardResult}) double-click the EAP file to open it.

%\usepackage{graphics} is needed for \includegraphics
\begin{figure}[!htbp]
\begin{center}
 \includegraphics[width=0.3\textwidth]{pics/moca/1DictionaryMetaModel/3-WizardResult}
  \caption{Eclipse workspace after creating the \texttt{Dictionary} project}
  \label{fig:moca-3-WizardResult}
\end{center}
\end{figure}

\item[$\blacktriangleright$] In EA, the project is already populated with the metamodel for our generic tree.
To differentiate this from other trees (ANTLR parse tree and abstract syntax tree, XML DOM tree, \ldots) we shall refer to it as \texttt{MocaTree} (Fig.~\ref{fig:moca-4-eapContainsMocatreeWithExportFalse}).
Note that the \texttt{MocaTree} package has a special tagged value \texttt{Moflon::Export} that is set to \texttt{false}\footnote{Tagged values can be viewed in the \texttt{Tagged Values} view in EA (Fig.~\ref{fig:moca-4-eapContainsMocatreeWithExportFalse}).}.
This ensures that the package is \emph{ignored} when exporting.
As with all standard metamodels (e.g. Ecore) the \texttt{MocaTre} package in EA should be regarded as read-only and is only required in the EA project so that SMDs can refer to the classes defined in the package.
The corresponding Java code is provided by our Eclipse plugin and is added automatically to the Java build path when necessary.

%\usepackage{graphics} is needed for \includegraphics
\begin{figure}[!htbp]
\begin{center}
 \includegraphics[width=0.4\textwidth]{pics/moca/1DictionaryMetaModel/4-eapContainsMocatreeWithExportFalse}
  \caption{\texttt{MocaTree} in default EA project}
  \label{fig:moca-4-eapContainsMocatreeWithExportFalse}
\end{center}
\end{figure}
\end{enumerate}

\clearpage

Go ahead and inspect the \texttt{MocaTree} metamodel (Fig.~\ref{fig:moca-tree}).
It basically combines concepts from a filesystem (folders and files), XML concepts like text nodes and attributes, and a general node containment hierarchy.

\begin{figure}[!htbp]
\begin{center}
 \includegraphics[width=\textwidth]{pics/moca/0Install/0-MocaTree}
  \caption{Mocatree}
  \label{fig:moca-tree}
\end{center}
\end{figure}
 
\begin{enumerate}

\item[$\blacktriangleright$] Add a new package \texttt{DictionaryLanguage} and model the required classes and relationships for our dictionary language (Fig.~\ref{fig:moca-5-DictionaryMM}).

Every dictionary has a title and consists of entries.
Entries have a content and a level that indicates how difficult the entry is.
Dictionaries can be organized in shelves that have a description and shelves can be collected in a library.
To make things interesting, each dictionary has an author.
Note that arbitrary many different dictionaries, irrespective of their shelves, can share the same author.

%\usepackage{graphics} is needed for \includegraphics
\begin{figure}[!htbp]
\begin{center}
 \includegraphics[width=\textwidth]{pics/moca/1DictionaryMetaModel/DictionaryLanguage}
  \caption{Dictionary metamodel}
  \label{fig:moca-5-DictionaryMM}
\end{center}
\end{figure}

\clearpage

\item[$\blacktriangleright$] For the moment, just add an empty package in EA for a \emph{code adapter} so that your EA workspace closely resembles Fig.~\ref{fig:moca-5-DictionaryMM-ProjectBrowser}.

According to our conventions and workflow, a code adapter is a package that contains the tree-to-model transformation logic.
This could of course be integrated directly in the corresponding metamodel (\texttt{Dictionary\-Language} in our case), but a separation makes sense as their could be \emph{different} code adapters for the same language.


%\usepackage{graphics} is needed for \includegraphics
\begin{figure}[!htbp]
\begin{center}
 \includegraphics[width=0.3\textwidth]{pics/moca/1DictionaryMetaModel/5-DictionaryMM-ProjectBrowser}
  \caption{EA workspace before exporting}
  \label{fig:moca-5-DictionaryMM-ProjectBrowser}
\end{center}
\end{figure}

\item[$\blacktriangleright$] Export as usual and ensure that your Eclipse workspace closely resembles Fig.~\ref{fig:moca-6-ExportToEclipse}.

%\usepackage{graphics} is needed for \includegraphics
\begin{figure}[!htbp]
\begin{center}
 \includegraphics[width=0.25\textwidth]{pics/moca/1DictionaryMetaModel/6-ExportToEclipse}
  \caption{Workspace after export to eclipse}
  \label{fig:moca-6-ExportToEclipse}
\end{center}
\end{figure}

\end{enumerate}