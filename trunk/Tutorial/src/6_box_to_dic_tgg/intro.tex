\chapter{Learning Box to Dictionary \emph{and} Back Again with TGGs}
\label{chap:Learning-Box-to-Dictionary-and-Back-Again}

If you are joining us directly in this chapter and are only interested in bidirectional model transformation with \emph{Triple Graph Grammars} then welcome!

To get eMoflon up and running, however, you should at least work through Chapter~\ref{chap:installation} on installation and set-up.
We try to assume as little as possible in the rest of this chapter and give appropriate references when we use terminology introduced in previous chapters.

Up until now in the tutorial, we have gotten to know what a learning box is, we have specified its \emph{abstract syntax} and \emph{static semantics} as a \emph{metamodel} and its \emph{dynamic semantics} with graph transformations (SDMs).
If the previous sentence could just as well have been in Chinese\footnote{Replace with Greek if you are Chinese.  If you are Chinese but speak fluent Greek then we give up - you get the point anyway right?} for you then please work through Chapter~\ref{chap:membox}.

We have also established a simple textual \emph{concrete syntax} for a dictionary and implemented \emph{model-to-text} and \emph{text-to-model} transformations with a merry mix of string grammars, graph transformations and templates (Chapter~\ref{chap:A-Dictionary-Language}).
As this was already in both directions (model$\Leftrightarrow$text) we have practically seen how to implement a \emph{bidirectional} transformation as two unidirectional transformations using a unidirectional transformation language (SDMs).
Although SDMs are crazily cool (don't you forget that!), this is rather unsatisfactory.  
If you take a careful and critical look at the forward (text-to-model) and the backward (model-to-text) transformations, you should be able to notice the following problems.
\begin{description}
  \item[Productivity:] We have to implement two transformations that are actually quite similar\ldots ~somehow.  This does not really feel productive.  Wouldn't it be nice to implement maybe the forward transformation and get the backward transformation for free?  Or vice-versa?  Or how about deriving forward \emph{and} backward transformations from a \emph{common} joint specification?
  \item[Maintainability:] Another, maybe even more important point, is that two separate transformations become quite a pain to maintain and keep \emph{consistent}.  If the forward transformation is adjusted to produce a different target structure, the backward transformation must be updated appropriately and vice-versa.  Again it would be great to get this for free or at least have some support.
  \item[Traceability] Finally, one often needs to identify the reason why a certain object has been created during a transformation process. 
This increases the trust in the specified transformation and is essential for working with systems that may actually do some harm (e.g., automotive or medical systems). 
With two separate transformations, \emph{traceability} would have to be supported manually and, once again, it would be nice to have this for free!
\end{description}

As we shall see in this chapter, Triple Graph Grammars (TGGs) are a \emph{bidirectional} transformation language that solves these problems.

After a brief introduction to TGGs, we shall specify a bidirectional transformation between learning boxes and dictionaries.
This transformation has already been motivated and explained in Chapter~\ref{chap:membox}, so lets move on. 