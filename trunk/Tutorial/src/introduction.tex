\chapter{Introduction}
\label{chap:introduction}

This tutorial has been engineered to be \emph{fun}.

If you work through it and, for some reason, do \emph{not} have a resounding
``I-Rule'' feeling afterwards, please send us an email and tell us how to improve it:
\url{contact@moflon.org}

%\usepackage{graphics} is needed for \includegraphics
\begin{figure}[htp]
\begin{center}
  \includegraphics[height=0.45\textheight]{pics/i-rule}
  \caption{How you should feel when you're done.}
  \label{i-rule}
\end{center}
\end{figure}

\newpage

To enjoy the experience, you should be fairly comfortable with Java or
a comparable object-oriented langauge, and know how to perform basic tasks
in Eclipse.  Although we assume this, we give references to help bring you up to
speed as necessary.  Last but not least, very basic knowledge of
common UML notation would be helpful.

Our goal is to give a \emph{hands-on} introduction to metamodelling and graph
transformations using our tool \emph{eMoflon}.
The idea is to \emph{learn by doing} and all concepts are introduced while
working on a concrete example.
The language and style used throughout is intentionally relaxed and
non-academic.
For those of you interested in further details and the mature
formalism of graph transformations, we give relevant references throughout the
tutorial.

The tutorial is divided into two main parts:  In the first part (Chapter
\ref{chap:installation}), we provide a very simple example and a few JUnit tests
to test the installation and configuration of eMoflon.

After working through this chapter, you should have an installed
and tested eMoflon working for a trivial example.
We also explain the general workflow and the different workspaces involved.

In the second part of the tutorial (Chapter \ref{chap:membox}), we go,
step-by-step, through a more realistic example that showcases most of the
features we currently support.

Working through this chapter should serve as a basic introduction to
model-driven engineering, metamodelling and graph
transformations.

One last thing -- at the moment we unfortunately only support Windows.
This should hopefully change in future releases.

That's it -- sit back, relax, grab a coffee and enjoy the ride!


