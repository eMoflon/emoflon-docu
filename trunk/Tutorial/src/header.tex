%
%
%Headerdatei der Diplomarbeit/Studienarbeit/Dissertation
%
%

\documentclass[accentcolor=tud1b,longdoc,bibliography=totoc,listof=totoc,oneside,nopartpage,openright,inverttitle,colorbacktitle,numbersubsubsec,noresetcounter,12pt]{tudreport}

% Babel-Paket f. neue deutsche Rechtschreibung
\usepackage[ngerman]{babel}

% Eingabekodierung auf latin1 setzen
\usepackage[latin1]{inputenc}
% Font-Encoding auf T1 setzen
\usepackage[T1]{fontenc}
% footmisc behebt u.a. Probleme mit Funoten in Abschnittstiteln
\usepackage[stable]{footmisc}

% Einbinden von Grafiken ermglichen
\usepackage{graphicx}

% Paket xtab ermglicht Umbrechen von langen Tabellen
\usepackage{xtab}

% picins erlaubt das Umflieen von Abbildungen durch Text
% Untenstehendes renewcommand behebt den picins-bug, dass Abbildungen
% nicht im Abbildungsverzeichnis auftauchen
\usepackage{picins}
\makeatletter
\renewcommand\piccaption{\@dblarg{\@piccaption}}
\makeatother

\usepackage{verbatim}

% Paket setspace erlaubt Umschalten auf 1.5fachen Zeilenabstand
\usepackage{setspace}

%%%%%%%%%%%%%%%%%%%%%%%%%%%%%%%%%%%%%%%%%%%%%%%%%%%%%%%%
%% Anpassungen von Literaturverzeichnis und Zitierweise%
%\usepackage[commabeforerest, authorformat=year, see, ibidem=strict, dotafter=bibentry]{jurabib}
%\AddTo\bibsgerman{%
%\renewcommand*{\ibidemname}{ebd.}
%\renewcommand*{\ibidemmidname}{ebd.}
%}
%Trennzeichen zwischen Autoren im Zitat
%\renewcommand*{\jbbtasep}{; }
%\renewcommand*{\jbbfsasep}{; }
%\renewcommand*{\jbbstasep}{; }

%Trennzeichen zwischen Autoren im Literaturverzeichnis
%\renewcommand*{\bibbtasep}{; }
%\renewcommand*{\bibbfsasep}{; }
%\renewcommand*{\bibbstasep}{; }

%Trennzeichen zwischen Herausgebern im Literaturverzeichnis
%\renewcommand*{\bibbtesep}{; }
%\renewcommand*{\bibbfsesep}{; }
%\renewcommand*{\bibbstesep}{; }

%Unterdrckt, dass bei mehr als drei Autoren im Literaturverzeichnis
%mit "et al." abgekrzt wird --> myjureco.bst-Datei wird zustzlich bentigt!
%\makeatletter
%\def\jb@use@fullcite{%
%\jbauthorfont{\jb@@author}\normalfont{\jbhowsepbeforetitle}\jb@@fulltitle}%
%\makeatother

%In: erscheint vor dem Titel von Zeitschriften, Konferenzbeitrgen, Sammelwerken
%und Herausgeberbnden
%\AddTo\bibsall{\def\incollinname{\textbf{In:}}}
%\renewcommand{\bibbtsep}{In: }
%\renewcommand*{\bibjtsep}{In: }

%Vor- und Nachname des Herausgebers werden nicht fett gedruckt
%\renewcommand*{\bibelnfont}{}
%\renewcommand*{\bibefnfont}{}

%ndert bei urldate das Prfix von "Zugriff am" zu "Abruf am"
%\AddTo\bibsgerman{\renewcommand*{\urldatecomment}{Abruf am }}

%Setzt ein Komma zwischen der URL und "Abruf am"
%\renewcommand*{\bibbudcsep}{, }

%Entfernt die Zeichen vor und nach der URL-Angabe im Literaturverzeichnis
%\renewcommand*{\biburlprefix}{}
%\renewcommand*{\biburlsuffix}{}

%Entfernt das Komma zwischen Jahrgang und Ausgabe
%und setzt die Ausgabe in Klammern
%\renewcommand*\artnumberformat[1]{\unskip\space (#1)}

%Entfernt das Komma zwischen Adresse und Verlag
%und setzt dafr ein Leerzeichen. Dies ist ntig,
%da die Reihenfolge von Adresse und Verlag in myjureco vertauscht wird
%und kein Zeichen nach der Adresse erscheinen soll.
%\renewcommand*\bpubaddr{}
%%%%%%%%%%%%%%%%%%%%%%%%%%%%%%%%%%%%%%%%%%%%%%%%%%%%%%%%

\usepackage{hyperref}
%Ermglicht Hyperlinks zwischen Textstellen und zu externen Dokumenten
%% breaklinks=true/false: Gibt an, ob Links umgebrochen werden drfen.
%% linktocpage=true/false: im Inhaltsverzeichnis sind nur die Seitenzahlen
%% links, nicht der Text
%% colorlinks=true/false: Links werden eingefrbt (Farben werden mit
%% linkcolor, anchorcolor \dots festgelegt)
%% linkcolor=Farbe: Farbe des verlinkten Textes, Dokument-interne Links
%% citecolor=Farbe: Farbe des verlinkten Textes, Links zum
%% Literaturverzeichnis
%% filecolor=Farbe: Farbe des verlinkten Textes, Links auf lokale Dateien
%% urlcolor=Farbe: Farbe des verlinkten Textes, externe URLs
%% frenchlinks=true/false: Links werden als smallcaps, anstatt farbig
%% dargestellt.
%% breaklinks=true/false: Gibt an, ob Links umgebrochen werden drfen.
\hypersetup{%
  linktocpage=true,
  breaklinks=true,
  colorlinks=true,
  citecolor=black,
  urlcolor=black,
  linkcolor=black,
  pdfpagemode=UseThumbs,
  pdftitle=Mustertitel,
  pdfauthor=Daniel Toegel,
  pdfsubject=Musterthema,
  %pdfkeywords=xy
}

% Anpassungen der Raender an die Vorgaben des Lehrstuhls
\geometry{left=3cm, right=2cm, top=2cm, bottom=2.5cm}

% PDF einbinden
\usepackage{pdfpages}

