\section{Executing a Model-To-Model Transformation}

Having specified a correspondence metamodel for our two languages in EA; we will generate code, do some hand-written completions and execute a model-to-model transformation in this chapter.

\begin{enumerate}
\item[$\blacktriangleright$] Export your metamodels by choosing ``\texttt{Extensions/\-MOFLON::\-Ecore Addin\-/Export all to Workspace}'' in EA and push \texttt{F5} on your Eclipse Metamodel project to trigger code generation.
\end{enumerate}

If you have done everything right while modelling your TGGs in EA, code generation terminates without any error and the structure of the \texttt{gen} folder in \texttt{LearningBox\-To\-Dictionary\-Integration} resembles Fig.~\ref{fig:gen_folder}.

\begin{figure}[htbp]
\begin{center}
  \includegraphics[width=\textwidth]{pics/tggBilder/transformation/tgg22}
  \caption{\texttt{LearningBox\-To\-Dictionary\-Integration/gen} after Code Generation}  
  \label{fig:gen_folder}
\end{center}
\end{figure} 

Due to our missing \texttt{indexToLevel} implementation, the class \texttt{Card\-To\-Entry\-Rule\-Impl} is marked as faulty.
In order to make our code compile, we will create the needed class by hand:

\begin{enumerate}
\item[$\blacktriangleright$] In \texttt{LearningBox\-To\-Dictionary\-Integration/src} create the package \texttt{csp.constraints} and add a new class \texttt{IndexToLevel}.
\item[$\blacktriangleright$] Implement the class with the code provided in Fig.~\ref{fig:indexToLevel}
\end{enumerate}

\begin{figure}[htbp]
\begin{center}
  \includegraphics[width=\textwidth]{pics/tggBilder/transformation/tgg23}
  \caption{\texttt{IndexToLevel} Class implementing our Attribute Constraint}  
  \label{fig:indexToLevel}
\end{center}
\end{figure} 

After providing an implementation for our self-defined attribute constraint \texttt{indexToLevel}, the integration project \texttt{LearningBox\-To\-Dictionary\-Integ\-ration} will compile. 
At this point, we recommend you to inspect the generated code, especially the classes with \texttt{Impl}-suffix, to get a feeling how the generated code performs a transformation.

In the next steps, we will instantiate\footnote{If you are not familiar with instantiating metamodels in Ecore, then refer to Chapter \ref{sect:instance}} one of our languages, transform it to an instance of the other language and watch the simultaneous evolution of the graph triple.

\begin{enumerate}
\item[$\blacktriangleright$] Open the \texttt{Dictionary\-Language/model/Dictionary\-Language.ecore} and create a dynamic instance of \texttt{Dictionary}, name it \texttt{Dictionary.xmi} and save it under \texttt{Learning\-Box\-To\-Dictionary\-Integration/instances} (Fig.~\ref{fig:create_instance_dict}).

\begin{figure}[htbp]
\begin{center}
  \includegraphics[width=0.5\textwidth]{pics/tggBilder/transformation/tgg24}
  \caption{Create a Dynamic Instance of \texttt{Dictionary}}  
  \label{fig:create_instance_dict}
\end{center}
\end{figure} 

\item[$\blacktriangleright$] Open \texttt{Dictionary.xmi}.
\item[$\blacktriangleright$] Set \texttt{English Numbers} as \texttt{Dictionary.title}
\item[$\blacktriangleright$] Create two \texttt{Entry} objects as child of \texttt{Dictionary}
\item[$\blacktriangleright$] Set \texttt{one : eins} as content and \texttt{beginner} as level of the first \texttt{Entry} (Fig.~\ref{fig:dictionaryxmi}).
\item[$\blacktriangleright$] Set \texttt{eleven : elf} as content and \texttt{advanced} as level of the second \texttt{Entry}.
\item[$\blacktriangleright$] Save \texttt{Dictionary.xmi}

\begin{figure}
\begin{center}
  \includegraphics[width=\textwidth]{pics/tggBilder/transformation/tgg26}
  \caption{Create an Instance of \texttt{DictionaryLanguage}}  
  \label{fig:dictionaryxmi}
\end{center}
\end{figure} 

\item[$\blacktriangleright$] Create a main class \texttt{TGGMain} under \texttt{LearningBox\-To\-Dictionary\-Integration\-/src} and implement it with the code provided in Fig.~\ref{fig:tggmain}.
\item[$\blacktriangleright$] Run the class.
\item[$\blacktriangleright$] Refresh the folder \texttt{LearningBox\-To\-Dictionary\-Integration/\-instances} and check the created source and correspondence models.

\begin{figure}[htbp]
\begin{center}
  \includegraphics[width=\textwidth]{pics/tggBilder/transformation/tgg25}
  \caption{\texttt{TGGMain} Class to perform a Transformation}  
  \label{fig:tggmain}
\end{center}
\end{figure} 


\item[$\blacktriangleright$] Open \texttt{LearningBox.xmi} in \texttt{instances} folder (Fig~\ref{fig:learningboxxmi}).

\begin{figure}
\begin{center}
  \includegraphics[width=\textwidth]{pics/tggBilder/transformation/tgg27}
  \caption{Generated Source Model \texttt{LearningBox.xmi}}  
  \label{fig:learningboxxmi}
\end{center}
\end{figure} 

\end{enumerate}

You will see that our \texttt{Dictionary} is translated to a \texttt{Box} with the same name (\texttt{English Numbers}) with three underlying \texttt{Par\-ti\-tions}, as we have specified in \texttt{Box\-To\-Dictionary\-Rule}.
You will also see that the two \texttt{Entry} objects are translatlated to \texttt{Card} objects as specified in \texttt{CartToEntryRule}.
The \texttt{face} and the \texttt{back} of the \texttt{Card}s are containing a prefix and the information from the \texttt{content} of the corresponding \texttt{Entry}, e.g. \texttt{card.face = ``Question : one''} and \texttt{card.back = ``Answer : eins''}. 
The \texttt{Partition} \texttt{index}es conform also to the \texttt{Entry level}, f.i. \texttt{0} for \texttt{beginner} and \texttt{1} for \texttt{advanced}.
Being aware of the triple graph structure, we recommend you also to have a look at the created corresponcence model \texttt{CorrespondenceModel.xmi}.  

Congratulations! You have done your first model-to-model transformation with TGGs using our MOFLON tool. 
That was a \emph{backward} transformation from \texttt{Dictionary\-Language} (target model) to \texttt{Learning\-Box\-Language} (source model). 
You have used hand-written Java code \texttt{TGGMain} to invoke the generated TGG implementation. 
But we provide you also with an integration environment, in which you can manipulate the source and target models and check the simultaneous evolution of the graph triple in a visual way.

(\ldots Use Integrator\ldots) 
