\subsection{Emptying a partition of all its cards}
\label{sec:empty}

The next SDM we shall specify should \emph{empty} a partition of all its cards,
deleting the cards in the process.  To do this we obviously need a construct for
repeating the action for all cards in the partition.  In SDM, this is
\note{For Each}
accomplished via a \emph{For Each} story node.  A for each story node performs
the specified actions for \emph{every} match of its story pattern.  To create a
for each story node, create the initial diagram and start node for the method
\texttt{Partition::empty} and quick create an activity node choosing
\texttt{ForEach} as its type (Fig.~\ref{fig:sdm_foreach}).

\begin{figure}[htbp]
\begin{center}
  \includegraphics[width=0.8\textwidth]{pics/sdmBilder/empty/sdm42RAW}
  \caption{A for each loop in SDM.}  
  \label{fig:sdm_foreach}
\end{center}
\end{figure}

A for each story node is visualised as a double node to indicate the
potential repetition (Fig.~\ref{fig:sdm_end}). Complete the story pattern as
indicated in Fig.~\ref{fig:sdm_end}.  Please note that the \texttt{card} that is
deleted in each match is unbound and both the \texttt{card} and link to
\texttt{this} are set to \texttt{destroy}.  Even more important, note that the
guard that terminates the for each story node has an \texttt{[end]} guard. 
Indeed, a for each story node \emph{must} have an end activity edge which is
taken when all matches for the story pattern have been handled.

There are two interesting points to note: First of all, how would the pattern be
interpreted if the story node where a normal story node and not a for each?
\note{Non-Determinism}  
Well, the pattern would specify that \emph{a} card should
be matched and deleted from the current partition.  Note that the \emph{exact}
card is not specified and indeed the actual choice of the card is
\emph{non-deterministic} or random.  This is a common property of graph
transformations and pattern matching and is something that takes some getting
used to. In general, there are no guarantees concerning the choice and order of
valid matches.

\begin{figure}[htbp]
\begin{center}
  \includegraphics[width=0.9\textwidth]{pics/sdmBilder/empty/sdm47}
  \caption{Complete story pattern with \texttt{[end]} guard.}  
  \label{fig:sdm_end}
\end{center}
\end{figure}

The second point is if we need to destroy the link between \texttt{this} and
\texttt{card}.  Would the pattern be interpreted differently if we just
destroyed \texttt{card} and left the link?  The answer is no, the pattern would
yield the same result, regardless of if the link is explicitly destroyed or not.
This is because the transformation engine we use\footnote{CodeGen2 which is
part of Fujaba \url{http://www.fujaba.de/}} ensures that there are never any
\note{Dangling Edges}
\emph{dangling edges} in a model.  As deleting only \texttt{card} would result
in a ``dangling edge'' attached to \texttt{this}, the link is deleted as well.
Explicitly destroying the link or not is therefore a matter of taste, but \ldots
why not be as explicit as possible?
