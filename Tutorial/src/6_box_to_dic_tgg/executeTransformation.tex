\section{TGGs in Action}
In this section, we shall export our TGG, implement our new constraint, and get our integration running!
 
\begin{enumerate}
\item[$\blacktriangleright$] Export your metamodels and TGG by choosing ``\texttt{Extensions/\-MOFLON::\-Ecore Addin\-/Export all to Workspace}'' as usual in EA and refreshing your Eclipse Metamodel project to trigger code generation.
\end{enumerate}

If you have done everything right, code generation should terminate without any error and the structure of the \texttt{gen} folder in \texttt{LearningBox\-To\-Dictionary\-Integration} should resemble Fig.~\ref{fig:gen_folder}.

\begin{figure}[htbp]
\begin{center}
  \includegraphics[width=0.75\textwidth]{pics/tggBilder/transformation/tgg22}
  \caption{Integration project after code generation}  
  \label{fig:gen_folder}
\end{center}
\end{figure} 

The Java compilation error in \texttt{Card\-To\-Entry\-Rule\-Impl} is due to our missing \texttt{indexToLevel} implementation, and just means we have to implement the required class by hand.

\begin{enumerate}
\item[$\blacktriangleright$] In \texttt{LearningBox\-To\-Dictionary\-Integration/src} create the Java package \texttt{csp.constraints} and add a new public class \texttt{IndexToLevel}.
\item[$\blacktriangleright$] Implement the class with the code provided in Fig.~\ref{fig:indexToLevel}.
If you take a look at the code it should be pretty clear how the declared adornments are checked and handled appropriately.
With this class, the generated code should now compile without any further errors.
\end{enumerate}

\begin{figure}[htbp]
\begin{center}
  \includegraphics[height=0.71\textheight]{pics/tggBilder/transformation/tgg23}
  \caption{Implementation of our attribute constraint}  
  \label{fig:indexToLevel}
\end{center}
\end{figure} 

In the next steps, we will instantiate\footnote{If you are not familiar with instantiating metamodels in Ecore, then refer to Chapter \ref{sect:instance}} one of our languages, transform it to an instance of the other language and watch the simultaneous evolution of the graph triple.

\begin{enumerate}
\item[$\blacktriangleright$] Open the \texttt{Dictionary\-Language/model/Dictionary\-Language.ecore} and create a dynamic instance of \texttt{Dictionary}, name it \texttt{Dictionary.xmi} and save it under \texttt{Learning\-Box\-To\-Dictionary\-Integration/instances} (Fig.~\ref{fig:create_instance_dict}).

\begin{figure}[htbp]
\begin{center}
  \includegraphics[width=0.5\textwidth]{pics/tggBilder/transformation/tgg24}
  \caption{Create a Dynamic Instance of \texttt{Dictionary}}  
  \label{fig:create_instance_dict}
\end{center}
\end{figure} 

\item[$\blacktriangleright$] Open \texttt{Dictionary.xmi}.
\item[$\blacktriangleright$] Set \texttt{English Numbers} as \texttt{Dictionary.title}
\item[$\blacktriangleright$] Create two \texttt{Entry} objects as child of \texttt{Dictionary}
\item[$\blacktriangleright$] Set \texttt{one : eins} as content and \texttt{beginner} as level of the first \texttt{Entry} (Fig.~\ref{fig:dictionaryxmi}).
\item[$\blacktriangleright$] Set \texttt{eleven : elf} as content and \texttt{advanced} as level of the second \texttt{Entry}.
\item[$\blacktriangleright$] Save \texttt{Dictionary.xmi}

\begin{figure}[htbp]
\begin{center}
  \includegraphics[width=\textwidth]{pics/tggBilder/transformation/tgg26}
  \caption{Create an Instance of \texttt{DictionaryLanguage}}  
  \label{fig:dictionaryxmi}
\end{center}
\end{figure} 

\item[$\blacktriangleright$] Create a main class \texttt{TGGMain} under \texttt{LearningBox\-To\-Dictionary\-Integration\-/src} and implement it with the code provided in Fig.~\ref{fig:tggmain}.
\item[$\blacktriangleright$] Run the class.
\item[$\blacktriangleright$] Refresh the folder \texttt{LearningBox\-To\-Dictionary\-Integration/\-instances} and check the created source and correspondence models.

\begin{figure}[htbp]
\begin{center}
  \includegraphics[width=\textwidth]{pics/tggBilder/transformation/tgg25}
  \caption{\texttt{TGGMain} Class to perform a Transformation}  
  \label{fig:tggmain}
\end{center}
\end{figure} 


\item[$\blacktriangleright$] Open \texttt{Box.xmi} in \texttt{instances} folder (Fig~\ref{fig:boxxmi}).

\begin{figure}[htbp]
\begin{center}
  \includegraphics[width=\textwidth]{pics/tggBilder/transformation/tgg27}
  \caption{Generated Source Model \texttt{Box.xmi}}  
  \label{fig:boxxmi}
\end{center}
\end{figure} 

\end{enumerate}

You will see that our \texttt{Dictionary} is translated to a \texttt{Box} with the same name (\texttt{English Numbers}) with three underlying \texttt{Par\-ti\-tions}, as we have specified in \texttt{Box\-To\-Dictionary\-Rule}.
You will also see that the two \texttt{Entry} objects are translatlated to \texttt{Card} objects as specified in \texttt{CartToEntryRule}.
The \texttt{face} and the \texttt{back} of the \texttt{Card}s are containing a prefix and the information from the \texttt{content} of the corresponding \texttt{Entry}, e.g. \texttt{card.face = ``Question : one''} and \texttt{card.back = ``Answer : eins''}. 
The \texttt{Partition} \texttt{index}es conform also to the \texttt{Entry level}, f.i. \texttt{0} for \texttt{beginner} and \texttt{1} for \texttt{advanced}.
Being aware of the triple graph structure, we recommend you also to have a look at the created corresponcence model \texttt{CorrespondenceModel.xmi}.  

Congratulations! You have done your first model-to-model transformation with TGGs using our MOFLON tool. 
That was a \emph{backward} transformation from \texttt{Dictionary\-Language} (target model) to \texttt{Learning\-Box\-Language} (source model). 
Now, you can also edit the source model and transform it \emph{forward}:

\begin{enumerate}
\item[$\blacktriangleright$] Open \texttt{Box.xmi} and create different \texttt{Card} objects in the \texttt{Partition}s (for example, a new \texttt{Card} with \texttt{Card.face = ``Question : two''}, \texttt{Card.back = ``Answer : zwei''} in \texttt{Partition 0}). 

\item[$\blacktriangleright$] Adjust \texttt{TGGMain.java} with the code provided in Fig.~\ref{fig:tggmainforward} and run the class again.

\begin{figure}[htbp]
\begin{center}
  \includegraphics[width=\textwidth]{pics/tggBilder/transformation/tgg28}
  \caption{\texttt{TGGMain} for\emph{forward} transformation}  
  \label{fig:tggmainforward}
\end{center}
\end{figure} 

\end{enumerate}

You will see that your newly created \texttt{Card} objects are transformed to \texttt{Entry} objects in \texttt{Dictionary.xmi}.
That was your first forward transformation using the same generated code!

The last thing we want to show you in this chapter is our integration environment which provides you a visualisation of the graph triple in a model-to-model transformation.

\begin{enumerate}
\item[$\blacktriangleright$] Do a right click on the correspondence model \texttt{Correspondence\-Model\-.xmi} and choose \texttt{Start Integrator} from the \texttt{eMoflon} context menu (Fig.~\ref{fig:startintegrator}).

\begin{figure}[htbp]
\begin{center}
  \includegraphics[width=\textwidth]{pics/tggBilder/transformation/tgg29}
  \caption{Starting the Integrator}  
  \label{fig:startintegrator}
\end{center}
\end{figure} 
\end{enumerate}

The eMoflon-Integrator window will be opened which shows you the objects from the source and target model in a treeview and visualises the links between them as depicted in Fig.~\ref{fig:emoflonintegrator}.

\begin{figure}[htbp]
\begin{center}
  \includegraphics[width=\textwidth]{pics/tggBilder/transformation/tgg30}
  \caption{eMoflon Integrator}  
  \label{fig:emoflonintegrator}
\end{center}
\end{figure} 

