\genHeader
\requiredTime{3h}

Welcome to Part III, an introduction to unidirectional model transformations with programmed graph transformations via Story Driven Modelling (SDM).
SDMs are used to describe behaviour, so the plan is to implement the methods declared in Part II with story diagrams. In other words,
this is where you'll complete your metamodel's dynamic semantics! Don't let the sheer size of this part frighten you off. We have included thorough
explanations (with an ample number of figures) to ensure the concepts are crystal clear.

In Part II, we learnt that we can implement methods in a fairly straightforward manner with injections and Java, so why bother with SDMs?

Overall, SDMs are a simpler, pattern-based way of specifying behavior. Rather than writing verbose Java code yourself, you can model each method and generate
the corresponding code. With the visual syntax, you'll be using familiar, easy-to-understand UML activity and object diagrams to establish your methods.
Textually, SDMs employ a simple Java-like syntax for imperative control flow, and a declarative pattern language with a syntax similar to list pattern matching
constructs common in many functional programming languages

If you're just joining us, read the next section for a brief overview of our running example so far, and how to download some files that will help you get
started right away. Alternatively, if you've just completed Part II, click the link below to continue right away with your constructed learning box metamodel.

\begin{center}\texttt{$\triangleright$ \hyperlink{explanation}{Continue from Part II\ldots}}\end{center}
