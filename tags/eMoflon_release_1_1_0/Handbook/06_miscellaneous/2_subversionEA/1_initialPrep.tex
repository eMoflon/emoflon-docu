\subsection{Initial preparation and setup}

Download and install Slik SVN (mandatory):
\begin{enumerate}
  \item[$\blacktriangleright$] x32: \small{\url{http://www.sliksvn.com/pub/Slik-Subversion-1.7.6-win32.msi}}\\\\
   x64: {\small \url{http://www.sliksvn.com/pub/Slik-Subversion-1.7.6-x64.msi}}
\end{enumerate}

For public/private key authentication, you also need Tortoise SVN:

\begin{enumerate}
  \item[$\blacktriangleright$] x32: {\small \begin{minipage}{.95\textwidth} 
  \url{http://sourceforge.net/projects/tortoisesvn/files/1.7.9/Application/TortoiseSVN-1.7.9.23248-win32-svn-1.7.6.msi/download}
    \end{minipage}}\\\\\\
  x64: {\small\begin{minipage}{.9\textwidth} 
  \url{http://downloads.sourceforge.net/project/tortoisesvn/1.7.9/Application/TortoiseSVN-1.7.9.23248-x64-svn-1.7.6.msi/download}\end{minipage}}
\end{enumerate}

If you do not want to have your private key password in plain text in an SVN configuration file, then also download Pageant:
\begin{enumerate}
  \item[$\blacktriangleright$] {\small \url{http://the.earth.li/~sgtatham/putty/latest/x86/pageant.exe}}
\end{enumerate}

With all of the tools installed, we now have to setup the SSH tunnel:

\begin{enumerate}
  \item[$\blacktriangleright$] Locate the file \texttt{\%APPDATA\%$\backslash$Subversion$\backslash$config} and open it with your favourite editor. Locate the
  \texttt{[tunnels]} section.
  \item[$\blacktriangleright$] If you do not want to install Pageant and do not mind entering your password in plain text enter the following command:\\
  \texttt{ssh = "<path/to/Tortoise/SVN>/bin/TortoisePlink.exe" -l \\<username> -pw <password for your private key> -i "<path to your private key>"}
  \item[$\blacktriangleright$] If you wish to use Pageant then the command can be simplified to:\\ \texttt{ssh = "<path/to/Tortoise/SVN>/bin/TortoisePlink.exe"
  -l \\<username>} as you can add your private key to Pageant.
  \item[$\blacktriangleright$] If you just use direct passwords for authentication then you can leave out the \texttt{-i} option in both cases.
\end{enumerate}
