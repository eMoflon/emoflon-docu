\newpage

\subsection{Configuration for code generation in Eclipse}
\genHeader

Since there is already generated code for the existing \texttt{GenModel} metamodel (provided via the Eclipse plugin), we do \emph{not} want to export our
incomplete subset of \texttt{GenModel} from EA. Instead, we need to configure Eclipse to access the elements specified in our partial metamodel from the
complete metamodel.

\begin{enumerate}

\item[$\blacktriangleright$] In EA, right-click your \texttt{GenModelLanguage} package and select ``Properties\ldots'' 

\item[$\blacktriangleright$] Navigate to ``Properties/Moflon'' in the dialogue window and update the tagged \texttt{Moflon::Export} value to \texttt{false}
(Fig.~\ref{fig_customNS}).

\vspace{0.5cm}

\begin{figure}[htb]
\begin{center}  \includegraphics[width=\textwidth]{ea_genModelExportFalse}
  \caption{Update the \texttt{GenModel} export option and create custom tags}  
  \label{fig_customNS}
\end{center}
\end{figure}

\newpage

\item[$\blacktriangleright$] Next we have to set the ``real'' name and URI of the project to be used in Eclipse so that the relevant references are exported
properly. In the same window, create new tagged values \texttt{Moflon::CustomNsPrefix} and \texttt{Moflon::CustomNsUri}.

\item[$\blacktriangleright$] Set their values to \texttt{genmodel} and \texttt{http://\-www.\-eclipse.\-org/\-emf/\-2002/\-GenModel} respectively, as shown in
Fig.~\ref{fig_customNS}. These values can be determined by inspecting the corresponding values in the existing .ecore file (i.e.,~the existing metamodel).

\item[$\blacktriangleright$] Validate and export all projects as usual to your Eclipse workspace, and update the metamodel project by pressing \texttt{F5} in
the package explorer.

\item[$\blacktriangleright$] In order to simplify setting the required dependencies for code generation,  convert the generated Eclipse project
\texttt{Ecore2GenModel} to a \emph{plug-in project} by right-clicking the project and selecting ``Configure/Convert to Plug-in Projects...''

\item[$\blacktriangleright$] Right-click \texttt{Ecore2GenModel} once more and navigate to ``Plug-in Tools/Open Manifest.'' The plug-in manager should have
opened in the editor with a series of tabs at the bottom for each option.

\item[$\blacktriangleright$] Switch to the \texttt{Dependencies} tab. Press \texttt{Add} and enter \texttt{org.\-eclipse.\-emf.\-codegen.\-ecore}. This plug-in
includes both the \texttt{Ecore} and \texttt{Gen\-Mod\-el} libraries we require.

\end{enumerate}

Although we have already specified the name and URI of the existing project (in this example, \texttt{GenModel}) as tagged project values, we now have to tell
eMoflon where to find the correct implementation (generated code) of the existing project.

\begin{enumerate}
  
\item[$\blacktriangleright$] Expand the \texttt{Ecore2GenModel} project folder and open the \texttt{mof\-lon.\-prop\-er\-ties.\-xmi} file tree. Right-click the
properties container, and create a new \texttt{Add\-it\-ion\-al Dep\-en\-den\-cies} child. Double click the element to open its properties tab below the
editor, and as shown in Fig.~\ref{eclipse:addDepChild}, update its \texttt{Value} to:\\
\end{enumerate}

\vspace{-1cm}
{\small \ttfamily  platform:/plugin/org.eclipse.emf.codegen.ecore/model/GenModel.ecore} \\
\vspace{-0.5cm}

\begin{figure}[htbp]
\begin{centering}
\includegraphics[width=\textwidth]{eclipse_additDepProps}
  \caption{Setting properties for the correct implementation code}  
  \label{eclipse:addDepChild}
\end{centering}
\end{figure} 

\begin{enumerate}

\item[$\blacktriangleright$] Similarly, add a second \texttt{Additional Used Gen Packages} child and set its value to: \\
\end{enumerate}

\vspace{-1cm}
{\small \ttfamily platform:\-/\-plugin/\-org.\-eclipse.\-emf.\-codegen.\-ecore/\-model/\-GenModel.\-genmodel}

\newpage
Finally, to compenstate for some cases where our naming conventions were violated, analogously add the following mapping as corrections:

\begin{enumerate}
\item[$\blacktriangleright$] Add an \emph{import mapping} child for correct generation of the required import, setting the key as \texttt{genmodel} (as
depicted in Fig.~\ref{eclipse:impMapValues}) and value to: \\
{\small \ttfamily \hspace*{1.5cm} org.\-eclipse.\-emf.\-codegen.\-ecore.\-genmodel}

\vspace{0.5cm}

\begin{figure}[htbp]
\begin{centering}
\includegraphics[width=0.9\textwidth]{eclipse_importMappingValues}
  \caption{Mapping properties from our metamodel to the existing \texttt{GenModel}}  
  \label{eclipse:impMapValues}
\end{centering}
\end{figure} 

\item [$\blacktriangleright$] Finally, add a \emph{factory mapping} to ensure that \texttt{GenModelFactory} is used as the factory for creating elements in the
transformation instead of \texttt{Genmodel\-Factory}, which would be the default convention. Set its key as \texttt{genmodel}, and its value to:
{\small \ttfamily GenModelFactory}.

\item [$\blacktriangleright$] Your completed \texttt{moflon.properties.xmi} file should now closely resemble Fig.~\ref{eclipse:finalPropTree}. Refresh your
workspace one more time to generate code for the project and ensure that the transformation behaves as expected via a JUnit test.

\newpage

\vspace*{2cm}

\begin{figure}[htbp]
\begin{centering}
\includegraphics[width=\textwidth]{eclipse_finalPropTree}
  \caption{Additional properties for code generation}  
  \label{eclipse:finalPropTree}
\end{centering}
\end{figure} 

\end{enumerate}
