\newpage
\texHeader

\hypertarget{validation tex}{} 
\subsection{eMoflon support with the MOSL Builder }

Our MOSL language is accompanied with its own builder that proves support for validating your metamodel.
Integrated with the Eclipse IDE, if there's an error in your files when you save, a message will appear in the console. Lets try to make an error, just to
see how this works.

Go to your \texttt{Partition} EClass and change the parameter type in \texttt{removeCard} from \texttt{Card} to \texttt{card}. Press save. An error should
immediately appear below the editor to inform you of your terrible mistake. Change your file back to the way it was, and the message should disappear.

In addition to validation on saving files, MOSL also lets you know whether or not your project has
recently been built (if code has been generated from the metamodels). You may have noticed this feature in the previous sections. Similar to how Eclipse
informs you that your file has changed by placing a \texttt{*} beside the name in the editor tab, MOSL places a \texttt{***} symbol beside your metamodel folder
title in the package explorer to remind you to build your project.
