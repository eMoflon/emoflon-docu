\newpage
\phantomsection
\addcontentsline{toc}{section}{Glossary}
\hypertarget{glossary}{}

\vspace{1cm}
{\Huge \bf Glossary}
\vspace{1cm}

\begin{description}

\item[\bf Activity]
Top-most element of an SDM.

\item[\bf Activity Edge]
A directed connection between activity nodes describing the control flow within an activity.

\item[\bf Activity Node]
Represents atomic steps in the control flow of an SDM. Can be either a story node or statement node.

\item[\bf Assignments]
Used to set attributes of object variables.

\item[\bf Attribute Constraint]
A non-structural constraint that must be satisfied for a story pattern to match. Can be either an assertion or assignment.

\item[\bf Binding State]
Can be either \emph{bound} or \emph{unbound/free}. See \emph{Bound vs Unbound}.

\item[\bf Binding Operator]
Determines whether a variable is to be \emph{checked}, \emph{created}, or \emph{destroyed} during pattern matching.

\item[\bf Binding Semantics]
Determines if an object variable \emph{must} exist (\emph{mandatory}), may not exist (\emph{negative}; see \emph{NAC}), or is \emph{optional} during
\emph{pattern matching}.

\item[\bf Bound vs Unbound]
Bound variables are completely determined by the current context, whereas unbound (free) variables have to be determined by the \emph{pattern matcher}.
\texttt{this} and parameter values are always bound.

\item[\bf Dangling Edges]
An edge with no target or source. Graphs with dangling edges are invalid, which is why dangling edges are avoided and automatically deleted by the pattern
matching engine.

\item[\bf EA]
Enterprise Architect; The UML visual modeling tool used as our visual frontend.

\item[\bf Edge Guards]
Refine the control flow in an activity by guarding activity edges with a condition that must be satisfied for the activity edge to be taken.

\item[\bf Link Variable]
Placeholders for links between matched objects.

\item[\bf Literal Expression]
Represents literals such as true, false, 7, or ``foo.'' See \hyperlink{expressionReview}{Section 12}.

\item[\bf MethodCallExpression]
Used to invoke any method. See \hyperlink{expressionReview}{Section 12}.

\item[\bf NAC]
Negative Application Condition; Used to specify structures that must not be present for a rule to be applied.
	
\item[\bf Object Variable]
Place holders for actual objects in the current model to be determined during pattern matching.

\item[\bf ObjectVariableExpression]
Used to reference other object variables. See \hyperlink{expressionReview}{Section 12}.

\item[\bf Parameter Expression]
Used to refer to method parameters. See \hyperlink{expressionReview}{Section 12}.

\item[\bf (Graph) Pattern Matching]
Process of assigning objects and links in a model to the object and link variables in a pattern in a type conform manner. This is also referred to as finding a
match for the pattern in the given model.

\item[\bf Statement Node]
Used to invoke methods as part of the control flow in an activity.

\item[\bf Story Node]
\emph{Activity node} that contains a \emph{story pattern}.

\item[\bf Story Pattern]
Specifies a structural change of the model.

\item[\bf Unification]
An extension of the Object Oriented ``Everything is an object'' principle, where everything is regarded as a \emph{model}, even the metamodel which defines
other models.

\end{description}
