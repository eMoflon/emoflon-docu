\genHeader
\section{Conclusion and next steps}

\vspace{0.5cm}

Congratulations -- you've reached the end of eMoflon's introduction to unidirectional model transformations! You've learnt that SDMs are declared as
\emph{activities}, which consist of \emph{activity nodes}, which are either \emph{story pattern}s or \emph{statement nodes} (for method calls).
\emph{Patterns} are made up of \emph{object} and \emph{link variables} with appropriate attribute constraints. Each of these variables can be given
different \emph{binding states}, binding operators, and can be marked as negative (for expressing NACs).

\vspace{0.5cm}

To further test your amazing story driven modelling skills, challenge yourself by:
\begin{itemize}
\item Adjusting \texttt{check} to eject the card from the box if it is guessed correctly and contained in the last partition (to signal it's been learnt). Do
you know how \texttt{check} currently handles this?
\item Editing \texttt{Partition.empty()} to include a method call to \texttt{removeCard}, thus reusing this previous SDM
\item Modifying the GUI source files to execute all methods
\end{itemize}

\vspace{0.5cm}
	
If you have any comments, suggestions, or concerns for this part, feel free to drop us a line at \href{mailto:contact@moflon.org}{contact@moflon.org}.
Otherwise, if you enjoyed this section, continue to Part IV to learn about Triple Graph Grammars, or Part V for Model-to-Text Transformations.
The final part of this handbook -- Part VI: Miscellaneous -- contains a full glossary, eMoflon hotkeys, and tips and tricks in EA which you might find
useful when creating SDMs in the future.

For more detailed information on each part, please refer to Part 0, which can be downloaded at \dlPartZero.
\vspace{0.5cm}

Cheers!
